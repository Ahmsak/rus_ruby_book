% Преамбула.
\documentclass[a4paper, 12pt, oneside, openany, book]{ncc}

\ChapterPrefixStyle{header,toc}
\PnumPrototype{999}
\usepackage[headings]{ncchdr} % Линейка в колонтитуле для стиля headings

\usepackage{rus_ruby_book}

% Тело документа.
\begin{document}

\author{Алекcандр Круглов}
\title{Ruby}
\date{\today}
\bookeditor{Ruby 1.9.3p194}
\maketitle

\frontmatter
  \chapter{От автора}

Когда я начинал изучение языка программирования Ruby, то был восхищен его красотой и богатством предоставляемых возможностей. 

К сожалению, на тот момент, существовало мало русскоязычных ресурсов, на которых была бы собрана исчерпывающая информация об его использовании. В этой книге я попытался объединить все те знания, почерпнутые из многочисленных статей и книг, которые получил в процессе обучения. 

Хотя моим намерением было написать полноценный учебник, на данном этапе, полученный результат может быть использован в основном в качестве справочника. Продвигаясь дальше я постараюсь своевременно обновлять и дополнять собранные данные, чтобы создать книгу, которая облегчит использование особенностей Ruby.

Связаться с автором можно по \href{mr.krugloff@gmail.com}{\underline{электронной почте}}. 

Исходный текст книги доступен по \href{http://github.com/Krugloff}{\underline{адресу}}. Вы можете оставить там свои замечания и исправления.

Если вам понравился результат или вы хотите поспособствовать развитию книги, вы можете сделать это с помощью WebMoney: R349517838989 или Qiwi: 89212870782. 
  \tableofcontents

\mainmatter

\part{Основы}
  \chapter{Начальные сведения}

\section{Краткое описание языка}

\epigraph
{Во имя эффективности - причем достигается она далеко не всегда - совершается больше компьютерных грехов, чем по любой другой причине, включая банальную глупость.}
{W. A. Wulf}

Язык программирования - это искусственно созданный язык, облегчающий управление компьютером. Текст, написанный с помощью языка программирования называется кодом.

Язык программирования Ruby начал разрабатываться 24 февраля 1993 года и вышел в свет в 1995. Создатель языка - Юкихиро Мацумото (Matz).
\begin{note}
  Я хотел получить скриптовый язык, который был бы мощнее чем Perl, и объектно-ориентированнее чем Python. Вот почему я решил разработать свой язык.
\end{note}

В противовес машинно-ориентированным языкам, работающим быстрее, целью Мацумото был язык, наиболее близкий к человеку. Основной целью Ruby он считает предоставление возможности программистам наслаждаться их работой.
\begin{note}
  Зачастую люди, в особенности компьютерные инженеры, фокусируют внимание на машинах. Они думают: "Сделаю это и машина будет работать быстрее. Сделаю то и машина будет работать эффективнее. Сделаю третье и машина будет пятое и десятое. Они помешаны на машинах. В действительности, необходимо больше думать о людях. О том чтобы людям было удобно создавать программы и пользоваться приложениями. Мы мастера. Они рабы.
\end{note}

Ruby позволяет создавать программы тем способом, который удобнее всего. Это подразумевает наличие нескольких путей выполнить одно и тоже действие, ведь и люди, хоть и схожие в целом имеют множество мелких различий. В том числе возможно применение нескольких парадигм программирования: процедурного, объектно-ориентированного и функционального.

Ruby освобождает пользователей от рутинной работы, которую компьютер может выполнять быстрее и качественнее. Особое внимание, в частности, уделялось будничным рутинным занятиям  (обработка текстов, администрирование), и для них язык подходит особенно хорошо. Гибкий и мощный синтаксис позволяет создавать программы, использующие термины предметной области (DSL).

\begin{note}
  Когда распространение языка только начиналось, очень часто его упрекали за низкую скорость. В последних версиях скорость выполнения была значительно увеличена. Обычно говорят, что Ruby достаточно быстр - скорость выполнения компенсируется скоростью разработки. Медленный код при этом принято переписывать на Си. Си API - одна из самых полезных и удобных особенностей языка.
\end{note}

Особенности Ruby:
\begin{itemize}
  \item высокоуровневый;
  \item интерпретируемый;
  \item объектно-ориентированный;
  \item динамическая неявная типизация.
\end{itemize}

Каждый пункт подробно описан в первой части этой книги.

Интерпретатор - это программа, переводящая код в машинные команды, понятные компьютеру.

\begin{keylist}{Интерпретаторы:}  
  \firstkey{MRI} - официальный интерпретатор, написанный на языке программирования Си с использованием виртуальной машины YARV (преимущество виртуальных машин в том, что код сначала интерпретируется полностью, и только затем выполняется).

  В этой книге описываются особенности работы именно этого интерпретатора;
  
  \key{Rubinius} - сторонняя реализация виртуальной машины. Написана как с помощью языка программирования C++, так и самого Ruby;
  
  \key{JRuby} - реализация языка для взаимодействия с виртуальной машиной Java;
  
  \key{IronRuby} - реализация языка для взаимодействия с платформой .Net.
\end{keylist}

Официальный \href{http://www.ruby-lang.org}{\underline{сайт}} языка. На нем можно узнать последние новости разработки интерпретатора, скачать исходный код и перейти к документации.

На этом \href{http://rubyinstaller.org}{\underline{сайте}} пользователи ОС Windows могут скачать установочный файл для своей операционной системы. 

В состав интерпретатора входят:
\begin{itemize}
  \item стандартная библиотека (наиболее часто используемые модули); 
  \item менеджер пакетов RubyGems; 
  \item интерактивный терминал irb (выполняет код на Ruby в режиме реального времени); 
  \item генератор документации RDoc; 
  \item программа ri для просмотра документации; 
  \item менеджер задач Rake; 
  \item шаблонизатор ERb.
\end{itemize} 

\section{Краткое описание кода}

Программа на Ruby - это код, сохраненный в текстовом файле с определенным расширением. Расширение - это группа символов, следующая за именем файла (после точки). Оно помогает компьютеру определить язык, на котором написана программа.

Для языка Ruby используются два расширения: \mono{.rb} (стандартное расширение) и \mono{.rbw} (используется в Windows для программ с графическим интерфейсом).

Любой интерпретатор понимает лишь код, написанный по заранее определенным правилам. Последовательность символов, имеющая смысл для интерпретатора называется лексемой. Лексические правила регулируют определение существующих лексем в коде. Встретив в тексте набор символов, не относящийся к известным лексемам, интерпретатор завершит обработку кода и вернет сообщение об ошибке.

\itemtitle{Типы лексем:}
\begin{itemize}
  \item элементарные типы данных - простейшие данные (числа, буквы, логические величины);

  \item идентификаторы - лексемы, использующиеся в двух случаях: для сохранения результатов выполнения выражения и для пометки различных синтаксических структур. И то и другое необходимо для повторного применения кода или данных. Лексема идентификатора - это группа символов состоящая из букв, цифр и знаков подчеркивания (\mono{_}). При этом идентификатор не может начинаться с цифры.

  Идентификаторы чувствительны к регистру. Интерпретатор по разному распознает строчные и прописные ASCII символы (принято использовать для идентификаторов именно ASCII символы, хотя в некоторых случаях это необязательное требование). Два идентификатора считаются идентичными только в том случае, если они состоят из одинакового набора байт;

  \item операторы - это знаки препинания, использующиеся в качестве разделителей и математичсекие символы, позволяющие выполнять различные вычисления;

  \item инструкции - слова, зарезервированные языком программирования. Их переопределение невозможно, а использование приведет к заранее определенному результату. Инструкции используется для создания различных синтаксических структур или для управления процессом выполнения программы.

  Список инструкций: __LINE__~; __ENCODING__~; __FILE__~; __END__~; BEGIN; END; =begin; =end; alias; and; begin; break; case; class; def; defined?; do; else; elsif; end; ensure; false; for; if; in; module; next; nil; not; or; redo; rescue; retry; return; self; super; then; true; undef; unless; until; when; while; yield;

  \item комментарии - текст на естественном языке, поясняющий процесс выполнения программы.
\end{itemize}

Из лексем создаются выражения, а из выражений - сложные выражения и предложения. Любая синтаксическая конструкция относится к выражениям. Вся программа может рассматриваться как последовательный набор выражений.

Выражение - это синтаксическая единица, возвращающая результат после выполнения. Простые выражения могут быть объединены в сложные, с помощью операторов, или даже в целые предложения, с помощью инструкций. Синтаксические правила регулируют использование выражений в коде. Нарушение синтаксических правил также приводит к завершению обработки кода интерпретатором. Синтаксис языка также содержит описание доступных видов выражений и предложений.

\begin{note}
  Кроме жестких правил существуют также соглашения, принятые в сообществе. Они необязательны, но крайне желательны к выполнению. Следования правилам позволяет понимать ваш код компьютеру, а следование соглашениям облегчает его понимание для людей.
\end{note}

\subsection{Синтаксис выражений}

Для упрощения разработки в Ruby присутствует множество синтаксического сахара. Синтаксический сахар - общее название дополнений (дополнительные выражения) к синтаксису языка, которые делают его использование более удобным, но не добавляют новых возможностей.

Минимальный набор синтаксических правил и соглашений описывает общие особенности употребления выражений в коде.

\itemtitle{Соглашения:}
\begin{itemize}
  \item Код принято разбивать на строки. Каждая строка обычно не превышает 80 символов (это облегчает чтение кода);

  \item На строках обычно располагаются отдельные выражения (простые или сложные). Выражение обычно составляют таким образом, чтобы результат его выполнения мог быть использован несколько раз.

  \item Вложенность выражений принято оформлять двумя пробелами.
\end{itemize}

\itemtitle{Синтаксические правила:}
\begin{itemize}
  \item Для разделения отдельных выражений используется символ перевода строки (невдимый символ, добавляемый текстовым редактором автоматически, при нажатии клавиши \mono{ENTER});

  \item Для разделения выражений на одной строке используется точка с запятой (\mono{;});

  \item Если выражение не помещается на строке, то каждую строку, содержащую фрагмент выражения, заканчивают символом обратной косой черты (backslash - \mono{\textbackslash}), отделяя его пробелом. (Таким образом он находится прямо перед символом перевода строки и экранирует его). Всякий раз, когда необходимо визуально разбить выражение на несколько строк, а вы не уверены в правильности его обработки - используйте обратную косую черту;

  \item Строки, начинающиеся с точки (\mono{.}), также считаются продолжением предыдущего выражения;

  \item Пробельные символы (пробел, отступ, перевод строки), не разделяющие выражения, игнорируются интерпретатором и могут использоваться для оформления кода. Однако стоит сохранять осторожность - в некоторых случаях они влияют на процесс выполнения программы.
\end{itemize}

\subsection{Комментарии}

Комментарии - это фрагменты текста на естественном языке, поясняющие задачу, которую решает код. 

Хорошие комментарии не повторяют и не объясняют код, а только цель, которую пытался достигнуть автор. Комментарии должны использовать более высокий уровень абстракции (ближайший к человеческому мышлению) чем код.

Обычно комментарий считается хорошим, если при перепроектировании кода необязательно изменять комментарий.

Хоть наличие комментариев и облегчает понимание кода, главный вклад все же вносится хорошим стилем программирования и следованием соглашениям. Никакой комментарий не спасет плохо написанной программы.

Следует помнить, что кроме пользы комментарии также приносят и вред - нарушают визуальное офрмление кода.

Комментарии не обрабатываются интерпретатором и не влияют на процесс выполнения программы.

В Ruby существует два способа создания комментариев:
\begin{itemize}
  \item Любой текст, начинающийся символом решетки и заканчивающийся переводом строки, считается комментарием.
  \begin{verbatim}
  # Это комментарий.
  # Это тоже комментарий.
  \end{verbatim}

  \item Любой текст между инструкциями \verb!=begin! и \verb!=end! на отдельных строках считается комментарием. Текст комментария начинается после первого пробельного символа.
  \begin{verbatim}
  =begin Это тоже комментарий.
    В нем можно записывать все что угодно.
    Обычно его называют встроенной документацией,
    а на первой строке записывают название программы
    для ее обработки.
  =end
  \end{verbatim}
\end{itemize}

\subsection{Кодировка символов}

Текст, который сохраняется в файле, для компьютера существует не в виде символов, а только как двоичные данные - числа из нулей и единиц. Каждый ноль или единица занимают один бит памяти. Восемь нулей или единиц занимают один байт. Обычно удобно записывать байты используя шестнадцатеричную систему счисления. Так ff в шестнадцатеричной системе соотвествует 255 в десятичной и 11111111 в двоичной.

Каждому символу соотвествует определенное число, сохраняемое в памяти компьютера. Это число также называется кодовой позицией (code point) символа. Таблица, в которой соотносятся кодовые позиции и символы называется кодовой таблицей.

Кодировка - это способ представления символов в памяти компьютера (в виде набора байт). Например пробел хранится как 10000 (32 в десятичной системе счисления, 20 в шестнадцатеричной). Без информации об используемой кодировке компьютер не сможет правильно отобразить сохраненный набор байт. Обычно понятия кодировки и кодовой таблицы взаимозаменяемы.

В начале появления компьютеров повсевместно использовалась кодировка ASCII, включающая кодовые позиции для 127 символов: цифр, знаков, букв латинского алфавита и спецсимволов. Для предоставления 127 различных кодовых позиций хватает 7 бит памяти, поэтому кодовая позиция в ASCII полностью аналогична байту, хранящемуся в памяти.

\itemtitle{Свойства ASCII:}
\begin{itemize}
  \item Кодовая позиция цифр (от 0 до 9) соотвествует их двоичному значению, перед которым стоит 00112;

  \item Кодовые позиции букв (от A до Z) верхнего и нижнего регистров различаются только одним битом, что упрощает преобразование регистра и проверку на диапазон. Кодовые позиции соотвествуют порядковым номерам букв в алфавите, записанным в двоичной системе счисления, перед которыми стоит 1002 (для букв верхнего регистра) или 1102 (для букв нижнего регистра).
\end{itemize}

В качестве минимальной единицы памяти обычно используется один байт. В одном байте может быть сохранено 255 различных кодовых позиций. Оставшиеся 128 чисел (от 128 до 255) использовались для представления национальных символов: букв национальных алфавитов и специфичных знаков.

Например слово hello хранится в памяти в виде набора байт 48 65 6c 6с 6a. Каждый байт одновременно является кодовой позицией символа.

Обилие национальностей и ограниченный набор различных вариаций битов привели к образованию огромного количества кодировок. Каждая кодировка по разному использовала оставшиеся кодовые позиции, представляя с их помощью разные символы. Это создавало сразу две проблемы: преобразования кодировок и ограниченности набора символов.

Для решения проблемы кодировок был создан стандарт Юникод (Unicode). Юникод был попыткой создать единый набор символов, который будет содержать в себе все символы всех языков на планете. В стандарте определены кодовые позиции символов, но не способ их хранения. Правила, согласно которым кодовые позиции преобразуются в байты (машинное представление), определяются Юникод-кодировками.

Кодовая позиция символа в Юникод записывается в формате U+xxxx, где x - это цифры в шестнадцатеричной системе счисления (может использоваться больше четырех цифр). Количество возможных кодовых позиций превышает миллион, что позволяет стандартизировать большинство существующих алфавитов. На данный момент стандарт содержит кодовые позиции около 100 тыс. символов.

Например, слово hello состоит из пяти кодовых позиций: U+0048 U+0065 U+006C U+006C U+006F.

Разные Юникод-кодировки довольно сильно отличаются. Слово hello может быть закодировано как в виде набора байт 00 48 00 65 00 6С 00 6С 00 6А, так и виде 48 00 65 00 6С 00 6С 00 6А 00. Существуют также кодировки, хранящие каждую кодовую позицию в четырех байтах.

В последнее время чаще всего используется Юникод-кодировка UTF-8. Она совместима с ASCII - для кодирования каждого символа, содержащегося в ASCII используется один байт (слово hello в UTF-8 кодируется так же как и в ASCII). Остальные символы кодируются двумя и более байтами. Это позволяет не хранить в памяти байты, содержащие только нули, и правильно обрабатывать ASCII символы.

Приступая к выполнению программы, интепретатор получает лишь набор байт. В зависимости от внутренней кодировки эти байты могут интерпретироваться по разному. По умолчанию интерпретатор считает, что внутри программы используется кодировка ASCII. Все лексические правила также определены относительно символов, содержащихся в ASCII. 

Если в коде или комментариях (ведь это тоже одна из лексем) используются символы, не входящие в ASCII (например кириллица), необходимо вручную задавать кодировку программы. 

Если текстовый редактор сохраняет код в кодировке, отличной от ASCII, то она также должна быть явно указана в качестве кодировки программы.

Кодировка устанавливается с помощью специального комментария, расположенного в самом начале программы: \verb!#coding: название_кодировки!

Существует несколько отдельных лексических правил для такого комментария:
\begin{itemize}
  \item Вместо coding также может быть использовано encoding;
  \item Вместо двоеточия также может использоваться знак равенства;
  \item Пробелы до и после двоеточия игнорируются;
  \item Весь комментарий не чувствителен к регистру;
  \item Перед coding также может использоваться набор символов -*-.
\end{itemize}

\section[Краткое описание ООП]{Объектно-ориентированная парадигма}

Объектно-ориентированная парадигма (ООП) - это парадигма программирования, в которой основными концепциями являются понятия объектов и классов. Большинство основных положений ООП было развито в языке программирования Smalltalk, сильно повлиявшем на Ruby. В настоящее время количество прикладных языков программирования, реализующих ООП, преобладает.

ООП - это не только набор конкретных методик, а также еще и философия проектирования приложений. Как и любая сложная парадигма, ООП состоит из нескольких уровней понимания. Следует заметить, что применение объектно-ориентированного языка Ruby не означает, что код автоматически становится объектно-ориентированным - требуется явная реализация и использование объектно-ориентированных концепций.

Основные понятия ООП - это абстракция, класс, объект, инкапсуляция, наследование и полиморфизм.

\paragraph*{Абстракция:}

Абстрагирование - это способ выделить существенные свойства и игнорировать несущественные. Соответсвенно, абстракция - это набор выделенных существенных свойств.

Существенные свойства - это свойства, которыми сущность должна обладать, чтобы быть именно этой сущностью. Несущественные свойства - свойства, обладание которыми необязательно.

С точки зрения сложности, главное достоинство абстракции в том, что она позволяет игнорировать несущественные детали (не имеющие значения для программы). Абстракция - это один из главных способов борьбы со сложностью реального мира.

\paragraph*{Класс:}

Класс - это абстрактная модель (абстрактный тип данных), еще не существующей сущности. Фактически класс является образцом для создания новых объектов (формулой или руководством по эксплуатации). Обычно классы относятся к статичным сущностям, существующим в коде и неизменным в процессе выполнения.

В отличии от объектов, классы обычно не содержат данных. Передача данных классу позволяет создавать объекты, описанного в классе типа (в ООП понятия тип данных и класс - синонимы).

Объект, созданный по образцу класса называют экземпляром этого класса. Основное предназначение классов - определять поведение своих экземпляров. Обычно классы создают таким образом, чтобы они описывали объекты предметной области (объекты реального мира).

\paragraph*{Объект:}

В основе ООП находится понятие объекта. Объект - это абстракция над содержимым памяти компьютера, появляющимся в результате создания экземпляра класса. Обычно объекты относятся к динамичным сущностям, создаваемым и изменяемым в процессе выполнения программы.

Объекты обладают состоянием и поведением. Состояние объекта зависит от значения его параметров (хранимых объектом данных). Поведение объекта зависит от набора доступных ему методов.

В Ruby любые данные (в том числе классы и элементарные типы данных) относятся к объектам. Сама программа представляет собой набор взаимодействующих объектов. Взаимодействие объектов обеспечивается вызовом ими методов друг друга. Это позволяет создавать модульные, разделяемые программы, облегчая их модификацию и позволяя работать с отдельными фрагментами по очереди.

\paragraph*{Инкапсуляция:}

Инкапсуляция - это свойство сущности объединять в себе данные и методы для работы с этими данными. Данные при этом скрыты от остальной программы, а методы доступны для взаимодействия объектов. Объект не считается отдельной сущностью, если его состояние может быть изменено без явного использования ссылки на объект.
\begin{itemize}
  \item Инкапсуляция поволяет распараллелить процессы создания программы, ускоряя разработку ПО;

  \item Инкапсуляция снижает сложность разработки, позволяя сосредоточиться на небольших фрагментах программы;

  \item Инкапсуляция помогает сокрытию деталей реализации, необходимых программе, но выходящих за рамки абстракции. Инкапсуляция помогает управлять сложностью, скрывая доступ к ней.
\end{itemize}

Инкапсуляция порождает понятие модуля. Модуль - это сущность, использующаяся для инкапсуляции. Любой класс может рассматриваться как модуль. Однако модули, в отличии от классов, не предназначены для создания объектов.

\paragraph*{Наследование:}

Наследование - это свойство сущности использовать структуру другой сущности, заимствуя и расширяя уже имеющуюся функциональность (например классы расширяют возможности модулей). Класс, который заимствуется (наследуется) называется базовым или суперклассом, а класс, который заимствует (наследует) - производным или подклассом. Экземпляры производных классов расширяют поведение экземпляров базовых классов. Все базовые и производные классы в общем создают иерархию классов программы.
\begin{itemize}
  \item Наследование снижает время на разработку за счет повторного использования кода;

  \item Наследование снижает сложность, позволяя использовать уже известные фрагменты программы. Однако при сложной иерархии повышается объем кода, с которым работает программист в отдельный момент времени;

  \item Наследование дополняет абстракцию, выделяя сущности с незначительным уровнем различий. Наследование позволяет создавать абстракции с различным уровнем реализации (дополнительными группами существенных свойств).
\end{itemize}

\paragraph*{Полиморфизм:}
Полиморфизм - это свойство сущностей с одинаковой структурой (одним и тем же базовым типом), иметь различную реализацию.
\begin{itemize}
  \item Полиморфизм повышает скорость разработки, позволяя быстро подстраиваться под требования заказчика;

  \item Полиморфизм снижает сложность, позволяя скрывать внутреннюю структуру объектов;

  \item Полиморфизм поддерживает возможность отдельной реализации базовых методов для производных классов.
\end{itemize}

\begin{note}
  Хоть ООП и является достаточно удобной парадигмой, не стоит забывать, что она далеко не единственная. В некоторых случаях применение объектного подхода не оправданно и снижает производительность и удобство создания программ. В любом случае необдуманное и чрезмерное применение концепций ООП усложняет и замедляет выполнение кода, превращая его в бесполезный, малопонятный набор выражений.
\end{note}

\paragraph*{Типизация данных:} строгая динамическая типизация.

При строгой типизации совместимость и границы использования типа объекта контролируются интерпретатором и каждый объект имеет тип (использование объектов не подходящего типа приводит к вызову ошибки).

При динамической (полиморфной) типизации тип объекта вычисляется во время выполнения и может произвольно изменяться в процессе. Динамическая типизация облегчает реализацию полиморфизма.

При неявной (утиной) типизации (подвид динамической типизации) совместимость и границы использования объекта ограничены его текущим набором методов и свойств, в противоположность наследованию от определенного класса. 

Для того чтобы в Ruby узнать возможность использования того или иного объекта проверяется не его класс, а его реакция на вызов определенных методов в текущий момент. Если реакция объекта удовлетворяет условию, то его использование разрешается. При прочтении этой книги вы встретите множество условий, требующих определенной реакции объекта, на вызов того или иного метода.

\begin{note}
  "Если что-то выглядит как утка, плавает как утка и крякает как утка, то, вероятно, это утка".
\end{note}

\section{Запуск программы}

Запуск программ выполняется из терминала. Общий вид команды:

{\medskip\noindent\verb!ruby [keys] [path] [args]!}

\begin{description}
  \item[keys] - заранее определенные \hyperlink{appbin}{\underline{спецсимволы или идентификаторы}}, влияющие на выполнение программы.; 
  
  \item[path] - путь к запускаемой программе. Поиск программы выполняется относительно текущего каталога. 

  Если имя программы не указано, или передается одиночный дефис, то интерпретатор будет выполнять код, полученный из стандартного потока для чтения информации (обычно связанный с терминалом).
  
  \item[args] - произвольный набор символов, которые будут переданы программе как элементы индексного массива ARGV.
\end{description}

Файлы, доступные для выполнения (скрипты или исполняемые файлы), могут содержать информацию об интерпретаторе в первом комментарии (shebang) программы:

\mono{\#!/usr/bin/env ruby [keys] [args]}

В этом случае для запуска программы требуется только ввести в терминале путь к ней.

Кроме переданных ключей на выполнение программы также могут влиять \hyperlink{appbin}{\underline{переменные окружения и предопределенные глобальные переменные}}.

Переменные окружения - это переменные, установленные операционной системой. В Linux список установленных переменных может быть получен с помощью команды \mono{env}.
  \chapter{Стандартные типы данных}

Каждый язык программирование поддерживает один или несколько встроенных типов данных. Экземпляры стандартных типов создаются без явного указания класса создаваемого объекта - он вычисляется автоматически на основе существующих лексических правил.

\section{Элементарные типы данных}

К элементарным типам данных относятся числа, текст, логические величины, объекты-идентификаторы и регулярные выражения. Элементарные типы данных - это базовые блоки для построения других типов данных. На их основе создаются все остальные классы.

\subsection{Числа}

Числа в Ruby описываются абстрактным классом Numeric и его подклассами.

\paragraph*{Целые числа:} абстрактный класс Integer и его подклассы Fixnum и Bignum.

Целые числа, занимающие в памяти не более 31-го бита, относятся к классу Fixnum. Целые числа, превышающие этот размер, относятся к классу Bignum. Преобразование между типами чисел происходит автоматически.

Лексема для чисел - это обычный набор цифр (1289). Для разделения разрядов может использоваться символ подчеркивания (\mono{_}), который будет игнорироваться интерпретатором. Однако этот знак нельзя использовать в начале или в конце лексемы (\verb!1_000_000! - соответствует одному миллиону). 

\itemtitle{Системы счисления:}
\begin{itemize}
  \item По умолчанию, все числа обрабатываются в десятичной системе счисления. Результат любых вычислений также преобразуется в десятичную систему;

  \item Числа, начинающиеся с приставки 0x или 0X, обрабатываются в шестнадцатеричной системе счисления (0х4AF);

  \item Числа, начинающиеся с приставки 0b или 0B, обрабатываются в двоичной системе счисления (0b0111).
\end{itemize}

Для записи отрицательных чисел используется знак "минус" ($-$). 

Для записи положительных чисел используется знак "плюс" ($+$).  По умолчанию все числа обрабатываются как положительные.

\paragraph*{Десятичные дроби:} класс Float.

Лексема для десятичных дробей - это группа цифр, разделенных десятичной точкой на две части: целую и дробную ($123.051$). Так же можно использовать научную или экспоненциальную нотацию. При этом после числа записывается символ экспоненты (e или Е), после которого следует отрицательное или положительное число, обозначающее показатель степени 10 (\verb!123е-10! - соответствует $123 * 10^{-10}$). 

При записи десятичных дробей для разделения разрядов может использоваться символ подчеркивания (\mono{_}), который будет игнорироваться интерпретатором.

\subsection{Текст}

Текст - это набор из одного или более символов. Символ - это любой отображаемый на экране знак. Работа с текстом в Ruby описывается классом String.
\begin{note}
  Обычно этот тип данных называют строками. Несмотря на то, что текст может содержать символ перевода строки, он все равно рассматривается как одна большая строка. Я решил использовать термин "текст" чтобы избежать путаницы между понятими строки кода (line), строки как объекта (string) и строки текста. В английском языке для каждой из строк существует отдельный термин, который при переводе на русский теряет смысловую нагрузку.
\end{note} 
  
\paragraph*{Простой текст:} группа символов, ограниченная одиночными кавычками (\verb!'Ruby'!).  

Простой текст обрабатывается в том виде, в котором записан. 

В простом тексте также распознается минимальный набор спецсимволов (называемых также управляющими или экранированными поселдовательностями). Спецсимвол - это группа из одного или более символов, теряющих своё индивидуальное значение, одновременно с приобретением этой группой нового значения. 

\begin{keylist}{Спецсимволы:}  
  \firstkey{\textbackslash '} - соответствует символу одиночной кавычки;
  
  \key{\textbackslash} - соответствует символу обратной косой черты.
\end{keylist}
  
\paragraph*{Составной текст:} группа символов, ограниченная двойными кавычками (\verb!"Ruby"!). 

Составной текст обрабатывается интерпретатором с распознаванием полного набора эмблем и поддержкой интерполяции.

Интерполяция - это выполнение фраментов кода \verb!#{выражение}! и замена их на результат выполнения выражения (\verb!"#{1+2}"! - соответствует тексту "3"). 

В составном тексте также разрешается использовать другие парные символы двойных кавычек.

\begin{keylist}{Спецсимволы (каждый спецсимвол также определен в кодовой таблице ASCII):}  
  \firstkey{\textbackslash *} - соответствует любому символу на месте *, который необходимо сохранить в тексте. Используется для экранирования символов. Поэтому спецсимволы также называют экранированными последовательностями - все они начинаются с обратной косой черты;
  
  \key{\textbackslash b} -  спецсимвол удаляет предшествующий символ;
  
  \key{\textbackslash r} - спецсимвол возвращает указатель курсора на начало строки, запись следующего символа удалит все предыдущие (символ возврата каретки). Спецсимвол используется для вставки новой строки вместо предыдущей. Это полезно для замены строк в различных программах (например создание прогресс бара в терминале).
  
  \key{\textbackslash n} - спецсимвол переводит указатель курсора на начало новой строки (символ перевода строки). Для операционной системы Windows в качестве символа перевода строки используется спецсимвол \verb!\r\n!.
  
  \key{\textbackslash t} - спецсимвол переводит указатель курсора вправо, создавая отступ (табуляцию);
  
  \key{\textbackslash ***} - соответствует символу, с указанной кодовой позицией из трех цифр в восьмеричной системе счисления;
  
  \key{\textbackslash **} - соответствует \verb!\0**!;
  
  \key{\textbackslash *} - соответствует \verb!\00*!;
  
  \key{\textbackslash x**} - соответствует символу, соотвествующему переданному байту;
  
  \key{\textbackslash x*} - соответствует \verb!\0*!;
  
  \key{\textbackslash u****} - соответствует символу, с указанной кодовой позицией в стандарте Юникод;
  
  \key{\textbackslash u\{*\}} – соответствует тексту, символы которого соответсвуют указанным кодовым позициям в стандарте Юникод.
\end{keylist}

\paragraph*{Специальная форма записи:} приставки \verb!%q! или \verb!%Q!.

Текст также может быть записан между двумя произвольными разделителями с использованием приставок \verb!%q! или \verb!%Q!. Разделитель – это символ или группа символов, которая служит границами текста. При использовании приставки \verb!%q! текст будет распознаваться как простой, а при использовании приставки \verb!%Q! - как составной (\verb!Q(Ruby)! - соответствует тексту "Ruby"). Вместо приставки \verb!%Q!  также можно использовать только знак процента (\verb!%(Ruby)!).

\paragraph*{Документы:} большие блоки текста, с многочисленными знаками препинания. 

Лексема документа начинается с символов \verb!<<! или \verb!<<-!. За ними следует группа символов, которая будет служить границой текста. 

Тело документа (текст) начинается со следующей строки. Объект создается, когда на отдельной строке будет использован разделитель. После создания объекта интерпретатор продолжит обработку кода с того места, на котором встретил начало лексемы (\verb!<<!). 

\begin{itemize}
  \item Если лексема начинается с \verb!<<!, то пробелы между началом строки и конечным разделителем не допускаются. Пробелы после конечного разделителя не допускаются никогда;

  \item Если начальный разделитель не ограничен кавычками, то лексема распознается как составной текст. Чтобы сохранить простой текст начальный разделитель ограничивают одинарными кавычками. Кавычки также позволяют использовать пробелы внутри разделителя (в данном случае в качестве разделителя будет выступать объект).
\end{itemize}
\begin{verbatim}
  <<- 'DOC'
    Здесь записан простой текст.
  DOC
  # -> 'Здесь записан простой текст'\
\end{verbatim}

\paragraph*{Одиночные символы:} символ, начинающийся со знака вопроса ?.

При использовании лексемы распознаются некоторые спецсимволы, в основном относящиеся к способам записи символов с помощью кодовых позиций (\verb!?A! – соответствует тексту 'A').

\subsection{Логические величины}

Логические величины используются для булевой алгебры, проверки различных условий или сравнении объектов.

\begin{keylist}{Список лексем:}
  
  \firstkey{Истина:} true ссылается на единственный экземпляр класса TrueClass;
  
  \key{Ложь:} false ссылается на единственный экземпляр класса FalseClass;
  
  \key{Отсутствие:} nil ссылается на единственный экземпляр класса NilClass. Она используется в том случае, если необходимо представить отсутствие объекта, подходящего под заданные условия.
\end{keylist}

При использовании в выражениях лексемы false и nil имеют логическое значение false, а все остальные объекты - логическое значение true.

\subsection{Объекты-идентификаторы}

Довольно часто для управления программой используются небольшие группы символов (слова). Из-за особенностей реализации использование для этого текстовых объектов снижает скорость выполнения программы. Вместо этого рекомендуется использовать экземпляры класса Symbol.

Лексема объекта-идентификатора - это группа символов, следующая за двоеточием (\verb!:green!). Также объект-идентификатор может быть записан между двумя произвольными разделителями, с использованием приставок \verb!%s! или \verb!%S!. При использовании приставки \verb!%s! группа символов распознается как простой текст, а при использовании \verb!%S! - как составной. 
(\verb!%s(Ruby)! соответствует объекту-идентификатору \verb!:'Ruby'!)

Для каждого объекта, кроме текста, сохраняется также цифровой код, вычисленный на его основе. При повторном использовании той же эмблемы, вместо создания нового объекта, по цифровому коду будет найден уже существующий. Использование цифровых кодов ускоряет поиск и сравнение объектов.
\begin{note}
  Однажды созданный объект-идентификатор будет существовать на всем промежутке времени выполнения программы, поэтому необходимо осторожно подходить к их использованию и делать это только по назначению (с целью управления процессом выполнения программы). Динамическое создание множества объектов идентификатров увеличивает количество памяти, занимаемой программой.
\end{note}

\subsection{Регулярные выражения}

Регулярные выражения -  это мощный инструмент для поиска по тексту. С помощью регулярных выражений составляются образцы, на основе которых выполняется поиск. Использование регулярных выражений в Ruby описывается классом Regexp.

Лексема регулярных выражений - это группа символов (называемая телом регулярного выражения), ограниченная двумя косыми чертами (slash - /). После конечного разделителя может быть использован необязательный модификатор, влияющий на механизм поиска (\verb!/Ruby/i!).

Тело регулярного выражения также может быть записано между двумя произвольными разделителями с использованием приставки \verb!%r!. Модификаторы в этом случае записываются после конечного разделителя (\verb!%r(Ruby)i! – соответствует \verb!/Ruby/i!).

Тело регулярного выражения обрабатывается как составной текст. Одиночные символы, не относящиеся к спецсимволам соответствуют их аналогам в тексте.
Поиск совпадений выполняется последовательно по каждому символу, слева направо.

Полный синтакси регулярных выражений описывается в \hyperlink{appregexp}{\underline{приложении}}.

\section{Составные типы данных}

Составные объекты -  это объекты, содержащие произвольный набор элементов (любых других объектов). Составные типы позволяет группировать объекты и рассматривать их на уровне группы.

\subsection{Индексные массивы}

Индексный массив (или просто массив) - это составной объект, содержащий упорядоченную группу элементов и позволяющий получить доступ к элементу, если известна его позиция (индекс элемента). Элементами массива могут быть любые объекты (даже другие массивы).

Использование массивов в Ruby описывает класс Array.

С точки зрения синтаксиса, индексные массивы - это группа объектов между двумя квадратными скобками. Сами объекты при этом разделяются запятыми: \verb![ 1, "Ruby", ?\u0048 ]!

Существует также специальный синтаксис записи массивов, в качестве элементов которых выступают короткие отрывки текста (состоящие из одного слова и не содержащие пробелов). Такие массивы могут быть записаны в виде группы элементов между двумя произвольными разделителями с использованием приставок \verb!%w! или \verb!%W!. Сами элементы при этом разделяются пробелами.

При использовании приставки \verb!%w! элементы массива будут рассматриваться как простой текст, а при использовании приставки \verb!%W! - как составной.
\\\verb!%W( Язык программирования Ruby )! соответствует массиву:
\\\verb![ "Язык", "программирования", "Ruby" ]!

\subsection{Ассоциативные массивы}

Ассоциативный массив - это составной объект, содержащий упорядоченную группу элементов (каждый из которых представляет собой пару ключ/объект) и позволяющий получить доступ к объекту, если известен его ключ (ключ ассоциируется с объектом). 

\begin{note}
  Для построения соответствий между ключами и объектами используется виртуальная таблица. 

  Объекты, выступающие в роли ключей, в таблице представлены в виде цифровых кодов (небольших целых чисел), получаемых в результате вызова метода .hash. Соответственно сам ключ может быть объектом любого типа, если для него определен этот метод.
\end{note}
  
В ассоциативном массиве можно хранить любые объекты (даже другие массивы).

Использование ассоциативных массивов в Ruby описывает класс Hash.

С точки зрения синтаксиса, ассоциативные массивы - это группа парных элементов ключ/объект между двумя фигурными скобками. Ключи от объектов отделяются символами \verb!=>!. Сами элементы при этом разделяются запятыми: \verb!{ "Ruby" => "language", "Вася" => "Человек" }!.

Один из наиболее распространенных способов использования объектов-идентификаторов - в качестве ключей ассоциативного массива. При этом ассоциативный массив записывают в коде программы, отделяя ключи от объектов двоеточием (при этом двоеточие перед ключом не используется).
\\\verb!{ Ruby: "language", Вася: "Человек" }!
\\ соответствует ассоциативному массиву:
\\\verb!{ :Ruby => "language",  :Вася => "Человек" }!

\subsection{Диапазоны}

Диапазон - это составной тип данных, содержащий упорядоченный набор объектов, располагающихся между заданными границами. 

Использование диапазонов в Ruby описывает класс Range.

С точки зрения синтаксиса, диапазоны - это два однотипых объекта, разделенные двумя или тремя точками. При использовании двух точек в диапазон включается конечная граница (\verb!1..3! содержит числа 1, 2, 3), а при использовании трех - нет (\verb!1...3! содержит числа 1 и 2).

\begin{note}
  Границы диапазона должны принадлежать к одному классу. В этом классе должен быть определен оператор \verb!<=>!, использующийся для сравнение объектов, входящих в диапазон, с его границами.
\end{note}
  \chapter{Переменные и константы}

Переменные и константы - это вид идентификаторов, использующихся для хранения данных (объектов). Объекты, которым присвоен идентификатор могут быть использованы повторно. Выражение, связывающее объекты и идентификаторы, называется выражением присваивания.

Переменные и константы также позволяют логически разделить используемые данные. Они облегчают понимание кода, позволяя перейти от терминов языка программирования к терминам решаемой задачи (проблемной области). Адекватность переменной (константы) во многом определяется ее именем. Название переменной можно рассматривать как высокоуровневый псевдокод, характеризующий ее содержимое.

Переменные (и константы) содержат адрес объекта, на который ссылаются. Адрес - это небольшое целое число, характеризующее область памяти, в которой хранится объект. В Ruby адрес объекта доступен только интерпретатору. Синтаксис использования идентификатора позволяет работать непосредственно с объектом, не отвлекая внимание на его адрес. Объект, на который ссылается переменная, также называют ее значением.

Константы от переменных отличаются только областью применения. Константы подразумевают единственное определение и не используются для изменения объектов. Обычно константы применяют для хранения постоянных неизменяющихся данных. В отличии от констант переменные подразумевают многократное использование. Переменная может использоваться как для изменения текущего объекта, так и связываться с новым объектом.

\begin{note}
  На самом деле, в Ruby, переопределение констант не приведет к завершению процесса выполнения программы. Вместо этого интерпретатор просто выведет обычное предупреждение. Скорей всего это сделано для облегчения возможности произвольного переопределения существуующих классов.
\end{note}

Для использования переменных и констант необходимо объявить интерпретатору об их существовании. Переменные и константы считаются существующими после выполнения первого выражения присваивания с их участием. Это выражение называют инициализацией. Процесс инициализации состоит из объявления (создания) переменной и определения (присваивания) переменной.

Переменные и константы также могут быть объявлены и без выражения присваивания. Встретив подходящую лексему интерпретатор создаст требуемую переменную или константу. В этом случае определение выполянется автоматически и идентификатор связывается с nil. Автоматическая инициализация предотвращает ошибки, возникающие при использовании переменных, не связанных с объектами. Это также позволяет сосредоточить внимание на объектах, а не на переменных.

\begin{note}
  Переменная (или константа) считается объявленной если код содержит ее лексему, даже если фрагмент кода не выполнялся (в этом случае она ссылается на nil). Это происходит из-за предварительной обработки кода для виртуальной машины.
\end{note}

\paragraph*{Область видимости:} фрагмент кода, в котором переменная (или константа) объявлена и может быть использована. 

На основе областей видимости также осуществляется классификация переменных. К базовым областям видимости относятся глобальная и локальные области. Глобальная область видимости распространяется на весь код. Глобальные переменные и константы существуют в любом месте кода (после их инициализации).	Локальная область видимости распространяется только на явно ограниченный фрагмент кода. Локальные переменные и константы существуют только в той части кода, в которой происходила их инициализация.

\begin{itemize}
  \item Лексемы локальных переменных начинаются с символа подчеркивания или строчной буквы (принято использовать только строчные буквы, разделяя слова знаком подчеркивания - змеиная_нотация). Использование не сущестующей локальной переменной обрабатывается интерпретатором как вызов метода;

  \item Лексемы глобальных переменных начинаются с знака доллара. Использование не существующей глобальной переменной, приводит к ее объявлению;

  \item Лексемы констант начинаются с прописной буквы (принято использовать только прописные буквы, разделяя слова знаком подчеркивания - НОТАЦИЯ_ГРЕМУЧЕЙ_ЗМЕИ). Область видимости констант основана на иерархии вложенности и иерархии классов. Использование не существующей константы приводит к вызову ошибки.
\end{itemize}

ООП вводит две дополнительные области видимости: область видимости класса и область видимости экземпляра класса. Это так же добавляет два вида переменных.
\begin{itemize}
  \item Лексема переменной экземпляра начинается с знака @ (принято использовать змеиную нотацию). Использование не существующей переменной экземпляра, приводит к ее объявлению;
  
  \item Лексема переменной класса начинается с @@ (принято использовать змеиную нотацию). Использование не существующей переменной класса, приводит к вызову ошибки.
\end{itemize}

\paragraph*{Сбор мусора:} удаление неиспользующихся объектов.

В Ruby ресурсы, используемые программой, определяются интерпретатором. Это позволяет избегать наиболее распространенных проблем работы с памятью. В то же время интерпретатор распознает только заранее определенные ситуации, что иногда приводит к довольно неприятным результатам. Дополнительная работа также увеличивает время интерпретации и выполнения программ. Баланс между этими двумя полюсами - важная задача для разработчиков интерпретатора.

Для автоматического управления памятью реализован механизм, называемый сбором мусора. Под мусором подразумеваются объекты, сохраненные в памяти, но при этом не использующиеся. Как только интерпретатор понимает, что объект не связан ни с одним идентификатором, он освобождает память, которую этот объект занимал. Это позволяет удалять одноразовые объекты сразу после их использования, а также сохранять синтаксичекие структуры на всем протяжении процесса выполнения.
  \chapter{Выражения}

Стандартные объекты и идентификаторы относятся к простым выражениям. В результате их выполнения возвращается объект. Для выполнения вычислений из простых выражений создаются сложные выражения и предложения (которые являются одной из разновидностей сложных выражений). Части сложного выражения соединяются с помощью операторов и инструкций.

\section{Операторы}

Простейший способ создания сложных выражений - объединение простых с помощью операторов. Лексема оператора - это группа математических символов или знаков препинания. Составные части сложного выражения называют операндами. Операнды, в свою очередь, также относятся к выражениям и могут быть как простыми выражениями так и сложными.

\itemtitle{Операторы делятся на:}
\begin{itemize}
	\item Унарные (У) 	- оперируют одним операндом;
	\item Бинарные (Б) 	- оперируют двумя операндами;
	\item Тернарные (Т) - оперируют тремя операндами.
\end{itemize}

Операторы также отличаются приоритетом и последовательностью выполнения операндов. 

В сложных выражениях, содержащих несколько операторов, операнды будут выполняться в порядке увеличения приоритета их операторов.

Если существует несколько операторов с одинаковым приоритетом (или только один оператор), то операнды выполняются в том порядке, в котором были записаны. При R-последовательности это будет происходит справа налево, а при L-последовательности - слева направо.

Чтобы изменить процесс выполнения выражения, операнды, которые необходимо выполнить в первую очередь, отделяют двумя круглыми скобками.

\pagebreak
\begin{longtable}{ | * {5} { l |}}
\hline
  Приор. & Лексема & Послед. & Тип & Название выражения \\ \hline

  1 & \verb#! ~ +# & R & У Б У & логическое отрицание; \\* &&&& побитовое отрицание; \\* &&&& унарный плюс \\ \hline

  2 & ** & R & Б & возведение в степень \\ \hline

  3 & \verb!-! & R & У & унарный минус \\ \hline

  4 & \verb!* / %! & L & Б & произведение (копирование); деление; \\* &&&& остаток от деления (форматирование) \\ \hline

  5 & \verb!+ -! & L & Б & сложение (объединение); \\* &&&& вычитание (удаление) \\ \hline

  6 & \verb!<< >>! & L & Б & побитовый сдвиг влево (добавление) \\* &&&& побитовый сдвиг вправо \\ \hline

  7 & \verb!&! & L & Б & побитовое И (пересечение множест) \\ \hline

  8 & \verb!| ^! & L & Б & побитовое ИЛИ (объединение множеств); \\* &&&& побитовое исключающее ИЛИ \\ \hline

  9 & < <= > >= & L & Б & отношение \\ \hline

  10 & <=> ! = \verb#=~# & L & Б & сравнение; неравенство; поиск совпадений \\ \hline

  10 & \verb#!~# == === & L & Б & отсутствие совпадений; равенство \\ \hline

  11 & \verb!&&! & L & Б & логическое И \\ \hline

  12 & || & L & Б & логическое ИЛИ \\ \hline

  13 & ?: & R & Т & логическое условие \\ \hline

  14 & = & R & Б & присваивание \\ \hline

  15 & not & R & У & логичекое отрицание \\ \hline

  16 & and or & L & Б & логичекое И; логическое ИЛИ \\ \hline  
\end{longtable}
\pagebreak

\begin{enumerate} % Перечисление по приоритету.

  \item %1
  \begin{operator}
    \define{!obj}{логическое отрицание}
    Возвращается ссылка на логическую величину, противоположную по значению.
    \begin{verbatim}
    !1 # -> false
    !nil # -> true\
    \end{verbatim}
  \end{operator}

  \begin{operator}
    \define{\textasciitilde\-, integer}{побитовое отрицание}
    Каждый бит числа изменяется на противоположный и дополняется до 1. В результате возвращается десятичное число, необходимое для дополнения. Аналогично выполнению выражения \verb!-number-1!.
    \begin{verbatim}
    ~1 # -> -2
    ~0b01 # -> -2\
    \end{verbatim}
  \end{operator}

  \begin{operator}
    \define{+ number}{унарный плюс}
    Возвращает число в десятичной системе счисления.
    \\\verb!+0b01 # -> 1!
  \end{operator}

  \item %2
  \begin{operator}
    \define{number**number}{возведение в степень}
    Возведение числа в степень. Первое число - основание степени, а второе - показатель.
    \\\verb!2**3 # -> 8!
  \end{operator}

  \item %3
  \begin{operator}
    \define{- number}{унарный минус}
    Знак числа изменяется на противоположный.
    \\\verb!-0b01 # -> -1!
  \end{operator}

  \item %4
  \begin{operator}
    \define{number * number}{произведение}
    Произведение двух чисел.
    \\\verb!1 * 2 # -> 2!
  \end{operator}

  \begin{operator}
    \define{string * integer}{копирование текста}
    Текст копируется указанное число раз и все копии соединяются.
    \\\verb!"R" * 3 # -> "RRR"!
  \end{operator}

  \begin{operator}
    \define{array * integer}{копирование массива}
    Элементы массива копируются указанное число раз и все копии объединяются.
    \\\verb![ 1, ?R ] * 2 # -> [ 1, "R", 1, "R" ]!
  \end{operator}

  \begin{operator}
    \define{[*object]}{извлечение элементов}
    Извлекаются элементы составного объекта.
    \begin{verbatim}
    a = [ 1, 2, 3 ]
    [*a] # -> [ 1, 2, 3 ]
    [ *a, 1 ] # -> [ 1, 2, 3, 1 ]
    b = \{ a: 1, b: 2 \}
    [*b] # -> [ [:a, 1], [:b, 2] ]
    c = 1..4
    [*c] # -> [ 1, 2, 3, 4 ]
    [*1] # -> [1]
    [*nil] # -> []
    [*?a] # -> ["a"]\
    \end{verbatim}    
  \end{operator}

  \begin{operator}
    \define{number / number}{деление}
    Возвращается частное от деления двух чисел.
    \\\verb!-6 / 3 # -> -2!
  \end{operator}

  \begin{operator}
    \define{number \% number}{остаток от деления}
    Возвращается остаток от деления двух чисел. Результат может изменяться в зависимости от знаков операндов.
    \begin{verbatim}
    7 % 3 # -> 1
    -7 % 3 # -> 2
    7 % -3 # -> -2\
    \end{verbatim}
  \end{operator}

  \begin{operator}
    \define{string \% object}{форматирование}
    Объект преобразуется в текст и форматируется согласно правилам, заданным \hyperlink{appformat}{\underline{форматной строкой}}. В качестве объектов могут быть использованы числа, текст, индексный и ассоциативный массивы.
  \end{operator}

  \item %5
  \begin{operator}
    \define{number + number}{сумма двух чисел}
    Сумма двух чисел.
    \\\verb!1 + 3 # -> 4!
  \end{operator}

  \begin{operator}
    \define{string + string}{объединение текста}
    Объединение двух текстов.
    \\\verb|"Ruby" + ?! # -> "Ruby!"|
  \end{operator}

  \begin{operator}
    \define{array + array}{объединение массивов}
    Объединение двух массивов.
    \\\verb![ 1, 2 ] + [ 3, 4 ] # -> [ 1, 2, 3, 4 ]!
  \end{operator}

  \begin{operator}
    \define{number - number}{вычитание}
    Возвращается разность двух чисел.
    \\\verb!2 - 1 # -> 1!
  \end{operator}

  \begin{operator}
    \define{array - array}{удаление элементов}
    Из первого массива удаляются все элементы, содержащиеся во втором массиве.
    \\\verb![ 1, 2, 2, ?R ] - [ 2, ?1 ] # -> [1, "R"]!
  \end{operator}

  \item %6
  \begin{operator}
    \define{number \twoless integer; number \twogreat integer}{побитовый сдвиг}
    Сдвиг влево или вправо каждого бита на указанное количество разрядов.
    \\\verb!1 << 2 # -> 4!
  \end{operator}

  \begin{operator}
    \define{string \twoless string}{добавление текста}
    В конец первого текста добавляется второй. 

    Если вместо второго операнда передается целое число, то оно обрабатывается как кодовоя позиция символа.
    \begin{verbatim}
    "Ruby" << ?! # -> "Ruby!"
    "Ruby" << 33 # -> "Ruby!"\
    \end{verbatim}
  \end{operator}

  \begin{operator}
    \define{array \twoless object}{добавление элемента}
    В массив добавляется новый объект.
    \\\verb![1] << 2 # -> [ 1, 2 ]!
  \end{operator}

  \item %7
  \begin{operator}
    \define{integer \& integer}{побитовое И}
    Возвращается результат сравнения каждого бита двух чисел.

    Если биты в одинаковых разрядах установлены в 1, то результирующий бит также устанавливается в 1.
    \\\verb!0b01 & 0b10 # -> 0!
  \end{operator}

  \begin{operator}
    \define{array \& array}{пересечение множеств}
    Возвращается массив, содержащий одну копию каждого элемента, входящего в оба массива.
    \\\verb![ 1, 2, 2, 3] & [ 2, 3 ] # -> [ 2, 3 ]!
  \end{operator}

  \item %8
  \begin{operator}
    \define{integer | integer}{побитовое ИЛИ}
    Возвращается результат сравнения каждого бита двух чисел.

    Если любой из битов в одинаковых разрядах установлены в 1, то результирующий бит также устанавливается в 1.
    \\\verb!0b01 | 0b10 # -> 3!
  \end{operator}

  \begin{operator}
    \define{array | array}{объединение множеств}
    Возвращается массив, содержащий одну копию каждого элемента, входящего в любой из массивов.
    \\\verb![ 1, 2, 2, 3 ] | [ 2, 3 ] # -> [ 1, 2, 3 ]!
  \end{operator}

  \begin{operator}
    \define{integer \textasciicircum\-, integer}{побитовое исключающее ИЛИ}
    Возвращается результат сравнения каждого бита двух чисел.

    Если один (и только один) из битов в одинаковых разрядах установлены в 1, то результирующий бит также устанавливается в 1.
    \\\verb!0b01 ^ 0b10 # -> 3!
  \end{operator}

  \item %9
  \begin{operator}
    \define{< <= >= >}{отношение}
    Проверка отношения двух объектов.

    Для чисел:
    \begin{verbatim}
    1 <= 2 # -> true
    1 > 2 # -> false\
    \end{verbatim}

    Для текста:

    При проверке текста последовательно проверяется каждый байт. Первый отрицательный результат станет результатом выполнения всего выражения.

    Каждая следующая буква алфавита считается больше, чем предшественница.

    Любая строчная буква считается больше, чем любая прописная буква.
    \begin{verbatim}
    "а" < "б" # -> true
    "а" < "Б" # -> false
    "1" <= "2" # -> true\
    \end{verbatim}
  \end{operator}

  \item %10
  \begin{operator}
    \define{object == object}{равенство}
    Проверка равенства двух объектов. Объекты считаются равными, если их значения и типы равны.
    \begin{verbatim}
    1 == 1.0 # -> true
    1 == "1" # -> false\
    \end{verbatim}
  \end{operator}

  \begin{operator}
    \define{object != object}{неравенство}
    Проверка равенства двух объектов. Возвращается ссылка на логическую величину, противоположную по значению.
    \begin{verbatim}
    1 != 1.0 # -> false
    1 != "1" # -> true\
    \end{verbatim}
  \end{operator}

  \begin{operator}
    \define{object === object}{равенство}
    Аналогично проверке на равенство, кроме случаев, которые будут описаны отдельно.
    \begin{verbatim}
    1 === 1.0 # -> true
    1 === "1" # -> false\
    \end{verbatim}
  \end{operator}

  \begin{operator}
    \define{string =\textasciitilde\-, regexp; regexp =\textasciitilde\-, string}{поиск совпадений}
    Поиск в тексте совпадений с \hyperlink{appregexp}{\underline{образцом}}. В результате возвращается индекс символа, с которого начинается найденное совпадение, или ссылка на nil, если совпадений не найдено.
  \end{operator}

  \begin{operator}
    \define{!\textasciitilde}{отсутствие совпадений}
    Проверка отсутствия совпадений.
  \end{operator}

  \begin{operator}
    \define{object <=> object}{сравнение}
    Сравнение двух объектов.
    \begin{verbatim}
          < = >
    # ->	-1 0 1\
    \end{verbatim}
    \begin{description}
      \item[-1:]  если первый операнд меньше второго;
      \item[0:]   если операнды равны;
      \item[1:]   если первый операнд больше второго.
      \item[nil:] если сравнение операндов невозможно (разные типы операндов).
    \end{description}

    Сравнить можно два числа, текста, индексных массива (последовательно сравнивается каждый элемент).
    \begin{verbatim}
    1 <=> 1.0 # -> 0
    1 <=> ?2 # -> nil\
    \end{verbatim}
  \end{operator}

  \item %11
  \begin{operator}
    \define{expression \&\& expression}{лоическое И}
    Выполняется первое выражение, если его логическое значение false, или второе выражение в другом случае.
    \begin{verbatim}
    4 && 2 - 1 # -> 1
    3 > 4 && 2 - 1 # -> false\
    \end{verbatim}
  \end{operator}

  \item %12
  \begin{operator}
    \define{expression || expression}{лоическое ИЛИ}
    Выполняется первое выражение, если его логическое значение true, или второе выражение в другом случае.
    \begin{verbatim}
    4 || 2 - 1 # -> 4
    3 > 4 || 2 - 1 # -> 1\
    \end{verbatim}
  \end{operator}

  \item %13
  \begin{operator}
    \define{expression ? expression : expression}{лоическое условие}
    Выполняется первое выражение. Если его логическое значение true, то выполняется второе выражение, или третье выражение в другом случае.
    \begin{verbatim}
    1 > 2 ? true : false # -> false
    1 < 2 ? true : false # -> true\
    \end{verbatim}
  \end{operator}

  \item %14
  \begin{operator}
    \define{identificator = object}{присваивание}
    \hyperlink{appequal}{\underline{Присваивание}} идентификаторов объектам.
  \end{operator}

  \begin{operator}
    \define{**= *= /= \%= += -= \textless\textless= \textgreater\textgreater= \&\&= \&= ||= |= \textasciitilde=}{псевдооператоры}
    Псевдооператоры - это операторы, получившиеся в результате объединения с оператором присваивания.
    \\\verb!object1 op= object2! аналогично \verb!object1 = object1 op object2!
  \end{operator}

  \item %15
  \begin{operator}
    \define{not object}{логическое отрицание}
    Аналогично \verb|!object| с меньшим приоритетом.
  \end{operator}

  \item %16
  \begin{operator}
    \define{expression and expression}{логическое И}
    Аналогично \verb!expression && expression! с меньшим приоритетом.
  \end{operator}

  \begin{operator}
    \define{experssion or expression}{логическое ИЛИ}
    Аналогично \verb!expression || expression! с меньшим приоритетом.
  \end{operator}
\end{enumerate}

\section{Предложения}

Предложения - это одна из разновидностей сложных выражений, создаваемая с помощью инструкций (поэтому количество возможных видов предложений строго ограничено). Каждое предложение начинается с инструкции, объявляющей тип предложения и заканчивается инструкцией end, объявляющей конец предложения. Между этими двумя инструкциями находится фрагмент кода, называемый телом предложения, процессом выполнения которого манипулирует предложение.

\subsection{Условное предложение}

Условные предложения управляют процессом выполнения в зависимости от логического значения переданного условия. В результате выполнения предложения возвращается результат выполнения последнего выражения в его теле (или nil).

\paragraph*{Синтаксис предложения:} тело предложения выполняется, если логическое значение условия true.

\begin{verbatim}
  if условие
    тело_предложения
  end\
\end{verbatim}

Тело предложения должно быть отделено от условия либо переводом строки, либо точкой с запятой, либо инструкцией then.

\paragraph*{Инструкция else:} содержит фрагмент кода, который выполняется, если логическое значение условия false.

\begin{verbatim}
  ...
    тело_предложения
  else
    код
  end\
\end{verbatim}

При необходимости записать подряд инструкцию else и новое условное предложение используется инструкция elsif.

\paragraph*{Краткий синтаксис:} \verb!тело_предложения if условие!

Тело предложения либо должно находиться на одной строке с условием, либо быть ограничено инстуркциями begin и end, либо быть ограничено двумя круглыми скобками.
\begin{verbatim}
  begin
    тело_предложения
  end if условие\
\end{verbatim}
или
\begin{verbatim}
  ( тело_предложения
  ) if условие\
\end{verbatim}

Если вместо инструкции if использовать инструкцию unless, то тело предложения будет выполняться, только если логическое значение условия false. Инструкция elsif в этом случае не используется.

\subsection{Разветвленное условие}

Разветвленные условия аналогичны условному предложению, но проверяют сразу несколько условий. Условия проверяются последовательно. Обычно наиболее вероятные варианты записывают раньше остальных - это уменьшит время выполнения предложения. В результате выполнения предложения возвращается результат выполнения последнего выражения в его теле (или nil).

\paragraph*{Синтаксис предложения:} выполняется фрагмент кода, логическое значение условия которого true.

\begin{verbatim}
  case
    when условие
      код
    when условие
      код
    ...
  end\
\end{verbatim}

Условие может состоять из нескольких выражений, разделенных запятыми. Любое из них может послужить причиной для выполнения кода.

\paragraph*{Специальный синтаксис:} проверяется равенство условия и выражений (===).

\begin{verbatim}
  case условие
  when выражение
    код
  when выражение
    код
  ...
  end\
\end{verbatim}

В любой форме можно использовать инструкцию else.

\subsection{Цикл}

Цикл - это предложение итеративного типа (заставляющее программу повторно выполнять некоторый фрагмент кода). Циклы используются в том случае, когда число итераций неизвестно заранее. В результате выполнения цикла возвращается ссылка на nil.

Цикл while не соотвествует слову "пока" в естественном языке. Условие цикла не проверяется непрерывно, а только до или после каждой итеарции.

\paragraph*{Синтаксис предложения:} тело предложения выполняется до тех пор пока логическое значение условия true (условие проверяется до итерации).

\begin{verbatim}
  while условие
    тело_цикла
  end\
\end{verbatim}

Тело цикла должно быть отделено от условия либо переводом строки, либо точкой с запятой, либо инструкицей do.

\paragraph*{Сокращенный синтаксис:} \verb!тело_цикла while условие!

Тело цикла и условие обязательно должны находиться на одной строке кода. Если тело цикла ограничено инструкциями begin и end, то условие проверяется после итерации. Также тело цикла может быть ограничено двумя круглыми скобками.
\begin{verbatim}
  begin 
    тело_цикла
  end while условие\
\end{verbatim}
или
\begin{verbatim}
  ( тело_цикла
  ) while условие\
\end{verbatim}
	
Если вместо инструкции while использовать инструкцию until, то тело цикла будет выполняться если логическое значение условия false. 

\subsection{Перебор элементов}

Перебор элементов - это предложение итеративного типа, выполняющее тело перебора для каждого элемента составного объекта.
\begin{verbatim}
  for параметр in объект
    тело_перебора
  end\
\end{verbatim}

Параметр в теле перебора ссылается на элементы составного объекта (может быть использовано несколько параметров, разделенных запятыми).

Для составного объекта должен существовать метод \method{.each}, который будет использоваться в ходе выполнения предложения.

Тело предложения должно быть отделено либо переводом строки, либо точкой с запятой, либо инструкцией do. 

\subsection{Процесс выполнения}

Процесс выполнения предложения может быть изменен с помощью специальных инструкций в его теле. Необязательный код, передаваемый инструкции возвращается в результате выполнения предложения. Если код отсутствует, то возвращается ссылка на nil.

\begin{keylist}{Список инструкций:}  
  \firstkey{return [код]} - завершает выполнение предложения и всех методов, в теле которых оно выполняется, продвигаясь вверх по областям видимости;
  
  \key{break [код]} - завершает выполнение предложения;
  
  \key{next [код]} - завершает текущую итерацию цикла и переходит к следующей;
  
  \key{redo} - еще раз выполняет тело предложения. Проверка условия при этом не выполняется.
\end{keylist}

\section{Триггеры}

В качестве условия могут быть использованы триггеры. Триггер - это сложное выражение, составленное с помощью операторов \mono{..} или \mono{...} . Эти операторы имеют приоритет выполнения больше, чем у оператора условия и меньше, чем у оператора логического ИЛИ (примерно 12.5). В условии может использоваться только один триггер. 

Как и обычные условия, триггеры имеют некоторое логическое значение. Его особенность в том, что оно может изменяться в зависимости от результатов предыдущих вычислений, т.е. сохраняет состояние выполнения.

Обычно триггеры применяются вместе с регулярными выражениями для обработки текста между начальными и конечными шаблонами.

\paragraph*{Синтаксис триггера:} \verb!условие..условие!; \verb!условие...условие!

Операнды триггера - это два условия, при проверке которых логичсекое значение триггера либо остается неизменным, либо изменяется на противоположное. Порядок проверки условий зависит от оператора, используемого для создания триггера.

Логическое значение триггера false до тех пор пока логическое значение первого условия false. После смены логического значения первого условия изменяется и логическое значение триггера. Оно будет сохраняться до тех пор пока логическое значение второго условия false. После смены логического значения второго условия, изменяется и логическое значение триггера и проверка условий начинается сначала.

\itemtitle{Процесс выполнения триггера:}
\begin{enumerate}
  \item Если логическое значение триггера false, то проверяется первое условие. В зависимости от его логического значения изменяется логическое значение триггера. После его возвращения для оператора \mono{..} также проверяется второе условие.

  \item Если логическое значение триггера true, то проверяется второе условие. Если логическое значение второго условия true, то логическое значение триггера меняется.
\end{enumerate}

\begin{note}
  По смыслу триггеры довольно похожи на диапазоны. Они верны до тех пор пока существующее состояние выполнения находится от достижения первого условия и до достижения второго условия. Оператор \mono{...} включает состояние достигнувшее второе условие, а оператор \mono{..} - нет. 
\end{note}
  \chapter{Реализация ООП}

Одна из главных особенностей Ruby - сильно выраженная поддержка объектно-ориентированного парадигмы при создании программ. Любые данные, в том числе и элементарные, относятся к объектам. Большинство операторов относится к методам. Множество синтаксического сахара облегчает использование основных концепций и сущностей ООП.
\itemtitle{Основные особенности:}
\begin{itemize}
  \item Любые данные хранятся в виде объектов;

  \item Вычисления выполняются путем взаимодействия (обмена данными) между объектами, при котором один из объектов требует выполнение некоторого действия от другого. Объекты взаимодействуют с помощью методов. Метод - это запрос на выполнение действия, дополненный набором аргументов, которые могут понадобиться при его выполнении;

  \item Каждый объект имеет независимую память, которая состоит из других объектов;

  \item Каждый объект является представителем класса, который определяет общие свойства объектов; 

  \item В классе также определяется поведение объекта - набор доступных методов. Все экземпляры класса могут выполнять одни и те же действия;

  \item Классы организованы в единую древовидную структуру с общим корнем, называемую иерархией наследования. Память и поведение экземпляров базового класса автоматически доступны в производном классе.
\end{itemize}

\section{Основные сущности}

\subsection{Модули}

Модули - это абстрактные сущности, использующиеся для инкапсуляции. Они могут применяться как для инкапсуляции программы или ее фрагментов (пространства имен), так и для инкапсуляции логически связанных фрагментов кода (обычно для объединения логически связанных методов или констант).
\begin{note}
  В качестве пространств имен модули обычно используются для инкапсуляции области видимости программы. При этом все классы или модули программы объявляются только в теле одного отдельного модуля. Это позволяет разным программам не засорять глобальную область видимости. Использование глобальной области видимости программой считается плохим тоном среди программистов, т.к. требует согласования использованных имен классов и модулей и может привести к непредсказуемым последсвиям.

  Явно выраженный пример инкапсуляции логически связанных фрагментов кода представляет модуль Math, объединяющий математические функции.
\end{note}

С точки зрения синтаксиса модуль - это фрагмент кода, связанный с константой. В отличии от классов модули не могут иметь экземпляров или подклассов, поэтому любой класс может считаться модулем, но не любой модуль - считаться классом. Кроме этого любой модуль может использоваться также как класс. Все модули относятся к экземплярам класса Module.

Создание модуля называется объявлением, а заполнение области видимости модуля - определением.
\begin{verbatim}
  module идентификатор_модуля
    тело_модуля
  end
\end{verbatim}
\begin{itemize}
  \item Инструкция module ожидает получения константы, сообщая интерпретатору область памяти в которую необходимо выполнить запись информации. Если соотвествующей области не существует, то автоматически объявляется новый модуль. В другом случае будет изменен уже существующий модуль;

  \item Идентификатор модуля должен начинаться с прописной буквы. Каждое отдельное слово также начинается с прописной буквы. Такой подход называют ВерблюжьейНотацией;

  \item В теле модуля создается отдельная область видимости. В ней могут создаваться любые другие сущности. Внутри тела модуля псевдопеременная self ссылается на модуль. Интерпретатор возвращает результат выполнения последнего выражения в теле модуля (обычно это nil).
\end{itemize}

Любой модуль, объявленный в теле другого модуля, добавляется в иерархию вложенности. С точки зрения иерархии вложенный модуль будет представлен в виде константы, инициализированной в теле основного модуля. Модули, определенные на верхнем уровне - константы в классе Object.

\subsection{Классы}

Класс - это абстрактная сущность, использующаяся для описания структуры и поведения ее экземпляров. Создание класса называется объявлением, а описание его экземпляров - определением. Любой класс (даже встроенный) может быть переопределен в любом месте кода. Все классы относятся к экземплярам класса Class.

\begin{verbatim}
  class идентификатор_класса
    тело_класса
  end
\end{verbatim}

\begin{itemize}
  \item Инструкция class ожидает получения константы, сообщая интерпретатору область памяти в которую необходимо выполнить запись информации. Если соотвествующей области не существует, то автоматически объявляется новый класс. В другом случае будет изменен уже существующий класс.

  \item Идентификатор класса должен начинаться с прописной буквы. Каждое отдельное слово также начинается с прописной буквы. Такой подход называют ВерблюжьейНотацией.

  \item В теле класса создается отдельная область видимости. В ней может определяться поведение экземпляров класса или любые другие сущности. Внутри тела класса псевдопеременная self ссылается на класс. Интерпретатор возвращает результат выполнения последнего выражения в теле класса (обычно это nil).
\end{itemize}

\subsection{Константы}

Константы определяются в теле класса и служат для хранения известных и не изменяющихся данных, общих для всего класса. 

Использовать константу можно с помощью бинарного оператора \mono{::}, имеющего наивысший приоритет и последовательность выполнения L. В качестве левого операнда используется идентификатор класса, а в качестве правого операнда - константа. Использование константы называют ее вызовом.
\verb!класс::константа!

Использование этого выражения с оператором присваивания \mono{=}, позволяет инициализировать константу в любом месте кода.

Для того, чтобы вызвать константу, интерпретатор должен определить какая именно константа должна быть использована. При полной записи выражения поиск осуществляется в указанном классе. При отсутствии левого операнда по умолчанию используется псевдопеременная self.

При наличии только константы, поиск будет выполняться в следующем порядке:
\begin{enumerate}
  \item в теле класса, в котором была вызвана константа;
  \item в теле модуля, в котором определяется класс;
  \item в теле модуля, добавленного к классу;
  \item вверх по иерархии классов с выполнением пунктов 1-3;
  \item вызов метода \method{::const_missing} для класса, в теле которого была вызвана константа.
\end{enumerate}

\subsection{Переменные} 

Переменные определяются в теле класса и служат для хранения изменяющихся данных: состояния класса и его экземпляров. Переменные могут существовать в двух различных областях видимости: области видимости класса и области видимости экземпляра класса.

\paragraph*{Переменные класса:} переменные, объявляемые в области видимости класса.

Переменные класса могут быть объявлены в любом фрагменте тела класса (в том числе и в теле метода). Лексема переменной начинается с символов @@. 

Переменная класса может быть использована в любом фрагменте тела класса (в том числе и в теле метода). Она ссылается на один и тот же объект для всех его экземпляров.

Переменные класса используются для хранения изменяющихся данных, общих для всех экземпляров класса. Обычно переменные класса инициализируются в теле класса и используются в теле его методов. Использование не существующей переменной класса приводит к вызову ошибки.

\paragraph*{Переменные экземпляра:} переменные, объявляемые в области видимости экземпляра класса.

Переменные экземпляра могут быть объявлены в теле метода экземпляров класса. Лексема переменной начинается с символа @. 

Переменная экземпляра может быть использована в теле любого метода экземпляров класса. Она ссылается на разные объекты для каждого отдельного экземпляра.

Переменные экземпляра используются для хранения состояния экземпляров класса. Обычно переменные экземпляра инициализируются в теле метода в момент его вызова. Использование не существующей переменной экземпляра приводит к ее автоматичсекому объявлению.

\subsection{Методы}

Методы - это абстрактные сущности, определяющие действия, которые может выполнять объект. Выполняемые действия делятся на два вида: функции и процедуры. В Ruby их четкого синтаксичекого разграничения не существует - процедуры входят в подмножество функций. С точки зрения языка программирования, метод относится к выражениям.
\begin{description}
  \item[Функция] - это метод, используемый для получения объекта (например математические функции);

  \item[Процедура] - это метод, используемый для выполнения действия (например сохранение данных в файл). В результате выполнения процедуры обычно возвращается статус выполнения (логическая величина) или ссылка на объект, для которого вызывалась процедура.
\end{description}

С точки зрения синтаксиса метод - это именованный фрагмент кода, выполняющий одну конкретную задачу. Методы определяются в теле класса. В результате определения возвращается nil. Каждый метод связан с тем объектом, для которого был определен и не может использоваться без ссылки на него. Использование метода также называют вызовом.

\subsubsection*{Определение метода}

\textbf{Методы экземпляров класса:} методы, определяющие поведение экземпляров класса.
\begin{verbatim}
  def идентификатор_метода(параметры)
    тело_метода
  end
\end{verbatim}

\textbf{Методы класса:} методы, определяющие поведение класса.
\begin{verbatim}
  def класс.идентификатор_метода(параметры)
    тело_метода
  end
\end{verbatim}

\begin{note}
  Вместо идентификатора класса может использоваться псевдопеременная self. Это повышает переносимость кода, определяя метод не для любого текущего класса, а не для какого-то конкретного.
\end{note}

\begin{keylist}{Фрагменты синтаксиса:}
  
  \firstkey{Идентификатор метода:} лексема идентификатора метода аналогична лексеме локальной переменной.

  Обычно имя метода выбирается в соответствии с его целью:
  \begin{itemize}
    \item Для именования процедур обычно используются глаголы;
    \item Имя функции обычно описывает объект, который она возвращает;
    \item Имя методов с побочным эффектом обычно заканчивается восклицательным знаком (добавляется только если существует версия метода без побочного эффекта);
    \item Предикаты - это методы, утверждающие или отрицающие что-либо об объекте. В результате выполнения предиката возвращается логическая величина, характеризующая истинность или ложность утверждения. Имя предиката обычно заканчивается вопросительным знаком.
  \end{itemize}
  
  \key{Параметры:} локальные переменные, которые служат для разделения аргументов (объектов, передаваемых методу при вызове).

  Несколько параметров разделяются запятыми. При вызове метода параметры будут последовательно инициализироваться переданными аргументами с помощью выражения присваивания.
  \begin{itemize}
    \item Если параметр инициализируется при объявлении метода, то он имеет значение по умолчанию. Это значение будет использоваться, если при вызове метода необходимый аргумент не передавался. Значения по умолчанию могут быть произвольными выражениями, переменными экземпляра или даже другими параметрами, объявленными ранее. Параметры, имеющие значение по умолчанию, должны объявляться последовательно;

    \item Если перед параметром используется оператор разименования (\mono{*}), то такой параметр принимает произвольное количество аргументов, которые будут сохранены в индексном массиве и присвоены параметру. Подобный тип параметров объявляется после параметров, имеющих значение по умолчанию, и должен быть единственным для каждого метода;

    \item Если перед последним параметром используется амперсанд (\mono{\&}), то такой параметр принимает блок (вызывая для аргумента метод \method{.to_proc}). В теле метода может использоваться как параметр, так и инструкция yield.
  \end{itemize}

  Для повышения читабельности параметров, ссылающихся на логическую величину может использоваться приставка is.

  \key{Тело метода:} фрагмент кода, выполняемый в момент вызова метода.

  Тело метода создает собственную область видимости. В ней доступны объявленные параметры и переменные экземпляра. В теле метода псевдопеременная self ссылается на объект, для которого метод был вызван.
  
\end{keylist}

\paragraph*{Синонимы:} копия метода, имеющая другой идентификатор.

Для метода может быть определено любое количество синонимов. Синонимы могут быть созданы только в теле класса, объявляющего метод.

Создание синонима выполняется с помощью инструкции alias:
\\\verb!alias синоним идентификатор_метода!

% ``` note
\begin{note}
Запомнить синтаксис выражения создания синонимов будет проще, если рассматривать его относительно выражения присваивания: \verb!синоним = метод!.
\end{note}
% ```

\paragraph*{Удаление метода:} выполняется с помощью инструкции undef. Ее можно использовать только в теле класса, определяющего данный метод или в теле его подклассов (в этом случае метод будет удален только для отдельного подкласса).
\\ \verb!undef идентификатор_метода!

\subsubsection*{Блоки}

Блоки - одна из особенных синтаксических конструкций в Ruby, предоставляющая мощные возможности по выполнению кода.

С точки зрения синтаксиса, блок - это фрагмент кода, связанный с группой параметров. Блоки не могут использоваться сами по себе, а только передаваться методам (они всегда должны передаваться методу последними). Если объект не использует блок, то его передача игнорируется.
\begin{note}
  Методы, позволяющие перебирать элементы составного объекта, не зависимо от их типа, называются итераторами. Перебор элементов выполняется с помощью блока, которому они передаются. Каждое отдельное выполнение блока называется итерацией. В результате выполнения итератора обычно возвращается ссылка на объект, для которого он был вызван.

  Вместо блока итераторы также могут принимать идентификатор метода, начинающийся с амперсанда. В этом случае метод будет вызван для каждого элемента составного объекта.

  Итераторы в Ruby - это синтаксический сахар, использующийся вместо предложения перебора (for in). 
\end{note}
\verb!{ |параметры| тело_блока }!
\\ или
\begin{verbatim}
  do |параметры|
    тело_блока
  end
\end{verbatim}

Открывающая фигурная скобка или инструкция do относятся к предыдущему выражению - остальные аргументы необходимо ограничивать круглыми скобками, иначе блок будет передан последнему аргументу.

\begin{keylist}{Синтаксис блока:}

  \firstkey{Параметры:} локальные переменные, которые служат для разделения аргументов (объектов, передаваемых блоку).

  Несколько параметров разделяются запятыми. При выполнении блока параметры будут последовательно инициализироваться переданными аргументами с помощью выражения присваивания.

  Параметры блока не могут иметь значений по умолчанию или принимать блоки. Однако они могут ссылаться на массив аргументов (с помощью оператора разыименования *).
  
  \key{Тело блока:} создает собственную область видимости. В нем определены объявленные параметры и переменные экземпляра. В теле блока псевдопеременная self ссылается на объект, для которого был вызван метод, принимающий блок. Интерпретатор возвращает результат выполнения последнего выражения.

  Блок относится к замыканиям - в теле блока существуют локаьные переменные, объявленные в окружающем коде. Чтобы явно указать переопределение локальных переменных, их идентификаторы отделяют от параметров блока точкой с запятой (\verb!| x, y; z, k, n |!). Переопределенные локальные переменные в теле блока будут ссылаться на nil.
\end{keylist}

\paragraph*{Выполнение блока:} блок выполняется в момент вызова.

Для вызова блока в теле метода используется инструкция yield, которой передаются аргументы, отправляемые в блок (интсрукция не может принимать блоки). Избыток аргументов игнорируется. Выполнив инструкцию, интерпретатор возвращает результат выполнения блока.

Процесс выполнения блока может изменяться с помощью специальных инструкций в его теле. Необязательный код, передаваемый инструкции становится результатом выполнения. Если он не указан, то в результате возвращается nil.
\begin{keylist}{Процесс выполнения:}
  
  \firstkey{return [код]} - завершает выполнение блока, принимающего блок метода, и метода, в теле которого это происходит;
  
  \key{break [код]} - завершает выполнение блока и принимающего блок метода;
  
  \key{next [код]} - завершает выполнение блока. Итератор, принимающий блок, может начать новую итерацию;
  
  \key{redo} - еще раз выполняет тело блока.
\end{keylist}

\subsubsection*{Вызов метода}

Методы вызываются с помощью бинарного оператора \mono{.} (или \mono{::}, но он обычно не используется или используется только для вызова методов класса), имеющего наивысший приоритет и последовательность выполнения L. В качестве левого операнда используется объект, а в качестве правого операнда — идентификатор метода.
\\\verb!объект.идентификатор_метода(аргументы)!

Аргументы - это группа объектов, инициализрующих параметры. Несколько объектов разделяются запятыми. Недостаток или избыток аргументов приводит к вызову ошибки. Количество требуемых аргументов соотвествует количеству объявленных параметров.

В результате вызова метода возвращается результат выполнения последнего выражения в его теле. Вызов метода без указания объекта вызывает его для псевдопеременной self. 

Аргументы классифицируют по способу их передачи. В Ruby все аргументы передаются по ссылке.
\begin{itemize}
  \item Передача по значению - метод копирует переданный аргумент. Изменение параметра не влияет на передаваемый аргумент;

  \item Передача по адресу - передаваемым значением является адрес, по которому можно найти значение переменной;

  \item Передача по ссылке - метод копирует не переданный аргумент, а его адрес, однако использует синтаксис, при котором работа выполняется с объектом, хранящимся по этому адресу. Изменение параметра также изменит переданный аргумент.
\end{itemize}

\itemtitle{Синтаксис вызова:}
\begin{itemize}
  \item использование круглых скобок при вызове метода не обязательно (кроме случаев, когда методу также передается блок);

  \item если последним аргументом метода передается ссылка на ассоциативный массив, то использование фигурных скобок для выделения массива не требуется;

  \item если методу передается аргумент, перед которым используется звездочка, то из объекта будут извлечены все его элементы (с помощью метода \method{.to_splat}). Это позволяет передавать аргументы в виде составного объекта;

  \item если методу передается аргумент, перед которым используется амперсанд (\mono{\&}), то он обрабатывается также как блок. Такой аргумент должен передаваться методу последним.  
\end{itemize}

Перед вызовом метода интерпретатор должен определить какой именно метод использовать. Поиск осуществляется в следующем порядке:
\begin{enumerate}
  \item в собственном классе объекта;
  \item в теле класса объекта;
  \item в теле модуля, добавленного к классу;
  \item вверх по иерархии классов с выполнением пунктов 2 и 3;
  \item вызов для объекта метода \method{.method_missing}. 
\end{enumerate}

\subsubsection*{Процесс выполнения}

Процесс выполнения метода может изменяться с помощью специальных инструкций в его теле. Необязательный код, передаваемый инструкции становится результатом выполнения. Если он не указан, то в результате возвращается nil.
\begin{keylist}{Процесс выполнения:}
  
  \firstkey{return [код]} - завершает вызов метода;
  
  \key{break [код]} - вызывает ошибку;
  
  \key{next [код]} - вызывает ошибку;
  
  \key{redo} - вызывает ошибку.
\end{keylist}

\subsection{Объекты}

\paragraph*{Экземпляры классов:} объекты, создаваемые с помощью классов.

Для создания объекта вызывается метод \method{::new} того класса, экземпляр которого необходимо получить. Этот метод выполняет два действия: создает область памяти для объекта, вызывая метод \method{.allocate}, и заполняет эту область, вызывая метод \method{.initialize}. 

Метод \method{.initialize} должен определяться классом или будет использован метод базового класса. Определяемый метод автоматически становится частным. Инициализируемые в нем переменные экземпляра доступны в теле любого другого метода экземпляров.

\begin{note}
  Метод .allocate уже определен в классе Class и его переопределение обычно не требуется.
\end{note}

Для создания экземпляра класса может быть определен любой другой метод, в результате. Такие методы называют конструкторами.

\paragraph*{Классы и модули:} могут рассматриваться с точки зрения экземпляров классов Class и Module соответственно.

К ним также применимы методы экземпляров, определенные в этих классах, синтаксис вызова которых аналогичен вызову методов класса. 

Если определить переменную экземпляра в теле класса, а не в теле метода, то она будет связана с классом, а не с его экземплярами. Такие переменные недоступны в теле методов и фактически относятся к переменным экземпляра объекта, принадлежащего к классу Class.

\begin{methodlist}
  \declare{Module::new \{ nil \}}{\# -> module} 
  Создание анонимного модуля. Необязательный блок выполняется в теле модуля. Модуль перестанет быть анонимным, если будет присвоен константе. 

  \declare{Class::new( class = Object )}{\# -> class} 
  Создание анонимного класса, наследующего переданному. Класс перестанет быть анонимным, если будет присвоен константе.
\end{methodlist}

\paragraph*{Собственный класс объекта:} отдельный класс для каждого объекта.

Каждый объект кроме класса, к экземплярам которого относится, имеет еще и свой собственный класс. С помощью собственных классов для объектов определяется уникальное поведение. Собственные классы также называют метаклассами (metaclass).
\begin{verbatim}
  class << объект
    тело_класса
  end
\end{verbatim}

Это предложение определяет собственный класс для переданного объекта. Объект рационально передавать с помощью переменной. В теле класса в этом случае используется псевдопеременная self.

При создании собственный класс помещается в начало иерархии наследования (т.е класс объекта становится базовым для собственного класса объекта). При этом хоть собственные классы и входят в иерархию наследования, но программисту в ней отображаться они не будут.

Интерпретатор возвращает результат выполнения последнего выражения в теле собственного класса объекта (обычно это nil).

\paragraph*{Собственный метод объекта:} метод, определенный в собственном классе объекта.

Собственные методы объекта могут быть определены либо в теле собственного класса (как методы экземпляров), либо с помощью отдельного предложения.

\begin{verbatim}
  def объект.метод
    тело_метода
  end
\end{verbatim}

Определение такого метода возможно только после создания объекта (объект также должен существовать в текущей области видимости). Очевидно, что объект должен быть представлен в виде переменной или константой.

В результате возвращается nil.

Поведение всего класса зависит от набора существующих методов класса. Поэтому, неудивительно, что методы класса также относятся к собственным методам объекта (которым в данном случае является класс). 

\itemtitle{Класс:}
\begin{itemize}
  \item методы экземпляров хранятся в теле класса;
  \item методы класса (собственные методы) хранятся в теле собственного класса объекта (объектом в данном случае является класс).
\end{itemize}

Следовательно главное отличие классов от обычных объектов в возможности хранить методы своих экземпляров.

\subsection{Примеры}

Определение простого класса в теле модуля.

Метод \method{.to_s} используется при отображении экземпляра класса.

\begin{verbatim}
  module OurKlass
    class User
      @@count = 0 # class var.

      # class method.
      def self.count
        @@count
      end

      def initialize( name, age )
        @name = name # instance var.
        @age = age
        @@count += 1
        @id += 1
      end

      # instance method.
      def to_s
        "#\{@name\}: #\{@age\} years old"
      end

    end # class User.
  end # module.
\end{verbatim}

Создание экземпляра класса. Метод \method{.to_a} еще не определен.
\begin{verbatim}
  user = OurKlass::User.new "Timmy", 22 # -> "Timmy: 22 years old"
  OurKlass::User.count # -> 1
  user.to_a # -> error!
\end{verbatim}

Определение метода \method{.to_a}.
\begin{verbatim}
  class OurKlass::User
    def to_a
      [ @name, @age ]
    end
  end

  user.to_a # -> ["Timmy", 22]
\end{verbatim}

Различные способы определения метода класса.
\begin{verbatim}
  class OurKlass::User
    def self.next_count
      @@count + 1
    end
  end

  OurKlass::User.next_count # -> 2

  class OurKlass::User
    class << self
      def prepend_count
        @@count - 1
      end
    end
  end

  OurKlass::User.prepend_count # -> 0
\end{verbatim}

Определение собственного метода объекта. При переопределении переменной для нового объекта метод определен уже не будет.
\begin{verbatim}
  class << user
    def id
      @id
    end
  end

  user.id # -> 1
  user = OurKlass::User.new "Tommy", 33 # -> "Tommy: 33 years old"
  user.id # -> error!
\end{verbatim}

\section{Основные принципы}

\subsection{Инкапсуляция}

Инкапсуляция обеспечивается ограничением области применения методов и ограничением доступа к состоянию объектов.

\subsubsection*{Контроль применения}

Контроль доступа позволяет ограничивать область применения методов.

Методы могут быть как чисто внутренними, обеспечивающими логику функционирования объекта, так и внешними, позволяющими объектам взаимодействовать, требуется реализовать их разделение. Доступ к внутренним методам при этом ограничивается областью видимости класса.

\itemtitle{Классификация методов:}
\begin{description}
  \item[public]    - общие методы. Общие методы позволяют объектам взаимодействовать друг с другом. Они могут быть вызваны в любой области видимости;

  \item[protected] - защищенные методы. Защищенные методы позволяют объектам одного типа взаимодействовать друг с другом. Они могут быть вызваны только в области видимости класса (и его подклассов) или экземпляров класса (и его подклассов) и только для экземпляров того же класса (и его подкласов);

  \item[private]   - частные методы. Частные методы реализуют внутреннюю логику объекта. Они могут быть вызваны только в области видимости  класса (и его подклассов) или экземпляров класса (и его подклассов) и только для текущего экземпляра (частные методы всегда вызываются для псевдопеременной self). Частные методы помогают скрывать реализацию работы программы и предоставлять только API для использования.
\end{description}

Область применения метода объявляется с помощью частных методов экземпляров из класса Module: \method{.public}, \method{.private} и \method{.protected}:
\begin{itemize}
  \item Интерпретатор ограничивает применение всех методов, идентификаторы которых переданы;

  \item Интерпретатор ограничивает применение всех методов, объявляемых после вызова метода без аргументов.
\end{itemize}

\begin{methodlist}
  \declare{module.private_class_method(*name)}{\# -> self} 
  Объявление методов класса частными. Обычно используется для инкапсуляции метода \method{::new}.

  \declare{module.public_class_method(*name)}{\# -> self} 
  Объявление методов класса общими. 

  \declare{module.module_function( *name = nil )}{\# -> self}
  Переданные методы экземпляров объявляются как частные. Для каждого метода также создается аналогичный метод класса. Это позволяет использовать их в теле производного класса без ссылки на модуль (например модуль Math). При вызове без аргументов влияет на все методы, объявленные далее.  
\end{methodlist}

Методы, определяемые вне тела явно определенного класса, относятся к частным методам класса Object.

\subsubsection*{Методы доступа}

По умолчанию состояние объекта не доступно вне тела объекта. Поэтому для получения или изменения значения переменных экземпляра или класса требуется явно определять методы доступа. Существует два вида методов доступа: метод, позволяющий получать состояние объекта (читать значение переменной), и метод, позволяющий изменять состояние объекта (определять значение переменной).

Методы чтения обычно просто возвращают текущее значение переменой. Для этого в конце тела метода, последним выражением используют идентификатор требуемой переменной.

Методы записи изменяют значение переменной. Для этого в конце тела метода, последним выражением используют выражение присваивания, в котором участвуют идентификатор требуемой переменной и переданный методу аргумент. Идентификатор метода доступа при этом обычно заканчивается знаком равенства (\mono{=}).

\paragraph*{Свойства:} синтаксический сахар над методами доступа.

Для облегчения определения методов доступа существуют свойства. Свойства - это методы доступа к состоянию экземпляров класса. При объявлении свойства автоматически объявляются соотвествующие переменные экземпляра и методы доступа к ним.

Объявление свойств выполняется с помощью частных методов экземпляров из класса Module. 
\begin{methodlist}
  \declare{.attr_accessor(*attribute)}{\# -> nil}
  Объявление переменной экземпляра и методов доступа для ее получения и изменения.

  \declare{.attr_reader(*attribute)}{\# -> nil}
  Объявление переменной экземпляра и метода доступа для ее получения.

  \declare{.attr_writer(*attribute)}{\# -> nil}
  Объявление переменной экземпляра и метода доступа для ее изменения.
\end{methodlist}

\subsection{Наследование}

Наследование обеспечивается возможностью изменять существующую иерархию наследования, добавляя к ней новые элементы.

\subsubsection*{Синтаксис наследования}

В Ruby реализовано единичное наследование с возможностью произвольного добавления модулей. Это означает, что любой класс может иметь только один базовый класс и бесконечное количестов добавленных модулей.

\paragraph*{Наследование класса:} может быть выражено словосочетанием "относится к" или "принадлежит к".

При наследовании одного класса другому в начало иерархии наследования добавляется новый класс, который, в свою очередь, будет наследовать классу Object.
\\\verb!class производный_класс < базовый_класс!

\paragraph*{Добавление модуля:} может быть выражено словом "содержит".

При добавлении модуля иерархия наследования он перемещается в начало иерархии наследования. В зависимости от способа добавления это может быть иерархия наследования обычных классов или иерархия наследования собственных классов.

\begin{methodlist}
  \declare{module.include(*a_module)}{\# -> self [PRIVATE]}
  Добавление модулей в начало иерархии наследования. Методы экземпляров, определенные в модуле, становятся методами экземпляров текущего модуля.

  \declare{object.extend(*a_module)}{\# -> object}
  Добавление модулей в начало иерархии наследования собственных классов. Методы экземпляров, определенные в модуле, становятся собственными методами объекта.

  \declare{module.extend_object(object)}{\# -> object [PRIVATE]}
  Аналогично выполнению \verb!object.extend self!
\end{methodlist}

Добавление модулей к иерархии наследования возможно только в теле модуля (или класса), а добавление модулей к иерархии наследования собственных классов в любом фрагменте кода (в теле класса методы экземпляров из модуля становятся методами класса).

\begin{note}
  Вызов \verb!extend self! в теле класса приведет к тому, что для каждого метода экземпляров автоматически объявляется соответствующий метод класса.
\end{note}

\subsubsection*{Иерархия наследования}

На вершине иерархии находится класс BaseObject, от него наследует класс Object, в теле которого выполняется программа. К классу Object добавлен модуль Kernel, в котором определено большинство основных методов. Любой модуль относится к экземплярам класса Module, который наследует классу Object. Методы экземпляров из класса Module могут вызываться в теле модулей. Любой класс относится к экземплярам класса Class, который наследует классу Module. Методы экземпляров из класса Class могут вызываться в теле классов. Методы экземпляров из класса Module могут вызываться в теле классов. Методы экземпляров из класса Class не могут вызываться в теле модулей.

\begin{note}
  Если собственные классы объектов влияют только на объект, то каким-образом наследуются методы класса? Дело в том, что при наследовании собственный класс производного класса наследует собственному классу базового класса. Образуется две равнозначные иерархии наследования - для обычных классов и для собственных.
\end{note}

$$
  \xymatrix{
  Classes \: +Modules &  Metaclasses\\
  BasicObject \ar[d] & <Class\colon BasicObject > \ar@{-->}[l] \ar[d] \\
  Object \ar[d] & <Class\colon Object> \ar@{-->}[l]^(0.75){+Kernel} \ar[d]\\
  Module \ar[d] & <Class\colon Module> \ar@{-->}[l] \ar[d]\\
  Class \ar@/^1pc/@{-->}[ruuu] & <Class\colon Class> \ar@{-->}[l] }
$$

\begin{itemize} 
  \item Переменные экземпляра не наследуются;

  \item Переменные класса наследуются его подклассами. При этом они ссылаются на тот же объект, что и в базовом классе;

  \item Константы наследуются производными классами. При этом они ссылаются на разные объекты для каждого подкласса;

  \item Методы наследуются производными классами. При этом для каждого класса существует своя копия метода.
\end{itemize}

\subsection{Полиморфизм}

Полиморфизм обеспечивается возможностью произвольно изменять унаследованные методы.

\paragraph*{Виртуальные методы:} методы, которые могут быть переопределены производным классом. В Ruby все методы относятся к виртуальным. Для переопределения метода, в теле класса, объявляют метод с тем же идентификатором. 

Чтобы из переопределяемого метода вызвать метод базового класса вызывают инструкцию super, которой передаются неоходимые аргументы. В другом случае методу базового класса будут переданы все аргументы, переданные текущему методу.

\paragraph*{Переопределение операторов:} одной из полезных особенностей в Ruby является то, что большинство операторов на самом деле относятся к методам. Поэтому поведение операторов для разных типов объектов отличается - оно зависит от определения метода в теле класса. Также разрешается определять собственное поведение для различных операторов и переопределять уже существующее. Выражение, составленное с помощью операторов, при этом аналогично вызову метода с тем же идентификатором.
\begin{verbatim}
  op object <-> object.op
  object1 op object2 <-> object1.op(object2)
\end{verbatim}
Операторы, не относящиеся к методам:
\verb!., ::, &&, ||, ?:, =, псевдооператоры, not, and, or, .., ... !

\paragraph*{Абстрактный метод:} метод, который был объявлен, но не определен. В Ruby все абстрактные методы считаются определенными и вовращают ссылку на nil.

\paragraph*{Абстрактный класс:} класс, соержащий хотя бы один виртуальный метод. Абстрактные классы используются только в иерархии наследования и не предназначены для создания экземпляров.

В Ruby создание экземпляров абстрактных классов не ограничивается, и фактически абстрактные классы не отличаются от обычных.
\begin{note}
  Класс Numeric, к которому относятся все числа - типичный абстрактный класс, а класс IO, к которому относятся файлы и потоки, абстрактным не является. В любом случае синтаксис определения этих классов не отличается.
\end{note}

Абстрактные классы можно рассматривать в качестве интерфейса к группе производных классов, но, в отличии от интерфейсов, абстрактные классы могут иметь определенные методы и свойства.

\paragraph*{Интерфейс:} модуль, содержащий только абстрактные методы. Интерфейсы описывают функциональность, предоставляемую классом, реализующим интерфейс. Класс, реализующий интерфейс, должен определять все его методы. Один класс может реализовывать несколько интерфейсов одновременно.
Пример интерфейса:
\begin{verbatim}
  module Openable
    def open; end
    def close; end
  end
\end{verbatim}

\part{Описание классов}
  \chapter{Числа}
$$
\xymatrix{
Numeric \: (+ \: Comparable) \ar[d]\ar[dr]\ar[drr]\ar[drrr] &&& \\
Complex & Rational & Float & Integer \ar[d]\ar[dr] & \\
&&& Fixnum & Bignum }
$$

Для математических расчетов в Ruby определен модуль Math.

\chapter{Числа}
$$
  \xymatrix{	
  Numeric \: (+ \: Comparable) \ar[d]\ar[dr]\ar[drr]\ar[drrr] &&& \\
  Complex & Rational & Float & Integer \ar[d]\ar[dr] & \\
  &&& Fixnum & Bignum }
$$

Для математических расчетов в Ruby определен модуль Math.

\section{Numeric}

Абстрактный класс, описывающий общие принципы работы с числами.

Добавленные модули: Comparable.

Если в результате выполнения какого-либо выражения интерпретатор возвращает число, то оно автоматически переводится в десятичную систему счисления.

\subsection*{Приведение типов}

\subsubsection{Неявное приведение}

Подклассы чисел за внешней схожестью имеют различную внутреннюю реализацию. Поэтому перед тем как вызвать метод, интерпретатор приводит переданные аргументы к одному классу. Делается это в соответствии с приведенным ниже списком:
\begin{enumerate}
  \item Если одно из чисел - комплексное, то и другие числа будут преобразованы в комплексные;

  \item Если одно из чисел - десятичная дробь, то и другие числа будут преобразованы в десятичные дроби;

  \item Если одно из чисел - рациональная дробь, то и другие числа будут преобразованы в рациональные дроби. 
\end{enumerate}

Результат будет экземпляром того же класса, к которому приводятся аргументы.

\subsubsection{Явное приведение}

\begin{methodlist}
  \declare{.coerce(number)}{\# -> array}
  Возвращает индексный массив вида \verb![ self, number ]!. 
  \begin{itemize}
    \item Если числа принадлежат к разным классам, то они преобразуются в десятичные дроби с помощью метода .Float;

    \item Текст преобразуется, если он содержит только цифры (поддерживаются двоичная и шестнадцатеричная системы счисления).
  \end{itemize}
  \begin{verbatim}
  1.coerce 2.1 # -> [ 2.1, 1.0 ]
  1.coerce "2.1" # -> [ 2.1, 1.0 ]
  1.coerce "0xAF" # -> [ 175.0, 1.0 ]
  1.coerce "q123" # -> error!\
  \end{verbatim}

  \declare{.i}{\# -> complex}
  Преобразует число в комплексное. Метод удален из класса Complex.
  \\\verb!1.i # -> (0+1i)!

  \declare{.to_c}{\# -> complex}
  Преобразует число в комплексное. 
  \\\verb!1.to_c # -> (1+0i)!

  \declare{.to_int}{\# -> integer}
  Преобразует число в целое с помощью метода \method{.to_i}.
  \\\verb!2.1.to_int # -> 2!
\end{methodlist}

\subsection*{Операторы}

\begin{methodlist}
  \declare{number1 \% number2}{}
  Синонимы: \method{modulo}

  Вычисление остатка от деления.

  Аналогично выполнению \verb!number1 – number2 * (number1 / number2).floor!

  \declare{+number}{(идентификатор метода +@)}
  Унарный плюс.

  \declare{-number}{(идентификатор метода -@)}
  Унарный минус.

  \declare{number1 <=> number2}{}
  Сравнение.
\end{methodlist}

\subsection*{Округление}

\begin{methodlist}
  \declare{.ceil}{\# -> integer}
  Возвращается наименьшее целое число, которое будет больше или равно объекту, для которого метод был вызван (округление в большую сторону).
  \\\verb!2.1.ceil # -> 3!

  \declare{.floor}{\# -> integer}
  Возвращается наибольшее целое число, которое будет меньше или равно объекту, для которого метод был вызван (округление в меньшую сторону).
  \\\verb!2.1.floor # -> 2!

  \declare{.round( precise = 0 )}{\# -> number}
  Округляет число с заданной точностью. Точность определяет разряд, до которого будет выполнено округление.
  \begin{verbatim}
  2.11355.round 4 # -> 2.1136
  2.round 4 # -> 2.0\
  \end{verbatim} 

  \declare{.truncate}{\# -> integer}
  Возвращает целую часть числа.
  \\\verb!2.1.truncate # -> 2!
\end{methodlist}

\subsection*{Математические функции}

\begin{methodlist}
  \declare{.abs2}{\# -> number}
  Возвращает квадрат числа.
  \\\verb!-2.1.abs2 # -> 4.41!

  \declare{.numerator}{\# -> integer}
  Возвращает числитель рациональной дроби, полученной с помощью метода \method{.to_r}.
  \\\verb!2.1.numerator # -> 4728779608739021!

  \declare{.denominator}{\# -> integer}
  Возвращает знаменатель рациональной дроби, полученной с помощью метода \method{.to_r}.
  \\\verb!2.1.denominator # -> 2251799813685248!

  \declare{.divmod(number)}{\# -> array}
  Возвращает ссылку на индексный массив вида \verb![ self / number, self  % number ]!. Деление и остаток от деления.
  \\\verb!1.divmod 3 # -> [ 0, 1 ]!

  \declare{.div(number)}{\# -> integer}
  Аналогично выполнению \verb!( self / number ).to_i!. Округленная разность.
  \\\verb!1.div 3 # -> 0!

  \declare{.fdiv(number)}{\# -> float}
  Аналогично выполнению \verb!( self / number ).to_f!.
  \\\verb!1.fdiv 3 # -> 0.3333333333333333!

  \declare{.quo(number)}{\# -> number2}
  Частное двух чисел. Для двух целых чисел результатом будет рациональная дробь.
  \\\verb!1.quo 3 # -> (1/3)!

  \declare{.remainder(number)}{\# -> number2}
  Аналогично выполнению \verb!self – number * ( self / number ).truncate!.
  \\\verb!1.remainder 3 # -> 1!

  \declare{.abs}{\# -> number}
  Возвращает модуль числа.
  \\\verb!-2.1.abs # -> 2.1!

  \declare{.arg}{\# -> number}
  Синонимы: \method{angle}, \method{phase}

  Угловое значение для полярной системы координат.

  Для действительных чисел возвращает ноль, если число не отрицательно. В другом случае возвращается ссылка на константу \constant{Math::PI}.
  \begin{verbatim}
  1.arg # -> 0
  -1.arg # -> 3.141592653589793\
  \end{verbatim}

  \declare{.polar}{\# -> array}
  Возвращает ссылку на индексный массив вида \verb![ self.abs, self.arg ]!. Число в полярной системе координат.
  \\\verb!1.polar # -> [ 1, 0 ]!

  \declare{.real}{\# -> number}
  Возвращает вещественную часть числа.
  \\\verb!1.real # -> 1!

  \declare{.imag}{\# -> 0}
  Синонимы: \method{imaginary}

  Возвращает мнимую часть числа.

  \declare{.rect}{\# -> array}
  Возвращает ссылку на индексный массив вида \verb![ self, 0 ]!. Вещественная и мнимая части числа. Число в прямоугольной системе координат.
  \\\verb!1.rect # -> [ 1, 0 ]!

  \declare{.conj}{\# -> self}
  Синонимы: \method{conjugate}

  Возвращает сопряженное число (используется для комплексных чисел).
\end{methodlist}

\subsection*{Предикаты}

\begin{methodlist}
  \declare{.integer?}{}
  Проверяет относится ли число к целым.
  \begin{verbatim}
  1.integer? # -> true
  2.1.integer? # -> false\
  \end{verbatim}

  \declare{.real?}{}
  Проверяет относится ли число к действительным (false возвращается только для комплексных чисел).
  \\\verb!1.real? # -> true!

  \declare{.nonzero?}{}
  Возвращает число, если оно не равно нулю. В другом случае возвращается nil.
  \begin{verbatim}
  1.nonzero? # -> 1
  0.0.nonzero? # -> nil\
  \end{verbatim}

  \declare{.zero?}{}
  Проверяет равно ли число нулю.
  \begin{verbatim}
  1.zero? # -> false
  0.0.zero? # -> true\
  \end{verbatim}
\end{methodlist}

\subsection*{Итераторы}

\begin{methodlist}
  \declare{.step( limit, step = 1) \{|number|\}}{\# -> self}
  Последовательно перебирает числа.

  Если одно из чисел не относится к целым, то все числа преобразуются в десятичные дроби. При этом число итераций соответствует \verb!n + n * Flt::EPSILON!, где \verb!n == limit – self / step!.
\end{methodlist}

\subsection*{Остальное}

\begin{methodlist}
  \declare{.singleton_method_added(*object)}{}
  Вызывает ошибку при попытке определить собственный метод для числа.
\end{methodlist}

\section{Integer}

Абстрактный класс, описывающий основы работы с целыми числами. Производные классы Fixnum и Bignum отличаются только внутренней реализацией.

\subsection*{Приведение типов}

\begin{methodlist}
  \declare{.to_i}{\# -> self}

  \declare{.to_r}{\# -> rational}
  Синонимы: \method{rationalize}

  Преобразует целое число в рациональную дробь.
  \\\verb!0.to_r # -> (0/1)!

  \declare{.to_f}{\# -> float [Fixnum и Bignum]}
  Преобразует целое число в десятичную дробь.
  \\\verb!1.to_f # -> 1.0!

  \declare{.to_s( numeral_system = 10 )}{\# -> string [Fixnum и Bignum]}
  Преобразует целое число в текст, используя указанную систему счисления (от 2 до 36).
  \begin{verbatim}
  16.to_s 16 # -> "10"
  0xF.to_s 16 # -> "f" 
  0x16.to_s 16 # -> "16" \
  \end{verbatim}
\end{methodlist}

\subsection*{Операторы: Fixnum и Bignum}

  \begin{methodlist}
  \declare{integer * integer2}{\# -> integer3}
  Произведение.

  \declare{integer / integer2}{\# -> integer3}
  Деление.

  \declare{integer**integer2}{\# -> integer3}
  Возведение в степень.

  \declare{integer + integer2}{\# -> integer3}
  Сумма.

  \declare{integer - integer2}{\# -> integer3}
  Разность.

  \declare{\textasciitilde\-, integer}{\# -> integer2}
  Побитовое отрицание.

  \declare{integer[index]}{\# -> 0 или 1}
  Возвращает указанный бит из двоичного представления числа.

  \declare{integer \twoless integer2}{\# -> integer3}
  Побитовый сдвиг влево.

  \declare{integer \twogreat integer2}{\# -> integer3}
  Побитовый сдвиг вправо.

  \declare{integer \& integer2}{\# -> integer3}
  Побитовое И.

  \declare{integer | integer2}{\# -> integer3}
  Побитовое ИЛИ.

  \declare{integer \textasciicircum\-, integer2}{\# -> integer3}
  Побитовое исключающее ИЛИ.
\end{methodlist}

\subsection*{Арифметические операции}

\begin{methodlist}
  \declare{.next}{\# -> integer}
  Синонимы: \method{succ}

  Аналогично выполнению \verb!self + 1!.
  \\\verb!1.next # -> 2!

  \declare{.pred}{\# -> integer}
  Аналогично выполнению \verb!self - 1!.
  \\\verb!1.pred # -> 0!

  \declare{.gcd(integer)}{\# -> integer2}
  Вычисляет наибольший общий делитель для двух целых чисел. Если одно из них равно нулю, то возвращается результат вызова метода \method{.abs} для другого.
  \\\verb!2.gcd 3 # -> 1!

  \declare{.lcm(integer)}{\# -> integer2}
  Вычисляет наименьшее общее кратное для двух целых чисел. Если одно из них равно нулю, то возвращается ноль.
  \\\verb!2.lcm 3 # -> 6!

  \declare{.gcdlcm(integer)}{\# -> array}
  	Интерпретатор возвращает ссылку на индексный массив вида
  \\\verb![ self.gcd(integer), self.lcm(integer) ]!
  \\\verb!2.gcdlcm 3 # -> [ 1, 6 ]!
\end{methodlist}

\subsection*{Предикаты}

\begin{methodlist}
  \declare{.even?}{}
  Проверяет относится ли число к четным.
  \begin{verbatim}
  0.even? # -> true
  1.even? # -> false
  2.even? # -> true\
  \end{verbatim}

  \declare{.odd?}{}
  Проверяет относится ли число к нечетным.
  \begin{verbatim}
  0.odd? # -> false
  1.odd? # -> true
  2.odd? # -> false\
  \end{verbatim}
\end{methodlist}

\subsection*{Итераторы}

\begin{methodlist}
  \declare{.upto(limit) \{|integer|\}}{\# -> self}
  Последовательно перебирает числа (включительно) с шагом +1. 

  \declare{.downto(limit) \{|integer|\}}{\# -> self}
  Последовательно перебирает числа (включительно) с шагом -1. 

  \declare{.times \{|integer|\}}{\# -> self}
  Последовательно перебирает числа из диапазона \verb!0...self!.
\end{methodlist}

\subsection*{Остальное}

\begin{methodlist}
  \declare{.chr( encode = "binary")}{\# -> string}
  Возвращает символ с переданной кодовой позицией. Передача неопределенной кодовой позиции приводит к вызову ошибки (\error{RangeError}).
  \\\verb!42.chr # -> "*"!

  \declare{.hash}{\# -> integer [Fixnum и Bignum]}
  Возвращает цифровой код объекта.
  \\\verb!1.hash # -> -861462684!

  \declare{.size}{\# -> integer [Fixnum и Bignum]}
  Возвращает количество байтов, занимаемых числом.
  \\\verb!1.size # -> 4!
\end{methodlist}

\section{Float}

Десятичные дроби реализованы в Ruby как числа с плавающей точкой.
\begin{keylist}{Константы:}
  
  \firstkey{Float::ROUNDS} - способ округления чисел по умолчанию;
  
  \key{Float::RADIX} - показатель степени для представления порядка числа;
  
  \key{Float::MANT_DIG} - количество цифр в мантиссе;
  
  \key{Float::DIG} - максимально возможная точность;
  
  \key{Float::MIN_EXP} - минимально возможный показатель степени 10;
  
  \key{Float::MAX_EXP} - максимально возможный показатель степени \verb!Float::RADIX**Float::MAX_EXP - 1!; 
  
  \key{Float::MIN_10_EXP} - минимально возможная экспонента;
  
  \key{Float::MAX_10_EXP} - максимально возможная экспонента;
  
  \key{Float::MIN} - минимально возможная десятичная дробь;
  
  \key{Float::MAX} - максимально возможная десятичная дробь;
  
  \key{Float::EPSILON} - минимальное число, при добавлении к которому единицы, в результате не возвращается 1.0;
  
  \key{Float::INFINITY} - используется для бесконечности;
  
  \key{Float::NAN} - инициализируется в результате выполнения выражения \verb!0.0 / 0.0!.
\end{keylist}

\subsection*{Приведение типов}

\begin{methodlist}
  \declare{.to_f}{\# -> float}

  \declare{.to_i}{\# -> integer}
  Возвращает целую часть десятичной дроби.
  \\\verb!2.1.to_i # -> 2!

  \declare{.to_r}{\# -> rational}
  Преобразует десятичную дробь в рациональную с максимально возможной точностью.
  \\\verb!2.1.to_r # -> ( 4728779608739021 / 2251799813685248 )!

  \declare{.rationalize( number = Flt::EPSILON )}{\# -> rational}
  Преобразует десятичную дробь в рациональную, так что 
  \\\verb!( self – number.abs ) <= rational and rational <= ( self + number.abs )!
  \\\verb!2.1.rationalize # -> (21/10)!

  \declare{.to_s}{\# -> string}
  Преобразует десятичную дробь в текст. Допускается возвращение \verb!"NaN"!, \verb!"+Infinity"! или \verb!"-Infinity"!.
  \\\verb!2.1.to_s # -> "2.1"!
\end{methodlist}

\subsection*{Операторы}

\begin{methodlist}
  \declare{float * number}{\# -> number2}
  Произведение.

  \declare{float / number}{\# -> number2}
  Деление.

  \declare{float**number}{\# -> number2}
  Возведение в степень.

  \declare{float + number}{\# -> number2}
  Сумма.

  \declare{float - number}{\# -> number2}
  Разность.
\end{methodlist}

\subsection*{Предикаты}

\begin{methodlist}
  \declare{.finite?}{}
  Проверяет относится ли десятичная дробь к конечным дробям.
  \\\verb!2.1.finite? # -> true!

  \declare{.infinite?}{\# -> -1, nil, 1}
  Возвращает направление бесконечности для десятичной дроби. Если дробь относится к конечным дробям, то, в результате, возвращается nil.
  \\\verb!2.1.infinite? # -> nil!

  \declare{.nan?}{}
  Проверяет ссылается ли десятичная дробь на константу \constant{Float::NAN}.
  \\\verb!2.1.nan? # -> false!
\end{methodlist}

\subsection*{Остальное}

\begin{methodlist}
  \declare{.hash}{\# -> integer}
  Возвращает цифровой код объекта.
  \\\verb!2.1.hash # -> 569222191!
\end{methodlist}

\section{Rational}

Рациональная дробь - это рациональное число, вида \verb!(a/b)!, где число a называют числителем, а число b - знаменателем. Косая черта обозначает деление двух чисел. Рациональные дроби используются чтобы избежать ошибок приближения при работе с десятичными дробями.

Для работы с рациональными дробями в Ruby предоставлен класс Rational. 

\begin{methodlist}
  \declare{Raional( nom, denom = 1 )}{\# -> rational} 
  Возвращает рациональную дробь, вида \verb!(nom/denom)!. 
  \begin{verbatim}
  Rational 2, 3 # -> (2/3) 
  Rational "2/3" # -> (2/3) 
  Rational 8, 3 # -> (8/3) 
  Rational 2.1, 3 # -> (4728779608739021/6755399441055744)\
  \end{verbatim}
\end{methodlist}

\subsection*{Приведение типов}

\begin{methodlist}
  \declare{.inspect}{\# -> string} 
  Преобразует рациональную дробь в текст. 
  \\\verb!Rational( 2, 3 ).inspect # -> "(2/3)"!

  \declare{.to_s}{\# -> string} 
  Преобразует тело рациональной дроби в текст. 
  \\\verb!Rational( 2, 3 ).to_s # -> "2/3"!

  \declare{.to_f}{\# -> float} 
  Преобразует рациональную дробь в десятичную. 
  \\\verb!Rational( 2, 3 ).to_f # -> 0.6666666666666666!

  \declare{.to_i}{\# -> integer} 
  Аналогично выполнению self.truncate.
  \\\verb!Rational( 2, 3 ).to_i # -> 0!

  \declare{.to_r}{\# -> rational} 
\end{methodlist}

\subsection*{Операторы} 

\begin{methodlist}
  \declare{rational * number}{\# -> number} 
  Произведение. 
 
  \declare{rational / number}{\# -> number2} 
  \alias{quo} 
  Деление. 
 
  \declare{rational**number}{\# -> number2}
  Возведение в степень. 
 
  \declare{rational + number}{\# -> number2} 
  Сумма. 
 
  \declare{rational - number}{\# -> number2} 
  Разность. 
\end{methodlist}

\subsection*{Остальное} 

\begin{methodlist}
  \declare{.rationalize( number = Flt::EPSILON )}{\# -> rational}
  Преобразует рациональную дробь, так что 
  \\\verb!self – number.abs <= rational && rational <= self + number.abs!
\end{methodlist}

\section{Complex}

Комплексные числа - это подвид вещественных чисел в виде суммы \verb!(a+b*i)!, где a и b - вещественные числа, а i - мнимая единица.

Для работы с комплексными числами в Ruby предоставлен класс Complex. В этом классе удалены некоторые базовые методы Numeric (все методы, относящиеся к округлению чисел, оператор \%, методы \method{.div}, \method{.divmod}, \method{.remainder}, итератор \method{.step}).

\begin{keylist}{Константы}
  \firstkey{Complex::I}{ - мнимая единица.}
\end{keylist} 

\begin{methodlist}
  \declare{Complex( real, imag = 0 )}{\# -> complex} 
  Создание комплексного числа.
  \begin{verbatim}
  Complex 2, 3 # -> (2+3i) 
  Complex "2/3" # -> ((2/3)+0i) 
  Complex 8, 3 # -> (8+3i) 
  Complex 2.1, 3 # -> (2.1+3i) 
  Complex ?i # -> (0+1i)\
  \end{verbatim}

  \declare{::polar( magnitude, angle = 0.0 )}{\# -> complex} 
  Возвращает комплексное число в полярной системе координат. 
  \\\verb!Complex.polar 2, 3 # -> (-1.9799849932008908+0.2822400161197344i)!
  \declare{::rect( real, imag = nil )}{\# -> complex} 
  \alias{rectangular} 
  Возвращает комплексное число в прямоугольной системе координат. 
  \\\verb!Complex.rect 2, 3 # -> (2+3i)!
\end{methodlist}

\subsection*{Приведение типов}

\begin{methodlist}
  \declare{.inspect}{\# -> string} 
  Преобразует комплексное число в текст. 
  \\\verb!Complex( 2, 3 ).inspect # -> "(2+3i)"!
 
  \declare{.to_s}{\# -> string} 
  Преобразует тело комплексного числа в текст. 
  \\\verb!Complex( 2, 3 ).to_s # -> "2+3i"!
 
  \declare{.to_f}{\# -> float} 
  Преобразует комплексное число в десятичную дробь, если такое преобразование возможно. 
  \\\verb!Complex( 2, 3 ).to_f # -> error!
 
  \declare{.to_i}{\# -> integer} 
  Преобразует комплексное число в целое, если такое преобразование возможно. 
  \\\verb!Complex( 2, 3 ).to_i # -> error!
 
  \declare{.to_r}{\# -> rational} 
  Преобразует комплексное число в рациональную дробь, если такое преобразование возможно. 
  \\\verb!Complex( 2, 3 ).to_r # -> error!
 
  \declare{.rationalize( number = Flt::EPSILON )}{\# -> rational} 
  Преобразует комплексное число в рациональную дробь, если такое преобразование возможно. Переданный аргумент игнорируется. 
  \\\verb!Complex(3).rationalize true # -> (3/1)!
\end{methodlist}

\subsection*{Операторы} 

\begin{methodlist}
  \declare{complex * number}{\# -> complex2} 
  Произведение. 
  \\\verb!Complex( 2, 3 ) * 2 # -> (4+6i)!
 
  \declare{complex / number}{\# -> complex2} 
  \alias{quo} 
  Деление. 
  \\\verb!Complex( 2, 3 ) / 2 # -> ((1/1)+(3/2)*i)!
 
  \declare{complex**number}{\# -> complex2} 
  Возведение в степень. 
  \verb!Complex(2, 3)**2 # -> (-5+12i)!
 
  \declare{complex + number}{\# -> complex2} 
  Сумма. 
  \\\verb!Complex( 2, 3 ) + 2 # -> (4+3i)!
 
  \declare{complex - number}{\# -> complex2} 
  Разность. 
  \\\verb!Complex( 2, 3 ) - 2 # -> (0+3i)!
 
  \declare{-complex}{\# -> complex2} 
  Унарный минус. 
  \\\verb!-Complex( 2, 3 ) # -> (-2-3i)!
\end{methodlist}


\subsection*{Математические функции}

\begin{methodlist}
  \declare{.abs2}{\# -> number} 
  Возвращает квадрат абсолютного значения. 
  \\\verb!Complex( 2, 3 ).abs2 # # -> 13!

  \declare{.numerator}{\# -> complex}
  Возвращает числитель возможной рациональной дроби. 
  \\\verb!Complex( 2, 3 ).numerator # -> (2+3i) !
 
  \declare{.denominator}{\# -> complex}
  Возвращает знаменатель возможной рациональной дроби (наименьшее общее кратное рациональной и мнимой частей). 
  \\\verb!Complex( 2, 3 ).denominator # -> 1!

  \declare{.fdiv(number)}{\# -> complex}
  Выполняет деление каждой части в виде десятичной дроби. 
  \\\verb!Complex( 2, 3 ).fdiv 2 # -> (1.0+1.5i)!

  \declare{.abs}{\# -> number} 
  \alias{magnitude} 
  Возвращает абсолютную часть в полярной системе координат. 
  \\\verb!Complex( 2, 3 ).abs # -> 3.605551275463989 !
 
  \declare{.arg}{\# -> float} 
  \alias{angle, phase} 
  Возвращает угловое значение в полярной системе координат. 
  \\\verb!Complex( 2, 3 ).arg # -> 0.982793723247329!

  \declare{.real}{\# -> number} 
  Возвращает вещественную часть числа. 
  \\\verb!Complex( 2, 3 ).real # -> 2 !
 
  \declare{.imag}{\# -> number} 
  \alias{imaginary} 
  Возвращает мнимую часть числа. 
  \\\verb!Complex( 2, 3 ).imag # -> 3 !
 
  \declare{.rect}{\# -> float} 
  \alias{rectangular} 
  Возвращает массив вида \verb![ self.real, self.imag ]!. Вещественная и мнимая части числа в прямоугольной системе координат.
  \\\verb!Complex( 2, 3 ).rect # -> [2, 3] !
 
  \declare{.conj}{\# -> complex}
  \alias{conjugate} 
  Возвращает сопряженное комплексное число. 
  \\\verb!Complex( 2, 3 ).conj # -> (2-3i) !
\end{methodlist}

\section{Math}

Модуль содержит определение различных математических функций. Все методы могут быть вызваны либо как методы класса (для модуля Math), либо как частные методы экземпляров (для любого класса, добавляющего модуль Math).

Переданные аргументы, преобразуются в десятичные дроби. Поэтому в результате вызова метода также возвращается десятичная дробь.

\begin{keylist}{Константы:}
  
  \firstkey{Math::PI} - число π (пи);
  
  \key{Math::E} - число e.
\end{keylist}

\begin{methodlist}
  \declare{.acos(number)}{\# -> float}
  Возвращает арккосинус числа.

  \declare{.acosh(number)}{\# -> float}
  Возвращает гиперболический косинус числа.

  \declare{.asin(number)}{\# -> float}
  Возвращает арксинус числа. 

  \declare{.asinh(number)}{\# -> float}
  Возвращает гиперболический синус числа.

  \declare{.atan(number)}{\# -> float}
  Возвращает арктангенс числа.

  \declare{.atan2( number, number2 )}{\# -> float}
  Возвращает арктангенс двух чисел.

  \declare{.atanh(number)}{\# -> float}
  Возвращает гиперболический тангенс числа.

  \declare{.cbrt(number)}{\# -> float}
  Возвращает кубический корень числа.

  \declare{.cos(number)}{\# -> float}
  Возвращает косинус числа (в радианах).

  \declare{.cosh(number)}{\# -> float}
  Возвращает гиперболический косинус числа (в радианах).

  \declare{.erf(number)}{\# -> float}
  Возвращает функцию ошибки из числа.

  \declare{.erfc(number)}{\# -> float}
  Возвращает дополнительную функцию ошибки из числа.

  \declare{.exp(number)}{\# -> float}
  Аналогично выполнению \verb!Math::E**number!.

  \declare{.frexp(number)}{\# -> array}
  Возвращает ссылку на индексный массив вида \verb![ float, integer]!, где \verb!float * 2**integer == number!.

  \declare{.gamma(number)}{\# -> float}
  Возвращает гамма функцию из числа.

  \declare{.hypot( number, number2 )}{\# -> float}
  Аналогично выполнению \verb!Math.sqrt(number**2 + number2**2)!.

  \declare{.ldexp( float, integer )}{\# -> float}
  Аналогично выполнению \verb!float * 2**integer!.

  \declare{.lgamma(number)}{\# -> array}
  Возвращает ссылку на индексный массив вида 
  \\\verb![ Math.log( Math.gamma(number).abs), Math.gamma(number)<0 ? -1: 1 ]!.

  \declare{.log( number, base = Math::E )}{\# -> float}
  Возвращает логарифм числа по заданному основанию.

  \declare{.log10(number)}{\# -> float}
  Возвращает десятичный логарифм числа.

  \declare{.log2(number)}{\# -> float}
  Возвращает логарифм числа по основанию 2.

  \declare{.sin(number)}{\# -> float}
  Возвращает синус числа (в радианах).

  \declare{.sinh(number)}{\# -> float}
  Возвращает гиперболический синус числа (в радианах).

  \declare{.sqrt(number)}{\# -> float}
  Возвращает квадратный корень числа.

  \declare{.tan(number)}{\# -> float}
  Возвращает тангенс числа (в радианах).

  \declare{.tanh(number)}{\# -> float}
  	Интерпретатор возвращает гиперболический тангенс числа (в радианах).
\end{methodlist}
\input{integer}
\input{float}
\input{rational}
\input{complex}
\input{math}

  \chapter{Текст}

\section{String}

Добавленные модули: Comparable

При работе с текстом следует помнить, что интерпретатор на стадии создания объекта не анализирует его значение. Поэтому для любого текста, всегда создается новый объект, даже если объект с таким же значением уже существует.

\begin{methodlist}
  \declare{::new( string = "" )}{\# -> string}
  Cоздает новый объект.
  \\\verb!String.new # -> ""!
\end{methodlist}

\subsection*{Приведение типов}

\begin{methodlist}
  \declare{::try_convert(object)}{\# -> string}
  Преобразует объект в текст, вызывая метод \method{.to_str}. Если для объекта этот метод не определен, то возвращается nil.
  \\\verb!String.try_convert [1] # -> nil!

  \declare{.to_s}{\# -> string}
  Синонимы: \method{to_str}

  \declare{.to_sym}{\# -> sym}
  Преобразует текст в объект-идентификатор.
  \begin{verbatim}
  "abc".to_sym # -> :abc
  "123a".to_sym # -> :"123a"
  \end{verbatim}  

  \declare{.to_i( numeral_system = 10 )}{\# -> integer}
  Преобразует текст в целое число, обрабатывая его в заданной системе счисления. Обработка продолжается до первого символа, не относящегося к цифрам. Если текст начинается с такого символа или преобразование невозможно, то возвращается 0.

  \declare{.to_r}{\# -> rational}
  Преобразует текст в рациональную дробь. Обработка продолжается до первого символа, не относящегося к цифрам. Если текст начинается с такого символа или преобразование невозможно, то возвращается (0/1). 

  \declare{.to_f}{\# -> float}
  Преобразует текст в десятичную дробь. Обработка продолжается до первого символа, не относящегося к цифрам. Если текст начинается с такого символа или преобразование невозможно, то возвращается 0.0. 

  \declare{.to_c}{\# -> complex}
  Преобразует текст в комплексное число. Обработка продолжается до первого символа, не относящегося к цифрам. Если текст начинается с такого символа или преобразование невозможно, то возвращается (0+0i).

  \declare{.hex}{\# -> integer}
  Преобразует текст, обрабатывая его как число в шестнадцатеричной системе счисления. Обработка продолжается до первого символа, не относящегося к цифрам. Если текст начинается с такого символа или преобразование невозможно, то возвращается 0.

  \declare{.oct}{\# -> integer}
  Преобразует текст, обрабатывая его как число в шестнадцатеричной системе счисления. Обработка продолжается до первого символа, не относящегося к цифрам. Если текст начинается с такого символа или преобразование невозможно, то возвращается 0.
{\noindent
\begin{Parallel}{\textwidth}{0.55\textwidth}
\ParallelLText{
\begin{alltt}
"1".to_i # -> 1
"1a".to_i # -> 1
"1x".to_i # -> 1
"1.2" .to_i # -> 1
"4/2" .to_i # -> 4
"1 + 2".to_i # -> 1
"1   2".to_i # -> 1
"1e2".to_i # -> 1
"1_2".to_i # -> 12
"0b01 ax".to_i # -> 0
"0x01 ax".to_i # -> 0
"1+1i".to_i # -> 1\
\end{alltt} }
\ParallelRText{
\begin{alltt}
"1".to_f # -> 1.0
"1a".to_f # -> 1.0
"1x".to_f # -> 1.0
"1.2" .to_f # -> 1.2
"4/2" .to_f # -> 4.0
"1 + 2".to_f # -> 1.0
"1   2".to_f # -> 1.0
"1e2".to_f # -> 100.0
"1_2".to_f # -> 12.0
"0b01 ax".to_f # -> 0.0
"0x01 ax".to_f # -> 0.0
"1+1i".to_f # -> 1.0\
\end{alltt} }
\ParallelPar
\ParallelLText{
\begin{alltt}
"1".hex # -> 1
"1a".hex # -> 26
"1x".hex# -> 1
"1.2".hex # -> 1
"4/2".hex # -> 4
"1 + 2".hex # -> 1
"1   2".hex # -> 1
"1e2".hex # -> 482
"1_2".hex # -> 18
"0b01 ax".hex # -> 2817
"0x01 ax".hex # -> 1
"1+1i".hex # -> 1\
\end{alltt} }
\ParallelRText{
\begin{alltt}
"1".oct # -> 1
"1a".oct # -> 1
"1x".oct # -> 1
"1.2" .oct # -> 1
"4/2" .oct # -> 4
"1 + 2".oct # -> 1
"1   2".oct # -> 1
"1e2".oct # -> 1
"1_2".oct # -> 10
"0b01 ax".oct # -> 1
"0x01 ax".oct # -> 1
"1+1i".oct # -> 1\
\end{alltt} }
\ParallelPar
\ParallelLText{
\begin{alltt}
"1".to_r # -> 1/1
"1a".to_r # -> 1/1
"1x".to_r # -> 1/1
"1.2" .to_r # -> 6/5
"4/2" .to_r # -> 2/1
"1 + 2".to_r # -> 1/1
"1   2".to_r # -> 1/1
"1e2".to_r # -> 100/1
"1_2".to_r # -> 12/1
"0b01 ax".to_r # -> 0/1
"0x01 ax".to_r # -> 0/1
"1+1i".to_r # -> 1/1\
\end{alltt} }
\ParallelRText{
\begin{alltt}
"1".to_c # -> 1+0i
"1a".to_c # -> 1+0i
"1x".to_c # -> 1+0i
"1.2" .to_c # -> 1.2+0i
"4/2" .to_c # -> 2/1+0i
"1 + 2".to_c # -> 1+0i
"1   2".to_c # -> 1+0i
"1e2".to_c # -> 100.0 + 0i
"1_2".to_c # -> 12+0i
"0b01 ax".to_c # -> 0+0i
"0x01 ax".to_c # -> 0+0i
"1+1i".to_c # -> 1+1i\
\end{alltt} }
\end{Parallel}}
\end{methodlist}

\subsection*{Элементы текста}

Любой текст может быть обработан как индексный массив, содержащий отдельные символы в качестве элементов. 

В классе String определены операторы \method{[]} и \method{[]=}, использующиеся для получения и изменения части текста. Индексация символов начинается с нуля. Если индекс отрицательный, то отсчет символов ведется справа налево, начиная с -1.

\subsubsection*{string.[*object]}

\alias{slice(*object)}

\begin{methodlist}
  \declare{string[integer]}{\# -> string2}
  Возвращает символ с указанным индексом. Если индекс выходит за пределы текста, то возвращается nil.
  \begin{verbatim}
  "abc"[2] # -> "c"
  "abc"[4] # -> nil
  \end{verbatim}

  \declare{string[ start, length ]}{\# -> string2}
  Возвращает часть текста (длиной length), начиная с символа с переданным индексом (start). 
  \begin{itemize}
    \item Если количество символов выходит за пределы текста, то возвращается текст до последнего символа;
    \item Если количество символов равно нулю, то возвращается пустой текст ("");
    \item Если количество символов отрицательно, то возвращается nil;
    \item Если индекс выходит за пределы текста, то возвращается пустой текст ("");
  \end{itemize}
  \begin{verbatim}
  "abc"[ 2, 1 ] # -> "c"
  "abc"[ 2, 2 ] # -> "c"
  "abc"[ 2, 0 ] # -> ""
  "abc"[ 2, -1 ] # -> nil
  "abc"[ 3, 1 ] # -> ""
  \end{verbatim}

  \declare{string[range]}{\# -> string2}
  Возвращает символы между индексами, заданными в качестве границ диапазона. 
  \begin{itemize}
    \item Если конечная граница выходит за пределы текста, то возвращается текст до последнего символа;

    \item Если конечная граница меньше, чем начальная, то возвращается пустой текст ("");

    \item Если начальная граница выходит за пределы текста, то возвращается nil.
  \end{itemize}
  \begin{verbatim}
  "abc"[ 1..3 ] # -> "bc"
  "abc"[ 1...3 ] # -> "bc"
  "abc"[ 1...5 ] # -> "bc"
  "abc"[ 1...0 ] # -> ""
  "abc"[ 5...9 ] # -> nil
  \end{verbatim}

  \declare{string[template]}{\# -> string2}
  Возвращает часть текста, совпадающую с образцом. Если совпадений не найдено, то возвращается nil.
  \begin{verbatim}
  "abc"[ /[b-z]+/ ] # -> "bc"
  "abc"[ /b-z+/ ] # -> nil
  \end{verbatim}

  \declare{string[ reg, group ]}{\# -> string2}
  Возвращает часть текста, совпадающую с группой в теле регулярного выражения (передается идентификатор группы). Если совпадений не найдено, то возвращается nil.
  \begin{verbatim}
  "abc"[ /(b)c/, 1 ] # -> "b"
  "abc"[ /(b)c/, 3 ] # -> nil
  \end{verbatim}
\end{methodlist}

\subsubsection*{string.[*object]=}

Метод изменяет объект, для которого был вызван. В результате вызова возвращается заменяемый фрагмент.

\begin{methodlist}
  \declare{string[integer] = string2}{\# -> string2}
  Изменяет символ с указанным индексом. Если индекс выходит за пределы текста, то вызывается ошибка.
  \begin{verbatim}
  "abc"[2] = "d" # -> "d"
  "abc"[4] = "d"  # -> error!
  \end{verbatim}

  \declare{string[ start, length ] = string2}{\# -> string2}
  Изменяет часть текста (длиной length), начиная с символа с переданным индексом (start).
  \begin{itemize}
    \item Если индекс выходит за пределы текста, то вызывается ошибка;
    \item Если количество символов выходит за пределы текста, то заменяются символы до конца текста;
    \item Если количество символов равно нулю, то выполняется вставка текста;
    \item Если количество символов отрицательно, то вызывается ошибка.
  \end{itemize}
  \begin{verbatim}
  "abc"[ 4, 1 ] = "d" # -> error!
  "abc"[ 3, 1 ] = "d" # -> "d"
  string # -> "abdc"
  "abc"[ 2, 1 ] = "d" # -> "d"
  string # -> "abd"
  "abc"[ 2, 2 ] = "d" # -> "d"
  string # -> "abd"
  "abc"[ 2, 0 ] = "d" # -> "d"
  string # -> "abdc"
  "abc"[ 2, -1 ] = "d" # -> error!
  \end{verbatim}

  \declare{string[range] = string2}{\# -> string2}
  Изменяет символы между индексами, заданными в качестве границ диапазона. 
  \begin{itemize}
    \item Если конечная граница выходит за пределы текста, то заменяются символы до конца текста;
    \item Если конечная граница меньше, чем начальная, то текст вставляется перед символом с индексом, заданным начальной границей диапазона;
    \item Если начальная граница выходит за пределы текста, то вызывается ошибка.
  \end{itemize}
  \begin{verbatim}
  "abc"[ 1...2 ] = "d" # -> "d"
  string # -> "adc"
  "abc"[1...5] = "d" # -> "d"
  string # -> "ad"
  "abc"[ 1...0 ] = "d" # -> "d"
  string # -> "adbc"
  "abc"[ 5...9 ] = "d" # -> error!
  \end{verbatim}

  \declare{string[template] = string2}{\# -> string2}
  Изменяет часть текста, совпадающую с образцом. Если совпадений не найдено, то вызывается ошибка.
  \begin{verbatim}
  "abc"[ /[b-z]+/ ] = "d" # -> "d"
  string # -> "ad"
  "abc"[ /b-z+/ ] = "d" # -> error!
  \end{verbatim}

  \declare{string[ reg, group ]}{\# -> string2}
  Изменяет часть текста, совпадающую с группой в теле регулярного выражения (передается идентификатор группы). Если совпадений не найдено, то вызывается ошибка.
  \begin{verbatim}
  "abc"[ /(b)c/, 1 ] = "d" # -> "d"
  a # -> "adc"
  "abc"[ /(b)c/, 3 ] = "d" # -> error!
  \end{verbatim}
\end{methodlist}

\subsubsection*{Остальное}

\begin{methodlist}
  \declare{.binslice( start, length = nil)}{\# -> string}
  \verb!(range) # -> string!

  Действие метода аналогично вызову \method{.slice}, но вместо индекса символа передается порядковый номер байта.

  \declare{.length}{\# -> integer}
  \alias{size}
  Возвращает количество символов в тексте.
  \\\verb!"abc".length # -> 3!

  \declare{.bytesize}{\# -> integer}
  Возвращает количество байтов, занимаемых текстом.
  \\\verb!"abc".bytesize # -> 3!

  \declare{.getbyte(index)}{\# -> integer}
  Возвращает байт с переданым индексом. Если индекс байта выходит за пределы текста, то возвращается nil.
  \\\verb!"abc".getbyte 0 # -> 97!

  \declare{.setbyte( index, byte )}{\# -> integer}
  Изменяет байт с переданным индексом. Если индекс байта выходит за пределы текста, то вызывается ошибка.
  \begin{verbatim}
  "abc".setbyte 0, 120 # -> 120
  string # -> "xbc"
  \end{verbatim}
\end{methodlist}


\subsection*{Операторы}

\begin{methodlist}
  \declare{string \% object}{\# -> string2}
  Форматирование объекта.

  \declare{string * integer}{\# -> string2}
  Копирование текста.

  \declare{string + string2}{\# -> string3}
  Объединение текста.

  \declare{string \twoless string2}{\# -> string}
  Добавление текста.

  \declare{string <=> object}{}
  Сравнение объектов.

  \declare{string =\textasciitilde\-, template}{\# -> integer}
  Выполняется поиск совпадений с образцом и возвращает индекс символа, с которого совпадение начинается. Если совпадений не найдено, то возвращается nil.

  Если переданный объект не относится к регулярным выражениям, то интерпретатор выполняет \verb!object =~ string! и возвращает результат выполнения.
\end{methodlist}

\subsection*{Изменение текста}

\subsubsection*{Изменение регистра}

\begin{methodlist}
  \declare{.capitalize}{\# -> string}
  Изменяет первый символ в тексте на прописной, а остальные символы - на строчные. Обрабатываются только ASCII символы.
  \\\verb!"aBc".capitalize # -> "Abc"!

  \declare{.capitalize!}{\# -> self}
  Версия предыдущего метода, изменяющая значение объекта.

  \declare{.upcase}{\# -> string}
  Изменяет все символы в тексте на прописные. Обрабатываются только ASCII символы.
  \\\verb!"aBc".upcase # -> "ABC"!

  \declare{.upcase!}{\# -> self}
  Версия предыдущего метода, изменяющая значение объекта.

  \declare{.downcase}{\# -> string}
  Изменяет все символы в тексте на строчные. Обрабатываются только ASCII символы.
  \\\verb!"aBc".downcase # -> "abc"!

  \declare{.downcase!}{\# -> self}
  Версия предыдущего метода, изменяющая значение объекта.

  \declare{.swapcase}{\# -> string}
  Изменяет регистр всех символов в тексте на противоположный. Обрабатываются только ASCII символы.
  \\\verb!"aBc".swapcase # -> "AbC"!

  \declare{.swapcase!}{\# -> self}
  Версия предыдущего метода, изменяющая значение объекта, для которого метод был вызван.
\end{methodlist}

\subsubsection*{Удаление символов}

\begin{methodlist}
  \declare{.clear}{\# -> self}
  Изменяет значение объекта, удаляя из текста все символы.
  \\\verb!"abc".clear # -> ""!

  \declare{.slice!(*object)}{\# -> string}
  Аналогично выполнению \verb!self[*object] = ""!. В результате возвращается удаленная часть текста.
  \begin{verbatim}
  "abc".slice! 2 # -> "c"
  string # -> "ab"
  \end{verbatim}

  \declare{.chomp( last = \$/ )}{\# -> string}
  Удаляет один символ из конца текста (по умолчанию - символ перевода строки).
  \\\verb!"abc".chomp ?c # -> "ab"!

  \declare{.chomp!( string = \$/ )}{\# -> self}
  Версия предыдущего метода, изменяющая значение объекта. Если ни один символ не был удален, то возвращается nil.

  \declare{.chop}{\# -> string}
  Удаляет последний символ в тексте.
  \\\verb!"abc".chop # -> "ab"!

  \declare{.chop!}{\# -> self}
  Версия предыдущего метода, изменяющая значение объекта. Если ни один символ не был удален, то возвращается nil.

  \declare{.strip}{\# -> string}
  Удаляет все пробельные символы (пробел, отступ, перевод строки) из начала и конца текста.
  \\\verb!" abc ".strip # -> "abc"!

  \declare{.strip!}{\# -> self}
  Версия предыдущего метода, изменяющая значение объекта. Если ни один символ не был удален, то возвращается nil.

  \declare{.lstrip}{\# -> string}
  Удаляет все пробельные символы (пробел, отступ, перевод строки) из начала текста.
  \\\verb!" abc ".lstrip # -> "abc "!

  \declare{.lstrip!}{\# -> self}
  Версия предыдущего метода, изменяющая значение объекта. Если ни один символ не был удален, то возвращается nil.

  \declare{.rstrip}{\# -> string}
  Удаляет все пробельные символы (пробел, отступ, перевод строки) из конца текста.
  \\\verb!" abc ".rstrip # -> " abc"!

  \declare{.rstrip!}{\# -> self}
  Версия предыдущего метода, изменяющая значение объекта. Если ни один символ не был удален, то возвращается nil.
\end{methodlist}

\subsubsection*{Добавление символов}

\begin{methodlist}
  \declare{.insert( index, string )}{\# -> self}
  Аналогично выполнению \verb!self[index] = string!.
  \begin{verbatim}
  "abc".insert 2, ?d # -> "abdc"
  string # -> "abdc"
  \end{verbatim}

  \declare{.prepend(string)}{\# -> self}
  Изменяет значение объекта, добавляя в начало переданный методу текст.
  \\\verb!"Ruby".prepend "Pure " # -> "Pure Ruby"!

  \declare{.center( length, string = "~" )}{\# -> string2}
  Добавляет в начало и конец текста недостающее количество символов (до length). Если ни один символ не был добавлен, то возвращается ссылка на объект, для которого метод был вызван.
  \\\verb|"abc".center 6, ?! # -> "!abc!!"|

  \declare{.ljust( length, string = "~" )}{\# -> string2}
  Добавляет в конец текста недостающее количество символов (до length). Если ни один символ не был добавлен, то возвращается ссылка на объект, для которого метод был вызван.
  \\\verb|"abc".ljust 6, ?! # -> "abc!!!"|

  \declare{.rjust( length, string = "~" )}{\# -> string2}
  Добавляет в начало текста недостающее количество символов (до length). Если ни один символ не был добавлен, то возвращается ссылка на объект, для которого метод был вызван.
  \\\verb|"abc".rjust 6, ?! # -> "!!!abc"|
\end{methodlist}

\subsubsection*{Экранирование символов}

\begin{methodlist}
  \declare{.dump}{\# -> string}
  Экранирует все спецсимволы. Сам текст при этом экранируется двойными кавычками. Символы, не относящиеся к ASCII кодировке заменяются на их кодовые позиции.
  \begin{verbatim}
  "3\\n/2".dump # -> "\"3\\\\n/2\""
  "3\\n/2л".dump # -> "\"3\\\\n/2\\u{43b}\""
  \end{verbatim}

  \declare{.inspect}{\# -> string}
  Экранирует все спецсимволы. Сам текст при этом экранируется двойными кавычками.
  \begin{verbatim}
  "3\verb!\n!/2".inspect # -> "\"3\\verb!\\n!/2\""
  "3\verb!\n!/2л".inspect # -> "\"3\\verb!\\n!/2л\""
  \end{verbatim}
\end{methodlist}

\subsubsection*{Остальное}

\begin{methodlist}
  \declare{.next}{\# -> string}
  \alias{succ}

  Увеличивает кодовую позицию последнего символа на единицу. При этом возможна цепная реакция.
  \\\verb!"xyz".next # -> "xza"!

  \declare{.next!}{\# -> self}
  Синонимы: \method{succ!}

  Версия предыдущего метода, изменяющая значение объекта.

  \declare{.reverse}{\# -> string}
  Переставляет символы в обратном порядке.
  \\\verb!"abc".reverse # -> "cba"!

  \declare{.reverse!}{\# -> self}
  Версия предыдущего метода, изменяющая значение объекта.

  \declare{.replace(string)}{\# -> self}
  Синонимы: \method{initialize_copy}

  Изменяет значение объекта на переданное.
  \\\verb!"abc".replace ?? # -> "?"!

  \declare{.unpack(string)}{\# -> array}
  Интерпретатор распаковывает двоичный текст, используя переданную \hyperlink{apppack}{\underline{форматную строку}}.
  \\\verb!"\xFF\xFE\xFD".unpack "C*" # -> [ 255, 254, 253 ]!
\end{methodlist}

\subsection*{Поиск совпадений}

\subsubsection*{Поиск}

\begin{methodlist}
  \declare{.count(*template)}{\# -> integer}
  Возвращает количество найденных символов. Для образца позволяется использовать спецсимволы \textasciicircum\-, (отрицание) и - (диапазон). 

  Если методу передается несколько объектов, то выполняется пересечения множеств.
  \\\verb!"abc".count "a-z", "^c" # -> 2!

  \declare{.index( template, start = 0 )}{\# -> integer}
  Возвращает индекс символа, с которого начинается совпадение. Поиск совпадений выполняется начиная с символа, имеющего переданный индекс.

  Если совпадений не найдено, то возвращается nil.
  \\\verb!"abbc".index /b/ # -> 1!

  \declare{.rindex( template, start = 0 )}{\# -> integer}
  Возвращает индекс символа, с которого начинается совпадение. Поиск совпадений выполняется справа налево, до символа с переданным индексом (start).

  Если совпадений не найдено, то возвращается nil.
  \\\verb!"abbc".rindex /b/ # -> 2!

  \declare{.match( template, start = 0 )}{\# -> match}
  \verb!( template, start = 0 ) { |match| } # -> object!

  Возвращает экземпляр класса MatchData, содержащий информацию о найденных совпадениях. Поиск совпадений начинается с символа, имеющего переданный индекс (start). Если совпадений не найдено, то возвращается nil.

  \declare{.partition(template)}{\# -> array}
  Возвращает индексный массив, состоящий из трех элементов: части текста до совпадения, части текста, совпадающей с образцом, и части текста, после совпадения.

  Если совпадений не найдено, то в качестве первого элемента возвращается весь текст, а вместо остальных элементов - пустой текст ("").
  \\\verb!"abbc".partition /b/ # -> [ "a", "b", "bc" ]!

  \declare{.rpartition(template)}{\# -> array}
  Возвращает индексный массив, состоящий из трех элементов: части текста до совпадения, части текста, совпадающей с образцом и части текста, после совпадения. Поиск совпадений происходит справа налево.

  Если совпадений не найдено, то в качестве первого элемента возвращается весь текст, а вместо остальных элементов - пустой текст ("").
  \\\verb!"abbc".rpartition /b/ # -> [ "ab", "b", "c" ]!

  \declare{.split( sep = \$;, size = nil )}{\# -> array}
  Возвращает индексный массив, состоящий из частей текста. Деление на фрагменты выполняется на основе переданного разделителя (по умолчанию пробел). Несколько пробельных символов в тексте при этом игнорируются.

  Если методу передается пустое регулярное выражение, то текст делится на фрагменты посимвольно.

  Также возможно ограничит размер массива.
  \begin{verbatim}
  "a  b  c".split # -> [ "a", "b", "c" ]
  "a  b  c".split // # -> [ "a", " ", " ", "b", " ", " ", "c" ]
  "a  b  c".split //, 2 # -> [ "a", "  b  c" ]
  \end{verbatim}
\end{methodlist}

\subsubsection*{Удаление совпадений}

\begin{methodlist}
  \declare{.delete(*template)}{\# -> string}
  Удаляет из текста все найденные символы. Для образца позволяется использовать спецсимволы \textasciicircum\-, (отрицание) и - (диапазон).

  Если методу передается несколько объектов, то выполняется пересечение множеств. 
  \\\verb!"abc".delete "a-z", "^A-Z" # -> ""!

  \declare{.delete!(*template)}{\# -> self}
  Версия предыдущего метода, изменяющая значение объекта. Если ни один символ не был удален, то возвращается nil.

  \declare{.squeeze(*template)}{\# -> string}
  Удаляет все найденные рядом стоящие дублированные символы. Для образца позволяется использовать спецсимволы \textasciicircum\-, (отрицание) и - (диапазон).

  Если методу передается несколько объектов, то выполняется пересечение множеств.
  \\\verb!"aabbcc".squeeze "a-z", "^A-Z" # -> "abc"!

  \declare{.squeeze!(*template)}{\# -> self}
  Версия предыдущего метода, изменяющая значение объекта. Если ни один символ не был удален, то возвращается nil.
\end{methodlist}

\subsubsection*{Замена совпадений}

\begin{methodlist}
  \declare{.gsub( template, replace )}{\# -> string}
  \verb!(template) { |match| } # -> string!

  Изменяет в тексте все найденные совпадения. При этом совпадающие фрагменты либо передаются в блок (и заменяются на результат выполнения блока), либо заменяются на объекты, ассоциируемые с соответствующими ключами (идентификаторами групп), либо заменяются на переданный методу текст, где текст также может иметь вид \verb!'\1'! или \verb!'\K <идентификатор>'!.
  \begin{verbatim}
  "abcab".gsub /(a)b/, '\1' # -> "aca"
  "abcab".gsub /(a)b/, 'ab' => ?y  # -> "ycy"
  "abcab".gsub( /(a)b/ ) \{ |match| match.next \} # -> "accac"
  \end{verbatim}

  \declare{.gsub!( template, replace )}{\# -> self}
  \verb!(template) { |match| } # -> self!

  Версия предыдущего метода, изменяющая значение объекта. Если ни один символ не был изменен, то возвращается nil.

  \declare{.sub( template, replace )}{\# -> string}
  \verb!(template) { |match| } # -> string!

  Изменяет в тексте первое найденное совпадение. При этом совпавший фрагменты либо передаются в блок (и заменяются на результат выполнения блока), либо заменяются на объекты, ассоциируемые с соответствующими ключами (идентификаторами групп), либо заменяются на переданный методу текст, где текст также может иметь вид \verb!'\1'! или \verb!'\K <идентификатор>'!.
  \begin{verbatim}
  "abcab".sub /(a)b/, '\1'  # -> "acab"
  "abcab".sub /(a)b/, 'ab' => ?y # -> "ycab"
  "abcab".sub( /(a)b/ ) \{ |match| match.next \} # -> "accab"
  \end{verbatim}

  \declare{.sub!( template, replace )}{\# -> self}
  \verb!(template) { |match| } # -> self!

  Версия предыдущего метода, изменяющая значение объекта. Если ни один символ не был изменен, то возвращается nil.

  \declare{.tr( template, replace )}{\# -> string}
  Изменяет все найденные символы. Для образца позволяется использовать спецсимволы \textasciicircum\-, (отрицание) и - (диапазон), а в заменяющем фрагменте только -.
  \\\verb!"abc".tr "^x-z", "X-Z" # -> "ZZZ"!

  \declare{.tr!( template, replace )}{\# -> self}
  Версия предыдущего метода, изменяющая значение объекта. Если ни один символ не был изменен, то возвращается nil.

  \declare{.tr_s( template, replace )}{\# -> string}
  Изменяет все найденные символы. При этом удаляются рядом стоящие дублированные символы. Для образца позволяется использовать спецсимволы \textasciicircum\-, (отрицание) и - (диапазон), а в заменяющем фрагменте только -.
  \\\verb!"aabbcc".tr_s "^x-z", "X-Z" # -> "Z"!

  \declare{.tr_s!( template, replace )}{\# -> self}
  Версия предыдущего метода, изменяющая значение объекта. Если ни один символ не был изменен, то возвращается nil.
\end{methodlist}

\subsection*{Предикаты}

\begin{methodlist}
  \declare{.empty?}{}
  Проверяет является ли текст пустым (\mono{""}).
  \\\verb!"abc".empty? # -> false!

  \declare{.ascii_only?}{}
  Проверяет содержит ли текст только ASCII символы.
  \\\verb!"Heлlo".ascii_only? # -> false!

  \declare{.include?(template)}{}
  Проверяет в тексте наличие совпадений. 
  \\\verb!"abc".include? "ab" # -> true!

  \declare{.end_with?(*template)}{}
  Проверяет наличие суффикса. 

  Если методу передается несколько объектов, то выполняется пересечение множеств.
  \\\verb!"abc".end_with? "a", "c" # -> true!

  \declare{.start_with?(*template)}{}
  Проверяет наличие приставки. 

  Если методу передается несколько объектов, то выполняется пересечение множеств.
  \\\verb!"abc".start_with? "a", "c" # -> true!
\end{methodlist}

\subsection*{Итераторы}

\begin{methodlist}
  \declare{.each_byte \{|byte|\}}{\# -> self}
  Синонимы: \method{bytes}

  Последовательно перебирает каждый байт текста.

  \declare{.each_char \{|char|\}}{\# -> self}
  Синонимы: \method{chars}

  Последовательно перебирает каждый символ в тексте.

  \declare{.each_line( sep = \$/ ) \{|line|\}}{\# -> self}
  Синонимы: \method{lines}

  Последовательно перебирает каждую строку в тексте. Также принимается произвольный разделитель для строк (по умолчанию - символ перевода строки).

  \declare{.each_codepoint \{|point|\}}{\# -> self}
  Синонимы: \method{codepoints}

  Последовательно перебирает кодовую позицию каждого символа в тексте.

  \declare{.upto( last, ending = false ) \{|string|\}}{\# -> self}
  Последовательно перебирает либо элементы диапазона \verb!self..last!, либо элементы диапазона \verb!self...last! (если методу передается логическая величина true).
\end{methodlist}

\subsection*{Кодировка символов}

\begin{methodlist}
  \declare{.valid_encoding?}{}
  Проверяет корректна ли информация о кодировке текста.

  \declare{.encoding}{\# -> encoding}
  Возвращает экземпляр класса Encoding, содержащий информацию о кодировке текста.
  \\\verb!"абв".encoding # ->  #<Encoding:UTF-8>!

  \declare{.force_encoding(encoding)}{\# -> self}
  Изменяет информацию о кодировке текста.

  \declare{.encode( encoding = Encoding.default_internal, options = \{\} )}{\# -> string}
  \verb!( encoding, result, options = {} ) # -> string!

  Перекодирует текст либо в переданную кодировку, либо из одной кодировки в другую. Элементы ассоциативного массива описаны в \hyperlink{appenocde}{\underline{приложении}}.

  \declare{.encode!( encoding = Encoding.default_internal, options = \{\} )}{\# -> self}
  \verb!( encoding, result, options = {} ) # -> self!

  Версия предыдущего метода, изменяющая значение объекта.
\end{methodlist}

\subsection*{Остальное}

\begin{methodlist}
  \declare{.ord}{\# -> integer}
  Возвращает первый байт в тексте. Если текст пуст, то вызывается ошибка.
  \\\verb!"abc".ord # -> 97!

  \declare{.crypt(salt)}{\# -> string}
  Кодирует текст с помощью переданного объекта, подходящего под образец вида \verb!/[\w\d./]{2,2}/!
  \\\verb!"abc".crypt "z1" # -> "z1Pgo5xjkEf8U"!

  \declare{.sum( salt = 16 )}{\# -> integer}
  Возвращает контрольную сумму. Контрольная сумма - это \verb!(сумма всех байт) % 2**salt - 1!.
  \\\verb!"abc".sum # -> 294!

  \declare{.hash}{\# -> integer}
  Возвращает цифровой код объекта.  
  \\\verb!"abc".hash # -> -913021130!

  \declare{.casecmp(object)}{}
  Аналогично выполнению \verb!self <=> object!. Регистр символов при этом не учитывается.
\end{methodlist}

\section{Регулярные выражения}

\subsection{Regexp}

\begin{keylist}{Константы}
  
  \firstkey{\constant{Regexp::IGNORECASE}} - регистр символов игнорируется (модификатор i);
  
  \key{\constant{Regexp::EXTENDED}} - пробельные символы и комментарии игнорируются (модификатор x);
  
  \key{\constant{Regexp::MULTILINE}} - многострочный режим (модификатор m);
  
  \key{\constant{Regexp::FIXEDENCODING}} - другая кодировка;
\end{keylist}

\begin{methodlist}
  \declare{::new( template, object = nil )}{\# -> regexp}
  \alias{compile}
  Создание нового регулярного выражения с помощью текста или другого регулярного выражения. Вторым аргументом передаются константы класса или произвольные объекты:
  \begin{itemize}
    \item если логическое значение аргумента true, то регистр символов будет игнорироваться;

    \item если передаются символы \verb!?n! или \verb!?N!, то в теле регулярного выражения используется ASCII кодировка.
  \end{itemize}
  \verb!Regexp.new "abc", 2 # -> /abc/x!

  \declare{::union( *template или array = nil )}{\# -> regexp}
  Создание регулярного выражения на основе переданных аргументов (выполняется объединение множеств). Без аргументов возвращается /!?/.
  \\\verb!Regexp.union [ ?0, ?1, ?2 ] # -> /0|1|2/!
\end{methodlist}

\subsubsection*{Преобразование типов}

\begin{methodlist}
  \declare{::try_convert(object)}{\# -> regexp}
  Преобразует объект в регулярное выражение с помощью метода \method{.to_regexp}. Если метод для объекта не определен, то возвращается nil.
  \\\verb!Regexp.try_convert "abc" # -> nil!

  \declare{.to_s}{\# -> string}
  Возвращает текст, содержащий тело регулярного выражения и его модификаторы.
  \\\verb!/(a-z)/i.to_s # -> "(?i-mx:(a-z))"!

  \declare{.inspect}{\# -> string}
  Возвращает текст, содержащий регулярное выражение.
  \\\verb!/(a-z)/i.inspect # -> "/(a-z)/i"!

  \declare{.source}{\# -> string}
  Возвращает текст, содержащий тело регулярного выражения c экранированными спецсимволами.
  \\\verb!/(a-z)/i.source # -> "(a-z)"!
\end{methodlist}

\subsubsection*{Операторы}

\begin{methodlist}
  \declare{regexp === string}{\# -> bool}
  Аналогично выполнению regexp =~ string.

  \declare{regexp =\textasciitilde\-, string}{\# -> integer}
  Поиск совпадений.

  \declare{\textasciitilde\-, regexp}{\# -> integer}
  Аналогично выполнению \verb!regexp =~ $_!.
\end{methodlist}

\subsubsection*{Поиск совпадений}

\begin{methodlist}
  \declare{::last_match}{\# -> match}
  \verb!(group) # -> string!

  Возвращается информация о последнем поиске совпадений. Если совпадений не найдено, то возвращается nil. 

  \declare{.match( text, start = 0 )}{\# -> match}
  \verb!( text, start = 0 ) { |match| } # -> object!

  Возвращает экземпляр класса MatchData содержащий информацию о найденных совпадениях. Поиск совпадений начинается с символа имеющего переданный индекс (start). Если совпадений не найдено, то возвращается nil.
\end{methodlist}

\subsubsection*{Кодировка символов}

\begin{methodlist}
  \declare{.encoding}{\# -> encoding}
  Возвращает используемую кодировку.
  \\\verb!/(a-z)/i.encoding # -> #<Encoding:US-ASCII>!

  \declare{.fixed_encoding?}{}
  Проверяет используется ли любая кодировка, кроме ASCII.
  \\\verb!/(a-z)/i.fixed_encoding? # -> false!
\end{methodlist}

\subsubsection*{Остальное}

\begin{methodlist}
  \declare{.casefold?}{}
  Проверяет игнорируется ли регистр символов (модификатор i).
  \\\verb!/(a-z)/i.casefold? # -> true!

  \declare{.named_captures}{\# -> hash}
  Возвращает массив идентификаторов групп, ассоциируемых с их позициями.
  \\\verb!/(?<group>a-z)/i.named_captures # -> {"group" => [1]}!

  \declare{.names}{\# -> array}
  Возвращает индексный массив, содержащий идентификаторы групп.
  \\\verb!/(?<group>a-z)/i.names # -> ["group"]!

  \declare{.options}{\# -> integer}
  Возвращает сумму чисел используемых модификаторов.
  \\\verb!/(a-z)/i.options # -> 1!

  \declare{.hash}{\# -> integer}
  Возвращает цифровой код объекта.  
  \\\verb!/(a-z)/i.hash # -> -145911848!

  \declare{::escape(text)}{\# -> string}
  \alias{quote}
  Экранирует спецсимволы. 
  
  При этом \verb!Regexp.new( Regexp.escape string ) =~ string # -> 0!
  \\\verb!Regexp.escape "'\*?{}.'" # -> "'\\*\\?\\{\\}\\.'"!
\end{methodlist}

\subsection{MatchData}
 
Экземпляры класса содержат полную информацию о найденных совпадениях.

\subsubsection*{Преобразование типов}

\begin{methodlist}
  \declare{.to_s}{\# -> string}
  Возвращает полный текст найденного совпадения.

  \declare{.inspect}{\# -> string}
  Возвращает текст, содержащий информацию об объекте.

  \declare{.to_a}{\# -> array}
  Возвращает индексный массив, содержащий все элементы найденного совпадения.
\end{methodlist}

\subsubsection*{Элементы поиска}

\begin{methodlist}
  \declare{match[index]}{\# -> string}
  Возвращает фрагмент, совпадающий с группой, имеющей переданный индекс (полный  текст совпадения - элемент с индексом 0).

  \declare{match[ first, last ]}{\# -> array}
  Возвращает индексный массив, содержащий фрагменты, совпадающие с группами, индекс которых входит между границами диапазона \verb!first..last!. Если границы диапазона отрицательны, то отсчет групп ведется справа налево.

  \declare{match[range]}{\# -> array}
  Возвращает индексный массив, содержащий фрагменты, совпадающие с группами, индекс которых входит между границами диапазона. Если границы отрицательны, то отсчет групп ведется справа налево.

  \declare{match[name]}{\# -> string}
  Возвращает фрагмент, совпадающий с группой, имеющей переданный идентификатор.

  \declare{.captures}{\# -> array}
  Возвращает индексный массив, содержащий фрагменты, совпадающие со всеми нумерованными группами.

  \declare{.offset(group)}{\# -> array}
  Возвращает индексный массив, содержащий индексы символов, с которых началось и которыми закончилось совпадение фрагмента с группой.

  \declare{.begin(group)}{\# -> integer}
  Возвращает индекс символа, с которого началось совпадение фрагмента с группой.

  \declare{.end(group)}{\# -> integer}
  Возвращает индекс символа, которым заканчивается совпадение фрагмента с группой.

  \declare{.pre_match}{\# -> string}
  Возвращает часть текста перед найденным совпадением.

  \declare{.post_match}{\# -> string}
  Возвращает часть текста после найденного совпадения.
\end{methodlist}

\subsubsection*{Остальное}

\begin{methodlist}
  \declare{.regexp}{\# -> regexp}
  Возвращает регулярное выражение, с которым происходило сравнение.

  \declare{.string}{\# -> string}
  Возвращает неизменяемую копию текста, в котором выполнялся поиск совпадений.

  \declare{.size}{\# -> integer}
  \alias{length} 
  Возвращает количество всех элементов поиска (включая полный текст совпадения).

  \declare{.names}{\# -> array}
  Возвращает индексный массив, содержащий идентификаторы групп регулярного выражения.

  \declare{.hash}{\# -> integer}
  Интерпретатор возвращает цифровой код объекта.
\end{methodlist}

\section{Кодировка}

\subsection{Кодировка текста}

Для работы с кодировками в Ruby предоставлен класс Encoding. Его экземпляры содержат информацию о кодировке, используемой в тексте.

В классе определены константы для каждой поддерживаемой кодировки. Вместо них также могут использоваться заранее определенные синонимы.
\\\verb!Encoding::UTF_8 # -> #<Encoding:UTF-8>! 

\subsubsection*{Поддерживаемые кодировки}

\begin{methodlist}
  \declare{::default_external}{\# -> encoding} 
  Возвращает внешнюю кодировку, используемую по умолчанию. 
 

  \declare{::default_external= (encoding)}{\# -> encoding} 
  Изменяет внешнюю кодировку, используемую по умолчанию. 

  \declare{::default_internal}{\# -> encoding} 
  Возвращает внутреннюю кодировку, используемую по умолчанию. 
 

  \declare{::default_internal= (encoding)}{\# -> encoding} 
  Изменяет внутреннюю кодировку, используемую по умолчанию. Передача nil удаляет информацию о кодировке.

  \declare{::locale_charmap}{\# -> string} 
  Возвращает системную кодировку. 
 
  \declare{::list}{\# -> array} 
  Возвращает массив всех поддерживаемых кодировок. 
 
  \declare{::name_list}{\# -> array} 
  Возвращает массив всех синонимов. 
 
  \declare{::aliases}{\# -> hash} 
  Возвращает массив синонимов, ассоциируемых с экземплярами класса. 
 
  \declare{::find(aliases)}{\# -> encoding} 
  Возвращает кодировку, соответствующую переданному синониму. Передача не поддерживаемого синонима приводит к вызову ошибки (для внутренней кодировки может возвращаться nil).

  Методу также могут быть переданы объекты: 
  \begin{description}
    \item["external"] - внешняя кодировка;
    \item["internal"] - внутренняя кодировка;
    \item["locale"] - локальная кодировка пользователя;
    \item["filesystem"] - кодировка файловой системы.
  \end{description}
 
  \declare{::compatible?( string, string2 )}{\# -> encoding} 
  Проверяет совместимость кодировок в переданных текстах. Если они совместимы, то в результате возвращается кодировка, поддерживаемая обоими. В другом случае возвращается nil. 
  \\\verb!Encoding.compatible? "асции", "utf-8" # -> #<Encoding:UTF-8>!
\end{methodlist}

\subsubsection*{Экземпляры}

\begin{methodlist}
  \declare{.inspect}{\# -> string} 
  Возвращает информацию об объекте. 
  \\\verb!Encoding::UTF_8.inspect # -> "#<Encoding:UTF-8>"!
 
  \declare{.name}{\# -> string} 
  \alias{to_s} 
  Возвращает информацию о кодировке. 
  \\\verb!Encoding::UTF_8.name # -> "UTF-8"!
 
  \declare{.names}{\# -> array} 
  Возвращает массив всех доступных синонимов.
  \begin{verbatim}
  Encoding::UTF_8.names 
  # -> ["UTF-8", "CP65001", "locale", "external", "filesystem"]
  \end{verbatim}  
 
  \declare{.ascii_compatible?}{} 
  Проверяет совместима ли кодировка с ASCII. 
  \\\verb!Encoding::UTF_8.ascii_compatible? # -> true!
 
  \declare{.dummy?}{} 
  Проверяет является ли кодировка фиктивной. Для фиктивных кодировок обработка символов не реализована должным образом. Метод используется для динамичных кодировок. 
  \\\verb!Encoding::UTF_8.dummy? # -> false!

  \declare{.replicate(name)}{\# -> encoding} 
  Создает новый объект, использующий ту же структуру. Использование уже существующего имени приведет к вызову ошибки.
  \\\verb!Encoding::UTF_8.replicate "утф8" # -> #<Encoding:утф8>!
\end{methodlist}

\subsection{Преобразование кодировок}

Для расширенного преобразования между различными кодировками в Ruby предоставлен класс Encoding::Converter.

\begin{keylist}{Константы:}
  
  \firstkey{Encoding::Converter::INVALID_MASK} - возврат ошибки при преобразовании некорректных байтов;
  
  \key{Encoding::Converter::INVALID_REPLACE} - замена символов при преобразовании некорректных байтов; 
  
  \key{Encoding::Converter::UNDEF_MASK} - возврата ошибки при преобразовании неопределенных символов; 
  
  \key{Encoding::Converter::UNDEF_REPLACE} - замена символов при преобразовании неопределенных символов; 
  
  \key{Encoding::Converter::UNDEF_HEX_CHARREF} - замена символов на байты \verb!&xHH! при преобразовании неопределенных символов; 
  
  \key{Encoding::Converter::UNIVERSAL_NEWLINE_DECORATOR} - замена CR (\verb!\r!) и CRLF (\verb!\r\n!) на LF (\verb!\n!); 
  
  \key{Encoding::Converter::CRLF_NEWLINE_DECORATOR} - замена LF (\verb!\n!) на CRLF (\verb!\r\n!); 
  
  \key{Encoding::Converter::CR_NEWLINE_DECORATOR} - замена LF (\verb!\n!) на CR (\verb!\r!); 
  
  \key{Encoding::Converter::XML_TEXT_DECORATOR}
  
  \key{Encoding::Converter::XML_ATTR_CONTENT_DECORATOR}
  
  \key{Encoding::Converter::XML_ATTR_QUOTE_DECORATOR}
  
  \key{Encoding::Converter::PARTIAL_INPUT} - обработка исходного текста как части другого объекта; 
  
  \key{Encoding::Converter::AFTER_OUTPUT} - цикличное преобразование исходного текста.
\end{keylist}

\begin{methodlist}
  \declare{::new( source_enc, dest_enc, options = nil )}{\# -> converter} 
  \verb!(conv_path) # -> converter!

  Создание объекта. Принимаются исходная кодировка и требуемая. Дополнительный объект влияет на \hyperlink{appencode}{\underline{преобразование}}.

  Если передается массив, то он считается путем преобразования и должен содержать двухэлементные подмассивы для каждого отдельного преобразования. Дополнительными элементами могут быть элементы, влияющие на преобразование (ассоциируемые с логической величиной).
\end{methodlist} 
 
\subsubsection*{Поиск необходимых кодировок}

\begin{methodlist}
  \declare{::asciicompat_encoding(encoding)}{\# -> encoding}
  Возвращает кодировку одновременно совместимую с переданной и с ASCII. Если это переданная кодировка, то возвращается nil.
  \begin{verbatim}
  Encoding::Converter.asciicompat_encoding "utf-8" # -> nil 
  Encoding::Converter.asciicompat_encoding "utf-16le" 
  # -> #<Encoding:UTF-8>
  \end{verbatim}

  \declare{::search_convpath( source_enc, dest_enc, options = nil )}{\# -> array}
  Возвращает путь преобразования.
  \begin{verbatim}
  Encoding::Converter.search_convpath "ISO-8859-1", "EUC-JP",
    universal_newline: true 
  # -> [ [ #<Encoding:ISO-8859-1>, #<Encoding:UTF-8> ],
  #   [ #<Encoding:UTF-8>, #<Encoding:EUC-JP> ],
  #   "universal_newline" ]
  \end{verbatim}
\end{methodlist}

\subsubsection*{Статистика}

\begin{methodlist}
  \declare{.inspect}{\# -> string} 
  Возвращает информацию об объекте.
  \begin{verbatim}
  Encoding::Converter.new( "ISO-8859-1", "EUC-JP" ).inspect 
  # -> "#<Encoding::Converter: ISO-8859-1 to EUC-JP>"
  \end{verbatim} 
 
  \declare{.convpath}{\# -> array} 
  Возвращает ссылку путь преобразования.
  \begin{verbatim}
  Encoding::Converter.new( "ISO-8859-1", "EUC-JP" ).convpath 
  # -> [ [ #<Encoding:ISO-8859-1>, #<Encoding:UTF-8> ],
  #   [ #<Encoding:UTF-8>, #<Encoding:EUC-JP> ] ]
  \end{verbatim}
 
  \declare{.source_encoding}{\# -> encoding} 
  Возвращает исходную кодировку.
  \begin{verbatim}
  Encoding::Converter.new("ISO-8859-1",
    "EUC-JP").source_encoding 
  # -> #<Encoding:ISO-8859-1>
  \end{verbatim}
 
  \declare{.destination_encoding}{\# -> encoding} 
  Возвращает требуемую кодировку. 
  \begin{verbatim}
  Encoding::Converter.new("ISO-8859-1",
    "EUC-JP").destination_encoding 
  # -> #<Encoding:EUC-JP>
  \end{verbatim}

  \declare{.replacement}{\# -> string} 
  Возвращает текст для замены некорректных байтов или неопределенных символов. 
  \begin{verbatim}
  Encoding::Converter.new( "ISO-8859-1", "EUC-JP" ).replacement
  # -> "?"
  \end{verbatim}
 
  \declare{.replacement= (string)}{\# -> string} 
  Изменяет текст для замены некорректных байтов или неопределенных символов. 
 
  \declare{.last_error}{\# -> error}
  Возвращает последнюю вызванную ошибку.
\end{methodlist}

\subsubsection*{Преобразование}

\begin{methodlist}
  \declare{.convert(text)}{\# -> string} 
  Преобразование текста. При обработке некорректных байтов или неопределенных символов в любом случае вызывается ошибка. 
 
  \declare{.primitive_convert( text, result, start = nil, bytesize = nil, options = nil )}{\# -> symbol}
  Преобразование из одного текста в другой. Позволяется ограничивать фрагмент, в который сохраняется результат (по умолчанию в конец текста). Дополнительный аргумент влияет на преобразование.

  \begin{keylist}{Дополнительные элементы:}
    
    \firstkey{partial_input: true}, исходный текст может быть частью другого объекта;          
    
    \key{after_output: true}, после получения результата, ожидается новый исходный текст.
  \end{keylist}
 
  Преобразование завершается при выполнении одного из следующих условий (в скобках указан возвращаемый результат):
  \begin{enumerate}
    \item Исходный текст содержит некорректные байты \\* (\verb!:invalid_byte_sequence!);
    \item Неожиданный конец исходного текста. Это возможно, \\* если \verb!:partial_input! не задан (\verb!:incomplete_input!);
    \item Исходный текст содержит неопределенные символы \\* (\verb!:undefined_conversion!);
    \item Данные выводятся до их записи. Это возможно, \\* если ключ \verb!:after_output! не задан (\verb!:after_output!);
    \item Буфер назначенного объекта полон. Это возможно, \\* если bytesize не ссылается nil (\verb!:destination_buffer_full!);
    \item Исходный текст пуст. Это возможно, \\* если \verb!:partial_input! не задан (\verb!:source_buffer_empty!);
    \item Преобразование завершено (\verb!:finished!).
  \end{enumerate}

  \declare{.primitive_errinfo}{\# -> array} 
  Возвращает информацию о последней ошибке преобразования в виде:
  \verb![ symbol, source_enc, dest_enc, ivalid_byte, undef_char ]!, где symbol - результат последнего вызова метода \method{.primitive_convert}. Другие элементы имеют смысл только для :invalid_byte_sequence, :incomplete_input или :undefined_conversion.
 
  \declare{.putback}{\# -> string} 
  Возвращает часть текста, которую будет преобразовывать при следующем вызове \method{.primitive_convert}. 
 
  \declare{.insert_output(string)}{\# -> nil} 
  Переаднный текст будет преобразован после завершения работы с объектом. 
  Если требуемая кодировка сохраняет свое состояние, то текст будет преобразован в соответствии с состоянием, которое будет обновлено после преобразования. Этот метод необходимо использовать только если при преобразовании между кодировками возникает ошибка. 
 
  \declare{.finish}{\# -> string} 
  Заканчивает преобразование и возвращает последний полученный результат. 
\end{methodlist}
  \chapter{Составные объекты}

Все составные объекты добавляют модуль Enumerable, содержащий набор методов, работающих на основе вызова метода \method{.each}.

\section{Array}

Индексные массивы.

Добавленные модули: Enumerable 

\begin{methodlist}
  \declare{::new( size = 0, object = nil )}{\# -> array}
  \verb!(array) # -> array!
  \verb!(size) { |index| } # -> array!

  Создание индексного массива. Размер массива может изменяться в зависимости от переданных аргументов. Элементы определяются в результате выполнения блока или ссылаются на дополнительно переданный методу объект.
  \begin{verbatim}
  Array.new 3, ?R # -> [ "R", "R", "R" ]
  Array.new [ 1, 2 ] # -> [ 1, 2 ]
  Array.new(3) \{ |index| index**2 \} # -> [ 0, 1, 4 ]\
  \end{verbatim}

  \declare{Array[*object]}{\# -> array}
  Возвращает индексный массив, содержащий переданные аргументы. 
  \\\verb!Array[ 1, 2, 3 ] # -> [ 1, 2, 3 ]!
\end{methodlist}

\subsection*{Приведение типов} 

\begin{methodlist}
  \declare{::try_convert(object)}{\# -> array}
  Преобразование объекта в индексный массив с помощью вызова метода \method{.to_ary}. Если для объекта этот метод не определен, то возвращается nil. 
  \\\verb!Array.try_convert 1 # -> nil!

  \declare{.to_s}{\# -> string} 
  \alias{inspect}
  Преобразование массива в текст. 
  \\\verb![ 1, 2, 3 ].to_s # -> "[1, 2, 3]"!

  \declare{.to_a}{\# -> array} 
  \alias{to_ary}

  \declare{.join( sep = \$, )}{\# -> string}
  Преобразует индексный массив в текст, используя переданный разделитель (по умолчанию nil).
  \begin{verbatim}
  [ 1, 2, 3 ].join # -> "123"

  [
     "#{msg}",
     "Class: <#\{e.class\}>",
     "Message: <#\{e.message.inspect\}>",
     "---Backtrace---",
     "#\{MiniTest::filter_backtrace(e.backtrace).join("\textbackslash n")\}",
     "---------------",
    ].join "\textbackslash n"\
  \end{verbatim}  

  \declare{.pack(format)}{\# -> string}
  Интерпретатор упаковывает массив в двоичный текст, используя переданную \hyperlink{apppack}{\underline{форматную строку}}.
  \\\verb![ -1, -2, -3 ].pack "C*" # -> "\xFF\xFE\xFD"!
\end{methodlist}

\subsection*{Элементы массива}

Для доступа к элементам используются операторы \method{[]} и \method{[]=}. Индексация элементов начинается с нуля. Если индекс отрицательный, то отсчет элементов ведется справа налево, начиная с -1. Наиболее частая проблема с массивами - передача индекса, выходящего за пределы массива.

\subsubsection*{array.[*object]} 
\alias{slice(*object)}

\begin{methodlist}
  \declare{array[index]}{\# -> object}
  \alias{at}
  Возвращает элемент с переданным индексом. Если индекс выходит за пределы массива, то возвращается nil.
  \begin{verbatim}
  [ 1, 2, 3 ][2] # -> 3 
  [ 1, 2, 3 ][4] # -> nil\
  \end{verbatim}

  \declare{array[ start, size ]}{\# -> array} 
  Возвращает часть массива.
  \begin{itemize}
    \item Если количество элементов выходит за пределы массива, то возвращается вся часть массива до последнего элемента;
    \item Если количество элементов равно нулю, то возвращается ссылка на пустой массив (\mono{[]});
    \item Если количество элементов отрицательно, то возвращается nil; 
    \item Если индекс выходит за пределы массива, то возвращается пустой массив.
  \end{itemize}
  \begin{verbatim}
  [ 1, 2, 3 ][ 2, 1 ] # -> [3] 
  [ 1, 2, 3 ][ 2, 2 ] # -> [3] 
  [ 1, 2, 3 ][ 2, 0 ] # -> [ ] 
  [ 1, 2, 3 ][ 2, -1 ] # -> nil 
  [ 1, 2, 3 ][ 3, 1 ] # -> [ ]\
  \end{verbatim}  

  \declare{array[range]}{\# -> object} 
  Возвращает элементы между индексами, заданными в качестве границ диапазона.
  \begin{itemize}
    \item Если конечная граница выходит за пределы массива, то возвращается вся часть массива до последнего элемента;
    \item Если конечная граница меньше, чем начальная, то возвращается пустой массив;
    \item Если начальная граница выходит за пределы массива, то возвращается nil. 
   \end{itemize} 
  \begin{verbatim}
  [ 1, 2, 3 ][ 1...3 ] # -> [ 2, 3 ] 
  [ 1, 2, 3 ][ 1...5 ] # -> [ 2, 3 ] 
  [ 1, 2, 3 ][ 1...0 ] # -> [ ] 
  [ 1, 2, 3 ][ 5...9 ] # -> nil\
  \end{verbatim}
\end{methodlist}

\subsubsection*{array.[*object] =}

Вызов метода приведет к изменению значения объекта. В результате возвращается измененный элемент или массив элементов. 

Если индекс выходит за пределы массива, то массив автоматически расширяется. При этом промежуточные элементы ссылаются на nil. 

Если индекс выходит за начало массива, то вызывается ошибка. 

\begin{methodlist}
  \declare{array[index] = object}{\# -> object}
  Изменяет элемент с переданным индексом. 
  \begin{verbatim}
  [ 1, 2, 3 ][2] = "d" # -> "d" 
  array # -> [ 1, 2, "d" ] 
  [ 1, 2, 3 ][4] = "d"  # -> "d" 
  array # -> [ 1, 2, 3, "d" ]\
  \end{verbatim}

  \declare{array[ start, size ] = object}{\# -> object} 
  Изменяет часть массива: если количество элементов равно нулю, то выполняется вставка элементов, а если количество элементов отрицательно, то вызывается ошибка.
  \begin{verbatim}
  [ 1, 2, 3 ][ 4, 1 ] = "d" # -> "d" 
  array # -> [ 1, 2, 3, nil, "d" ] 
  [ 1, 2, 3 ][ 3, 1 ] = "d" # -> "d" 
  array # -> [ 1, 2, 3, "d" ] 
  [ 1, 2, 3 ][ 2, 1 ] = "d" # -> "d" 
  array # -> [ 1, 2, "d" ] 
  [ 1, 2, 3 ][ 2, 2 ] = "d" # -> "d" 
  array # -> [ 1, 2, "d" ] 
  [ 1, 2, 3 ][ 2, 0 ] = "d" # -> "d" 
  array # -> [ 1, 2, "d", 3 ] 
  [ 1, 2, 3 ][ 2, -1 ] = "d" # -> error!\
  \end{verbatim}

  \declare{array[range] = object}{\# -> object} 
  Изменяет элементы между индексами, заданными в качестве границ диапазона. 
  Если конечная граница меньше, чем начальная, то элементы добавляются перед индексом, заданным начальной границей диапазона.
  \begin{verbatim}
  [ 1, 2, 3 ][1...2] = "d" # -> "d" 
  array # ->[ 1, "d", 3 ] 
  [ 1, 2, 3 ][1...5] = "d"# -> "d" 
  array # -> [ 1, "d" ] 
  [ 1, 2, 3 ][1...0] = "d"# -> "d" 
  array # -> [ 1, "d", 2, 3 ] 
  [ 1, 2, 3 ][5...9] = "d"# -> "d" 
  array # -> [ 1, 2, 3, nil, nil, "d" ]\
  \end{verbatim}
\end{methodlist}

\subsubsection*{Остальное}

\begin{methodlist}
  \declare{.fetch(index, object)}{\# -> object2}
  \verb!(index) { |index| } # -> object!

  Аналогично выполнению \verb!array[index]!. Дополнительный аргумент используется, если индекс выходит за пределы массива.
  \\\verb![ 1, 2, 3 ].fetch 3, 4 # -> 4!

  \declare{.values_at(*object)}{\# -> array} 
  Аналогично выполнению \verb!array[*object]! для каждого переданного объекта.
  \\\verb![ 1, 2, 3 ].values_at 1, 1 # -> [ 2, 2 ]!

  \declare{.sample( size = nil )}{\# -> object} 
  Возвращает ссылку на случайный элемент массива (или на массив случайных элементов). Для пустых массивов возвращается соответственно nil или пустой массив. Если аргумент равен или превышает размеры массива, то в результате элементы просто перестраиваются в случайном порядке.
  \\\verb![ 1, 2, 3 ].sample 4 # -> [ 2, 1, 3 ]!

  \declare{.last(size = 1)}{\# -> object}
  Возвращает ссылку либо на последний элемент массива, либо на массив последних элементов.
  \\\verb![ 1, 2, 3 ].last 2 # -> [ 2, 3 ]!

  \declare{.index(object)}{\# -> integer}
  \verb!{ |object| } # -> integer!

  Возвращает индекс элемента либо равного переданному объекту, либо с логическим значением итерации true. 
  \\\verb![ 1, 2, 3 ].index { |elem| elem < 3 } # -> 0!

  \declare{.rindex(object)}{\# -> integer}
  \verb!{ |object| } # -> integer!

  Возвращает индекс элемента либо равного переданному объекту, либо с логическим значением итерации true. Поиск элементов выполняется справа налево. 
  \\\verb![ 1, 2, 3 ].rindex { |elem| elem < 3 } # -> 1!
\end{methodlist}

\subsection*{Операторы}

\begin{methodlist}
  \declare{array * integer}{\# -> array2}
  Копирование. 

  \declare{array * sep}{\# -> string} 
  Преобразует тело массива в текст, используя переданный разделитель. 
  \\\verb![ 1, 2, 3 ] * ?? # -> "1?2?3"!

  \declare{array + array2}{\# -> array3} 
  Объединение элементов. 

  \declare{array - array2}{\# -> array3} 
  Удаление элементов. 

  \declare{array \twoless object}{\# -> array}
  Добавление элемента.  Изменяется значение объекта.

  \declare{array \& array2}{\# -> array3} 
  Пересечение множеств.

  \declare{array | array2}{\# -> array3} 
  Объединение множеств. 
\end{methodlist}

\subsection*{Изменение массивов}

\subsubsection*{Добавление элементов}

\begin{methodlist}
   \declare{.push(object)}{\# -> self}
  Изменяет значение объекта, добавляя в конец массива переданный методу объект.
  \\\verb![ 1, 2, 3 ].push 2 # -> [ 1, 2, 3, 2 ]!

  \declare{.unshift(object)}{\# -> self}
  Изменяет значение объекта, добавляя в начало массива переданный методу объект. 
  \\\verb![ 1, 2, 3 ].unshift 2 # -> [ 2, 1, 2, 3 ]!
\end{methodlist}

\subsubsection*{Удаление элементов} 

\begin{methodlist}
  \declare{.pop( size = 1 )}{\# -> object} 
  Изменяет значение объекта, удаляя элементы из конца массива. В результате возвращается либо удаленный объект, либо массив удаленных объектов. 
  \\\verb![ 1, 2, 3 ].pop 2 # -> [ 2, 3 ]!

  \declare{.shift( size = 1 )}{\# -> object} 
  Изменяет значение объекта, удаляя элементы из начала массива. В результате возвращается либо удаленный объект, либо массив удаленных объектов. 
  \\\verb![ 1, 2, 3 ].shift 2 # -> [ 1, 2 ]!

  \declare{.clear}{\# -> self}
  Изменяет значение объекта, удаляя все элементы.
  \\\verb![ 1, 2, 3 ].clear # -> [ ]!

  \declare{.compact}{\# -> array}
  Удаляет из массива все элементы, ссылающиеся на nil. 
  \\\verb![ 1, 2, 3 ].compact # -> [ 1, 2, 3 ]!

  \declare{.compact!}{\# -> self} 
  Версия предыдущего метода, изменяющая значение объекта. 

  \declare{.uniq}{\# -> array}
  Удаляет из массива все повторяющиеся элементы. 
  \\\verb![ 1, 2, 3, 3, 2, 1 ].uniq # -> [ 1, 2, 3 ]!

  \declare{.uniq!}{\# -> array} 
  Версия предыдущего метода, изменяющая значение объекта. Если ни один элемент не был удален, то возвращается nil. 

  \declare{.slice!(*object)}{\# -> object} 
  Изменяет значение объекта, удаляя из массива элементы \verb!array[*object]!.
  \\\verb|[ 1, 2, 3 ].slice! 1, 1 # -> [ 2 ]|

  \declare{.delete(object) \{ nil \}}{\# -> object}
  Изменяет значение объекта, удаляя из массива все элементы, равные переданному аргументу. В результате возвращается удаленный элемент. 
  Если ни один элемент не был удален, то возвращается либо nil, либо результат выполнения необязательного блока.
  \begin{verbatim}
  [ 1, 2, 3 ].delete(4) \{ "error!" \} # -> "error!"
  [ 1, 2, 3, 1 ].delete 1 # -> 1
  array # -> [ 2, 3 ]
  array = [ 2, 3, 4 ]
  [ 1, 2, 3 ].each { |elem| array.delete elem } # -> [ 1, 2, 3 ]
  array # -> [4]\
  \end{verbatim}

  \declare{.delete_at(index)}{\# -> object} 
  Изменяет значение объекта, удаляя элемент с переданным индексом. В результате возвращается удаленный объект. Если ни один элемент не был удален, то возвращается nil. 
  \begin{verbatim}
  [ 1, 2, 3 ].delete_at 1 # -> 2 
  array # -> [ 1, 3 ]\
  \end{verbatim}

  \declare{.delete_if \{ |object| \}}{\# -> self}
  \alias{reject!}
  Изменяет значение объекта, удаляя из массива все элементы, с логическим значением итерации true. Если ни один элемент не был удален, то возвращается nil.
  \\\verb![ 1, 2, 3 ].delete_if { |elem| elem < 3 } # -> [3]!

  \declare{.select! \{ |object| \}}{\# -> self} 
  Изменяет значение объекта, удаляя из массива все элементы, с логическим значением итерации false. Если ни один элемент не был удален, то возвращается nil.
  \\\verb?[ 1, 2, 3 ].select! { |elem| elem < 3 } # -> [ 1, 2 ]?
\end{methodlist}

\subsubsection*{Замена элементов} 

\begin{methodlist}
  \declare{.replace(array)}{\# -> self}
  \alias{initialize_copy} 
  Изменяет значение объекта, заменяя его переданным аргументом. 
  \\\verb![ 1, 2, 3 ].replace [ ] # -> [ ]!
 
  \declare{.insert( *(index, *object) )}{\# -> self}
  Аналогично выполнению \verb!array[integer] = *object! для каждой пары переданных методу объектов. Результат возвращается в виде объекта или массива объектов.
  \\\verb![ 1, 2, 3 ].insert 1, 2, 3 # -> [ 1, 2, 3, 2, 3 ]!
 
  \declare{.fill( object, start = 0, size = self.size )}{\# -> self}
  \verb!( start = 0, size = self.size ) { |index| } # -> self!
  \\\verb!( object, range ) # -> self!
  \\\verb!(range) { |index| } # -> self!

  Изменяет значение объекта, заменяя элементы \verb!array[ start, size ]! или \verb!array[range]! либо на переданный объект, либо на результат выполнения блока. 
  \\\verb![ 1, 2, 3 ].fill 1 # -> [ 1, 1, 1 ]!
\end{methodlist}

\subsubsection*{Остальное}

\begin{methodlist}
  \declare{.flatten( deep = nil )}{\# -> array}
  Извлекает элементы вложенных массивов до заданного уровня. По умолчанию извлекаются все элементы. 
  \\\verb![ [[1]], [[2]], [[3]] ].flatten # -> [ 1, 2, 3 ]!
 
  \declare{.flatten!(deep)}{\# -> self}
  Версия предыдущего метода, изменяющая значение объекта. 

  \declare{.rotate( step = 1 )}{\# -> array} 
  Вращает индексный массив на заданное число позиций. Если передано положительное число, то вращение происходит слева направо, если отрицательное, то справа налево.
  \begin{verbatim}
  [ 1, 2, 3 ].rotate  # -> [ 2, 3, 1 ]
  [ 1, 2, 3 ].rotate -1  # -> [ 3, 1, 2 ]\
  \end{verbatim}

  \declare{.rotate!( step = 1 )}{\# -> array}
  Версия предыдущего метода, изменяющая значение объекта. 
\end{methodlist}

\subsection*{Сортировка массива}

\begin{methodlist}
  \declare{.reverse}{\# -> array}
  Переставляет элементы индексного массива в обратном порядке. 
  \\\verb![ 1, 2, 3 ].reverse # -> [ 3, 2, 1 ]!

  \declare{.reverse!}{\# -> self}
  Версия предыдущего метода, изменяющая значение объекта. 

  \declare{.shuffle}{\# -> array}
  Переставляет элементы индексного массива в случайном порядке. 
  \\\verb![ 1, 2, 3 ].shuffle # -> [ 2, 3, 1 ]!

  \declare{.shuffle!}{\# -> self}
  Версия предыдущего метода, изменяющая значение объекта. 

  \declare{.sort!}{\# -> self}
  \verb!{ |object, object2| } # -> self!

  Изменяет значение объекта, сортируя элементы либо с помощью оператора \method{<=>}, либо с помощью результатов выполнения блока (-1, 0, 1 или nil).

  \declare{.sort_by! \{ |object, object2| \}}{\# -> self}
  Изменяет значение объекта, сортируя элементы массива в восходящем порядке, на основе результатов их итерации. 
\end{methodlist}

\subsection*{Итераторы}

\begin{methodlist}
  \declare{.each \{ |object| \}}{\# -> self} 
  Последовательно перебирает элементы индексного массива. 

  \declare{.each_index \{ |index| \}}{\# -> self}
  Последовательно перебирает индексы массива.

  \declare{.collect! \{ |object| \}}{\# -> self}
  \alias{map!}
  Изменяет значение объекта, заменяя элементы массива на результат их итерации.
  \begin{verbatim}
  [ 1, 2, 3 ].collect! { |elem| elem + 1 } # -> [ 2, 3, 4 ]
  [1,2,3].collect! &:to_s # -> ["1", "2", "3"]\
  \end{verbatim}
  

  \declare{.combination(size) \{ |array| \}}{\# -> self} 
  Последовательно перебираются все возможные массивы заданного размера, созданные на основе элементов объекта. Различный порядок элементов при этом игнорируется.
  \begin{itemize}
    \item Если методу передается ноль, то итерируется \verb![[]]!;
    \item Если переданное число больше, чем размер объекта, то итерируется пустой массив.
  \end{itemize}
  
  \declare{.repeated_combination(size) \{ |array| \}}{\# -> self}
  Версия предыдущего метода, в которой один и тот же элемент может использоваться несколько раз.

  \declare{.permutation( size = self.size ) \{ |array| \}}{\# -> self} 
  Версия метода, учитывающая различный порядок элементов.

  \declare{.repeated_permutation( size = self.size ) \{ |array| \}}{\# -> self} 
  Версия предыдущего метода, в которой один и тот же элемент может использоваться несколько раз. 
\end{methodlist}

\subsection*{Ассоциативные массивы}

\begin{methodlist}
  \declare{.assoc(key)}{\# -> array}
  Сравнивает переданный объект с первыми элементами вложенных подмассивов. В результате возвращается вложенный массив, содержащий первое найденное совпадение. Если совпадений не найдено, то возвращается nil. 
  \\\verb![ [ :a, 1 ], [ :b, 2 ], [ :a, 3] ].assoc :a # -> [:a, 1]!

  \declare{.rassoc(object)}{\# -> array} 
  Сравнивает переданный объект с вторыми элементами вложенных подмассивов. В результате возвращается вложенный массив, содержащий первое найденное совпадение. Если совпадений не найдено, то возвращается nil. 
  \\\verb![ [ :a, 1 ], [ :b, 2 ], [ :a, 3] ].rassoc :a # -> nil!
 
  \declare{.transpose}{\# -> array}
  Обрабатывает индексный массив, состоящий из вложенных подмассивов. В результате возвращается объект, состоящий из двух вложенных массивов. Первый содержит все первые элементы, а второй - оставшиеся элементы. 
  \\\verb![ [:a, 1], [:b, 1] ].transpose # -> [ [:a, :b ], [1, 1 ] ]!
\end{methodlist} 

\subsection*{Остальное}

\begin{methodlist}
  \declare{.product( *array = nil )}{\# -> array2}
  \verb!( *array = nil ) { |array2| } # -> array3!

  Создает массив из всех возможных подмассивов размером self.size, созданных на основе всех участвующих элементов. Учитывается разный порядок элементов. Каждый элемент может быть использован в подмассиве только один раз.
  \begin{itemize}
    \item При вызове без аргументов, подмассивы будут состоять из одного элемента;
    \item Если методу передается пустой массив, то в результате также возвращается пустой массив.
  \end{itemize}
  \begin{verbatim}
  [ 1, 2 ].product [3] # -> [ [1, 3], [2, 3] ]
  [ 1, 2, 3 ].product # -> [ [1], [2], [3] ] 
  [ 1, 2, 3 ].product [] # -> []\
  \end{verbatim}

  \declare{.hash}{\# -> integer} 
  Возвращает цифровой код объекта. 
  \\\verb![ 1, 2, 3 ].hash # -> -831861323!

  \declare{.empty?}{}
  Проверяет пуст ли массив. 
  \\\verb![ 1, 2, 3 ].empty? # -> false!

  \declare{.size}{\# -> integer}
  \alias{length}
  Возвращает количество элементов в индексном массиве. Результат всегда на единицу больше, чем индекс последнего элемента. 
  \\\verb![ 1, 2, 3 ].size # -> 3!
 \end{methodlist}

\section{Hash}

Ассоциативные массивы.

Добавленные модули: Enumerable 

\begin{methodlist}
  \declare{::new( object = nil )}{\# -> hash} 
  \verb!{ |hash, key| } # -> hash!

  Создание массива. Значение по умолчанию определяется с помощью дополнительного аргумента.

  \declare{Hash[ *(key, object) ]}{\# -> hash} 
  \verb![ *[key,object] ] # -> hash!
  \\\verb![object] # -> hash!

  Создание массива на основе переданных объектов.
  \begin{verbatim}
  Hash[ :Ruby, "languages", :Ivan, "man" ] 
  # -> \{ Ruby: "languages", Ivan: "man" \} 
  
  Hash[ [ [:Ruby, "languages"], [:Ivan, "man"] ] ] 
  # -> \{ Ruby: "languages", Ivan: "man" \} 
  
  Hash[ Ruby: "languages", Ivan: "man" ] 
  # -> \{ Ruby: "languages", Ivan: "man" \}\
  \end{verbatim}
\end{methodlist} 

\subsection*{Приведение типов}

\begin{methodlist}
  \declare{::try_convert(object)}{\# -> hash} 
  Преобразование объекта в массив, с помощью метода \method{.to_hash}. Если для объекта этот метод не определен, то возвращается nil. 
  \\\verb!Hash.try_convert[1] # -> nil!
 
  \declare{.to_s}{\# -> string} 
  \alias{inspect} 
  Преобразование массива в текст. Спецсимволы экранируются. 
  \\\verb!{ a: ?a, "b" => '\n' }.to_s # -> "{:a=>\"a\", \"b\"=>\"\\\\n\"}"!

  \declare{.to_a}{\# -> array }
  Преобразование ассоциативного массива в индексный вида \verb![ *[key, object] ]!. Спецсимволы экранируются.
  \\\verb!{ a: ?a, "b" => '\n' }.to_a # -> [ [ :a, "a" ], [ "b", "\\n" ] ]!

  \declare{.to_hash}{\# -> hash} 
\end{methodlist}

\subsection*{Элементы массива} 

\begin{methodlist}
  \declare{hash[key]}{\# -> object} 
  Возвращает объект, ассоциируемый с ключом. Если ключ не найден, то возвращается значение по умолчанию. 
  \\\verb!{ a: ?a, "b" => 1 }[:a] # -> "a"!
  
  \declare{.values_at(*key)}{\# -> array}
  Возвращает индексный массив, содержащий все объекты, ассоциируемые с переданными ключами. 
  \\\verb!{ a: ?a, "b" => 1 }.values_at :a, :b, ?b # ->  [ "a", nil, 1 ]!
  
  \declare{.select \{ | key, object | \}}{\# -> hash}
  Возвращает ассоциативный массив, содержащий все элементы, с логическим значением итерации true. 
  \\\verb!{ a: ?a, "b" => 1 }.select { |key| key == ?b } # -> { "b"=>1 }!
  
  \declare{.key(object)}{\# -> key}
  Возвращает ключ, ассоциированный с заданным объектом. Если ассоциируемый объект не найден, то возвращается nil. 
  \\\verb!{ a: ?a, "b" => 1 }.key ?a # -> :a!

  \declare{.keys}{\# -> array} 
  Возвращает индексный массив, содержащий все ключи. 
  \\\verb!{ a: ?a, "b" => 1 }.keys # -> [:a, "b"]!
 
  \declare{.values}{\# -> array} 
  Возвращает индексный массив, содержащий все ассоциируемые объекты. 
  \\\verb!{ a: ?a, "b" => 1 }.values # -> [ "a", 1 ]!
 
  \declare{hash[key] = (object)}{\# -> object} 
  \alias{store}
  Изменяет значение объекта заменяя или добавляя элемент в ассоциативный массив. 
  \begin{verbatim}
  \{ a: ?a, "b" => 1 \}[:a] = 2 # -> 2 
  hash # -> \{ :a => 2, "b" => 1 \}\
  \end{verbatim}
   
  \declare{.fetch( key, default = nil )}{\# -> object} 
  \verb!(key) { |key| } # -> object!

  Возвращает объект, ассоциируемый с ключом. Дополнительный аргумент используется если ключ не найден (иначе вызывается ошибка). 
  \\\verb!{ a: ?a, "b" => 1 }.fetch :b, ?a # -> "a"!
\end{methodlist}

\subsection*{Изменение массивов}

\subsubsection*{Добавление элементов}

\begin{methodlist} 
  \declare{.merge(hash)}{\# -> hash2}
  \verb!(hash) { | key, object, object2 | } # -> hash2!

  Объединение двух массивов. Для одинаковых ключей ассоциируемыми объектами станут либо объекты из переданного массива, либо объекты, возвращаемые в результате выполнения блока.
  \\\verb!{ a: ?a, "b" => 1 }.merge( { "b" => ?b } ) # -> { a: "a", "b" => "b" }!

  \declare{.merge!(hash)}{\# -> self} 
  \verb!(hash) { | key, object, object2 | } # -> self!

  \alias{update} 
  Версия предыдущего метода, изменяющая значение объекта. 
\end{methodlist}

\subsubsection*{Удаление элементов}

\begin{methodlist}
  \declare{.clear}{\# -> self} 
  Изменяет значение объекта, удаляя все элементы. 
  \\\verb!{ a: ?a, "b" => 1 }.clear # -> { }!
 
  \declare{.shift}{\# -> array} 
  Изменяет значение ассоциативного массива, удаляя первый элемент. В результате возвращается индексный массив вида \verb![ key, object ]!. 
  \\\verb!{ a: ?a, "b" => 1 }.shift # -> [ :a, "a" ]!
 
  \declare{.delete(key)}{\# -> object}
  \verb!(key) { |key| } # -> object!

  Изменяет значение объекта, удаляя указанный элемент. В результате возвращается ассоциируемый объект. Если ключ не найден, то возвращается значение по умолчанию или результат выполнения необязательного блока. 
  \begin{verbatim}
  \{ a: ?a, "b" => 1 \}.delete :a # -> "a"
  hash # -> \{ "b"=>1 \}\
  \end{verbatim}
 
  \declare{.delete_if \{ | key, object | \}}{\# -> self}
  Изменяет значение объекта, удаляя все элементы, с логическим значением итерации true. 
  \\\verb!{ a: ?a, "b" => 1 }.delete_if { |key| key == ?b } # -> { a: "a" }!
 
  \declare{.reject \{ | key, object | \}}{\# -> self}
  Аналогично выполнению \verb!self.delete_if { | key, object | }!. Значение объекта при этом не изменяется. 
  \\\verb!{ a: ?a, "b" => 1 }.reject { |key| key == ?b } # -> { a: "a" }!

  \declare{.reject! \{ | key, object | \}}{\# -> self}
  Аналогично выполнению \verb!self.delete_if { | key, object | }!. Если ни один объект не был удален, то возвращается nil. 
  \\\verb/{ a: ?a, "b" => 1 }.reject! { |key| key == ?c } # -> nil/
 
  \declare{.keep_if \{ | key, object | \}}{\# -> self}
  Изменяет значение объекта, удаляя все элементы, с логическим значением итерации false. 
  \\\verb!{ a: ?a, "b" => 1 }.keep_if { |key| key == ?b } # -> { "b"=>1 }!
 
  \declare{.select! \{ | key, object | \}}{\# -> self}
  Аналогично выполнению \verb!self.keep_if { | key, object | }!. Если ни один элемент не был удален, то возвращается nil. 
  \\\verb/{ a: ?a, "b" => 1 }.select! { |key| key == ?b } # -> { "b"=>1 }/
\end{methodlist}

\subsubsection*{Остальное} 

\begin{methodlist}
  \declare{.replace(hash)}{\# -> self}
  \alias{initialize_copy}
  Изменяет значение объекта, копируя переданный аргумент. 
  \\\verb!{ a: ?a, "b" => 1 }.replace( { } ) # -> { }!
 
  \declare{.invert}{\# -> hash}
  Возвращает ассоциативный массив, в котором ключи и объекты меняются местами.
  \\\verb!{ a: ?a, "b" => 1 }.invert # -> { "a" => :a, 1 => "b" }!
\end{methodlist}

\subsection*{Предикаты}

\begin{methodlist}
  \declare{.has_key?(key)}{} 
  \alias{include?, key?, member?} 
  Проверяет содержит ли массив элемент с переданным ключом. 
  \\\verb!{ a: ?a, "b" => 1 }.has_key? :a # -> true!
 
  \declare{.has_value?(object)}{} 
  \alias{value?} 
  Проверяет содержит ли массив элемент с переданным ассоциируемым объектом. 
  \\\verb!{ a: ?a, "b" => 1 }.has_value? :a # -> false!
 
  \declare{.compare_by_identity?}{} 
  Проверяет сравниваются ли все ключи по их объектам-идентификаторам. 
 
  \declare{.empty?}{} 
  Проверяет пуст ли ассоциативный массив. 
  \\\verb!{ a: ?a, "b" => 1 }.empty? # -> false!
\end{methodlist} 

\subsection*{Итераторы} 

\begin{methodlist}
  \declare{.each \{ | key, object | \}}{\# -> self} 
  \alias{each_pair} 
  Последовательно перебирает элементы. 
 
  \declare{.each_key \{ |key| \}}{\# -> self} 
  Последовательно перебирает ключи. 
 
  \declare{.each_value \{ |object| \}}{\# -> self} 
  Последовательно перебирает ассоциируемые объекты. 
\end{methodlist}

\subsection*{Индексные массивы}

\begin{methodlist}
  \declare{.assoc(key)}{\# -> array} 
  Возвращает индексный массив вида \verb![ key, object ]!. Если ключ не найден, то возвращается nil. Сравнение ключей выполняется с помощью оператора \method{==}. 
  \\\verb!{ a: ?a, "b" => 1 }.assoc :a # -> nil!
 
  \declare{.rassoc(object)}{\# -> array} 
  Возвращает индексный массив вида \verb![ key, object ]!. Если объект не найден, то возвращаетсяnil. Сравнение объектов выполняется с помощью оператора ==. 
  \\\verb!{ a: ?a, "b" => 1 }.rassoc ?a # -> [ :a, "a" ]!

  \declare{.flatten( deep = 0 )}{\# -> array} 
  Возвращает индексный массив, содержащий элементы ассоциативного. Все вложенные индексные массивы будут извлекаться до заданного уровня.
  \begin{verbatim}
  \{ 1 => "one", 2 => [ [2], ["two"] ], 3 => "three" \}.flatten 3 
  # -> [ 1, "one", 2, 2, "two", 3, "three" ]\
  \end{verbatim}
\end{methodlist}

\subsection*{Остальное}

\begin{methodlist}
  \declare{.compare_by_identity}{\# -> self} 
  Изменяет информацию об элементах ассоциативного массива, так что доступны будут только те элементы, ключами для которых служат объекты-идентификаторы. (То есть все ключи будут сравниваться по их объектам-идентификаторам). 
  \begin{verbatim}
  \{ a: ?a, "b" => 1 \}.compare_by_identity
  # -> \{ :a => "a", "b" => 1 \} 
  hash[:a] # -> "a" 
  hash["b"] # -> nil
  hash[:b] # -> nil
  hash.key 1 # -> "b"\
  \end{verbatim}

  \declare{.size}{\# -> integer} 
  \alias{length} 
  Возвращает количество элементов в ассоциативном массиве. 
  \\\verb!{ a: ?a, "b" => 1 }.size # -> 2!

  \declare{.default}{\# -> object} 
  Интерпретатор возвращает значение по умолчанию для ассоциативного массива. 
 
  \declare{.default_proc}{\# -> proc} 
  Интерпретатор возвращает замыкание, выполняемую как значение по умолчанию для ассоциативного массива. Если ее нет, то возвращается nil.

  \declare{.default = (object)}{\# -> object} 
  Объявление значения по умолчанию. 
 
  \declare{.default = proc}{\# -> proc} 
  Выполнение замыкания в качестве значения по умолчанию.

  \declare{.hash}{\# -> integer} 
  Возвращает цифровой код объекта. 
  \\\verb!{ a: ?a, "b" => 1 }.hash # -> -3034512!
 
  \declare{.rehash}{\# -> hash} 
  Интерпретатор заново вычисляет цифровые коды ключей для виртуальной таблицы.
\end{methodlist}

\section{Range} 

Диапазоны.

Добавленные модули: Enumerable 

Диапазоны требуется ограничивать круглыми скобками. В противном случае метод будет вызван только для конечной границы.

\begin{methodlist}
  \declare{::new( first, last, include_last = false )}{\# -> range} 
  Возвращает диапазон с переданными начальной и конечной границами. Логическая величина влияет на обработку конечной границы.
  \\\verb!Range.new 1, 5, 0 # -> 1...5!
\end{methodlist}

\subsection*{Приведение типов} 

\begin{methodlist}
  \declare{.inspect}{\# -> string} 
  Преобразование диапазона в текст, с помощью вызова метода \method{.inspect} для каждой границы. 
  \\\verb!(1..3).inspect # -> "1..3"!
 
  \declare{.to_s}{\# -> string} 
  Преобразует диапазон в текст. 
  \\\verb!(1..3).to_s # -> "1..3"!
\end{methodlist}

\subsection*{Элементы диапазона}

\begin{methodlist}
  \declare{.begin}{\# -> object} 
  Возвращает первый элемент диапазона. 
  \\\verb!(1..3).begin # -> 1!
 
  \declare{.end}{\# -> object} 
  Возвращает последний элемент диапазона. 
  \\\verb!(1..3).end # -> 3!
 
  \declare{.last( size = nil )}{\# -> array} 
  Возвращает последний элемент диапазона или массив конечных элементов. 
  \\\verb!(1..3).last 2 # -> [ 2, 3 ]!
\end{methodlist}
  
\subsection*{Операторы}

\begin{methodlist}
  \declare{range === object}{} 
  \alias{cover?, member?, include?}  
  Проверяет входит ли объект в диапазон.
  \\\verb!(1..3) === 2 # -> true!
\end{methodlist}

\subsection*{Итераторы}

\begin{methodlist}
  \declare{.each \{ |object| \}}{\# -> self}
  Последовательно перебирает элементы диапазона с помощью метода \method{.succ}. 
 
  \declare{.step( step = 1 ) \{ |object| \}}{\# -> self}
  Последовательно перебирает элементы диапазона с указанным шагом, либо прибавляя его после каждой итерации, либо используя метод \method{.succ}.
\end{methodlist} 

\subsection*{Остальное}

\begin{methodlist}
  \declare{.exclude_end?}{} 
  Проверяет входит ли конечная граница в диапазон. 
  \\\verb!(1..3).exclude_end? # -> false!

  \declare{.hash}{\# -> integer} 
  Возвращает цифровой код объекта. 
  \\\verb!(1..3).hash # -> -337569967!
\end{methodlist}

\section{Перечни (Класс Enumerator)}

Добавленные модули: Enumerable 

Перечни - это составные объекты, содержащие информацию о хранящихся элементах и о методе, вызов которого привел к их группировке. Индексация элементов начинается с нуля. 

Если вызвать метод, отправляющий элементы в блок и не передать ему блока, то в результате возвращается ссылка на перечень, содержащий все отправленные элементы. 

\begin{methodlist}
  \declare{::new( object, method, *arg )}{\# -> enum} 
  \verb!{ |enum| } # -> enum!

  Создание перечня с помощью переданного метода. 

  Результат также может быть передан в блок. В теле блока предоставляется возможность добавлять элементы в перечень с помощью выражения \verb!enum << object! (как синоним для yield). Тело блока будет выполняться в момент использования перечня.
  \begin{verbatim}
  Enumerator.new( [ 1, 2, ?R ], :delete_at, 2 ) 
  # -> #<Enumerator: [1, 2, "R"]:delete_at(2)> 
  Enumerator.new { |enum| enum << 3 } 
  # -> #<Enumerator: <Enumerator::Generator:0x87378e8>:each>\
  \end{verbatim}

  \declare{object.enum_for( method = :each, *arg = nil )}{\# -> enum} 
  \alias{to_enum}
  Аналогично выполнению \verb!Enumerator.new( object, method, *arg )!. 
  \\\verb![ 1, 2, ?R ].enum_for # -> #<Enumerator: [1, 2, "R"]:each>!
\end{methodlist}

\subsection*{Приведение типов}

\begin{methodlist}
  \declare{.inspect}{\# -> string} 
  Преобразует перечень в текст. 
  \begin{verbatim}
  Enumerate.new( [ 1, 2, ?R ], :delete_at, 2 ).inspect 
  # -> "#<Enumerator: [1, 2, \"R\"]:delete_at(2)>"\
  \end{verbatim}
\end{methodlist}

\subsection*{Элементы перечня}

\begin{methodlist}
  \declare{.next}{\# -> object} 
  Возвращает следующий элемент перечня, сохраняя его позицию. При достижении конца перечня вызывается событие \error{StopIteration}. 
 
  \declare{.next_values}{\# -> array} 
  Возвращает индексный массив, содержащий следующий элемент перечня, сохраняя его позицию. При достижении конца перечня вызывается событие \error{StopIteration}. Этот метод может быть использован для различия между инструкциями yield и yield nil. 
 
  \declare{.peek}{\# -> object} 
  Возвращает следующий элемент перечня. При достижении конца перечня вызывается событие \error{StopIteration}. 
 
  \declare{.peek_values}{\# -> array} 
  Возвращает индексный массив, содержащий следующий элемент перечня. При достижении конца перечня вызывается событие \error{StopIteration}. Этот метод может быть использован для различия между инструкциями yield и yield nil. 
 
  \declare{.rewind}{\# -> enum} 
  Обнуляет позицию последнего извлеченного элемента.
\end{methodlist}

\subsection*{Итераторы}

\begin{methodlist}
  \declare{.each( start = 0 ) \{ |object| \}}{\# -> self} 
  Последовательно перебирает элементы перечня, используя информацию о создавшем его методе. 
 
  \declare{.with_index( start = 0 ) \{ | object, index | \}}{\# -> self} 
  Последовательно перебирает элементы перечня вместе с их индексами. Перебор начинается с элемента, имеющего индекс, переданный методу.
 
  \declare{.with_object(object) \{ | object2, object | \}}{\# -> object} 
  Аналогично выполнению 
  \\\verb!.each_with_object(object) { | object2, object | }!.
\end{methodlist}

\subsection*{Остальное}

\begin{methodlist}
  \declare{.feed( object = nil )}{\# -> nil} 
  Устанавливает результат следующей итерации перечня. При вызове без аргументов, использование инструкции yield возвращает nil.
\end{methodlist}

\section{Enumerable} 

Модуль содержит методы для работы с составными объектами, реализованные на основе итератора \method{.each}. 

Также для некоторых методов может понадобиться определение оператора \method{<=>}. 

Ассоциативные массивы преобразуются в индексные с помощью метода \method{.to_a}. 

\subsection*{Приведение типов} 

\begin{methodlist}
  \declare{.to_a}{\# -> array}
  \alias{entry} 
  Преобразует составной объект в индексный массив.
\end{methodlist}

\subsection*{Элементы объекта}

\begin{methodlist}
  \declare{.first( size = nil )}{\# -> object}
  Возвращает либо первый элемент составного объекта, либо начальную часть массива. Если методу передается ноль, то возвращается пустой массив. 
  \\\verb![ 1, 2, 3 ].first 2 # -> [ 1, 2 ]!
 
  \declare{.take(size)}{\# -> array}
  Возвращает массив элементов из начала массива. Если методу передается ноль, то возвращается пустой массив. 
  \\\verb![ 1, 2, 3 ].take 2 # -> [ 1, 2 ]!
 
  \declare{.drop(size)}{\# -> array}
  Возвращает индексный массив, содержащий все элементы составного объекта, кроме первых. 
  \\\verb![ 1, 2, 3 ].drop 2 # -> [ 3 ]!
\end{methodlist}

\subsection*{Сортировка и группировка}

\begin{methodlist}
  \declare{.sort}{\# -> array} 
  \verb!{ | object, object2 | } # -> array!

  Сортировка элементов либо с помощью оператора \method{<=>}, либо с помощью блока возвращающего -1, 0 , 1 или nil. 
  \\\verb!{ a: 1, b: 2, c: 3 }.sort # -> [ [ :a, 1 ], [ :b, 2 ], [ :c, 3 ] ]!
 
  \declare{.sort_by \{ | object, object2 | \}}{\# -> array} 
  Сортировка элементов в восходящем порядке в соответствии с результатами их итераций.
  \begin{verbatim}
  \{ a: 1, b: 2, c: 3 \}.sort_by \{ |array| -array[1] \}
  # -> [ [:c, 3], [:b, 2], [:a, 1] ]\
  \end{verbatim}  
 
  \declare{.group_by \{ |object| \}}{\# -> hash} 
  Группировка элементов. В качестве ключей будут использоваться результаты итерации элементов, а в качестве ассоциируемых объектов - индексные массивы, содержащие элементы составного объекта. 
  \\\verb![ 1, 2 ].group_by { |elem| elem > 4 } # -> { false => [ 1, 2 ] }!

  \declare{.zip(*object)}{\# -> array} 
  \verb!(*object) { |array| } # -> nil!

  Группировка элементов с одинаковыми индексами. Группы элементов могут передаваться в необязательный блок. 

  Количество групп равно размеру объекта, для которого метод был вызван. Остальные объекты при необходимости дополняются элементами, ссылающимися на nil.
  \begin{verbatim}
  \{ a: 1, b: 2 \}.zip [ 1, 2 ], [1]  
  # -> [ [ [:a, 1], 1, 1 ], [ [:b, 2], 2, nil ] ]\
  \end{verbatim}

  \declare{.chunk \{ |object| \}}{\# -> enum} 
  \verb!(buffer) { | object, buffer | } # -> enum!

  Группировка элементов с одинаковыми результатами итераций. Создаваемый перечень содержит информацию о группе элементов в индексном массиве и объединившем их результате итерации. 

  Результат итерации блока для дальнейшего использования может быть сохранен с помощью дополнительного аргумента. 

  Блок может возвращать специальные объекты:
  \begin{description}
    \item[nil или :_] - игнорировать элемент;
    \item[:_alone] - элемент будет единственным в группе.
  \end{description}   
 
  \declare{.slice_before(object)}{\# -> enum} 
  \verb!{ |object| } # -> enum!
  \\\verb!(buffer  { | object, buffer | } # -> enum!

  Группировка элементов. 

  Новая группа начинается с элемента равного переданному аргументу (сравнение выполняется с помощью оператора \method{===}), или если логическое значение итерации элемента true. 

  Первый элемент игнорируется. 

  В перечне группа объектов сохраняется в виде индексного массива. 
\end{methodlist}

\subsection*{Поиск элементов}

\begin{methodlist}
  \declare{.count( object = nil )}{\# -> integer} 
  \verb!{ |object| } # -> integer!

  Возвращает количество элементов, либо равных переданному аргументу, либо с логическим значением итерации true. При вызове без аргументов возвращает количество элементов. 
  \\\verb![ 1, 2, 3 ].count { |elem| elem < 4 } # -> 3!

  \declare{.grep(object)}{\# -> array} 
  \verb!(object) { |object| } # -> array!

  Поиск всех элементов, равных переданному аргументу (сравнение выполняется с помощью оператора \method{===}).

  Если методу передается блок, то вместо элементов возвращается результат их итерации.
  \\\verb![ 1, 2, 3 ].grep(2) { |elem| elem > 4 } # -> [false]!

  \declare{.find_all \{ |object| \}}{\# -> array} 
  \alias{select} 
  Поиск всех элементов с логическим значением итерации true. 
  \\\verb![ 1, 2, 3 ].find_all { |elem| elem > 4 } # -> [ ]!

  \declare{.reject \{ |object| \}}{\# -> array} 
  Поиск всех элементов с логическим значением итерации false. 
  \\\verb![ 1, 2, 3 ].reject { |elem| elem > 4 } # -> [ 1, 2, 3 ]!

  \declare{.partition \{ |object| \}}{\# -> array} 
  Поиск элементов с различными логическими значениями итераций. 
  \\\verb![ 1, 2, 3 ].partition { |elem| elem > 2 } # -> [ [3], [1, 2] ]!

  \declare{.detect( default = nil ) \{ |object| \}}{\# -> obj2}
  \alias{find} 
  Поиск первого элемента с логическим значением итерации true. Если искомый элемент не найден, то возвращается либо nil, либо переданный методу объект.
  \\\verb![ 1, 2, 3 ].detect { |elem| elem > 4 } # -> nil!

  \declare{.find_index( object = nil )}{\# -> index} 
  \verb!{ |object| } # -> index!

  Поиск индекса первого элемента, либо равного переданному аргументу, либо с логическим значением итерации true. Если элемент не найден, то возвращается nil. 
  \\\verb![ 1, 2, 3 ].find_index { |elem| elem > 4 } # -> nil!
\end{methodlist}

\subsection*{Сравнение элементов}

Сравнение выполняется с помощью оператора \method{<=>}.

\begin{methodlist}
  \declare{.min}{\# -> object}
  Возвращает наименьший элемент.
  \\\verb![ 1, 2, 3 ].min # -> 1!
  
  \declare{.max}{\# -> object}
  Возвращает наибольший элемент. 
  \\\verb![ 1, 2, 3 ].max # -> 3!
  
  \declare{.minmax}{\# -> array} 
  Возвращает наименьший и наибольший элементы.
  \\\verb![ 1, 2, 3 ].minmax # -> [ 1, 3 ]!
  
  \declare{.min_by \{ |object| \}}{\# -> object} 
  Возвращает элемент с наименьшим результатом итерации.
  \\\verb![ 1, 2, 3 ].min_by { |elem| -elem } # -> 3!
  
  \declare{.max_by \{ |object| \}}{\# -> object}
  Возвращает элемент с наибольшим результатом итерации.
  \\\verb![ 1, 2, 3 ].max_by { |elem| -elem } # -> 1!
  
  \declare{.minmax_by \{ |object| \}}{\# -> array} 
  Возвращает элементы с наименьшим и наибольшим результатами итерации.  
  \\\verb![ 1, 2, 3 ].minmax_by { |elem| -elem } # -> [ 3, 1 ]!
\end{methodlist}

\subsection*{Предикаты}

\begin{methodlist}  
  \declare{.include?(object)}{} 
  \alias{member?} 
  Проверяется наличие элемента, равного переданному аргументу (сравнение выполняется с помощью оператора \method{==}). 
  \\\verb![ 1, 2, 3 ].include? 4  # -> false!
  
  \declare{.all? \{ |object| \}}{} 
  Проверяется наличие элементов только с логическим значением итерации true. При вызове без аргументов последовательно проверяет каждый элемент. 
  \\\verb![ 1, 2, 3 ].all? { |elem| elem < 4 } # -> true!
  
  \declare{.any? \{ |object| \}}{} 
  Проверяется наличие любого элемента с логическим значением итерации true. При вызове без аргументов последовательно проверяет каждый элемент. 
  \\\verb![ 1, 2, 3 ].any? { |elem| elem < 4 } # -> true!
 
  \declare{.one? \{ |object| \}}{}
  Проверяется наличие только одного элемента с логическим значением итерации true. При вызове без аргументов последовательно проверяет каждый элемент. 
  \\\verb![ 1, 2, 3 ].one? { |elem| elem < 4 } # -> false!
 
  \declare{.none? \{ |object| \}}{}
  Проверяется отсутствие элементов с логическим значением итерации true. При вызове без аргументов последовательно проверяет каждый элемент. 
  \\\verb![ 1, 2, 3 ].none? { |elem| elem < 4 } # -> false!
\end{methodlist}

\subsection*{Итераторы}

\begin{methodlist}
  \declare{.collect \{ |object| \}}{\# -> array} 
  \alias{map, collect_concat, flat_map} 
  Последовательно перебирает элементы, сохраняя результат итерации в индексном массиве.
  \begin{verbatim}
  [ 1, 2, 3 ].collect { |elem| elem < 4 } # -> [ true, true, true ]
  (1..3).collect(&:next) * ?| # -> "2|3|4"\
  \end{verbatim}  

  \declare{.reverse_each( *arg = nil ) \{ |object| \}}{\# -> self} 
  Последовательно перебирает элементы в обратном порядке.
  
  \declare{.each_with_index \{ | object, index | \}}{\# -> self}
  Последовательно перебирает элементы вместе с их индексами. 
  
  \declare{.each_with_object(object) \{ | object2, object | \}}{\# -> object} 
  Последовательно перебирает элементы вместе с переданным аргументом. 
  
  \declare{.each_slice(size) \{ |array| \}}{\# -> nil} 
  Последовательно перебирает элементы, группируя их в индексные массивы заданного размера. 
  
  \declare{.each_cons(size) \{ |array| \}}{\# -> nil} 
  Последовательно перебирает элементы составного объекта, группируя их в индексные массивы заданного размера. 

  После каждой итерации из начала группы будет удален элемент, а в конец будет добавлен следующий элемент составного объекта. 
  
  \declare{.each_entry \{ |object| \}}{\# -> nil} 
  Последовательно перебирает элементы. Несколько объектов, переданных инструкции yield в теле метода \method{.each}, сохраняются в индексном массиве. 

  \declare{.drop_while \{ |object| \}}{\# -> array}
  Последовательно перебирает все элементы составного объекта, кроме первого, до итерации с логическим значением false.

  Возвращаются элементы, итерация которых не выполнялась. 
  \\\verb![ 1, 2, 3].drop_while { |elem| elem < 4 } # -> [ ]!

  \declare{.take_while \{ |object| \}}{\# -> array} 
  Последовательно перебирает все элементы составного объекта, кроме первого, до итерации с логическим значением false.

  Возвращаются элементы, итерация которых выполнялась. 
  \\\verb![ 1, 2, 3].take_while { |elem| elem < 4 } # -> [ 1, 2, 3 ]!

  \declare{.cycle( step = nil ) \{ |object| \}}{\# -> nil} 
  Последовательно перебирает элементы в бесконечном цикле. Выполнение может быть остановлено с помощью инструкций или ограничения количества выполнений цикла переборов.

  Если методу передано отрицательное число, то вызов метода завершается до выполнения.

  \declare{.inject(method)}{\# -> object}
  \verb!( first, method ) # -> object!
  \\\verb!( first = nil ) { | buffer, first | } # -> buffer!

  \alias{reduce}
  Преобразует объект, объединяя его элементы либо с помощью указанного метода, либо передавая результат каждой итерации первому параметру блока.

  Выполнение начинается с первого элемента или с дополнительно переданного аргумента. 
  \\\verb![ 1, 2, 3 ].inject( 100, :+ ) # -> 106!
\end{methodlist} 
  \chapter{Объекты}

\section{Интроспекция}

Интроспекция - это возможность определять тип и структуру объектов в процессе выполнения программы.

Интроспекция объектов выполняется с помощью методов экземпляров из классов Object, Class, Module.

В Ruby вся программа может рассматриваться в качестве объекта. Интроспекция программы выполняется с помощью методов классов Class и Module, и методов из модуля Kernel.

\subsection{Проверка выражений}
Для проверки выражений используется инструкция \method{defined?}, принимающая выражение. 
\begin{keylist}{Принимаемые аргументы:}
  
  \firstkey{Выражение}\# -> "expression";
  
  \key{Глобальная переменная}\# -> "global-variable"; 
  
  \key{Локальная переменная}\# -> "local-variable"; 
  
  \key{Переменная класса}\# -> "class-variable"; 
  
  \key{Переменная экземпляра}\# -> "instance-variable"; 
  
  \key{Константа}\# -> "constant";
  
  \key{true}\# -> "true";
  
  \key{false}\# -> "false";
  
  \key{nil}\# -> "nil"; 
  
  \key{self}\# -> "self";
  
  \key{yield}\# -> "yield";
  
  \key{super}\# -> "super"; 
  
  \key{Выражение присваивания}\# -> "assignment"; 
  
  \key{Вызов метода}\# -> "method". 
\end{keylist}

\subsection{Тип объекта}

\begin{methodlist}
  \declare{object.class}{\# -> class} 
  Возвращает класс объекта. 
  \\\verb!1.class # -> Fixnum!
 
  \declare{object.singleton_class}{\# -> class} 
  Возвращает собственный класс объекта.

  Для nil, true, false возвращаются NilClass, TrueClass, FalseClass соответственно. 

  Для чисел и объектов-идентификаторов вызывается ошибка. 
  \\\verb!?1.singleton_class # -> #<Class:#<String:0x921114c>>!

  \declare{object.nil?}{}
  Проверка отсутствия подходящего объекта. Только для nil возвращается true.

  \declare{module === object}{} 
  Проверка типа объекта.
  \begin{verbatim}
  Fixnum === 1 # -> true 
  Integer === 1 # -> true 
  Comparable === 1 # -> true\
  \end{verbatim}
  Этот метод обычно используется в альтернативном синтаксисе предложения case, позволяя проверять сразу несколько объектов.

  \declare{object.is_a?(module)}{} 
  \alias{kind_of?}
  Проверка типа объекта.
  \begin{verbatim}
  1.is_a? Fixnum # -> true 
  1.is_a? Integer # -> true 
  1.is_a? Comparable # -> true\
  \end{verbatim}  

  \declare{object.instance_of?(module)}{}
  Проверка типа объекта. 
  \begin{verbatim}
  1.instance_of? Fixnum # -> true 
  1.instance_of? Integer # -> false
  1.instance_of? Comparable # -> false\
  \end{verbatim}

  \declare{object.respond_to?( method, include_private = false )}{}
  Проверка реакции объекта на вызов метода с переданным идентификатором. Логическая величина влияет на необходимость поиска среди частных методов.

  Если метод для объекта не определен, то вместо него вызывается метод \method{.respond_to_missing?}. 
\end{methodlist}

\subsection{Иерархия наследования}

\begin{methodlist}
  \declare{class.superclass}{\# -> class} 
  Возвращает базовый класс. Для BasicObject возвращается nil. 
  \\\verb!Fixnum.superclass # -> Integer!

  \declare{module.name}{\# -> string} 
  Возвращает название модуля. Для анонимных модулей возвращается nil. 
  \\\verb!Fixnum.name # -> "Fixnum"!
 
  \declare{module.to_s}{\# -> string} 
  Для встроенных классов возвращается название модуля. Для анонимных модулей возвращается nil. 
  \\\verb!Fixnum.to_s # -> "Fixnum"!
 
  \declare{module.ancestors}{\# -> array} 
  Возвращает основную иерархию наследования, включая добавленные модули.
  \begin{verbatim}
  Fixnum.ancestors 
  # -> [Fixnum, Integer, Numeric, Comparable,
    Object, Kernel, BasicObject]\
  \end{verbatim}
 
  \declare{module.included_modules}{\# -> array} 
  Возвращает список всех добавленных модулей из основной иерархии наследования.
  \\\verb!Fixnum.included_modules # -> [ Comparable, Kernel ]!  

  \declare{module.include?(module)}{} 
  Проверяет наличие переданного модуля в иерархии наследования. 
  \\\verb!Fixnum.include? Comparable # -> true!
 
  \declare{module <=> module}{\# -> -1. 0, 1 или nil}
  Сравнение положения в иерархии наследования.
  
  Интерпретатор возвращает:
  \begin{description}
    \item[-1] - если первый модуль добавлен ко второму; 
    \item[0] - если два модуля ссылаются на один объект; 
    \item[1] - если второй модуль добавлен к первому; 
    \item[nil] - если два модуля не относятся к одной иерархии наследования.
  \end{description} 

  \declare{module < module}{} 
  Отношение в иерархии наследования. Если два модуля не относятся к одной иерархии наследования, то возвращается nil.
  \begin{verbatim}
  Integer < Numeric # -> true
  Numeric < Comparable # -> true
  Integer < Comparable # -> true\
  \end{verbatim}

  \declare{module > module}{} 
  Отношение в иерархии наследования. Если два модуля не относятся к одной иерархии наследования, то возвращается nil.
  \begin{verbatim}
  Integer > Numeric # -> false
  Numeric > Comparable # -> false
  Integer > Comparable # -> false\
  \end{verbatim}

  \declare{module <= module}{} 
  Отношение в иерархии наследования. Если два модуля не относятся к одной иерархии наследования, то возвращается nil.
  \begin{verbatim}
  Integer <= Integer # -> true 
  Numeric <= Comparable # -> true
  Integer <= Comparable # -> true\
  \end{verbatim}

  \declare{module >= module}{} 
  Отношение иерархии наследования. Если два модуля не относятся к одной иерархии наследования, то возвращается nil.
  \begin{verbatim}
  Integer >= Integer # -> true 
  Numeric => Comparable # -> false
  Integer => Comparable # -> false\
  \end{verbatim}
\end{methodlist}

\subsection{Состояние объекта}

\begin{methodlist}
  \declare{object.to_s}{\# -> string} 
  Возвращает класс и цифровой идентификатор объекта.

  \declare{object.inspect}{\# -> string} 
  Аналогично выполнению \verb!object.to_s!. 

  \declare{object.object_id}{\# -> integer} 
  Возвращает цифровой идентификатор объекта. 
  \\\verb!1.object_id # -> 3!
 
  \declare{object.__id__}{\# -> integer} 
  Возвращает цифровой идентификатор объекта.
  \\\verb!1.__id__ # -> 3!
\end{methodlist}

\subsubsection*{Состояние программы}

Так как вся программа по сути является объектом, то интроспекция также позволяет узнать о состоянии выполнения программы.

\begin{methodlist}
  \declare{global_variables}{\# -> array [PRIVATE: Kernel]} 
  Возвращает список идентификаторов глобальных переменных. 

  \declare{local_variables}{\# -> array [PRIVATE: Kernel]} 
  Возвращает список идентификаторов локальных переменных.

  \declare{Module::constants}{\# -> array} 
  Возвращает список идентификаторов всех констант в теле программы (в теле Object).

  \declare{Module::nesting}{\# -> array} 
  Возвращает иерархию наследования, в которой вызывается метод (вверх по иерархии пространства имен).

  \begin{keylist}{Переменные и константы:}
    
    \firstkey{RUBY_PATCHLEVEL} - версия интерпретатора;
    
    \key{RUBY_PLATFORM} - название используемой системы;
    
    \key{RUBY_RELEASE_DATE} - дата выпуска интерпретатора;
    
    \key{RUBY_VERSION} - версия языка;
    
    \key{\$PROGRAM_NAME (\$0)}  - имя выполняемой программы (по умолчанию - имя файла с расширением);
    
    \key{__FILE__} - имя выполняемого файла; 
    
    \key{__LINE__} - номер выполняемой строки кода;
    
    \key{__Encoding__} - кодировка программы.
  \end{keylist}

  \declare{caller( offset = 1 )}{\# -> array}
  Возвращает текущую позицию выполнения в виде массива, содержащего: \verb!"файл:строка_кода"! или \verb!"файл: строка_кода in метод"!. Число игнорируемых строк кода может быть ограничено. Если оно больше, чем количество выполненных строк кода, то возвращается nil. 

  \declare{__method__}{\# -> symbol} 
  \alias{__callee__}
  Возвращает идентификатор текущего метода. Вне тела метода возвращается nil.
\end{methodlist}

\subsubsection*{Константы} 

\begin{methodlist}
  \declare{module.constants( inherited = true )}{\# -> array} 
  Возвращает список идентификаторов всех констант в теле модуля. Логическая величина влияет на наличие унаследованных констант.

  \declare{module.const_defined?( name, inherited = true )}{}
  Проверяет существование константы. Логическая величина влияет на наличие унаследованных констант.
  \\\verb!Comparable.const_defined? :Fixnum # -> true!

  \declare{module.const_get( name, inherited = true )}{\# -> objject}
  Возвращает значение константы. Если константы не существует, то вызывается ошибка \error{NameError}. Логическая величина влияет на наличие унаследованных констант.
\end{methodlist}

\subsubsection*{Переменные класса} 

\begin{methodlist}
  \declare{.class_variables}{\# -> array} 
  Возвращает список идентификаторов всех существующих переменных класса.

  \declare{.class_variable_defined?(name)}{} 
  Проверяет существование переменной класса. 
 
  \declare{.class_variable_get(name)}{\# -> object} 
  Возвращает значение переменной класса. Если переменной не существует, то вызывается ошибка \error{NameError}. 
\end{methodlist}

\subsubsection*{Переменные экземпляра}

\begin{methodlist}
  \declare{.instance_variables}{\# -> array} 
  Возвращает список идентификаторов всех существующих (инициализированных) переменных экземпляра.

  \declare{.instance_variable_defined?(name)}{}
  Проверяет существование переменной экземпляра. 
 
  \declare{.instance_variable_get(name)}{\# -> object}
  Возвращает значение переменной экземпляра. Если переменной не существует, то вызывается ошибка \error{NameError}. 
\end{methodlist} 

\subsection{Поведение объекта}

\begin{methodlist}
  \declare{object.singleton_methods( inherited = true )}{\# -> array} 
  Возвращает список идентификаторов всех существующих собственных методов.

  \declare{object.public_methods( inherited = true )}{\# -> array} 
  Возвращает список идентификаторов всех существующих общих методов класса.

  \declare{object.protected_methods( inherited = true )}{\# -> array} 
  Возвращает список идентификаторов всех существующих защищенных методов класса.

  \declare{object.private_methods( inherited = true )}{\# -> array} 
  Возвращает список идентификаторов всех существующих частных методов класса.

  \declare{module.instance_methods( inherited = false )}{\# -> array}
  Возвращает список идентификаторов всех существующих общих и защищенных методов экземпляров.

  \declare{module.public_instance_methods( inherited = true )}{\# -> array}
  Возвращает список идентификаторов всех существующих общих методов экземпляров.

  \declare{module.protected_instance_methods( inherited = true)}{\# -> array}
  Возвращает список идентификаторов всех существующих защищенных методов экземпляров.

  \declare{module.private_instance_methods( inherited = true )}{\# -> array}
  Возвращает список идентификаторов всех существующих частных методов экземпляров.

  \declare{module.method_defined?(name)}{}
  Проверяет существование общего или защищенного метода экземпляров.
  \\\verb!Fixnum.method_defined? :next -> true!

  \declare{module.public_method_defined?(name)}{}
  Проверяет существование общего метода экземпляров. 
  \\\verb!Fixnum.public_method_defined? :next -> true!

  \declare{module.protected_method_defined?(name)}{}
  Проверяет существование защищенного метода экземпляров. 
  \\\verb!Fixnum.protected_method_defined? :next -> false!

  \declare{module.private_method_defined?(name)}{}
  Проверяет существование частного метода экземпляров. 
  \\\verb!Fixnum.private_method_defined? :next -> false!

  \declare{block_given?}{} 
  \alias{iterator?} 
  Проверяет передан ли методу блок.
\end{methodlist} 

\section{Метапрограммирование}

Метапрограммирование - это вид программирования, связанный с созданием программ, в результате работы которых создаются новые программы или программ, изменяющих себя в процессе выполнения.

\subsection{Выполнение произвольного кода}

\begin{methodlist}
  \declare{binding}{\# -> binding [PRIVATE: Kernel]}
  Создание экземпляра класса Binding, содержащего информацию о текущем состоянии выполнения программы. Этот объект может быть передан методу \method{.eval}.
 
  \declare{binding.eval( code, file = nil, line = nil )}{\# -> object }
  Выполнение произвольного кода в отдельной локальной области видимости в контексте, существовавшим при создании экземпляра класса Binding. Дополнительные аргументы используются при вызове ошибки в выполняемом фрагменте кода.
 
  \declare{eval( code, binding = nil, file = nil, line = nil )}{\# -> object [PRIVATE: Kernel]}
  \verb!{ } # -> object!

  Выполнение произвольного кода в отдельной локальной области видимости в контексте, существовавшим при создании экземпляра класса Binding. Дополнительные аргументы используются при вызове ошибки в выполняемом фрагменте кода.
 
  \declare{module.module_exec( *arg = nil ) \{ |*arg| \}}{\# -> self}
  \alias{class_exec}
  Выполнения блока кода в теле модуля. 
 
  \declare{module.module_eval( code, file = nil, line = nil )}{\# -> self}
  \verb!{ } # -> self!

  \alias{class_eval} 
  Выполнение произвольного кода или блока в теле модуля. Дополнительные аргументы используются при вызове ошибки в выполняемом фрагменте кода.
  
  \declare{object.instance_exec( *arg = nil ) \{ |*arg| \}}{\# -> self}
  Выполнение блока в области видимости объекта (псевдопеременная self ссылается на объект).

  \declare{object.instance_eval( code, file = nil, line = nil )}{\# -> obj}
  \verb!{ } # -> self!

  Выполнение произвольного кода или блока в области видимости объекта (псевдопеременная self ссылается на объект). Дополнительные аргументы используются при вызове ошибки в выполняемом фрагменте кода.

  \declare{object.tap \{ |object| \}}{\# -> object} 
  Выполняется произвольный блок, принимающий объект и возвращающий его после выполнения.
\end{methodlist}

\subsection{Вызов метода}

\begin{methodlist}
  \declare{object.send( name, *arg = nil )}{} 
  \alias{__send__} 
  Вызов переданного метода. Вызов не существующего метода приводит к возникновению ошибки. 
 
  \declare{object.public_send( name, *arg = nil )}{} 
  Вызов переданного общего метода. Вызов не существующего метода приводит к возникновению ошибки.
\end{methodlist}

\subsection{Перехват выполнения}

Перехват выполнения - это технология, позволяющая изменить стандартное поведение интерпретатора при выполнение тех или иных действий.

Перехват выполнения осуществляется с помощью объявления специальных методов, вызываемых автоматически.

\begin{methodlist}
  \declare{.method_missing( name, *arg = nil )}{} 
  Выполняется при отсутствии вызываемого метода. 

  \declare{.respond_to_missing?( name, include_private = nil )}{} 
  Выполняется если при вызове \method{.respond_to?} необходимый метод не будет найден.

  \declare{.const_missing(name)}{} 
  Выполняется при использовании не существующей константы.

  \declare{.singleton_method_added(name)}{} 
  Выполняется при объявлении собственного метода объекта. 
 
  \declare{.singleton_method_removed(name)}{} 
  Выполняется при удалении собственного метода объекта. 
 
  \declare{.singleton_method_undefined(name)}{} 
  Выполняется при запрете вызова собственного метода объекта.    

  \declare{.method_added(name)}{} 
  Выполняется при объявлении метода. 
 
  \declare{.method_removed(name)}{} 
  Выполняется при удалении метода.
 
  \declare{.method_undefined(name)}{} 
  Выполняется при запрете вызова метода. 

  \declare{.inherited(class)}{} 
  Выполняется при наследовании классу.
 
  \declare{.append_features(module)}{} 
  Выполняется при добавлении модуля в одну из иерархий наследования. 
 
  \declare{.extended(module)}{} 
  Выполняется при добавлении модуля в иерархию наследования собственных классов. 
 
  \declare{.included(module)}{} 
  Выполняется при добавлении модуля в иерархию наследования обычных классов. 
\end{methodlist}

\subsection{Изменение состояния}

\begin{methodlist}
  \declare{module.const_set( name, object )}{\# -> object} 
  Определение константы.

  \declare{module.remove_const(name)}{\# -> object [PRIVATE]} 
  Удаление константы и возвращение ее значения. Встроенные классы и модули не могут быть удалены.

  \declare{module.class_variable_set( sym, obj )}{\# -> obj} 
  Определение переменной класса.

  \declare{module.remove_class_variable(name)}{\# -> object} 
  Удаление переменной класса и возвращение ее значения. 

  \declare{object.instance_variable_set( name, object )}{\# -> object}
  Определение переменной экземпляра.   

  \declare{object.remove_instance_variable(name)}{\# -> object [PRIVATE]} 
  Удаление переменной экземпляра и возвращение ее значения.
\end{methodlist}

\subsection{Изменение поведения}

\begin{methodlist}
  \declare{object.define_singleton_method( name, block )}{\# -> block} 
  Определение собственного метода объекта. 

  \declare{module.define_method( name, block )}{\# -> block [PRIVATE]}
  Определение метода экземпляров.

  \declare{module.alias_method( new_name, old_name )}{\# -> self [PRIVATE]}
  Объявление синонима для метода. 

  \declare{module.remove_method(name)}{\# -> self [PRIVATE]} 
  Удаление метода. Унаследованные методы при этом могут быть вызваны. 
 
  \declare{module.undef_method(name)}{\# -> self [PRIVATE]} 
  Запрет вызова метода.  
\end{methodlist}

\section{Остальное}

\subsection{Приведение типов}

Приведение типа - это создание на основе переданного объекта экземпляра другого класса.
\begin{description}
  \item{Неявное приведение:}

  Для приведения типов при реализации неявной типизации используются методы \method{.to_i}, \method{.to_s}, \method{.to_a}, \method{.to_f}, \method{.to_c}, \method{.to_r} и т.д. Эти методы возвращают целые числа, текст, индексные массивы, десятичные дроби, комплексные числа и рациональные дроби в том виде, в котором это удобно интерпретатору.

  \item {Явное приведение:}

  Для того, чтобы получить требуемый экземпляр, отформатированный в удобном для человека виде реализуют методы \method{.to_int}, \method{.to_str}, \method{.to_ary}, \method{.to_sym}, \method{.to_regexp}, \method{.to_proc}, \method{.to_hash} и т.д. Эти методы возвращают соответственно целые числа, текст, индексные массивы, объекты-идентификаторы, регулярные выражения, замыкания и ассоциативные массивы.
\end{description}

Для непосредственного приведения типов вызываются частные методы экземпляров из модуля Kernel. Если приведение типов невозможно, то возвращается nil.
\begin{methodlist}
  \declare{Array(object)}{\# -> array}
  Интерпретатор выполняет выражения:
  \begin{enumerate}
    \item \verb!object.to_ary!;
    \item \verb!object.to_a!.
  \end{enumerate}

  \declare{Float(object)}{\# -> float}
  Интерпретатор выполняет выражение \verb!object.to_f!.

  \declare{Integer( object, numeral_system = 10 )}{\# -> integer}
  Интерпретатор выполняет выражения:
  \begin{enumerate}
    \item \verb!object.to_int!;
    \item \verb!object.to_i!.
  \end{enumerate}

  Если первым аргументом передается текст, то второй аргумент объявляет систему счисления (от 2 до 36). Для двоичной и шестнадцатеричной систем допускаются приставки 0b (0B) и 0x (0X). В другом случае передаваемый текст должен содержать только десятичные цифры.

  \declare{String(object)}{\# -> string}
  Интерпретатор выполняет выражение \verb!object.to_s!.

  \declare{format( format, *object )}{\# -> string}
  \alias{sprintf}
  Аналогично выполнению \verb!format % [*object]!.
\end{methodlist}

\subsection{Сравнение объектов}

\subsubsection*{Проверка равенства}

\begin{methodlist}
  \declare{object == object}{} 
  \alias{===} 
  Проверка равенства с приведением типов. 
  \\\verb!1 == 1.0 -> true!

  \declare{object.eql? object}{} 
  Проверка равенства без приведения типов. 
  \\\verb!1.eql? 1.0 -> false!
 
  \declare{object.equal? object}{} 
  Идентичность двух ссылок.
  \begin{verbatim}
  1.equal? 1.0 -> false 
  "1".equal? "1" -> false 
  1.equal? 1 -> true 
  :a.equal? :a -> true\
  \end{verbatim}
\end{methodlist}

\subsubsection*{Comparable}

В модуле определены опреаторы для проверки отношения и равенства объектов (<, <=, >, >=, ==), основанные на работе оператора \method{<=>}.

\begin{methodlist}
  \declare{.between?( first, last )}{}
  Проверяет входит ли объект между двумя заданными границами.
\end{methodlist}

\subsection{Копирование объектов}

\subsubsection*{Внутри программы}

Так как все аргументы передаются по ссылке, то изменение их в теле метода может привести к непредсказуемым последствиям. Для решения этой проблемы перед использованием аргумента обычно создают его копию.

\begin{methodlist}
  \declare{object.clone}{\# -> object2} 
  Создание копии объекта, сохраняющей все его модификаторы. 

  \declare{object.dup}{\# -> object2} 
  Создание копии объекта, разрешенной к изменению.
\end{methodlist}

\subsubsection*{Вне программы}

Маршализация позволяет сохранять произвольные объекты, извлекать их из выполняемой программы и восстанавливать в другой программе (программы должны использовать одну версию интерпретатора).

Для маршализации используется модуль Marshal.

\begin{keylist}{Константы:}
  
  \firstkey{Marshal::MAJOR_VERSION} - мажорная версяи интерпретатора;
  
  \key{Marshal::MINOR_VERSION} - минорная версия интерпретатора.
\end{keylist}

\begin{methodlist}
  \declare{::dup( object, io = nil, deep = -1 )}{\# -> string}
  Маршализация объекта. Принимается глубина вложенности маршализуемых свойств (по умолчанию не сохраняется). Если передается поток, то результат будет записан в поток.

  Невозможна маршализация:
  \begin{itemize}
    \item анонимных модулей или классов;
    \item объектов, связанных с ОС (файлы, каталоги, потоки и т.д);
    \item экземпляров MatchData, Data, Method, UnboundMethod, Proc, Thread, ThreadGroup, Continuation;
    \item объектов, определяющих собственные методы.
  \end{itemize}
  \verb!Marshal.dump ?$ # -> "\x04\bI\"\x06$\x06:\x06ET"!

  \declare{::load(marshal_data)}{\# -> object}
  \verb!(marshal_data) { |result| } # -> object!
  
  \alias{restore}
  Восстановление объекта с помощью текста или потока (используется метод \method{.to_str}). Результат может быть передан в необязательный блок.
  \\\verb!Marshal.load Marshal.dump(?$) # -> "$"!
\end{methodlist}
  \chapter{Подпрограммы}

Подпрограммы - это именованный фрагмент программы, содержащий описание определенного набора действий. К подпрограммам относятся замыкания (процедуры и функции), методы и сопрограммы.

\section{Замыкания}

С помощью класса Proc предоставляется возможность создавать замыкания. 

Замыкания - это подпрограммы, в теле которых существуют локальные переменные, объявленные в окружающей программе.

Замыкания создаются заново каждый раз в момент выполнения.

Замыкания, так же как и объекты служат для инкапсуляции функциональности и данных.

Замыкания могут быть переданы методам как обычные блоки. Для этого перед аргументом записывают амперсанд.

Замыкания разделяются на процедуры (ведущие себя как блоки, но в действительности также относящиеся к функциям) и лямбда-функции (ведущие себя как методы).

\subsection{Процедуры}

\begin{methodlist}
  \declare{::new \{ |*params| \}}{\# -> proc}
  Создание нового объекта. В теле метода может принимать переданный ему блок.
  \begin{verbatim}
  def proc_from
    Proc.new
  end
  proc = proc_from \{ "hello" \}
  proc.call # -> "hello"
  \end{verbatim}

  \declare{proc \{ |*params| \}}{\# -> proc}
  Частный метод экземпляров из модуля Kernel, аналогичный предыдущему.
\end{methodlist}

Инструкция return в теле процедуры приводит к завершению выполнения метода, в теле которого объект был создан. Если выполнение метода уже завершено, то вызывается ошибка \error{LocalJumpError}.

Процедуры принимают аргументы также как и обычные блоки.

\subsection{Лямбда-функции}

\begin{methodlist}
  \declare{lambda \{ |*params| \}}{\# -> lambda}
  Создание объекта.

  \declare{->(*params) \{ \}}{\# -> lambda}
  Создание объекта. Круглые скобки не обязательны. Позволяется определять параметры, имеющие значения по умолчанию.
\end{methodlist}

Инструкция return в теле лямбда-функций приводит к завершению выполнения функции.

Лямбда-функции принимают аргументы также как и обычные методы.

\subsection{Использование замыканий}

\subsubsection*{Приведение типов}

\begin{methodlist}
  \declare{.to_proc}{\# -> self} 
 
  \declare{.to_s}{\# -> string} 
  Возвращает информацию об объекте. 
  \\\verb!Proc.new { }.to_s # -> "#<Proc:0x8850f18@(irb):31>"!
\end{methodlist}

\subsubsection*{Операторы}

\begin{methodlist}
  \declare{proc == proc2}{}
  \alias{eql?} 
  Два замыкания равны, если относятся к копиям одного и того же объекта. 
  \\\verb!proc {} == proc {} # -> true!

  \declare{proc === arg}{\# -> object} 
  Выполнение функции.
  \\\verb!proc {} === 1 # -> nil!
\end{methodlist}

\subsubsection*{Выполнение замыкания}

\begin{methodlist}
  \declare{.call(*arg)}{\# -> object} 
  \alias{yield, .(*arg), [*arg]} 
  Выполнение замыкания. 
  \\\verb!-> x {x**2}.(5) # -> 25!
 
  \declare{.curry( count = nil )*arg}{\# -> proc}
  Подготовка замыкания для выполнения.

  Если передано достаточное количество аргументов, то замыкание выполняется.

  В другом случае замыкание сохраняет информацию об аргументах.

  Дополнительный аргумент ограничивает количество передаваемых аргументов (остальным параметрам присваивается nil).
  \begin{verbatim}
  b = proc \{ | x, y, z | x + y + z \} 
  b.curry[1][2][3] # -> 6 
  b.curry[1, 2][3, 4] # -> 6 
  b.curry[1][2][3][4][5] # -> 0 
  b.curry(5) [ 1, 2 ][ 3, 4][5] # -> 6 
  b.curry(1) [1] # -> type_error

  b = lambda \{ | x, y, z | x + y + z \} 
  b.curry[1][2][3] # -> 6 
  b.curry[1, 2][3, 4] # -> argument_error 
  b.curry(5) # -> argument_error 
  b.curry(1) # -> argument_error
  \end{verbatim}
\end{methodlist}

\subsubsection*{Остальное}

\begin{methodlist}
  \declare{.arity}{\# -> integer}
  Возвращает количество принимаемых аргументов. Для произвольного количества возвращается отрицательное число. Его инверсия с помощью оператора \verb!~! в результате возвращает количество обязательных аргументов.
  \begin{verbatim}
  proc \{\}.arity # -> 0 
  proc \{ || \}.arity # -> 0 
  proc \{ |a| \}.arity # -> 1 
  proc \{ | a,b | \}.arity # -> 2 
  proc \{ | a, b, c | \}.arity # -> 3 
  proc \{ |*a| \}.arity # -> -1 
  proc \{ | a, *b | \}.arity # -> -2 
  proc \{ | a, *b, c | \}.arity # -> -3
  \end{verbatim} 
 
  \declare{.parameters}{\# -> array} 
  Возвращает информацию о параметрах.
  \begin{verbatim}
  proc = lambda \{ | x, y = 42, *other | \} 
  proc.parameters
  # -> [ [:req, :x],  [:opt, :y], [:rest, :other] ]
  \end{verbatim}

  \declare{.binding}{\# -> binding} 
  Возвращает экземпляр класса Binding, содержащий информацию о состоянии выполнения программы для замыкания.

  \declare{.lambda?}{} 
  Проверяет относится ли объект к лямбда-функциям. 
  \\\verb!proc {}.lambda? # -> false!
 
  \declare{.source_location}{\# -> array} 
  Возвращает местоположение создания замыкания в виде \verb![ filename, line ]!. Для замыканий, создаваемых не на Ruby, возвращается nil.
  \\\verb!proc {}.source_location # -> [ "(irb)", 19 ]!

  \declare{.hash}{\# -> integer} 
  Возвращает цифровой код объекта. 
  \\\verb!proc {}.hash # -> -259341767!
\end{methodlist}

\section{Методы}

Подпрограммы могут быть созданы на основе уже существующих методов с помощью классов Method и UnboundMethod.

\subsection{Method}

Экземпляры класса сохраняют информацию о методе вместе с объектом, для которого он вызывается.

\begin{methodlist}
  \declare{object.method(name)}{\# -> method}
  Создание объекта для метода с переданным идентификатором. Если метод для объекта не определен, то вызывается ошибка.
  \\\verb!12.method(?+) # -> #<Method: Fixnum#+>!

  \declare{object.public_method(name)}{\# -> method}
  Создание объекта для общего метода с переданным идентификатором. Если метод для объекта не определен, то вызывается ошибка. 
  \\\verb!12.public_method(?+) # -> #<Method: Fixnum#+>!
\end{methodlist}

\subsubsection*{Операторы}

\begin{methodlist}
  \declare{method == method2}{}
  \alias{eql?}
  Проверка на равенство. Объекты равны, если связаны с одним и тем же объектом и содержат информацию об одном и том же методе.
  \\\verb!12.method(?+) == 13.method(?+) # -> false!
\end{methodlist}

\subsubsection*{Приведение типов}

\begin{methodlist}
  \declare{.to_proc}{\# -> lambda}
  Создание лямбда-функции.
  \\\verb!12.method(?+).to_proc # -> #<Proc:0x88cdd60 (lambda)>!

  \declare{.to_s}{\# -> string}
  Возвращает информацию об объекте.
  \\\verb!12.method(?+).to_s # -> "#<Method: Fixnum#+>"!

  \declare{.unbind}{\# -> umethod}
  Удаление информации об объекте, для которого вызывается метод.
  \\\verb!12.method(?+).unbind -> #<UnboundMethod: Fixnum#+>!
\end{methodlist}

\subsubsection*{Вызов метода}

\begin{methodlist}
  \declare{.call(*arg)}{\# -> object}
  Вызов метода с переданными аргументами.
  \\\verb!12.method(?+).call 3 # -> 15!
\end{methodlist}

\subsubsection*{Остальное}

\begin{methodlist}
  \declare{.arity}{\# -> integer}
  Возвращает количество принимаемых аргументов. Для произвольного количества возвращается отрицательное число. Его инверсия с помощью оператора \verb!~! в результате возвращает количество обязательных аргументов. Для методов, определенных без помощи Ruby возвращается -1.
  \\\verb!12.method(?+).arity # -> 1!

  \declare{.name}{\# -> symbol}
  Возвращает идентификатор метода.
  \\\verb!12.method(?+).name # -> :+!

  \declare{.owner}{\# -> module}
  Возвращает модуль, в котором объявлен метод.
  \\\verb!12.method(?+).owner # -> Fixnum!

  \declare{.parameters}{\# -> array}
  Возвращает массив параметров метода.
  \\\verb!12.method(?+).parameters # -> [ [:req] ]!

  \declare{.receiver}{\# -> object}
  Возвращает объект, для которого метод вызывается.
  \\\verb!12.method(?+).receiver # -> 12!

  \declare{.source_location}{\# -> array}
  Возвращает местоположение объявления метода в виде массива \verb![ filename, line ]!. Для методов, определенных без помощи Ruby, возвращается nil.  
  \\\verb!12.method(?+).source_location # -> nil!

  \declare{.hash}{\# -> integer}
  Возвращает цифровой код объекта.
  \\\verb!12.method(?+).hash # -> -347045594!
\end{methodlist}

\subsection{UnboundMethod}

Экземпляры класса сохраняют информацию только о методе.

\begin{methodlist}
  \declare{module.instance_method(name)}{\# -> umethod} 
  Создание объекта для метода с переданным идентификатором. Если метод для объекта не определен, то вызывается ошибка. 
  \\\verb!Math.instance_method :sqrt # -> #<UnboundMethod: Math#sqrt>!
 
  \declare{module.public_instance_method(name)}{\# -> umethod} 
  Создание объекта для метода с переданным идентификатором. Если метод для объекта не определен, то вызывается ошибка.
  \\\verb!Math.public_instance_method :sqrt # -> error!
\end{methodlist}

\subsubsection*{Операторы}

\begin{methodlist}
  \declare{umethod == umethod2}{}
  \alias{eql?}
  Проверка на равенство. Объекты равны, если содержат информацию об одном и том же методе.
  \\\verb!Math.instance_method(:sin) == Math.instance_method(:sin) # -> true!
\end{methodlist}

\subsubsection*{Приведение типов}

\begin{methodlist}
  \declare{.to_s}{\# -> string}
  \alias{inspect}
  Возвращает информацию об объекте.
  \\\verb!Math.instance_method(:sqrt).to_s # -> "#<UnboundMethod: Math#sqrt>"!

  \declare{.bind(object)}{\# -> method}
  Добавление информации об объекте, для которого метод вызывается. Если такая информация уже существовала, то объекты должны принадлежать к одному классу.
  \begin{verbatim}
  12.method(?+).unbind.bind 1 # -> #<Method: Fixnum#+>
  12.method(?+).unbind.bind 1.0 # -> error
  \end{verbatim}  
\end{methodlist}

\subsubsection*{Остальное}

\begin{methodlist}
  \declare{.arity}{\# -> integer}
  Возвращает количество принимаемых аргументов. Для произвольного количества возвращается отрицательное число. Его инверсия с помощью оператора \verb!~! в результате возвращает количество обязательных аргументов. Для методов, определенных без помощи Ruby возвращается -1.
  \\\verb!Math.instance_method(:sqrt).arity # -> 1!

  \declare{.name}{\# -> symbol}
  Возвращает идентификатор метода.
  \\\verb!Math.instance_method(:sqrt).name # -> :sqrt!

  \declare{.owner}{\# -> module}
  Возвращает модуль, в котором объявлен метод.
  \\\verb!Math.instance_method(:sqrt).owner # -> Math!

  \declare{.parameters}{\# -> array}
  Возвращает массив параметров метода.
  \\\verb!Math.instance_method(:sqrt).parameters # -> [ [:req] ]!

  \declare{.source_location}{\# -> array}
  Возвращает местоположение объявления метода в виде массива \verb![ filename, line ]!. Для методов, определенных без помощи Ruby, возвращается nil.  
  \\\verb!Math.instance_method(:sqrt).source_location # -> nil!

  \declare{.hash}{\# -> integer}
  Возвращает цифровой код объекта.
  \\\verb!Math.instance_method(:sqrt).hash # -> 563385534!
\end{methodlist}

\section{Сопрограммы}

Сопрограмма - это фрагмент кода, поддерживающий несколько входных точек и остановку или продолжение выполнения с сохранением состояния выполнения. 

Для работы с сопрограммами используется класс Fiber.

Для управления сопрограммами создаются контрольные точки с помощью метода \method{::yield} и осуществляется последовательный переход между ними с помощью метода \method{.resume}.

\begin{methodlist}
  \declare{::new \{ |*params| \}}{\# -> fiber}
  Создание объекта. Блок при этом не выполняется. 

  \declare{::yield(*temp_result)}{\# -> object}
  Создание контрольной точки. Переданные аргументы возвращаются в результате вызова \method{.resume}.

  В результате возвращаются объекты, переданные при вызове метода \method{.resume} или полученные в результате выполнения последнего выражения в теле сопрограммы.
 
  \declare{.resume(*args)}{\# -> temp_result}
  Выполнение блока до следующей контрольной точки.

  Если метод был вызван впервые, то переданные аргументы отправляются в сопрограмму. В другом случае они возвращаются в результате вызова метода \method{::yield} в теле блока. 

  Если сопрограмма уже выполнена, то вызывается ошибка \error{FiberError}.
\end{methodlist}

Сопрограммы также могут использоваться для реализации многопоточности на уровне программы (в действительности код выполняется в единственном потоке выполнения). Такие потоки также называют "green thread". Использование сопрограмм позволяет уменьшить накладные расходы на переключение и обмен данными, так ка не требует взаимодействия с ядром ОС.

Проблема при использовании сопрограмм в том, что выполнение системного вызова будет блокировать процесс выполнения программы. Сопрограммы должны использовать специальные методы ввода/вывода, не блокирующие процесс выполнения. Также стоит заметить что управление переключением сопрограмм выполняется вручную и требует дополнительных затрат при разработке.
  \chapter{Псевдослучайные числа}

Для генерации псевдослучайных чисел в Ruby используется алгоритм "Вихрь Мерсена", предложенный в 1997 году Мацумото и Нисимурой. Его достоинствами являются колоссальный период \verb!(2 ** 19937 - 1)!, равномерное распределение в 623 измерениях, быстрая генерация случайных чисел (в 2-3 раза быстрее, чем стандартные генераторы). Однако, существуют алгоритмы, распознающие последовательность, порождаемую "Вихрем Мерсенна", как неслучайную.

Для получения псевдослучайных чисел в Ruby предоставлен класс Random.

\begin{methodlist}
  \declare{::new( seed = Random.new_seed )}{\# -> random} 
  Создание генератора. Объект, переданный методу, необходим для исключения повторяющихся чисел. Его также называют "соль". 
  \\\verb!Random.new # -> #<Random:0xa0a1fa4>! 

  \declare{::rand( number = 0 )}{\# -> number2} 
  Проверяет результат \verb!number.to_i.abs!: если результат выполнения равен нулю или ссылается на nil, то в результате возвращается псевдослучайная десятичная дробь в диапазоне \verb!0.0...1.0!. В другом случае возвращается псевдослучайное число в диапазоне \verb!0...number.to_i.abs!. 
  \\\verb!Random.rand # -> 0.8736231696463861!

  \declare{rand( number = 0 )}{\# -> number2}
  Частный метод экземпляров из модуля Kernel, аналогичный предыдущему.

  \declare{::new_seed}{\# -> integer} 
  Возвращает новое число для генерации псевдослучайных чисел. 
  \\\verb!Random.new_seed # -> 69960780063826734370396971659065074316!
 
  \declare{::srand( number = 0 )}{\# -> number} 
  Возвращает новое число для генерации псевдослучайных чисел. Если аргумент равен нулю, то используется время вызова, идентификатор процесса и порядковый номер вызова. С помощью аргумента программа может быть детерминирована во время тестирования. В результате возвращается предыдущее значение.

  \declare{srand( number = 0 )}{\# -> number}
  Частный метод экземпляров из модуля Kernel, аналогичный предыдущему.
\end{methodlist}

\subsection*{Генераторы}

\begin{methodlist}
  \declare{.bytes(bytesize)}{\# -> string} 
  Возвращает случайный двоичный текст. 
  \\\verb!Random.new.bytes 2 -> "\xA1W"!

  \declare{.rand( object = nil )}{\# -> number}
  \begin{itemize}
    \item Если передано целое число, то возвращается псевдослучайное число в диапазоне \verb!0...object!;
    \item Если передано отрицательное число или ноль, то вызывается ошибка;
    \item Если передана десятичная дробь, то возвращается псевдослучайная десятичная дробь в диапазоне \verb!0.0...object!;
    \item Если передан диапазон, то возвращается случайный элемент диапазона. Для начальной и конечной границ должны быть определены операторы - (разность) и + (сумма);
    \item В остальных случаях вызывается ошибка.
  \end{itemize}   

  \declare{.seed}{\# -> integer} 
  Возвращает число, использующееся для генерации псевдослучайных чисел. 
  \\\verb!Random.new.seed # -> 173038287409845379387953855893202182131!
\end{methodlist}
  \chapter{Дата и время}

Экземпляры класса Time - это абстрактные объекты, содержащие информацию о времени. Время хранится в секундах, начиная с 01.01.1970 00:00 UTC. Системы отсчета времени GMT (время по Гринвичу) и UTC (универсальное время) трактуются как эквивалентные.

При сравнении разных объектов необходимо помнить, что различные пояса могут иметь смещения по времени от UTC.

Добавленные модули: Comparable

\begin{keylist}{Аргументы:}
  
  \firstkey{year} - год; 
  
  \key{month} - месяц: либо целое число от 1 до 12, либо текст, содержащий первые три буквы английского названия месяца (аббревиатуру); 
  
  \key{day} - день месяца: целое число от 1 до 31; 
  
  \key{wday} - день недели: целое число от 0 до 6, начиная с воскресенья; 
  
  \key{yday} - день года: целое число от 1 до 366; 
  
  \key{isdst} - летнее время: логическая величина; 
  
  \key{zone} - временная зона: текст; 
  
  \key{hour} - час: целое число от 0 до 23; 
  
  \key{min} - минуты: целое число от 0 до 59; 
  
  \key{sec} - секунды: целое число или десятичная дробь от 0 до 60; 
  
  \key{usec} - микросекунды: целое число или десятичная дробь от 0 до 999;
\end{keylist}

\begin{methodlist}
  \declare{::new}{\# -> time}
  \begin{alltt}
  (year, month = 1, day = 1, hour = 0, min = 0, sec = 0, usec = 0, zone)
  # -> time\
  \end{alltt}
  Создание объекта. При вызове без аргументов возвращается текущее системное время. Последним аргументом передается смещение относительно UTC в виде текста "+00:00" или количества секунд. По умолчанию берется системное смещение часового пояса.
  \begin{verbatim}
  Time.new 1990, 3, 31, nil, nil, nil, "+04:00" 
  # -> 1990-03-31 00:00:00 +0400\
  \end{verbatim} 
   
  \declare{::now}{\# -> time} 
  Создание объекта для текущего системного времени. 
  \\\verb!Time.now -> 2011-09-17 10:36:26 +0400!
 
  \declare{::at(time)}{\# -> time} 
  \verb!( sec, usec = nil ) # -> time!

  Создание объекта. Принимаются секунды и микросекунды, прошедшие с начала точки отсчета UTC. Время вычисляется с учетом смещения часового пояса. 
  \\\verb!Time.at 1 -> 1970-01-01 03:00:01 +0300!

  \declare{::utc( year, month = 1, day = 1, hour = 0, min = 0, sec = 0, usec = 0)}{\# -> time} 
  \verb!( sec, min, hour, day, month, year, wday, yday, isdst ) # -> time!
  \alias{gm} 
  Создание объекта.   
  \\\verb!Time.utc 1990, 3, 31 -> 1990-03-31 00:00:00 UTC!
 
  \declare{::local( year, month = 1, day = 1, hour = 0, min = 0, sec = 0, usec = 0, zone )}{\# -> time} 
  \verb!(sec, min, hour, day, month, year, wday, yday, isdst, zone) # -> time! 

  \alias{time} 
  Версия предыдущего метода, вычисляющая время с учетом смещения часового пояса. 
  \\\verb!Time.local 1990, 3, 31 -> 1990-03-31 00:00:00 +0400!
\end{methodlist}

\subsection*{Приведение типов} 

\begin{methodlist}
  \declare{.to_s}{\# -> string} 
  \alias{inspect} 
  Преобразование в текст. 
  \\\verb!Time.local( 1990, 3, 31 ).to_s -> "1990-03-31 00:00:00 +0400"!
 
  \declare{.to_a}{\# -> array} 
  Возвращает индексный массив вида:
  \begin{verbatim}
  [ self.sec, self.min, self.hour,
    self.day, self.month, self.year,
    self.wday, self.yday,
    self.isdst, self.zone ]\
  \end{verbatim}   
  \verb!Time.local( 1990, 3, 31 ).to_a # -> [ 0, 0, 0, 31, 3, 1990, 6, 90, true, "MSD" ]!
 
  \declare{.to_i}{\# -> integer} 
  \alias{tv_sec} 
  Возвращает количество секунд прошедших начиная с 1970-01-01 00:00:00 UTC. 
  \\\verb!Time.local( 1990, 3, 31 ).to_i # -> 638827200!
 
  \declare{.to_r}{\# -> rational} 
  Возвращает количество секунд прошедших начиная с 1970-01-01 00:00:00 UTC в виде рациональной дроби.
  \\\verb!Time.local( 1990, 3, 31 ).to_r # -> (638827200/1)!
 
  \declare{.to_f}{\# -> float} 
  Возвращает количество секунд прошедших начиная с 1970-01-01 00:00:00 UTC в виде десятичной дроби.
  \\\verb!Time.local( 1990, 3, 31 ).to_f # -> 638827200.0!
\end{methodlist}

\subsection*{Операторы}

\begin{methodlist}
  \declare{time + sec}{\# -> time2} 
  Прибавляет переданное количество секунд. 
  \\\verb!Time.local( 1990, 3, 31 ) + 3600 # -> 1990-03-31 01:00:00 +0400!
 
  \declare{time - time2}{\# -> float}
  Возвращает разницу в секундах.
  \\\verb!Time.local( 1990, 3, 31 ) - Time.new( 1990, 3, 31 ) -> 0.0!

  \declare{time - sec}{\# -> time2} 
  Отнимает переданное количество секунд. 
  \\\verb!Time.local( 1990, 3, 31 ) - 3600 -> 1990-03-30 23:00:00 +0400 !
 
  \declare{time <=> time2}{} 
  Сравнение 
  \\\verb!Time.local( 1990, 3, 31 ) <=> Time.new( 1990, 3, 31 ) -> 0!
\end{methodlist}

\subsection*{Форматирование}

\begin{methodlist}
  \declare{.strftime(format)}{\# -> string}
  Форматирование времени на основе переданной \hyperlink{appdatetime}{\underline{форматной строки}}.
\end{methodlist}

\subsection*{Изменение времени}

\begin{methodlist}
  \declare{.getutc}{\# -> time}
  \alias{getgm} 
  Возвращает время относительно UTC (без смещения часовых поясов). 
  \\\verb!Time.local( 1990, 3, 31 ).getutc # -> 1990-03-30 20:00:00 UTC!
 
  \declare{.utc}{\# -> self} 
  \alias{gmtime} 
  Версия предыдущего метода, изменяющая значение объекта. 

  \declare{.getlocal( zone = nil )}{\# -> time} 
  Возвращает время, учитывая смещение часового пояса. Смещение может быть явно передано методу (по умолчанию используется системное смещение). 
  \\\verb!Time.local(1990, 3, 31).getutc.getlocal # -> 1990-03-31 00:00:00 +0400!
 
  \declare{.localtime( zone = nil )}{\# -> self} 
  Версия предыдущего метода, изменяющая значение объекта.
\end{methodlist}

\subsection*{Статистика}

\begin{methodlist}
  \declare{.asctime}{\# -> string} 
  \alias{ctime} 
  Возвращает время. 
  \\\verb!Time.local( 1990, 3, 31 ).asctime # -> "Sat Mar 31 00:00:00 1990"!
 
  \declare{.utc_offset}{\# -> integer}
  \alias{gmt_offset, gmtoff} 
  Интерпретатор возвращает смещение часового пояса относительно UTC в секундах. 
  \\\verb!Time.local( 1990, 3, 31 ).utc_offset # -> 14400!
 
  \declare{.zone}{\# -> string}
  Возвращает название временной зоны. 
  \\\verb!Time.local( 1990, 3, 31 ).zone # -> "MSD"!

  \declare{.year}{\# -> integer} 
  Возвращает год. 
  \\\verb!Time.local( 1990, 3, 31 ).year # -> 1990!
 
  \declare{.month}{\# -> integer} 
  \alias{mon} 
  Возвращает номер месяца. 
  \\\verb!Time.local( 1990, 3, 31 ).month # -> 3!
 
  \declare{.yday}{\# -> integer} 
  Возвращает номер дня в году от 1 до 366. 
  \\\verb!Time.local( 1990, 3, 31 ).yday # -> 90!
 
  \declare{.day}{\# -> integer} 
  \alias{mday} 
  Возвращает число. 
  \\\verb!Time.local( 1990, 3, 31 ).day # -> 31!
 
  \declare{.wday}{\# -> integer} 
  Возвращает день недели (от 0 до 6, начиная с воскресенья). 
  \\\verb!Time.local( 1990, 3, 31 ).wday # -> 6!

  \declare{.hour}{\# -> integer} 
  Возвращает час дня (число от 0 до 23). 
  \\\verb!Time.local( 1990, 3, 31 ).hour # -> 0!
 
  \declare{.min}{\# -> integer} 
  Возвращает количество минут (число от 0 до 59). 
  \\\verb!Time.local( 1990, 3, 31 ).min # -> 0!
 
  \declare{.sec}{\# -> integer} 
  Возвращает количество секунд (число от 0 до 60). 
  \\\verb!Time.local( 1990, 3, 31 ).sec # -> 0!
 
  \declare{.subsec}{\# -> integer} 
  Возвращает дробную часть секунд. 
  \\\verb!Time.local( 1990, 3, 31 ).subsec # -> 0!
 
  \declare{.usec}{\# -> integer} 
  \alias{tv_usec} 
  Возвращает количество микросекунд. 
  \\\verb!Time.local( 1990, 3, 31 ).usec # -> 0!
 
  \declare{.time}{}
  \alias{tv_nsec} 
  Возвращает количество наносекунд. 
  \\\verb!Time.local( 1990, 3, 31 ).nsec # -> 0!
\end{methodlist}

\subsection*{Предикаты}

\begin{methodlist}
  \declare{.monday?}{} 
  Проверяет является ли понедельник днем недели. 
  \\\verb!Time.local( 1990, 3, 31 ).monday? # -> false!

  \declare{.tuesday?}{} 
  Проверяет является ли вторник днем недели. 
  \\\verb!Time.local( 1990, 3, 31 ).tuesday? # -> false!
 
  \declare{.wednesday?}{} 
  Проверяет является ли среда днем недели. 
  \\\verb!Time.local( 1990, 3, 31 ).wednesday? # -> false!
 
  \declare{.thursday?}{} 
  Проверяет является ли четверг днем недели. 
  \\\verb!Time.local( 1990, 3, 31 ).saturday? # -> false!
 
  \declare{.friday?}{} 
  Проверяет является ли пятница днем недели. 
  \\\verb!Time.local( 1990, 3, 31 ).friday? # -> false!
 
  \declare{.saturday?}{} 
  Проверяет является ли суббота днем недели. 
  \\\verb!Time.local( 1990, 3, 31 ).saturday? # -> true!
 
  \declare{.sunday?}{} 
  Проверяет является ли воскресенье днем недели. 
  \\\verb!Time.local( 1990, 3, 31 ).saturday? # -> false!

  \declare{.utc?}{}
  \alias{gmt?} 
  Проверяет используется ли время относительно UTC. 
  \\\verb!Time.local( 1990, 3, 31 ).utc? # -> false!
 
  \declare{.dst?}{}
  \alias{isdst} 
  Проверяет используется ли переход на летнее время. 
  \\\verb!Time.local( 1990, 3, 31 ).dst? # -> true!
\end{methodlist}
 
\subsection*{Остальное}

\begin{methodlist}
  \declare{.hash}{\# -> integer} 
  Возвращает цифровой код объекта. 
  \\\verb!Time.local( 1990, 3, 31 ).hash # -> -494674000!

  \declare{.round( precise = 0 )}{ -> time}
  Округляет количество секунд с заданной точностью. Точность определяет размер дробной части. 
  \\\verb!Time.local( 1990, 3, 31 ).round 2 # -> 1990-03-31 00:00:00 +0400!
\end{methodlist}
  \chapter{Типы данных}

\section{Логические величины}

Классы TrueClass, FalseClass и NilClass - это собственные классы объектов true, false и nil соответственно. Также существуют константы TRUE, FALSE, NIL, которые могут быть переопределены (но зачем?).

\begin{longtable}{ | * {3} { l |}}
\hline
NilClass & FalseClass & TrueClass \\ \hline
nil \& obj -> false & false \& obj -> false & true \& bool -> bool \\ \hline
nil \textasciicircum\-, bool -> bool & false \textasciicircum\-, bool -> bool & true \textasciicircum\-, bool -> !bool \\ \hline
nil | bool -> bool & false | bool -> bool & true | object -> true \\ \hline
nil.to_s -> "" & false.to_s -> "false" & true.to_s -> "true" \\ \hline
\end{longtable}

\begin{methodlist}
  \declare{nil.nil?}{\# -> true}

  \declare{nil.inspect}{\# -> "nil"}

  \declare{nil.rationalize}{\# -> (0/1)}
  \alias{to_r}

  \declare{nil.to_a}{\# -> []}

  \declare{nil.to_c}{\# -> (0+0i)}

  \declare{nil.to_f}{\# -> 0.0}

  \declare{nil.to_i}{\# -> 0}
\end{methodlist}

\section{Symbol}

Добавленные модули: Comparable

Большинство методов сначала преобразует объект в текст.

\begin{methodlist}
  \declare{::all_symbols}{\# -> array}
  Возвращает массив всех существующих экземпляров класса.
\end{methodlist}

\subsection*{Приведение типов}

\begin{methodlist}
  \declare{.to_s}{\# -> string}
  \alias{id2name}
  Преобразует значение в текст.
  \\\verb!:Ruby.to_s # -> "Ruby"!

  \declare{.inspect}{\# -> string}
  Преобразует объект в текст.
  \\\verb!:Ruby.inspect # -> ":Ruby"!

  \declare{.to_sym}{\# -> symbol}
  \alias{intern}

  \declare{.to_proc}{\# -> proc}
  Преобразует в замыкание, соответствующую методу с используемым идентификатором.
  \begin{verbatim}
  1.next # -> 2
  :next.to_proc.call(1) # -> 2\
  \end{verbatim}  
\end{methodlist}

\subsection*{Операторы}

\begin{methodlist}
  \declare{symbol <=> object}{} 
  Выполняемое выражение: \verb!symbol.to_s <=> object!.

  \declare{symbol =~ object}{} 
  Выполняемое выражение: \verb!symbol.to_s =~ object!.

  \declare{symbol[*object]}{} 
  \alias{slice}
  Выполняемое выражение: \verb!symbol.to_s[*object]! или \verb!symbol.to_s.slice(*object)!.
\end{methodlist}

\subsection*{Изменение регистра}

\begin{methodlist}
  \declare{.capitalize}{\# -> symbol} 
  Выполняемое выражение: \verb!self.to_s.capitalize.to_sym!.

  \declare{.swapcase}{\# -> symbol} 
  Выполняемое выражение: \verb!self.to_s.swapcase.to_sym!.

  \declare{.upcase}{\# -> symbol} 
  Выполняемое выражение: \verb!self.to_s.upcase.to_sym!.

  \declare{.downcase}{\# -> symbol} 
  Выполняемое выражение: \verb!self.to_s.downcase.to_sym!.
\end{methodlist}

\subsection*{Остальное}

\begin{methodlist}
  \declare{.encoding}{\# -> encoding}
  Возвращает кодировку.
  \\\verb!:Ruby.encoding # -> #<Encoding:US-ASCII>!

  \declare{.empty?}{} 
  Выполняемое выражение: \verb!self.to_s.empty?!.

  \declare{.length}{\# -> integer}
  \alias{size}
  Выполняемое выражение: \verb!self.to_s.length! или \verb!self.to_s.size!.

  \declare{.casecmp(object)}
  Выполняемое выражение: \verb!self.to_s.casecmp(object)!. 

  \declare{.next}{\# -> symbol}
  \alias{succ}
  Выполняемое выражение: \verb!self.to_s.next.to_sym! или \verb!self.to_s.succ.to_sym!
\end{methodlist}

\section{Структуры}

Добавленные модули: Enumerable 

Структуры данных - это объект, имеющий только свойства. Иногда структуры также используют при необходимости передавать множество аргументов.

Для облегчения создания структур в Ruby предоставлен класс Struct.

\begin{methodlist}
  \declare{::new( name = nil, *attribute )}{\# -> class} 
  Создание класса структуры в теле Struct и объявление переданных свойств (ссылающихся на nil). Если название класса не передано, то создается анонимный класс. Анонимный класс получит собственный идентификатор, если ему будет присвоена константа.

  Для нового класса определяется конструктор, принимающий объекты для инициализации свойств.
  \begin{verbatim}
  Struct.new "Kлюч", :объект # -> Struct::Kлюч 
  # только для примера. Никогда так больше не делайте :) 
  Struct::Kлюч.new [ 1, 2, 3 ] 
  # ->  #<struct Struct::Kлюч объект=[1, 2, 3]> \
  \end{verbatim}
\end{methodlist}
  
\subsection*{Приведение типов} 

\begin{methodlist}
  \declare{.to_s}{\# -> string} 
  \alias{inspect} 
  Возвращает информацию о структуре.
  \begin{verbatim}
  Struct::Kлюч.new( [ 1, 2, 3 ] ).to_s 
  # -> "#<struct Struct::Kлюч объект=[1, 2, 3]>"\
  \end{verbatim} 

  \declare{.to_a}{\# -> array} 
  \alias{values}
  Возвращает массив свойств структуры. 
  \\\verb!Struct::Kлюч.new( [ 1, 2, 3 ] ).to_a # -> [ [ 1, 2, 3 ] ]!
\end{methodlist}

\subsection*{Элементы структур}

Элементами структур считаются свойства. Для доступа к свойствам определены операторы \verb![]! и \verb![]=!. Они принимают идентификаторы или индексы свойств. Индексация свойств начинается с 0 и соответствует порядку, использовавшемуся при создании структуры. Индекс может быть отрицательным (индексация, начинается с -1).

Если требуемое свойство не найдено, то вызывается ошибка. 

\begin{methodlist}
  \declare{struct[attr]}{\# -> object} 
  Возвращает значение свойства структуры.
  \begin{verbatim}
  Struct::Kлюч.new( [ 1, 2, 3 ] )[ "объект" ] # -> [ 1, 2, 3 ]
  Struct::Kлюч.new( [ 1, 2, 3 ] )[ 0 ] # -> [ 1, 2, 3 ] 
  Struct::Kлюч.new( [ 1, 2, 3 ] )[ -1 ] # -> [ 1, 2, 3 ]\
  \end{verbatim}

  \declare{struct[attr]=(object)}{\# -> object}
  Изменяет значение свойства структуры.
  \begin{verbatim}
  Struct::Kлюч.new( [ 1, 2, 3 ] )[ "объект" ] = :array # -> :array
  Struct::Kлюч.new( [ 1, 2, 3 ] )[ 0 ] = :array # -> :array 
  Struct::Kлюч.new( [ 1, 2, 3 ] )[ -1 ] = :array # -> :array\
  \end{verbatim} 
 
  \declare{struct.values_at(*integer)}{\# -> array}
  Возвращает массив значений свойств с переданными индексами. При вызове без аргументов возвращается пустой массив.
  \begin{verbatim}
  Struct::Kлюч.new( [ 1, 2, 3 ] ).values_at 0, -1 
  # -> [ [ 1, 2, 3 ], [ 1, 2, 3 ] ]\
  \end{verbatim} 
\end{methodlist}

\subsection*{Итераторы}

\begin{methodlist}
  \declare{.each \{ |object| \}}{\# -> self}
  Последовательно перебирает значения свойств. 
 
  \declare{.each_pair \{ | name, object | \}}{\# -> self} 
  Последовательно перебирает идентификаторы свойств и их значения. 
\end{methodlist}

\subsection*{Остальное} 

\begin{methodlist}
  \declare{.hash}{\# -> integer} 
  Возвращает цифровой код объекта. 
  \\\verb!Struct::Kлюч.new( [ 1, 2, 3 ] ).hash # -> -764829164!

  \declare{.size}{\# -> integer} 
  \alias{length} 
  Возвращает количество свойств. 
  \\\verb!Struct::Kлюч.new( [ 1, 2, 3 ] ).size # -> 1!

  \declare{.members}{\# -> array} 
  Возвращает массив идентификаторов свойств. 
  \\\verb!Struct::Kлюч.new( [ 1, 2, 3 ] ).members # -> [ :объект ]!
\end{methodlist}

\part{Работа программы}
  \chapter{Выполнение программы}

\section{Процесс выполнения}

Процесс выполнения - это непосредственное выполнение кода программы.

Процесс выполнения может быть разделен на три этапа: начало выполнения, выполнение программы и завершение выполнения.

BEGIN и END - это предложения, выполняющиеся в начале выполнения и перед завершением выполнения программы. Тело каждого предложения определяет собственную локальную область видимости и выполняется строго один раз.

\begin{description}
  \item[BEGIN \{ \}] - код выполняется в начале выполнения программы. Если в коде программы используется несколько таких предложений, то они выполняются последовательно в порядке записи;

  \item[END \{ \}] - код выполняется перед завершением выполнения программы. Если в коде программы используется несколько таких предложений, то они выполняются последовательно в обратном порядке. Для выполнения этого предложения используется экземпляр File, на который ссылается константа DATA;

  \item[__END__] - аналогично END; 

  \item[at_exit \{ \}] - аналогично END.
\end{description}

Для запуска программы в теле цикла (бесконечного выполнения программы) существует частный метод экземпляров из модуля Kernel.

\begin{methodlist}
  \declare{loop \{ \}}{}
  Тело блока итерируется до тех пор пока не будет вызвана ошибка \error{StopIteration} или одна из инструкций.
\end{methodlist}

\subsection{Перехват выполнения}

Перехватить выполнения программы можно объявив код (с помощью частных методов экземпляров из модуля Kernel), выполняемый при возникновении определенных событий.

\begin{methodlist}
  \declare{set_trace_func( proc = nil )}{}
  Объявление замыкания для перехвата выполнения. Перехват выполнения в теле замыкания при этом не выполняется.
  
  Замыканию передаются: идентификатор события, имя файла, номер строки кода, цифровой идентификатор объекта, экземпляр класса Binding и идентификатор класса объекта.

  Передача nil отменяет перехват выполнения.
  
  Возможные события:
  \begin{description}
    \item["c-call"] - вызов Си функции;
    \item["c-return"] - завершение выполнения Си функции;
    \item["call"] - вызов Ruby метода;
    \item["return"] - завершение выполнения Ruby метода;
    \item["class"] - начало определения класса или модуля;
    \item["end"] - завершение определения класса или модуля;
    \item["line"] - выполнение новой строки кода;
    \item["raise"] - вызов ошибки. 
  \end{description}

  \declare{trace_var( name, code )}{\# -> nil}
  \verb!(name) { |object| } -> nil!
  
  Объявление кода для перехвата изменения глобальной переменной. В блок передается новое значение.

  \declare{untrace_var( name, code = nil )}{\# -> array}
  Отмена перехвата выполнения при изменении глобальной переменной. Возвращается массив, содержащий выполняемый при перехвате код.
\end{methodlist}

\subsection{Завершение программы}

Для завершения выполнения программы используются частные методы экземпляров из модуля Kernel (методы влияют на любой поток выполнения, в теле которого взываются).

\begin{methodlist}
  \declare{sleep( sec = nil )}{\# -> sec}
  Останавливает выполнение (по умолчанию навсегда). В результате возвращается время фактического ожидания.

  \declare{exit( state = true )}{}
  Завершает выполнение, вызывая ошибку \error{SystemExit}. 

  \declare{exit!( state = false )}{}
  Немедленно завершает выполнение.

  \declare{abort( mesage = nil )}{} 
  Немедленно завершает выполнение. Аргумент записывается в стандартный поток для вывода ошибок. Аналогично выполнению \verb!exit false!.
\end{methodlist}

\section{Вызов системных команд}

На Ruby довольно часто создают небольшие скрипты, облегчающие вызов различных системных команд. Для этого используются частные методы экземпляров из модуля Kernel.

\begin{methodlist}
  \declare{Kernel.`(code)}{\# -> string}
  Текст между двумя "обратными" кавычками (\verb!`ruby --help`!) обрабатывается как составной и передается для выполнения операционной системе. 
  
  Тот же эффект достигается при ограничении текста произвольными разделителями с использованием приставки \verb!%x! (\verb!%x[ruby --help]!).

  \declare{exec( env, command, options )}{}
  Заменяет текущий процесс, выполняя системный вызов. Код, записанный после вызова метода не выполняется. Если команда не может быть выполнена, то вызывается ошибка.

  \begin{keylist}{Принимаемые аргументы:}

    \firstkey{env (hash):} управление переменными окружения.
    \begin{description}
      \item[name:] значение для переменной окружения;
      \item[name:] nil, удаление переменной окружения.
    \end{description}

    \key{command:} системный вызов.
    \begin{description}
      \item[string] - текст команды для используемой оболочки: по умолчанию в Unix - это \verb!"/bin/sh"!, а в Windows - \verb!ENV["RUBYSHELL"]! или \verb!ENV["COMSPEC"]!;
      \item[string, *arg] - текст команды и передаваемые аргументы;
      \item[{[ string, first_arg ], *arg}] - текст команды, первый аргумент и остальные аргументы. 
    \end{description}

    \key{option (hash):} дополнительный аргумент.
    \begin{description}
      \item[unsetenv_others:] true, удаление всех переменных окружения, кроме переданных методу;
      \item[pgroup:] группировка процессов:
        \begin{itemize}
          \item true для создания новой группы;
          \item integer для сохранения процесса в соответствующей группе;
          \item nil для отмены группировки.
        \end{itemize}
      \item[chdir:] путь к текущему рабочему каталогу;
      \item[umask:] права доступа для создаваемых файлов или каталогов. 
    \end{description}
  \end{keylist}

  \declare{syscall( number, *object = nil )}{}
  Вызывает функцию операционной системы с переданным цифровым идентификатором (для Unix систем идентификаторы и функции описаны в файле syscall.h).

  Дополнительно (не более девяти аргументов) методу передаются либо текст, содержащий указатель на последовательность байт, либо размер указателя в битах.

  Если вызов системной функции невозможен, то вызывается ошибка \error{SystemCallEror}.
  Если вызов метода невозможен, то вызывается ошибка \error{NotImplementedError}.

  Метод непереносим и небезопасен в использовании.
\end{methodlist}
  \chapter{Чтение и запись данных}

\input{stream.tex}
\input{file.tex}
\input{dir.tex}
\input{filestat.tex}

  \chapter{Обработка аргументов}

Аргументы, переданные при запуске программы, сохраняются в массиве ARGV.

\section{Файлы}

Программы, работающие с файлами, могут принимать как по одному файлу, так и сразу несколько. Для работы с одним файлом используется массив ARGV, а для работы с несколькими файлами - поток ARGF.

\subsection{ARGV}

Для чтения файла используются частные методы экземпляров из модуля Kernel.

\begin{methodlist}
  \declare{.gets( sep = \$/, bytesize = nil )}{\# -> string}
  Чтение следующей строки из файла, найденного в ARGV (размер строки может быть ограничен). Переданный разделитель обрабатывается в качестве символа перевода строки (если передается пустой текст, то обрабатывается "/n/n"). Если достигнут конец файла, то возвращается nil (если в массиве ARGV больше нет файлов).

  Полученная в результате строка связывается с глобальной переменной \$_. 
 
  \declare{.readline( sep = \$/, bytesize = nil )}{\# -> string}
  Версия метода вызывающая ошибку при достижении конца файла. 
\end{methodlist}

\subsection{ARGF}

ARGF (\verb!$<!) - это поток, открываемый для файлов, содержащихся в ARGV. При этом подразумевается, что ARGV содержит только пути к файлам. 

Файлы обрабатываются в том порядке, в котором они содержатся в ARGV. После обработки путь к файлу удаляется автоматически.

Файлы обрабатываются последовательно в качестве одного виртуального файла. Концом файла считается конец всех объединяемых файлов, а не конец отдельных элементов. 

Если ARGV ссылается на пустой массив, то ARGF ссылается на стандартный поток для ввода.

Добавленные модули: Enumerable

\subsubsection{Управление потоком}

\begin{methodlist}
  \declare{::binmode}{\# -> self} 
  Переключение в двоичный режим. Возврат в текстовый режим после этого невозможен. Преобразование кодировок и символа перевода строки отменяется, а содержимое потока обрабатывается в ASCII кодировке. 

  \declare{::close}{\# -> self} 
  Завершение обработки текущего файла и переход к следующему. Если все файлы уже обработаны, то вызывается ошибка \error{IOError}. 

  \declare{::skip}{\# -> self}
  Пропуск обработки текущего файла и переход к следующему. При отсутствии файлов ничего не выполняется.

  \declare{::argv}{\# -> ARGV}

  \declare{::binmode?}{} 
  Проверяет используется ли двоичный режим. 

  \declare{::closed?}{} 
  Проверяет завершена ли обработка текущего файла. 

  \declare{::eof?}{} 
  \alias{eof} 
  Проверяет достигнут ли конец текущего файла. 

  \declare{::filename}{\# -> path}
  \alias{path} 
  Возвращает относительный путь к обрабатываемому файлу. При взаимодействии с стандартным потоком для ввода возвращается \verb!"-"!. Аналогично использованию \verb!$FILENAME!. 
\end{methodlist}

\subsubsection{Кодировка}

\begin{methodlist}
  \declare{::external_encoding}{\# -> encoding} 
  Возвращает внешнюю кодировку.
 
  \declare{::internal_encoding}{\# -> encoding} 
  Возвращает внутреннюю кодировку если она указана. В другом случае возвращает nil. 

  \declare{::set_encoding( *encoding, options = nil )}{\# -> self} 
  Объявление внешней и внутренней кодировок (объекты или текст). Дополнительный аргумент используется при \hyperlink{appencode}{\underline{преобразовании данных}}. 
\end{methodlist}

\subsubsection{Приведение типов}

\begin{methodlist}
  \declare{::to_io}{\# -> io}
  \alias{file}  
  Возвращает файл или поток для обрабатываемого файла (или стандартного потока для ввода).

  \declare{::to_i}{\# -> integer} 
  \alias{fileno} 
  Возвращает дескриптор обрабатываемого файла. При отсутствии файлов вызывается ошибка \error{ArgumentError}.

  \declare{::to_s}{\# -> "ARGF"}

  \declare{::to_a( sep = \$/, size = nil )}{\# -> array} 
  \alias{readlines}  
  Чтение строк и сохранение их в индексном массиве (размер массива может быть ограничен). Переданный разделитель обрабатывается в качестве символа перевода строки (если передается пустой текст, то обрабатывается "/n/n").
 
  \declare{::to_write_io}{\# -> io} 
  Интерпретатор возвращает поток, доступный для записи (только если используется режим редактирования файлов - ключ \verb!-i!).
\end{methodlist}

\subsubsection{Чтение данных}

{\bf Фрагмент}

\begin{methodlist}
  \declare{::read( bytesize = nil, buffer = nil )}{\# -> buffer}
  Чтение данных из потока (по умолчанию полностью). Если размер фрагмента ограничен, то чтение выполняется в двоичном режиме. Дополнительный аргумент служит для хранения полученных данных.
  \begin{itemize}
    \item Если передается ноль, то возвращается пустой текст;
    \item Если в начале чтения достигнут конца файла, то возвращается ссылка nil (если размер фрагмента ограничен). В другом случае возвращается пустой текст. 
  \end{itemize}

  \declare{::read_nonblock( bytesize, buffer = nil )}{\# -> buffer}
  Чтение данных, не блокирующее процесс выполнения. В этом случае при ошибке чтения, она будет вызвана немедленно.
 
  \declare{::readpartial( bytesize, buffer = nil )}{\# -> buffer}
  Чтение данных блокирующее процесс выполнения:
  \begin{itemize}
    \item если буфер пуст;
    \item если поток пуст;
    \item если поток не достиг конца файла;
  \end{itemize}
  
  После блокировки ожидается получение данных или вызов сообщения о достижении конца файла.
\end{methodlist}

{\bf Строки}

\begin{methodlist}
  \declare{::gets( sep = \$/, bytesize = nil )}{\# -> string}
  Чтение следующей строки из потока (размер строки может быть ограничен). Переданный разделитель обрабатывается в качестве символа перевода строки (если передается пустой текст, то обрабатывается "/n/n"). Если достигнут конец файла, то возвращается nil.

  Полученная в результате строка связывается с глобальной переменной \$_. 
 
  \declare{::readline( sep = \$/, bytesize = nil )}{\# -> string}
  Версия метода вызывающая ошибку при достижении конца файла. 
 
  \declare{::lineno}{\# -> integer}
  Возвращает позицию (порядковый номер) извлекаемой строки. Вызов метода аналогичен вызову глобальной переменной \$. 
 
  \declare{::lineno=(pos)}{\# -> integer}
  Объявление позиции (порядкового номера) извлекаемой строки. Позиция обновляется при последующем чтении данных из потока.  
 
  \declare{::seek( offset, object = IO::SEEK_SET )}{\# -> 0} 
  Объявление позиции (порядкового номера) извлекаемой строки относительно текущего положения и переданного смещения.

  Константы:
  \begin{description}
    \item[IO::SEEK_CUR] -> новая_позиция = текущая_позиция + offset
    \item[IO::SEEK_END] -> новая_позиция = конец_файла + offset
    \item[IO::SEEK_SET] -> новая_позиция = offset
  \end{description}   

  \declare{::rewind}{\# -> 0} 
  Сброс позиции (порядковый номер) извлекаемой строки.
\end{methodlist}

{\bf Символы}

\begin{methodlist}
  \declare{::getc}{\# -> string}
  Чтение следующего символа из потока. Если достигнут конец файла, возвращается nil.

  \declare{::readchar}{\# -> string}
  Версия метода вызывающая ошибку при достижении конца файла.
\end{methodlist}

{\bf Байты}

\begin{methodlist}
  \declare{::getbyte}{\# -> integer} 
  Чтение следующего байта из потока. Если достигнут конец файла, возвращается nil. 
 
  \declare{::readbyte}{\# -> integer} 
  Версия метода вызывающая ошибку при достижении конца файла. 
 
  \declare{::pos}{\# -> integer} 
  \alias{tell} 
  Возвращает позицию (порядковый номер) извлекаемого байта. 
 
  \declare{::pos=(pos)}{\# -> integer} 
  Объявление позиции (порядкового номера) извлекаемого байта. Позиция обновляется при последующем чтении данных из потока.
\end{methodlist}

{\bf Итераторы}

\begin{methodlist}
  \declare{::each( sep = \$/ ) \{ |string| \}}{\# -> self} 
  \verb!(size) { |string| } # -> self!
  \\\verb!( sep, size) { |string| } # -> self!

  \alias{each_line, lines} 
  Последовательный перебор строк для каждого файла в потоке. Количество строк может быть ограничено. Переданный разделитель обрабатывается в качестве символа перевода строки.
 
  \declare{::bytes \{ |byte| \}}{\# -> self}
  \alias{each_byte} 
  Последовательно перебирает байты для каждого файла в потоке. 

  \declare{::chars \{ |char| \}}{\# -> self}
  \alias{each_char} 
  Последовательно перебирает символы для каждого файла в потоке.
\end{methodlist}

\subsubsection{Запись данных}

Запись данных с помощью ARGF возможна только при запуске программы с ключом \verb!-i!.

\begin{methodlist}
  \declare{::inplace_mode}{\# -> string} 
  Возвращает расширение, используемое для создания резервных копий изменяемых файлов.

  \declare{::inplace_mode=(ext)}{\# -> self}
  Изменяет расширение, используемое для создания резервных копий изменяемых файлов.

  \declare{::write(object)}{\# -> object.to_s.bytesize}
  Запись \verb!object.to_s! в обрабатываемый файл.

  \declare{::print( *object = \$_ )}{\# -> nil}
  Запись \verb!object.to_s! в обрабатываемый файл для всех переданных аргументов.
  \begin{itemize}
    \item Если глобальная переменная \$,, отвечающая за разделение элементов, не ссылается на nil, то она будет использоваться для разделения аргументов. 
    \item Если глобальная переменная \$\textbackslash, отвечающая за разделение данных, не ссылается на nil, то она будет использована после записи всех объектов. 
  \end{itemize}
 
  \declare{::printf( format, *object = nil )}{\# -> nil}
  Запись отформатированных объектов в обрабатываемый файл: \\* \verb!string % [*object]!. 
   
  \declare{::putc(object)}{\# -> object}
  Запись \verb!object.to_s! в обрабатываемый файл. Переданное число считается кодовой позицией.
 
  \declare{::puts( *object = \$\\ )}{\# -> nil}
  Запись \verb!object.to_s! в обрабатываемый файл. Аргументы разделяются с помощью перевода строки. Из индексного массива извлекаются все элементы.
\end{methodlist}
  \chapter{Использование библиотек кода}

Для облегчения повторного использования кода программу принято разделять на библиотеки. Библиотека - это файл, содержащий набор модулей и классов (хотя обычно каждая библиотека содержит только один модуль или класс).

Для использования библиотеки требуется явно объявить это интерпретатору. В свою очередь, интерпретатор автоматически находит и извлекает содержимое библиотеки. Обычно используемые библиотеки объявляются в начале программы.

Поиск всех объявленных библиотек происходит в каталогах, хранящихся в глобальном массиве \verb!$LOAD_PATH ($:)!. Поиск файла выполняется, начиная с первого элемента (начиная с первого каталога).

\begin{methodlist}
  \declare{.require(path)}{\# -> bool}
  Однократное использование библиотеки. Названия библиотек сохраняются в массиве \verb!$LOAD_FEAUTURES ($”)!. Для каждой библиотеки может существовать только одно объявление. Уровень безопасности объявляемой библиотеки должен быть равен 0.

  В названии файла библиотеки расширение обычно не указывается. По умолчанию обрабатывается расширение \verb!.rb!. Если файла с таким расширением не найдено, то будет произведен поиск бинарного файла с тем же именем (например, с расширениями \verb!.so! или \verb!.dll!).
   
  Возвращается логическое значение. 

  \declare{.require_relative(path)}{\# -> bool}
  Версия метода, аналогичная предыдущему. Вызывается для поиска библиотек в базовом каталоге программы.
 
  \declare{.load( path, anonym = false )}{} 
  Многократное использование библиотеки (извлечение и выполнение кода). В имени файла должно быть указано его расширение.

  Код библиотеки может быть выполнен в теле анонимного модуля. В этом случае она не будет влиять на глобальную область видимости основной программы. 
 
  \declare{.autoload( name, path )}{\# -> nil}
  Данный метод используется для автоматизации. Поиск библиотеки выполняется только при вызове переданной константы.
 
  \declare{.autoload?(name)}{\# -> path} 
  Возвращает название библиотеки, которая будет использована при вызове переданной константы. Если такая библиотека не объявлена, то возвращается nil. 
 
  \declare{module.autoload( name, path )}{\# -> nil} 
  Данный метод используется для автоматизации. Поиск библиотеки выполняется только при вызове в теле модуля переданной константы.
 
  \declare{module.autoload?(name)}{\# -> path} 
  Возвращает название библиотеки, которая будет использована при вызове в теле модуля переданной константы. Если такая библиотека не объявлена, то возвращается nil.
\end{methodlist}
  \chapter{Обработка событий}

Событие - это сообщение, вызываемое в различных точках выполнения программы при переходе объектов из одного состояния в другое. События предназначены для управления реакцией программы на изменение состояния.
  
Один из подвидов событий - это ошибки или исключения, возникающие в процессе выполнения программы (подразумеваются допустимые ошибки, возникновение которых ожидаемо). Для обработки ошибок в Ruby предусмотрен класс Exception. Обычно название каждого подкласса, содержит полную информацию о причине вызова ошибки.

Каждая ошибка обычно связана с определенным текстовым сообщением, поясняющим причину и место вызова ошибки.

В множество ошибок также входят системные ошибки, имеющие стандартный цифровой код. Модуль Errno динамически связывает полученные от операционной системы цифровые коды с подклассами Exception. При этом для каждой ошибки создается собственный подкласс \error{SystemCallError}, на которые ссылаются константы модуля Errno. Цифровой код ошибки может быть получен с помощью константы Errno (\constant{Errno::\italy{ErrorKlass}::Errno}). 

\section{Иерархия исключений}

\begin{description}
  \item[Exception] - базовый класс для всех исключений.
  \begin{description}
    \item[NoMemoryError] - выделение памяти не может быть выполнено;

    \item[ScriptError]- базовый класс для ошибок интерпретации программы;

    \begin{description}
      \item[LoadError] - файл не может быть загружен;
      \item[NotImplemenetedError] - метод не поддерживается системой;
      \item[SyntaxError] - ошибка в синтаксисе;
    \end{description}

    \item[SecuirityError] - нарушение требований безопасности;

    \item[SignalException] - получение сигнала от системы;
    \begin{description}
      \item[Interrupt] - сигнал прервать процесс выполнения (обычно Ctrl+C);
    \end{description}

    \item[SystemExit] - завершение выполнения программы системой;

    \item[SystemStackError] - переполнение стека;

    \item[StandardError] - базовый класс для стандартных ошибок выполнения;
    \begin{description}
      \item[Math::DomainError] - объекты не принадлежат области определения функции;

      \item[ArgumentError] - ошибка при передаче аргументов;

      \item[EncodingError] - базовый класс для ошибок, связанных с кодировкой;
      \begin{description}
        \item[Encoding::CompatibilityError] - исходная кодировка не совместима с требуемой;
        \item[Encoding::ConverterNotFoundError] - требуемая кодировка не поддерживается;
        \item[Encoding::InvalidByteSequenceError] - текст содержит некорректные байты;
        \item[Encoding::UndefinedConversionError] - текст содержит неопределенные символы;
      \end{description}

      \item[FiberError] - ошибка при работе с управляемыми блоками;

      \item[IOError] - возникновение ошибки при работе с потоками;
      \begin{description}
        \item[EOFError] - достигнут конец файла; 
      \end{description}

      \item[IndexError] - индекс не найден;
      \begin{description}
        \item[KeyError] - ключ не найден; 
        \item[StopIteration] - завершение итерации;
      \end{description}

      \item[LocalJumpError] - блок не может быть выполнен;

      \item[NameError] - неизвестный идентификатор;
      \begin{description}
        \item[NoMethodError] - неизвестный метод;
      \end{description}

      \item[RangeError] - выход за границы диапазона;
      \begin{description}
        \item[FloatDomainError] - попытка преобразования констант для определения специальных чисел (NaN и т.д.);
      \end{description}

      \item[RegexpError] - ошибка в регулярном выражении;

      \item[RuntimeError] - универсальный класс для ошибок выполнения;

      \item[SystemCallError] - базовый класс для системных ошибок;

      \item[ThreadError] - ошибка при работе с процессами;

      \item[TypeError] - неправильный тип объекта;

      \item[ZeroDivisionError] - деление целого числа на ноль.
    \end{description}
  \end{description}
\end{description}
 
\section{Методы}

\subsection*{Exception}

\begin{methodlist}
  \declare{::exception( message = nil )}{\# -> exception} 
  Создание объекта. Для аргумента вызывается метод \method{.to_str}.

  \declare{::new( mesage = nil )}{\# -> exception} 
  Создание объекта. 
 
  \declare{.exception( message = nil )}{\# -> exception}
  Новый экземпляр класса. Для аргумента вызывается метод \method{.to_str}.

  \declare{.backtrace}{\# -> array} 
  Возвращает позицию выполнения программы при вызове ошибки. Каждый элемент имеет вид:
  \\\verb!"имя_файла:номер_строки: in 'идентификатор_метода'"!
  или 
  \\\verb!"имя_файла:номер_строки"!. 

  \declare{.set_backtrace(array)}{\# -> array} 
  Изменяет позицию выполнения программы, возвращаемую при вызове ошибки. Каждый элемент индексного массива должен иметь вид: 
  \\\verb!"имя_файла:номер_строки: in 'идентификатор_метода'"!
  или 
  \\\verb!"имя_файла:номер_строки"!.

  \declare{.to_s}{\# -> string} 
  \alias{message}
  Возвращает сообщение об ошибке (или идентификатор класса). 
 
  \declare{.inspect}{\# -> string}
  Возвращает идентификатор класса.
\end{methodlist}

\subsection*{SystemExit}

\begin{methodlist}
  \declare{::new( status = 0 )}{\# -> exception} 
  Создание нового объекта. 

  \declare{.status}{\# -> integer} 
  Возвращает статус выхода. 
 
  \declare{.success?}{} 
  Проверяет удалось ли завершение программы.
\end{methodlist}

\subsection*{Encoding::InvalidByteSequenceError}

\begin{methodlist}
  \declare{.destination_encoding}{\# -> encoding} 
  Возвращает требуемую кодировку 
 
  \declare{.destination_encoding_name}{\# -> string}
  Возварщает название требуемой кодировки.

  \declare{.source_encoding}{\# -> encoding} 
  Возвращает исходную кодировку. При нескольких преобразованиях исходной будет считаться последняя стабильная кодировка. 
 
  \declare{.source_encoding_name}{\# -> string} 
  Возвращает название исходной кодировки. При нескольких преобразованиях исходной будет считаться последняя стабильная кодировка.

  \declare{.error_bytes}{\# -> string} 
  Возвращает байт из-за которого была вызвана ошибка. 

  \declare{.incomplete_input?}{} 
  Проверяет была ли ошибка вызвана преждевременным завершением текста. 
 
  \declare{.readagain_bytes}{\# -> string} 
  Интерпретатор возвращает байт, обрабатываемый в момент вызова ошибки.
\end{methodlist}

\subsection*{Encoding::UndefinedConversionError}

\begin{methodlist}
  \declare{.destination_encoding}{\# -> encoding} 
  Возвращает требуемую кодировку 
 
  \declare{.destination_encoding_name}{\# -> string}
  Возварщает название требуемой кодировки.

  \declare{.source_encoding}{\# -> encoding} 
  Возвращает исходную кодировку. При нескольких преобразованиях исходной будет считаться последняя стабильная кодировка. 
 
  \declare{.source_encoding_name}{\# -> string} 
  Возвращает название исходной кодировки. При нескольких преобразованиях исходной будет считаться последняя стабильная кодировка.

  \declare{.error_char}{\# -> string} 
  Возвращает символ из-за которого была вызвана ошибка. 
\end{methodlist}

\subsection*{StopIteration}

\begin{methodlist}
  \declare{.result}{\# -> object} 
  Возвращает результат итерации.
\end{methodlist}

\subsection*{LocalJumpError}

\begin{methodlist}
  \declare{.exit_value}{\# -> object} 
  Возвращает аргумент, передача которого привела к вызову ошибки. 
 
  \declare{.reason}{\# -> symbol}
  Возвращает идентификатор инструкции, выполнение которой привело к вызову ошибки (:break, :redo, :retry, :next, :return, или :noreason).
\end{methodlist}

\subsection*{NameError}

\begin{methodlist}
  \declare{::new( message, name = nil )}{\# -> exception} 
  Создание нового объекта. 

  \declare{.name}{\# -> name}
  Возвращает идентификатор, использование которого привело к ошибке.
\end{methodlist}

\subsection*{NoMethodError}

\begin{methodlist}
  \declare{::new( message, name, *args = nil )}{\# -> exception}
  Создание нового объекта. 

  \declare{.args}{\# -> object}
  Возвращает аргументы, переданные отсутствующему методу.
\end{methodlist}

\subsection*{SystemCallError}

\begin{methodlist}
  \declare{::new( message, integer )}{\# -> exception}
  Создание нового экземпляра класса из модуля Errno (если методу передан известный системе цифровой код ошибки) или класса SystemCallError. 

  \declare{.errno}{\# -> integer} 
  Возвращает цифровой код ошибки. 
\end{methodlist}

\section{Вызов и обработка событий}

\subsection{Вызов}

Вызов события выполняется с помощью частного метода экземпляров из модуля Kernel.

\begin{methodlist}
  \declare{raise( message = nil )}{\# -> exception} 
  \verb!( exc = RuntimeError, message = nil, pos = caller ) # -> exception!

  \alias{fail}
  Возвращает значение \verb/$!/ или новый экземпляр класса RuntimeError, если \verb/$!/ ссылается на nil.

  В другом случае методу передаются любой объект, отвечающий на вызов метода \method{.exception}, сообщение о событии и текущая позиция выполнения программы.
\end{methodlist}

\subsection{Обработка событий}

Обработка событий выполняется с помощью предложения rescue, которое может использоваться только в теле предложений begin, def, class, или module.

События обрабатываются в том порядке, в котором записаны в коде. При вызове события интерпретатор ищет для него обработчик, продвигаясь вверх по областям видимости. Таким образом события обрабатываются в обход процесса выполнения.

Если событие вызвано в результате обработки другого события, то поиск выполняется заново. 

\subsubsection*{Полный синтаксис}

\begin{verbatim}
  begin 
    тело_предложения
  rescue 
    тело_обработчика 
  else 
    code 
  ensure 
    code 
  end\
\end{verbatim}

\begin{itemize}
  \item Тело обработчика выполняется после вызова события в теле предложения. Переменная \verb/$!/ при этом ссылается на объект события.
 
  Чтобы присвоить объекту события локальную переменную используют инструкцию \verb!rescue => локальная_переменная!.

  \item По умолчанию обрабатываются подклассы StandardError.

  Для ограничения обрабатываемых событий используют инструкцию \verb!rescue class! или \verb!rescue class => локальная_переменная!. Несколько классов разделяются запятыми.

  \item Инструкция else выполняется если вызова события не произошло. При этом события, вызванные в инструкции не обрабатываются.

  \item Инструкция ensure выполняется после выполнения всего предложения. Результат ее выполнения не влияет на результат выполнения предложения (кроме случаев использования инструкций return, break и т.д)
\end{itemize}

\subsubsection*{Краткий синтаксис:}

\verb!код rescue тело_обработчика!

Если в коде будет вызвана ошибка, то выполняется тело обработчика. Обрабатываются только подклассы StandardError. 

\subsection{Catch и Throw}

В других языках программирования обработка событий обычно выполняется с помощью пары инструкций catch и throw. В Ruby существуют чатсные методы экземпляров из модуля Kernel, ведущие себя сходным образом.

\begin{methodlist}
  \declare{catch(name = nil) \{ |name| \}}{\# -> object}
  Тело блока выполняется до тех пор пока интерпретатор не встретит вызов метода \method{throw} с тем же идентификатором. При вызове без аргументов новый случайный идентификатор передается блоку.

  \declare{throw( name, *object = nil )}{}
  Завершает выполнение блока, переданного методу \method{catch} с тем же идентификатором (иначе вызывается ошибка). Поиск метода выполняестя вверх по иерархии вложенности. Дополнительные аргументы возвращаются методом \method{catch}.
\end{methodlist}
  \chapter{Тестирование}

\epigraph
{Если вы хотите улучшить программу, вы должны не тестировать больше, а программировать лучше.}{}

Любая, даже самая маленькая программа не может гарантировать правильной работы. Тестирование программ и исправление обнаруженных ошибок - один из самых трудоемких этапов разработки. Она не заканчивается и после публикации приложения. 

Исправив самые распространенные ошибки на этапе разработки программист сохранит огромное количество нервов и времени будущим пользователям. 

Обзор кода часто эффективнее, чем тестирование, но выполнение тестов позволяет проверить влияние внесенных изменений.

Главная задача тестирования - это выявление ошибок, а не их устранение. Устранение дефектов - это самый дорогой и длительный этап разработки. Легче сразу создать высококачественную программу, чем создать низкокачественную и исправлять ее. Повышение качества программы снижает затраты на ее разработку.

Существует множество подходов к тестированию приложения, но в основном грамотное тестирование - процесс прежде всего творческий.

В общем случае тестирование приложения разделяется на четыре уровня:
\begin{description}
  \item[Модульное тестирование] - тестирование минимально возможного фрагмента кода (он же блочный или unit test); 
  \item[Интеграционное тестирование] - тестирование взаимодействиями между различными элементами приложения; 
  \item[Функциональное тестирование] - тестирование задач, выполняемых с помощью приложения; 
  \item[Тестирование производительности.]
\end{description}

Одна из популярных техник тестирования - разработка приложения через его тестирование (TDD, Test-Drive Development). Использование этой техники разделено на следующие этапы: 
\begin{enumerate}
  \item Написание кода, тестирующего часть приложения; 
  \item Выполнение теста. Получение отрицательного результата. Это необходимо для проверки корректности теста; 
  \item Написание части кода приложения; 
  \item Выполнение теста. Получение положительного результата.
\end{enumerate}

Создание тестов перед кодом фокусирует внимание на требованиях к программе (т.е. понимания для чего она создается).

Для тестирования программы может быть использована команда \verb!testrb! (исполняемый файл на Ruby, использующий стандартную библиотеку Test::Unit). 

Программа принимает путь к файлу, выполняет содержащуюся в нем программу и выводит информацию о выполнении (время выполнения, количество выполненных тестов, результат их выполнения). 

Для проверки выполнения небольших кусков кода может быть использован интерактивный терминал irb. 
  \chapter{Многопоточность и параллелизм}

Процесс - максимальная единица планирования ядра ОС. Ресурсы для процессов выделяются системой (каждый процесс использует отдельные ресурсы). При запуске программы создается новый процесс, в пространстве которого она выполняется. В связи с этим процессом также иногда называют непосредственное выполнение кода программы.

Поток выполнения - это минимальная единица планирования. Внутри каждого процесса существует по крайней мере один поток выполнения. Потоки выполнения, находящиеся внутри одного процесса совместно используют его адресное пространство и состояние выполнения.

% ``` note
\begin{note}
Потоки выполнения (thread) легко перепутать с потоками данных (io). Поток выполнения - это упорядоченная группа выполняемых действий (инструкций, выражений, команд и т.д.), а поток данных - это упорядоченная группа данных, которые могут быть записаны или прочитаны.
\end{note}
% ```

Идея распараллеливания вычислений основана на том, что большинство задач (процессов) может быть разделено на набор меньших задач (потоков выполнения), которые могут быть выполнены одновременно.

Параллельные вычисления стали доминирующей парадигмой после появления многоядерных процессоров.

Создание программ для параллельного выполнения сложнее, чем для последовательного из-за возникновения конкуренции за ресурсы. Взаимодействие и синхронизация между процессами повышают сложность проектирования параллельных программ.

Использование потоков облегчает распараллеливание так как переключение между потоками легче, чем переключение между процессами и потоки, в отличии от процессов, используют единое адресное пространство и единое состояние.

В Ruby реализована поддержка потоков, но при этом используется глобальная блокировка интерпретатора (GIL), запрещающая выполнение двух или более потоков одновременно, фактически ограничивая параллелизм на многоядерных системах, заменяя его многопоточностью.

Многопоточность - это широко распространенная модель проектирования и выполнения кода, позволяющая нескольким потокам выполняться в рамках одного процесса. Процессор переключается между разными потоками выполнения. Это переключение выполняется достаточно часто, чтобы восприниматься как одновременное выполнение.

Глобальная блокировка необходима, потому что управление памятью в Ruby реализовано в расчете на последовательное выполнение программ и не подразумевает встроенной защиты от изменения различными потоками (а это не безопасно). 

Реализация многопоточности в Ruby позволяет реагировать на ввод данных. Обычно если основной поток выполнения заблокирован, то выполнение программы останавливается. Вместо этого, после блокировки основного потока, позволяется переключаться на выполнение производного потока, избегая ожидания реакции системы.

\begin{note}
  Блокировка потока отличается от завершения выполнения. После блокировки выполнение потока может быть продолжено с места выполнения блокировки (точки останова). Таким образом блокировка потока - это пауза, а не стоп.
\end{note}

\section{Потоки выполнения}

\epigraph
{Хотя кажется, что потоки выполнения — это небольшой шаг от последовательных вычислений, по сути они представляют собой огромный скачок. Они отказываются от наиболее важных и привлекательных свойств последовательных вычислений: понятности, предсказуемости и детерминизма. Потоки выполнения, как модель вычислений, являются потрясающе недетерминированными, и работа программиста становится одним из обрезков этого недетерминизма.}
{Edward A. Lee}

Для создания и управления потоками выполнения используется класс Thread. Однако многопоточность также может быть реализована с помощью класса Fiber.

Поток выполнения, создаваемый при запуске программы, называется основным, а потоки выполнения, создаваемые программой - производными.

После завершения основного потока выполнения, процесс выполнения программы прекращается, даже если производные потоки еще не выполнены.

Производные потоки используют ту же глобальную область видимости, что и основной процесс, но многие глобальные переменные отличаются для разных процессов.

Состояние потоков:
\begin{description}
  \item[Выполняющийся] - поток, выполнение которого еще не завершено; 
  \item[Ожидающий] - поток, бездействующий до выполнения определенного условия (ограничения по времени, получения ответа от системы);
  \item[Выполненный] - поток, выполнение которого завершено. Выполнение может быть завершено нормально, с ошибкой или преждевременно.
\end{description}

Потоки выполнения создаются в состоянии выполнения. Текущий поток выполняется до тех пор пока не будет переведен в режим ожидания или выполнен. После этого начинается выполнение следующего потока. Переключение между потоками совершается при выполнении системных вызовов (например ввода/вывода). 

\subsection{Thread}

Класс Thread реализует стандарт POSIX для реализации потоков выполнения.

\begin{methodlist}
  \declare{::start(*arg) \{ |*arg| \}}{\# -> thread} 
  \alias{fork, new} 
  Создание нового потока.
  \\\verb!Thread.start { } # -> #<Thread:0x962817c run>!

  \declare{::list}{\# -> array} 
  Возвращает массив, содержащий все созданные потоки выполнения. 
  \\\verb!Thread.list # -> [#<Thread:0x94da0a4 run>, #<Thread:0x964b4d8 sleep>]!

  \declare{::current}{\# -> thread} 
  Возвращает ссылку на текущий поток выполнения. 
  \\\verb!Thread.current # -> #<Thread:0x94da0a4 run>!
 
  \declare{::main}{\# -> thread} 
  Возвращает ссылку на основной поток выполнения. 
  \\\verb!Thread.main # -> #<Thread:0x94da0a4 run>!
\end{methodlist}

\subsubsection*{Управление текущим потоком}

\begin{methodlist}
  \declare{::pass}{\# -> nil}
  Интерпретатору передается сообщение о необходимости переключения потока. Переключение выполняется в зависимости от операционной системы и процессора (т.е. не обязателен). 
  \\\verb!Thread.start { Thread.pass } # -> #<Thread:0x962e554 run>!

  \declare{::stop}{\# -> nil} 
  Перевод текущего потока выполнения в режим ожидания.

  \declare{::exit}{\# -> thread} 
  \alias{kill}
  Завершение выполнения текущего потока. Если выполнение уже завершено, то возвращается ссылка на класс Thread.
\end{methodlist}

\subsubsection*{Управление произвольным потоком}

\begin{methodlist}
  \declare{.join( sec = nil )}{\# -> thread}
  Блокировка текущего потока до тех пор пока не будет выполнен поток, для которого метод был вызван (выполнение потока может быть приостановлено через переданное количество секунд - в этом случае возвращается nil).
  \begin{verbatim}
  a = Thread.new { print ?a; sleep(10); print ?b; print ?c } 
  x = Thread.new { print ?x; Thread.pass; print ?y; print ?z } 
  x.join 
  # -> "axyz" 
  y = Thread.new { 4.times { sleep 0.1; puts 'tick... ' } } 
  puts "Waiting" until y.join 0.15 
  # -> 
  "tick... 
  Waiting 
  tick... 
  Waiting 
  tick... 
  tick..."
  \end{verbatim}

  \declare{.value}{\# -> object} 
  Выполняет поток (с помощью метода \method{.join}) и возвращает результат выполнения (результат последнего выполненного выражения) .

  \declare{.run }{\#-> self} 
  Переключение потока в режим выполнения. Текущий процесс при этом переводится в режим ожидания, после чего начинается выполнение процесса, для которого метод был вызван. 

  \declare{.wakeup}{\# -> self} 
  Переключение потока в режим выполнения (при этом процесс может быть заблокирован).

  \declare{.exit(status)}{\# -> self} 
  \alias{kill, terminate}
  Завершение выполнения потока с переданным статусом. Если выполнение уже завершено, то возвращается ссылка на класс Thread. 
 
  \declare{.priority}{\# -> integer}
  Возвращает приоритет выполнения поток. Основной поток выполняется с нулевым приоритетом. Производные потоки, наследуют приоритет от базовых. 

  Потоки выполнения с большим приоритетом выполняются раньше, чем потоки с меньшим приоритетом. Работа данного метода зависит от операционной системы. 

  \declare{.priority= (integer)}{\# -> self} 
  Изменяет приоритет выполнения для потока.

  \declare{.add_trace_func( proc = nil )}{\# -> proc} 
  \alias{set_trace_func} 
  Объявление замыкания для перехвата выполнения при изменении состояния потока. Если передается nil, то перехват выполнения прекращается.
\end{methodlist}

\subsubsection*{Локальные переменные}

\begin{methodlist}
  \declare{thread[name]= (object)}{\# -> object} 
  Инициализирует локальную переменную для потока выполнения.
  \\\verb!Thread.main[:local] = 4 # -> 4!
 
  \declare{thread[name]}{\# -> object}
  Возвращает значение локальной переменной. Если переменная не существует, то возвращается nil.
  \\\verb!Thread.main[:local] # -> 4!
 
  \declare{.keys}{\# -> array}
  Возвращает массив идентификаторов всех локальных переменных.
  \\\verb!Thread.main.keys # -> [:local]!
 
  \declare{.key?(name)}{} 
  Проверяет существование локальной переменной.
  \\\verb!Thread.main.key? :global # -> false!
\end{methodlist}

\subsubsection*{Обработка ошибок}

\begin{methodlist}
  \declare{::abort_on_exception}{\# -> bool} 
  Возвращает глобальные настройки по обработке ошибок. По умолчанию - false. 
 
  \declare{::abort_on_exception= (bool)}{\# -> bool}
  При вызове ошибки в любом из производных потоков, выполнение основного потока прекращается. Подобное поведение также применяется если глобальная переменная \verb!$DEBUG! ссылается на true (программа запущена с ключом \verb!-d!). Основной поток завершается с помощью выражения \verb!Thread.main.exit(0)!. 

  \declare{.abort_on_exception}{\# -> bool} 
  Возвращает глобальные настройки по обработке ошибок. По умолчанию - false. 
 
  \declare{.abort_on_exception= (bool)}{\# -> bool} 
  При вызове ошибки в производном потоке, выполнение основного процесса прекращается. Подобное поведение также применяется если глобальная переменная \verb!$DEBUG! ссылается на true (программа запущена с ключом \verb!-d!). Основной поток завершается с помощью выражения \verb!Thread.main.exit(0)!.

  \declare{.raise( message = nil )}{}
  \verb!( exc, message = nil, pos = nil )!

  Вызов ошибки для потока. Метод не может быть вызван для текущего потока выполнения. По умолчанию вызов ошибки в производном потоке не приводит к завершению основного потока.
\end{methodlist}

\subsubsection*{Остальное}

\begin{methodlist}
  \declare{.group}{\# -> thgroup} 
  Возвращает группу, в которую входит поток выполнения. Если поток не входит ни в одну из существующих групп, то возвращается nil.
  \\\verb!Thread.main.group # -> #<ThreadGroup:0x94d9d98>!

  \declare{.inspect}{\# -> string} 
  Возвращает информацию об объекте. 
  \\\verb!Thread.main.inspect -> "#<Thread:0x94da0a4 run>"!

  \declare{.status}{\# -> object}
  Возвращает статус выполнения потока.
  \begin{itemize}
    \item Для выполняющихся потоков возвращается "run";
    \item Для ожидающих потоков возвращается "sleep";
    \item Для потоков, завершенных нормально возвращается false;
    \item Для потоков, завершенных с ошибкой возвращается nil;
    \item Для потоков, завершенных преждевременно возвращается "aborting".
  \end{itemize}   
  \verb!Thread.main.status -> "run"!

  \declare{.alive?}{} 
  Проверяет будет ли выполняться поток (выполнение потока еще не завершено).
  \\\verb!Thread.main.alive? -> true!

  \declare{.stop?}{} 
  Проверяет остановлено ли выполнение потока (поток не находится в режиме выполнения). 
  \\\verb!Thread.main.stop? -> false!
 
  \declare{.safe_level}{\# -> int} 
  Возвращает уровень безопасности. 
  \\\verb!Thread.main.safe_level -> 0!

  \declare{.backtrace}{\# -> array} 
  Возвращает состояние выполнения потока.
\end{methodlist}

\subsection{Группировка потоков}

Для группировки потоков выполнения в Ruby предоставлен класс ThreadGroup. 

Каждый поток выполнения может одновременно входить только в одну группу. При добавлении потока в другую группу, он автоматически удаляется из текущей группы. Производные потоки выполнения входят в ту же группу, что и базовые.

\begin{methodlist}
  \declare{::Default}{\# -> thgroup}
  Группа, создаваемая по умолчанию. Основной поток выполнения будет относиться к этой группе.

  \declare{::new}{\# -> thgroup}
  Создание новой группы.
  \\\verb!ThreadGroup.new # -> #<ThreadGroup:0x9711728>!

  \declare{.add(thread)}{\# -> thgroup}
  Добавляет поток выполнения в группу, удаляя его из текущей группы.
  \begin{verbatim}
  Thread.main.group # -> #<ThreadGroup:0x94d9d98>
  ThreadGroup.new.add Thread.main # -> #<ThreadGroup:0x96f0154>
  Thread.main.group # -> #<ThreadGroup:0x96f0154>
  \end{verbatim}

  \declare{.enclose}{\# -> thgroup}
  Блокирует группу, запрещая добавлять или удалять содержащиеся в ней потоки выполнения.
  \begin{verbatim}
  Thread.main.group.enclose # -> #<ThreadGroup:0x96f0154>
  ThreadGroup.new.add Thread.main # -> error
  \end{verbatim}

  \declare{.enclosed?}{}
  Проверяет заблокирована ли группа.
  \\\verb!Thread.main.group.enclosed? # -> true!

  \declare{.list}{\# -> array}
  Возвращает массив потоков выполнения, входящих в группу.
  \begin{verbatim}
  Thread.main # -> #<Thread:0x94da0a4 run>
  Thread.main.group.list # -> [#<Thread:0x94da0a4 run>]
  \end{verbatim}
\end{methodlist}

\subsection{Синхронизация потоков}

Синхронизация потоков требуется, чтобы избежать ошибок при использовании потоками одних и тех же данных или устройств (состояние гонки при обновлении данных).

Простейшая синхронизация выполняется с помощью экземпляров класса Mutex.

Поток выполнения, который в настоящее время работает с данными, блокирует экземпляр класса Mutex. После выполнения работы, блокировка снимается. В зависимости от наличия блокировки изменяется реакция конкурирующих потоков.

\begin{methodlist}
  \declare{::new}{\# -> mutex} 
  Создание нового объекта. 

  \declare{.lock}{\# -> mutex}
  Блокировка объекта. Если объект уже заблокирован текущим потоком, то вызывается ошибка \error{ThreadError}. 

  \declare{.try_lock}{\# -> bool}
  Попытка блокировки объекта. Возвращается результат блокировки.

  \declare{.locked?}{} 
  Проверяет заблокирован ли объект. 

  \declare{.unlock}{\# -> mutex} 
  Снятие блокировки. Если объект заблокирован другим потоком выполнения, то вызывается ошибка \error{ThreadError}.

  \declare{.sleep( sec = nil )}{\# -> sec} 
  Снятие блокировки и переключение в режим ожидания. Если объект заблокирован другим потоком выполнения, то вызывается ошибка \error{ThreadError}. 
\end{methodlist}

\section{Процессы}

Процесс - это программа в стадии ее выполнения. Ruby позволяет манипулировать процессами, используя низкоуровневые возможности системы.

Использование процессов позволяет достигать параллельного выполнения кода на многоядерных системах, но увеличивает сложность программы и количество потребляемой памяти. 

Характеристики процессов:
\begin{description}
  \item[PID] - идентификатор процесса;
  \item[PPID] - идентификатор базового процесса;
  \item[]имя владельца процесса;
  \item[UID] - реальный идентификатор владельца;
  \item[EUID] - действующий идентификатор владельца;
  \item[GUID] - реальный идентификатор группы владельцев;
  \item[EGUID] - действующий идентификатор группы владельцев;
  \item[]приоритет;
  \item[]терминал.
\end{description}

Для манипуляции процессами используются модуль Process и модуль Kernel. Также управление процессами происходит с помощью передачи сигналов.

\begin{methodlist}
  \declare{fork \{nil\}}{\# -> status}
  Создание нового подпроцесса, который является копией процесса, выполняющего этот вызов. Подпроцессы обычно используются для выполнения системных вызовов, загружающих в пространство подпроцесса новую программу. Однако ничто не мешает использовать подпроцессы для выполнение параллельных задач.

  Для производного процесса создается собственная таблица файловых дескрипторов, копирующая таблицу базового процесса. Изменения не синхронизируются.

  В Linux страницы памяти базового процесса копируются производным, только после их изменения. Это позволяет уменьшить время создания процессов и количество потребляемой памяти.

  Для производного процесса копируется только текущий поток выполнения.

  Переданный блок выполняется в теле производного процесса. После выполнения производного процесса блок закрывается и возвращается 0.

  В другом случае код программы после вызова метода, выполняется дважды - для базового процесса и для производного процесса:

  \begin{itemize}
    \item Для базового процесса в результате вызова метода возвращается идентификатор производного процесса;
    \item Для производного процесса в результате вызова метода возвращается nil.
  \end{itemize}

  Базовый процесс должен обрабатывать статусы завершения производных процессов с помощью методов \method{::wait} или \method{::detach}, иначе процессы могут превратиться в зомби.

  Если создание подпроцессов не реализовано для ОС, то выполнение \verb!Process.respond_to?(:fork)! вернет false.

  \declare{fork \{nil\}}{\# -> status}
  Частный метод экземпляров из модуля Kernel, аналогичный предыдущему.
\end{methodlist}

\section{Обработка сигналов}

Сигналы - это способ передачи сообщений между процессами.

Для работы с сигналами используется модуль Signal.

\begin{methodlist}
  \declare{::list}{\# -> hash}
  Возвращает массив названий сигналов, ассоциируемых с цифровыми кодами.

  \declare{::trap( name, command )}{\# -> object}
  \verb!(name) { } # -> object!

  Обработка сигнала после его получения. Метод принимает либо название сигнала (приставка SIG может быть пропущена), либо его цифровой код. В результате возвращается предыдущий обработчик.

  command:
  \begin{itemize}
    \item блок, выполняемый при получении сигнала;
    
    \item текст:
    \begin{description}
      \item["IGNORE" ("SIG_IGN")] - полученный сигнал игнорируется;
      \item["DEFAULT" ("SIG_DFL")] - сигнал обрабатывается как обычно;
      \item["SYSTEM_DEFAULT"] - сигнал обрабатывается в зависимости от операционной системы;
      \item["EXIT" (0)] - завершение выполнения программы. 
    \end{description}

    \item системный вызов:
    \begin{description}
      \item[string] - текст команды для используемой оболочки: по умолчанию в Unix - это \verb!"/bin/sh"!, а в Windows - \verb!ENV["RUBYSHELL"]! или \verb!ENV["COMSPEC"]!;
      \item[string, *arg] - текст команды и передаваемые аргументы;
      \item[{[ string, first_arg ], *arg}] - текст команды, первый аргумент и остальные аргументы. 
  \end{description}
  \end{itemize}

  \declare{trap( name, command )}{\# -> object}
  \verb!(name) { } # -> object!

  Частный метод экземпляров из модуля Kernel, аналогичный предыдущему.
\end{methodlist}
  \chapter{Безопасность}
 
Реализация собственной системы безопасности - одна из спорных и неоднозначных особенностей в Ruby. 

Безопасность кода в Ruby обеспечивается с помощью применения различных модификаторов, запрещающих изменение значения объекта или маркирующих небезопасные данные.

Интерпретатор автоматически считает небезопасными:
\begin{enumerate}
  \item аргументы, переданные при запуске программы (элементы массива ARGV);
  \item переменные окружения (элементы массива ENV);
  \item любые данные, извлекаемые из файлов, сокетов или потоков;
  \item объекты, создаваемые на основе небезопасных данных также не считаются безопасными.
\end{enumerate}

\verb!$SAFE! ссылается на значение текущего уровня безопасности. Переменная локальна для каждого отдельного блока кода или процесса. Уровень безопасности основного процесса объявляется при запуске программы с помощью ключа \verb!-T! (по умолчанию - 0).

При нарушении ограничений безопасности вызывается ошибка \error{SecurityError}.

\section{Уровни безопасности}

Более высокие уровни безопасности наследуют ограничения более низких.

\subsection{Уровень 0}

Не предполагает ограничений.

\subsection{Уровень 1}

Запрещается:
\begin{itemize}
  \item подключать библиотеки, если их название небезопасно;
  \item выполнять произвольный код, если текст кода небезопасен;
  \item открывать файлы, если их название небезопасно;
  \item соединяться с хостом, если его название небезопасно;
  \item запускать программу с ключами \verb!-e!, \verb!-i!, \verb!-l!, \verb!-r!, \verb!-s!, \verb!-S!, \verb!-X!;
  \item выполнять методы из классов Dir, IO, File, FileTest, передавая им небезопасные аргументы;
  \item выполнять методы \method{test}, \method{eval}, \method{require}, \method{load}, \method{trap}, передавая им небезопасные аргументы.
\end{itemize}
 
Также игнорируются переменные окружения RUBYLIB и RUBYOPT и текущий каталог при поиске подключаемых библиотек.

\subsection{Уровень 2}

Запрещается выполнять методы \method{fork}, \method{syscall}, \method{exit!}, \method{Process.equid=}, \method{Process.fork}, \method{Process.setpgid}, \method{Process.setsid}, \method{Process.kill}, \method{Process.seeprioprity} передавая им небезопасные аргументы.

\subsection{Уровень 3}

Все объекты, кроме предопределенных считаются небезопасными. Вызов метода \method{.untaint} запрещается.

\subsection{Уровень 4}

Запрещается:
\begin{itemize}
  \item изменять значение безопасных объектов;
  \item подключать библиотеки с помощью \method{require} или \method{load} (разрешается в теле анонимного модуля);
  \item использовать метапрограммирование;
  \item работать с любыми процессами кроме текущего;
  \item завершать выполнение процесса;
  \item читать или записывать данные;
  \item изменять переменные окружения;
  \item вызывать методы \method{.srand} и \method{Random.srand}.
\end{itemize}

Разрешается вызывать метод \method{.eval}, передавая ему небезопасные аргументы.

\section{Модификаторы}

Модификаторы в данном случае объявляются с помощью методов.

\begin{methodlist}
  \declare{object.freeze}{\# -> self} 
  Запрещает изменение значения объекта. 
 
  \declare{object.frozen?}{} 
  Проверяет запрещено ли изменение значения объекта. 
 
  \declare{object.taint}{\# -> self} 
  Маркировка небезопасных данных. Этим модификатором отмечаются полученные внешние данные. 
 
  \declare{object.untaint}{\# -> self} 
  Маркировка безопасных данных. 
 
  \declare{object.tainted?}{} 
  Проверяет безопасно ли использование объекта. 
 
  \declare{object.untrust}{\# -> self} 
  Маркировка данных, которым программист не доверяет. Этим модификатором отмечаются данные, полученные при выполнении кода с уровнем безопасности 4. 
  \declare{object.trust}{\# -> self} 
  Маркировка данных, которым программист доверяет. 
 
  \declare{object.untrusted?}{} 
  Проверяет доверяет ли программист объекту.
\end{methodlist} 

\appendix

\part*{Приложения}
\addcontentsline{toc}{part}{Приложения}
  \hypertarget{appbin}{}
\chapter{Запуск программы}

\begin{keylist}{Ключи:}
  \firstkey{\twominus copyright} - отображение сведений о копирайте; 
  
  \key{\twominus version} - отображение версии интерпретатора;
  
  \key{-(-h)elp} - отображение справочной информации;

  \key{-0[\italy{CODEPOINT}]} - символ перевода строки при чтении из потока (\verb!$/!). Кодовая позиция задается в восьмеричной системе счисления.
  \begin{itemize}
    \item Если число не указано, то строки разделятся не будут; 
    \item Если используется -00, то в качестве разделителя будут использоваться два символа перевода строки, идущих один за другим;
    \item Если используется -0777, то файлы буду обрабатываться как одна большая строка.
  \end{itemize}
   
  \key{-C (-X) \italy{DIR}} - базовый каталог;
  
  \key{\twominus encoding (-E) \italy{EXTERNAL[:INTERNAL]}} - внешняя и внутренняя кодировки;

  \key{-F \italy{PATTERN}} - разделитель для частей текста (\verb!$;!), использующийся при вызове метода \method{.split}. В качестве образца передаются символы или тело регулярного выражения;

  \key{-I \italy{DIRS}} - каталоги, добавляемые в начало \verb!\$LOAD_PATH ($:)! и использующиеся для поиска подключаемых библиотек. Каталоги разделяются двоеточием (Linux) или точкой с запятой (Windows);

  \key{-K \italy{ENCODING}} - внешняя и внутренняя кодировки:
  \begin{description}
    \item[e] - EUC-JP;
    \item[s] - Windows-31J (CP932);
    \item[u] - UTF-8;
    \item[n] - ASCII-8BIT (BINARY).
  \end{description} 
  
  \key{-S} - поиск программы с помощью переменной окружения PATH;
  
  \key{-T[\italy{SECURITY}]} - уровень безопасности для программы (по умолчанию 1);
  
  \key{-U} - UTF-8 в качестве внутренней кодировки;

  \key{-W[\italy{VERBOSE}]} - степень подробности отладочной информации:
  \begin{description}
    \item[0] - без предупреждений, \verb!$VERBOSE! ссылается на nil;
    \item[1] - средний уровень, \verb!$VERBOSE! ссылается на false;
    \item[2 (по умолчанию)] - выводятся все предупреждения, \verb!$VERBOSE! ссылается на true.
  \end{description}    
  
  \key{-a} - автоматическое разделение строк при использовании ключей \verb!-n! или \verb!-p!. В начале каждой итерации цикла выполняется код: \verb/$F = $_.split!/;

  \key{-c}- проверка синтаксиса. Если ошибок не найдено, то в стандартный поток для вывода передается \verb!"Syntax OK"!; 

  \key{-(-d)ebug} - режим отладки, \verb!$DEBUG! ссылается на true. Ключ позволяет писать код для отладки программы, который будет выполняться только если \verb!$DEBUG! ссылается на true;
  \\\verb!{verbatim} if $DEBUG! 
  
  \key{-e \italy{verbatim}} - выполнение произвольного кода;

  \key{-i [\italy{EXT}]} - режим редактирования, позволяющий записывать данные с помощью ARGF. Расширение используется для создания резервных копий; 

  \key{-l} - автоматическое изменение строк кода. При этом, во-первых, \verb!$\! копирует \verb!$/!, и, во-вторых, при чтении строк для каждой вызывается метод \method{.chop!};

  \key{-n} - программа выполняется в теле цикла (каждая строка программы выполняется отдельно):  
  \begin{verbatim}
  while gets 
    # code 
  end
  \end{verbatim}    
  
  \key{-p} - программа выполняется в теле цикла (каждая строка программы выполняется отдельно, результат выполнения передается в стандартный поток для вывода):  
  \begin{verbatim}
  while gets 
    # code 
  end
  print
  \end{verbatim}
  
  \key{-r \italy{LIB}} - подключение библиотеки перед выполнением с помощью \method{.require};

  \key{-s} - предварительная обработка аргументов, начинающихся с дефиса. Обработка выполняется до любого обычного аргумента или символов --. Обработанные аргументы будут удалены из ARGV.
  \begin{itemize}
    \item Для аргументов вида \verb!-x=y!, \verb!$x! в теле программы будет ссылаться на y;
    \item Для аргументов вида \verb!-x!, \verb!$x! в теле программы будет ссылаться на true.
  \end{itemize}    
  
  \key{-(-v)erbose} - программа запускается с ключом -w. Если имя программы не указано, то отображается версия интерпретатора. Ключ позволяет писать код для отладки программы, который будет выполняться только если \verb!$VERBOSE! ссылается на true;
  \\\verb!{verbatim} if $VERBOSE!; 

  \key{-w} - отображаются все возможные предупреждения, \verb!$VERBOSE! ссылается на true;

  \key{-x[\italy{DIR}]} - отображает код программы, начиная со строки, начинающейся с \verb/!#/ и содержащей \mono{ruby}. Конец программы должен быть указан с помощью EOF, \textasciicircum D (control-D), \textasciicircum Z (control-Z), или \verb!__END__!. Переданный каталог используется вместо базового;

  \key{-(-y)ydebug} - используется для отладки интерпретатора;

  \key{\twominus disable-... или \twominus enable-...} - включение или отключение:  
  \begin{description}
    \item[gems] - поиск подключаемых библиотек в пакетах;
    \item[rubyopt] - использования переменной окружения RUBYOPT;
    \item[all] - выполнение двух перечисленных действий.
  \end{description}
  
  \key{--dump \italy{TARGET}} - используется для отладки интерпретатора.
\end{keylist}

\begin{keylist}{Переменные окружения:}
  \firstkey{RUBYLIB} - список каталогов, используемых для поиска подключаемых библиотек; 

  \key{RUBYOPT} - список ключей, используемых для запуска программ. Могут быть добавлены только ключи -d, -E, -I, -K, -r, -T, -U, -v, -w, -W, \twominus debug, \twominus disable-... и \twominus enable-...; 

  \key{RUBYPATH} - список каталогов, в которых выполняется поиск программы; 
  
  \key{RUBYSHELL} - путь к оболочке ОС. Переменная действительна только для mswin32, mingw32, и OS/2; 
  
  \key{PATH} - путь к интерпретатору.
\end{keylist}

Доступ к перемнным окружения может быть получен с помощью ENV, объекта, подобный ассоциативному массиву. Для объекта определен метод \method{.to_hash}, преобразующий его в настоящий ассоциативный массив.

\begin{keylist}{Глобальные переменные:}
  \firstkey{\$F} - последний результат выполнения выражения \verb!$_.split!. Переменная определена, если программа запущена с ключами -a, -n или -p;
  \key{\$-W} - степень подробности предупреждений; 
  \key{\$-i} - расширения, используемое для резервных копий;
  \key{\$-d} -> bool; 
  \key{\$-l} -> bool; 
  \key{\$-v} -> bool; 
  \key{\$-p} -> bool; 
  \key{\$-a} -> bool;
  \key{\$; (\$-F)} - разделитель частей текста при вызове метода \method{.split} (по умолчанию nil);
  \key{\$/ (\$-0)} - символ перевода строки (по умолчанию - \verb!"\n"!);
  \key{\$\textbackslash} - разделитель, добавляемый при передаче объектов в поток (по умолчанию nil); 
  \key{\$,} - разделитель записываемых элементов(по умолчанию nil).
  \key{\$\textasciitilde} - экземпляр класса MatchData для последнего поиска совпадений; 
  \key{\$\&} - текст найденного совпадения; 
  \key{\$`} - текст перед найденным совпадением; 
  \key{\$'} - текст после найденного совпадения; 
  \key{\$+} - текст совпадения с группой; 
  \key{\$1, \$2, \$3, \$4, \$5, \$6, \$7, \$8, \$9} - текст совпадения с группой, имеющей соответствующий индекс;
  \key{\$LOAD_FEAUTURES (\$")} - массив всех подключенных библиотек;
  \key{\$LOAD_PATH (\$:, \$-I)} - массив каталогов для поиска подключаемых библиотек;
  \key{\$.} - позиция последней прочитанной строки из потока; 
  \key{\$_} - последняя прочитанная строка из потока;
  \key{\$!} - последнее вызванное событие; 
  \key{\$@} - расположение последнего вызванного события;
  \key{\$DEBUG} - true, если программа запущена с ключом \verb!-(-d)ebug!;
 
  \key{\$VERBOSE (\$-v, \$-w)}
  \begin{itemize}
    \item nil, если программа запущена с ключом \verb!-W0!;
    \item true, если программа запущена с ключами \verb!-w! или \verb!-(-v)erbose!;
    \item false в остальных случаях.
  \end{itemize}    
  
  \key{\$\$} - идентификатор текущего процесса;
  \key{\$?} - статус последнего выполненного процесса;
  \key{\$FILENAME} - относительный путь к текущему файлу в ARGF. При взаимодействии с стандартным потоком для ввода возвращается \verb!"-"!;
  \key{\$SAFE} - текущий уровень безопасности.
\end{keylist}    
  \hypertarget{appregexp}{}
\chapter{Синтаксис регулярных выражений}

\begin{keylist}{Модификаторы:}
  
  \firstkey{i} – поиск будет выполняться без учета регистра символов;
  
  \key{m} – поиск будет выполняться в многострочном режиме. Точка в теле регулярного выражения будет соответствовать также и символу перевода строки;
  
  \key{x} – пробельные символы (пробел, отступ, перевод строки) в теле регулярного выражения будут игнорироваться интерпретатором;
  
  \key{o} – интерполяция в теле регулярного выражения будет выполняться только один раз, перед началом поиска;
  
  \key{u, e, s, n} – тело регулярного выражения будет обрабатываться в указанной кодировке. Соответственно: u - UTF-8, e - EUC-JP , s - Windows-31J , n - ASCII-8BIT.
\end{keylist}

\begin{keylist}{ASCII символы:}
  
  \firstkey{.} - соответствует любому символу в тексте (кроме символа перевода строки в однострочном режиме поиска);
  
  \key{\textbackslash w} - соответствует любой букве, цифре или знаку подчеркивания;  
  \key{\textbackslash W} - соответствует любому символу, кроме букв, цифр или знаков подчеркивания;
  
  \key{\textbackslash s} - соответствует любому пробельному символу (пробел, отступ, перевод строки);
  
  \key{\textbackslash S} - соответствует любому символу, кроме пробельных;
  
  \key{\textbackslash d} - соответствует любой десятичной цифре;
  
  \key{\textbackslash D} - соответствует любому символу, кроме десятичных цифр;
  
  \key{\textbackslash h} - соответствует любой шестнадцатеричной цифре;
  
  \key{\textbackslash H} - соответствует любому символу, кроме шестнадцатеричных цифр.
\end{keylist}

\begin{keylist}{Юникод символы:}

\firstkey{[[:класс:]]} -  соответствует любому символу, входящему в класс:
  \begin{description}
    \item[alnum]  – буквы и цифры;
    \item[alpha]  – буквы;
    \item[ascii]  – ASCII символы;
    \item[blank]  – пробел и отступ;
    \item[cntrl]  – эмблемы составного текста;
    \item[digit]  – десятичные цифры;
    \item[graph]  – буквы, цифры и знаки препинания;
    \item[lower]  – строчные буквы;
    \item[print]  – буквы, цифры, знаки препинания и пробел;
    \item[punct]  – знаки препинания;
    \item[space]  – пробельные символы (пробел, отступ, перевод строки);
    \item[upper]  – прописные буквы;
    \item[word]   – буквы, цифры и специальные знаки препинания (знак подчеркивания);
    \item[xdigit] – шестнадцатеричные цифры;
  \end{description}

\key{\textbackslash p\{класс\}} - соответствует любому символу, входящему в класс.

{\bf \textbackslash p\{\textasciicircum класс\}} - соответствует любому символу, кроме входящих в класс.
  \begin{description}
    \item[Alnum]    – буквы и цифры; 
    \item[Alpha]    – буквы;
    \item[Any]      – Unicode символы;
    \item[ASCII]    – ASCII символы;
    \item[Assigned] – свободные цифровые коды;
    \item[Blank]    – пробел и отступ;
    \item[Cntrl]    – эмблемы составного текста;
    \item[Digit]    – десятичные цифры;
    \item[Graph]    – буквы, цифры и знаки препинания;
    \item[Lower]    – строчные буквы;
    \item[Print]    – буквы, цифры, знаки препинания и пробел;
    \item[Punct]    – знаки препинания;
    \item[Space]    – пробельные символы (пробел, отступ, перевод строки);
    \item[Upper]    – прописные буквы;
    \item[Word]     – буквы, цифры и специальные знаки препинания (знак подчеркивания);
    \item[Xdigit]   – шестнадцатеричные цифры.
  \end{description}
\end{keylist}

Юникод-классы символов:

C - остальные символы; Cc - спецсимволы; Cf - спецсимволы, влияющие на форматирование; Сn – свободные цифровые коды; Co - логотипы; Cs - символы-заменители;

L – буквы; Ll - строчные буквы; Lm - особые символы; Lo - остальные символы; Lt - буквы в начале слова; Lu - прописные буквы;

M – символы, использующиеся в связке; Mn - символы, изменяющие другие символы; Mc - специальные модификаторы, занимающие отдельную позицию в тексте; Me - символы, внутри которых могут находиться другие символы;

N - цифры; Nd - десятичные цифры; Nl - римские цифры; No - остальные цифры;

P - знаки препинания; Pc - специальные знаки препинания; Pd - дефисы и тире; Ps - открывающие скобки; Pe - закрывающие скобки; Pi - открывающие кавычки; Pf - закрывающие кавычки; Po - остальные знаки препинания;

S - декоративные символы; Sm - математические символы; Sc - символы денежных единиц;	Sk - составные декоративные символы; So - остальные декоративные символы;

Z - разделители, не имеющие графического представления; Zs - пробелы; Zl - перевод строки; Zp - перевод параграфа.

Также можно указать класс, определяющий алфавит. Поддерживаемые алфавиты: Arabic, Armenian, Balinese, Bengali, Bopomofo, Braille, Buginese, Buhid, Canadian_Aboriginal, Carian, Cham, Cherokee, Common, Coptic, Cuneiform, Cypriot, Cyrillic, Deseret, Devanagari, Ethiopic, Georgian, Glagolitic, Gothic, Greek, Gujarati, Gurmukhi, Han, Hangul, Hanunoo, Hebrew, Hiragana, Inherited, Kannada, Katakana, Kayah_Li, Kharoshthi, Khmer, Lao, Latin, Lepcha, Limbu, Linear_B, Lycian, Lydian, Malayalam, Mongolian, Myanmar, New_Tai_Lue, Nko, Ogham, Ol_Chiki, Old_Italic, Old_Persian, Oriya, Osmanya, Phags_Pa, Phoenician, Rejang, Runic, Saurashtra, Shavian, Sinhala, Sundanese, Syloti_Nagri, Syriac, Tagalog, Tagbanwa, Tai_Le, Tamil, Telugu, Thaana, Thai, Tibetan, Tifinagh, Ugaritic, Vai, and Yi.

\begin{keylist}{Группировка символов:}
  
  \firstkey{...} - cоответствует любому символу из ограниченных квадратными скобками. Внутри квадратных скобок могут быть использованы диапазоны символов (a-z);
  
  \key{\textasciicircum...} - соответствует любому символу, кроме ограниченных квадратными скобками.
  
  \key{(?:...)} - символы объединяются в группу и используются как одна логическая единица;
  
  \key{(...)} - символы объединяются в группу и используются как одна логическая единица. Группе будет присвоен порядковый номер (от 1 до 9);
  
  \key{(?<идентификатор>...)} - символы объединяются в группу и используются как одна логическая единица. Группе будет присвоен указанный идентификатор.
\end{keylist}

Если лексема регулярного выражения использовалось на месте левого операнда, тогда, после выполнения выражения, идентификаторы групп ссылаются на текст совпадения, или на nil, если совпадений не найдено. Идентификаторы групп объявляются как локальные переменные.

Глобальные переменные от \$1 до \$9 также ссылаются на текст совпадения с группой, имеющей указанный номер.

\begin{keylist}{Группы символов:}
  
  \firstkey{\textbackslash целое_число} - соответствует тексту совпадения с группой, имеющей указанный номер;
  
  \key{\textbackslash K <идентификатор>} - соответствует тексту совпадения с группой, имеющей указанный идентификатор;
  
  \key{\textbackslash g <...>} - соответствует тексту совпадения с группой, имеющей указанный порядковый номер или идентификатор.
\end{keylist}

\begin{keylist}{Повторы:}
  
  \firstkey{…?} - соответствует от 0 до 1 повторам символа или группы;
  
  \key{...*} - соответствует 0 и более повторам символа или группы. Результат поиска содержит максимально возможное совпадение (жадный алгоритм);
  
  \key{...*?} - соответствует 0 и более повторам символа или группы. Результат поиска содержит минимально возможное совпадение (не жадный алгоритм);
  
  \key{...+} - соответствует 1 и более повторам символа или группы. Результат поиска содержит максимально возможное совпадение (жадный алгоритм);
  
  \key{...+?} - соответствует 1 и более повторам символа или группы. Результат поиска содержит минимально возможное совпадение (не жадный алгоритм);
  
  \key{...\{a, b\}} - соответствует от a до b повторам символа или группы. Результат поиска содержит максимально возможное совпадение (жадный алгоритм);
  
  \key{...\{a, b\}?} - соответствует от a до b повторам символа или группы. Результат поиска содержит минимально возможное совпадение (не жадный алгоритм).
  
  В обоих случаях допускается отсутствие a, b, или запятой.
\end{keylist}

\begin{keylist}{Положение в тексте:}
  
  \firstkey{\textasciicircum...} - соответствует символу или группе в начале строки;
  
  \key{...\$} - соответствует символу или группе в конце строки;
  
  \key{\textbackslash A...} - соответствует символу или группе в начале текста;
  
  \key{...\textbackslash z} - соответствует символу или группе в конце текста;
  
  \key{...\textbackslash Z} - соответствует символу или группе в конце текста или перед последним символом перевода строки, замыкающим текст;
  
  \key{...\textbackslash b} - соответствует символу или группе в конце слова;
  
  \key{\textbackslash b...} - соответствует символу или группе в начале слова;
  
  \key{...\textbackslash B} - соответствует символу или группе в любом месте, кроме конца слова;
  
  \key{\textbackslash B...} - соответствует символу или группе в любом месте, кроме начала слова.
\end{keylist}

\begin{keylist}{Логические условия:}
  
  \firstkey{№1|№2} - соответствует либо символам №1, либо символам №2;
  
  \key{№1(?=№2)} - соответствует символам №1, если символы №2 встречаются далее по тексту (позитивное заглядывание вперед);
  
  \key{№1(?!№2)} - соответствует символам №1, если символы №2 не встречаются далее по тексту (негативное заглядывание вперед);
  
  \key{№1(?<=№2)} - соответствует символам №1, если символы №2 встречаются в предыдущей части текста (позитивное заглядывание назад);
  
  \key{№1(?<!№2)} - соответствует символам №1, если символы №2 не встречаются в предыдущей части текста (негативное заглядывание вперед).
\end{keylist}

\begin{keylist}{Остальное:}
  
  \firstkey{(?\#...)} - игнорируемый комментарий;
  
  \key{(?>...)} - в любом случае соответствует указанным символам;
  
  \key{(?№1-№2)} - применяет модификаторы №1 и отменяет модификаторы №2 для дальнейшего поиска;
  
  \key{(?№1-№2:...)} - применяет модификаторы №1 и отменяет модификаторы №2 для символов или групп.
\end{keylist}
  \hypertarget{appformat}{}
\chapter{Форматные строки}

Форматная строка - это текст, в котором обычные символы перемешаны с специальными синтаксическими конструкциями. Обычные символы переносятся в результат без изменений, а специальные влияют на форматируемые объекты.

Спецсимволы классифицируются по типам влияния на результат. Из них только устанавливающие тип форматирования являются обязательными.

Синтаксис спецсимволов (пробелы между ними не используются, а добавлены только для наглядности):
\\\verb!%модификатор размер .точность тип_форматирования!

Размер влияет на количество символов в результате. По умолчанию в начало текста добавляются дополнительные пробелы.
\\\verb!"%5d"%2 ->"    2"!

Точность влияет на количество цифр после десятичной точки (по умолчанию - 6).

\subsubsection*{Типы форматирования:}

\begin{keylist}{Для целых чисел:}

  \firstkey{b} – преобразует число в десятичную систему счисления. Для отрицательных чисел будет использоваться необходимое дополнение до 2 с символами \mono{..} в качестве приставки.
  \begin{verbatim}
  "%b" % 2 # -> "10"
  "%b" % -2 # -> "..10"
  \end{verbatim}
  
  \key{B} – аналогично предыдущему, но с приставкой 0B в альтернативной нотации.
  
  \key{d (или i или u)} - преобразует число в двоичную систему счисления.
  \\\verb!"%d" % 0x01 # -> "1"!
  
  \key{o} – преобразует число в восьмеричную систему счисления. Для отрицательных чисел будет использоваться необходимое дополнение до 2 с символами .. в качестве приставки.
  \begin{verbatim}
  "%o" % 2 # -> "2"
  "%o" % -2 # -> "..76"
  \end{verbatim}
  
  \key{x} – преобразует число в шестнадцатеричную систему счисления. Для отрицательных чисел будет использоваться необходимое дополнение до 2 с символами .. в качестве приставки.
  \begin{verbatim}
  "%x" % 2 -> "2"
  "%x" % -2 -> "..fe"
  \end{verbatim}
  
  \key{X} – аналогично предыдущему, но с приставкой 0X,  в альтернативной нотации.
\end{keylist}

\begin{keylist}{Для десятичных дробей:}
  \firstkey{e} – преобразует число в экспоненциальную нотацию.
  \\\verb!"%e" % 1.2 # -> "1.200000e+00"!
  
  \key{E} – аналогично предыдущему, но с использованием символа экспоненты E.
  
  \key{f} – округление числа.
  \\\verb!"%.3f" % 1.2 # -> "1.200"!
  
  \key{g} – преобразует число в экспоненциальную нотацию, если показатель степени будет меньше -4 или больше либо равен точности. В других случаях точность определяет количество значащих цифр.
  \begin{verbatim}
  "%.1g" % 1.2 # -> "1"
  "%g" % 123.4 # -> "1e+02"
  \end{verbatim}
  
  \key{G} – аналогично предыдущему, но с использованием символа экспоненты E.

  \key{a} - преобразование числа в виде: знак числа, число в шестнадцатеричной системе счисления с приставкой 0x, символ p, знак показателя степени и показатель степени в десятичной системе счисления.
  \\\verb!"%.3a" % 1.2 # -> "0x1.333p+0"!
  
  \key{A} – аналогично предыдущему, но с использованием приставки 0X и символа~P.
\end{keylist}

\begin{keylist}{Для других объектов:}
  \firstkey{c} - результат будет содержать один символ.
  \\\verb!"%с" % ?h # -> "h"!
  
  \key{p} – вызов метода .inspect для объекта.
  \\\verb!"%p" % ?h # -> "\"h\""!
  
  \key{s} – преобразование текста. Точность определяет количество символов.
  \\\verb!"%.3s" % "Ruby" # -> "Rub"!
\end{keylist}

\begin{keylist}{Модификаторы:}  
  \firstkey{-} - пробелы будут добавляться не в начало, а в конце результата.
  \\\verb!"%-4b" % 2 # -> "10  "!

  \key{0 (для чисел)} - вместо пробелов в результате будут использоваться нули.
  \\\verb!"%04b" % 2 # -> "0010"!
  
  \key{+ (для чисел)} - результат будет содержать знак плюса для положительных чисел. Для oxXbB используется обычная запись отрицательных чисел.
  \\\verb!"%+b" % -2 # -> "-10"!
  
  \key{\# (для эмблем bBoxXaAeEfgG)} - использование альтернативной нотации.
  \begin{itemize}
    \item Для o повышается точность результата до тех пор пока первая цифра не будет нулем, если не используется дополнительное форматирование.
    \\\verb!"%#o" % 2 # -> "02"!

    \item Для xXbB используются соответствующие приставки.
    \\\verb!"%#X" % 2 # -> "0X2"!

    \item Для aAeEfgG используется десятичная точка даже если в этом нет необходимости.
    \\\verb!"%#.0E" % 2 # -> "2.E+00"!

    \item Для gG используются конечные нули.
    \\\verb!"%#G" % 1.2 # -> "1.20000"!
  \end{itemize}
  
  \key{*} - соответствующий спецсимволу объект используется для определения размера. Для отрицательных чисел пробелы добавляются в конце результата.
  \\\verb!"%*d" % [ -2, 1 ] # -> "1 "!  
\end{keylist}

Если форматная строка содержит несколько спецсмволов, то они будут последовательно использоваться для форматируемых объектов, которые должны храниться в индексном или ассоциативном массивах.

Для индексных массивов модификатор \verb!цифра$! изменяет форматирование элемента с соответствующей позицией (начиная с 1). При этом модификатор должен быть указан для каждого спецсимвола.
\\\verb!"%3$d, %1$d, %1$d" % [ 1, 2, 3 ] # -> "3, 1, 1"!

Для ассоциативных массивов модификатор <идентификатор> после приставки применяет форматирование для элемента с соответсвующим ключом.
\\\verb!"%<two>d, %<one>d" % { one: 1, two: 2, three: 3 } # -> "2, 1"!
  \include{appassign}
  \hypertarget{appencode}{}
\chapter{Преобразование кодировок}

\begin{keylist}{Принимаемые элементы:}
  
  \firstkey{replace:} текст, использующийся для замены символов. По умолчанию используется \mono{"uFFFD"} для символов Unicode и \mono{"?"} для других символов;
  
  \key{:invalid => :replace} - заменяются ошибочные байты;
  
  \key{:undef => :replace} - заменяются отсутствующие символы;
  
  \key{:fallback => encoding} - изменяется кодировка отсутствующие символов;
  
  \key{:xml => :text} - символы из XML CharData экранируются. Результат может быть использован в HTML 4.0: 
    \begin{itemize}
      \item \verb!'&'! на \verb!'&amp;'! 
      \item \verb!'<'! на \verb!'&lt;'!
      \item \verb!'>'! на \verb!'&gt;'! 
      \item отсутствующие символы на байты вида \verb!&x**! (где * - цифра в шестнадцатеричной системе счисления).
    \end{itemize}
  
  \key{:xml => :attr} - символы из XML AttValue экранируются. Результат ограничивается двойными кавычками и может быть использован для значений свойств в HTML 4.0: 
    \begin{itemize}
      \item \verb!'&'! на \verb!'&amp;'! 
      \item \verb!'<'! на \verb!'&lt;'!
      \item \verb!'>'! на \verb!'&gt;'! 
      \item \verb!'""! на \verb!’&quot;'!
      \item отсутствующие символы на байты вида \verb!&x**! (где * - цифра в шестнадцатеричной системе счисления).
    \end{itemize}
  
  \key{cr_newline: true} - символы LF (\verb!\n!) заменяются на CR (\verb!\r!);
  
  \key{crlf_newline: true} - символы LF (\verb!\n!) заменяются на CRLF (\verb!\r\n!);
  
  \key{universal_newline: true} - символы CR (\verb!\r!) и CRLF (\verb!\r\n!) заменяются на LF (\verb!\n!).  
\end{keylist}
  \hypertarget{apppack}{}
\chapter{Упаковка данных}

Произвольные данные могут быть представлены в виде двоичного текста. Для этого элементы массива (данные) описывают с помощью форматных строк (набора спецсимволов).

\itemtitle{Синтаксис форматных строк}
\begin{itemize}
  \item Пробелы внутри форматной строки игнорируются;

  \item Цифра после спецсимвола соответствует количеству элементов, на которые распространяется его действие;
  
  \item Символ звездочки (*) после спецсимвола распространяет его на все оставшиеся элементы;
  
  \item Спецсимволы sSiIlL могут начинаться с знака подчеркивания или восклицательного знака, означающих что размер типа данных зависит от операционной системы;

  \item Добавление символов > или < позволяет использовать старший или младшие порядки байтов соответственно (l_> или L!<).
\end{itemize}

\begin{keylist}{Целые числа:}
  
  \firstkey{C} - 8-битное целое число без знака (unsigned integer или unsigned char); 
  
  \key{c} - 8-битному целое число со знаком (signed integer или signed char); 
  
  \key{S} - 16-битное целое число без знака (uint_16t); 
  
  \key{s} - 16-битное целое число со знаком (int_16t); 
  
  \key{L} - 32-битное целое число без знака (uint_32t); 
  
  \key{l} - 32-битное целое число со знаком (int_32t); 
  
  \key{Q} - 64-битное целое число без знака (uint_64t); 
  
  \key{S_ (S!)} - целое число без знака минимального размера (unsigned short);
  
  \key{s_ (s!)} - целое число со знаком минимального размера (signed short);
  
  \key{I (I_ или I!)} - целое число без знака (unsigned integer);
  
  \key{i (i_ или i!)} - целое число со знаком (signed integer);
  
  \key{L_ (L!)} - целое число без знака максимального размера (unsigned long);
  
  \key{l_ (l!)} - целое число со знаком максимального размера (signed long);
  
  \key{N} - 32-битное целое число без знака со старшим порядком байтов (для сетей);
  
  \key{n} - 16-битное целое число без знака со старшим порядком байтов (для сетей);
  
  \key{V} - 32-битное целое число без знака с младшим порядком байтов (VAX);
  
  \key{v} - 16-битное целое число без знака с младшим порядком байтов (VAX);
  
  \key{U} - кодовая позиция UTF-8 символа;
  
  \key{w} - BER-кодированное целое число.
\end{keylist}

\begin{keylist}{Десятичные дроби:}
  
  \firstkey{D, d} - число с плавающей точкой двойной точности; 
  
  \key{F, f} - число с плавающей точкой; 
  
  \key{E} - число с плавающей точкой двойной точности, с младшим порядком байтов; 
  
  \key{e} - число с плавающей точкой с младшим порядком байтов; 
  
  \key{G} - число с плавающей точкой двойной точности, со старшим порядком байтов; 
  
  \key{g} - число с плавающей точкой со старшим порядком байтов. 
\end{keylist}

\begin{keylist}{Текст:}
  
  \firstkey{A} - произвольный двоичный текст с удаленными конечными нулями и ASCII пробелами; 
  
  \key{a} - произвольный двоичный текст; 
  
  \key{Z} - произвольный двоичный текст заканчивающемуся нулем;
  
  \key{B} - произвольный двоичный текст (MSB первый); 
  
  \key{b} - произвольный двоичный текст (LSB первый); 
  
  \key{H} - шестнадцатеричный текст (hight nibble первый); 
  
  \key{h} - шестнадцатеричный текст (low nibble первый); 
  
  \key{u} - UU-кодированный тексту; 
  
  \key{M} - MIME-кодированный тексту (RFC2045); 
  
  \key{m} - base64-кодированный текст (RFC2045, если заканчивается 0, то RFC4648); 
  
  \key{P} - указатель на контейнер (текст фиксированной длины); 
  
  \key{p} - указатель на текст, заканчивающийся нулем. 
\end{keylist}

\begin{keylist}{Остальное:}
  
  \firstkey{@} - интерпретатор пропускает указанное целым числом количество байтов; 
  
  \key{X} - интерпретатор продвигается вперед на один байт;
  
  \key{x} - интерпретатор возвращается назад на один байт.
\end{keylist}

  \hypertarget{appdatetime}{}
\chapter{Форматирование времени}

Форматная строка состоит из групп символов вида:

\medskip\noindent\verb!%[модификатор][размер][спецсимвол]!. 

Любой текст, не относящийся к спецсимволам или модификаторам переносится в результат без изменений. 

Размер влияет на количество символов в результате. Если размер меньше, чем необходимое количество символов, то он игнорируется интерпретатором.

\begin{keylist}{Модификаторы:}
  \firstkey{-} - ограничение размера игнорируется;
  \key{_} - для выделения используются пробелы;
  \key{0} - для выделения используются нули (по умолчанию);
  \key{\textasciicircum} - результат в верхнем регистре;
  \key{\#} - изменяет регистр символов.
\end{keylist}

\begin{keylist}{Год:}
  \firstkey{\%Y} - номер года с веком;  
  \begin{verbatim}
  Time.local( 1990, 3, 31 ).strftime "Год: %Y" # -> "Год: 1990"
  Time.local( 1990, 3, 31 ).strftime "Год: %7Y" # -> "Год: 0001990"
  Time.local( 1990, 3, 31 ).strftime "Год: %-7Y" # -> "Год: 1990" 
  Time.local( 1990, 3, 31 ).strftime "Год: %_7Y" # -> "Год:    1990" 
  Time.local( 1990, 3, 31 ).strftime "Год: %07Y" # -> "Год: 0001990"
  \end{verbatim}    
     
  \key{\%G} - номер года с веком. В качестве первого дня в году обрабатывается понедельник; 
  \\\verb!Time.local( 1990, 3, 31 ).strftime "Год: %G" # -> "Год: 1990"!
    
  \key{\%y} - остаток от деления номера года на 100 (от 00 до 99);  
  \\\verb!Time.local( 1990, 3, 31 ).strftime "Год: %y" # -> "Год: 90"!
    
  \key{\%g} - остаток от деления номера года на 100 (от 00 до 99). Первым днем года считается понедельник; 
  \\\verb!Time.local( 1990, 3, 31 ).strftime "Год: %g" # -> "Год: 90"!
    
  \key{\%C} - номер года, разделенный на 100 (20 в 2011);  
  \\\verb!Time.local( 1990, 3, 31 ).strftime "Век: %C" # -> "Век: 19"!
\end{keylist}

\begin{keylist}{Месяц:}
  \firstkey{\%b (\%h)} - аббревиатура названия месяца (три первые английские буквы);   
  \\\verb!Time.local( 1990, 3, 31 ).strftime "Месяц: %b" -> "Месяц: Mar"!    
  
  \key{\%B} - полное названию месяца;
  \begin{verbatim}
  Time.local(1990, 3, 31).strftime "Месяц: %B" # -> "Месяц: March"
  Time.local(1990, 3, 31).strftime "Месяц: %^B" # -> "Месяц: MARCH"
  Time.local(1990, 3, 31).strftime "Месяц: %#B" # -> "Месяц: MARCH  "
  \end{verbatim}    
   
  \key{\%m} - номер месяца (от 01 до 12);  
  \\\verb!Time.local( 1990, 3, 31 ).strftime "Месяц: %m" -> "Месяц: 03"!
\end{keylist}

\begin{keylist}{Неделя:}
  \firstkey{\%U} - номер недели (от 00 до 53). Первое воскресенье года считается началом первой недели;  
  \\\verb!Time.local( 1990, 3, 31 ).strftime "Неделя: %U" -> "Неделя: 12"! 
   
  \key{\%W} - номер недели (от 00 до 53). Первый понедельник года считается началом первой недели;  
  \\\verb!Time.local( 1990, 3, 31 ).strftime "Неделя: %W" -> "Неделя: 13"! 
   
  \key{\%V} - номер недели (от 01 до 53) (номер недели в формате ISO 8601);  
  \\\verb!Time.local( 1990, 3, 31 ).strftime "Неделя: %V" -> "Неделя: 13"! 
\end{keylist}

\begin{keylist}{День:}
  \firstkey{\%j} - день года (от 001 до 366);  
  \\\verb!Time.local( 1990, 3, 31 ).strftime "День: %j" # -> "День: 090"!
   
  \key{\%d} - день месяца;  
  \\\verb!Time.local( 1990, 3, 3 ).strftime "День: %d" # -> "День: 03"!
   
  \key{\%e} - день месяца;  
  \\\verb!Time.local( 1990, 3, 3 ).strftime "День: %e" # -> "День:  3"!
   
  \key{\%a} - аббревиатура дня недели (три первые английские буквы);  
  \\\verb!Time.local( 1990, 3, 31 ).strftime "День: %a" # -> "День: Sat"!
   
  \key{\%A} - полное название дня недели;  
  \\\verb!Time.local( 1990, 3, 31 ).strftime "День: %A" # -> "День: Saturday"!
   
  \key{\%u} - номер дня недели (от 1 до 7, понедельник - 1);  
  \\\verb!Time.local( 1990, 3, 31 ).strftime "День: %u" # -> "День: 6"!
   
  \key{\%w} - номер дня недели (от 0 до 6, воскресенье - 0);  
  \\\verb!Time.local( 1990, 3, 31 ).strftime "День: %w" # -> "День: 6"!
\end{keylist}

\begin{keylist}{Час:}
  \firstkey{\%H} - час дня в 24 часовом формате (от 00 до 23);  
  \\\verb!Time.local( 1990, 3, 31 ).strftime "Час: %H" # -> "Час: 00"!    
   
  \key{\%k} - час дня в 24 часовом формате (от 0 до 23);  
  \\\verb!Time.local( 1990, 3, 31 ).strftime "Час: %k" # -> "Час:  0"!    
   
  \key{\%I} - час дня в 12 часовом формате (от 01 до 12);  
  \\\verb!Time.local( 1990, 3, 31 ).strftime "Час: %I" # -> "Час: 12"!    
   
  \key{\%l} - час дня в 12 часовом формате, с приставкой в виде знака пробела (от 0 до 12);  
  \\\verb!Time.local( 1990, 3, 31 ).strftime "Час: %l" # -> "Час: 12"!    
   
  \key{\%p} - индикатор меридиана ("AM" или "PM"); 
  \\\verb!Time.local( 1990, 3, 31 ).strftime "%I %p" # -> "12 AM"!    
   
  \key{\%P} - индикатор меридиана ("am" или "pm");  
  \\\verb!Time.local( 1990, 3, 31 ).strftime "%I %P" # -> "12 am"!   
\end{keylist}

\begin{keylist}{Минуты и секунды:}
  \firstkey{\%M} - количество минут (от 00 до 59);  
  \\\verb!Time.local( 1990, 3, 31 ).strftime "мин: %M" # -> "мин: 00"!    
   
  \key{\%S} - количество секунд (от 00 до 60);  
  \\\verb!Time.local( 1990, 3, 31 ).strftime "сек: %S" # -> "сек: 00"!    
   
  \key{\%N} - дробная часть секунд.
  \begin{description}
    \item[\%3N] - миллисекунды;
    \item[\%6N] - микросекунды;
    \item[\%9N] - наносекунды (по умолчанию);
    \item[\%12N] - пикосекунды.
  \end{description} 
  
  \key{\%L} - количество миллисекунд (от 000 до 999); 
  \\\verb!Time.local( 1990, 3, 31 ).strftime "мс: %L" # -> "мс: 000"!    
   
  \key{\%s} - количество секунд, прошедших начиная с 1970-01-01 00:00:00 UTC;  
  \\\verb!Time.local( 1990, 3, 31 ).strftime "%s" # -> "638827200"!   
\end{keylist}
  
\begin{keylist}{Форматы:}
  \firstkey{\%D} - форматная строке \verb!"%m/%d/%y"!;
  
  \begin{verbatim}
  Time.local( 1990, 3, 31 ).strftime "Дата: %D"
  # -> "Дата: 03/31/90"
  \end{verbatim} 
     
  \key{\%F} - форматная строке \verb!"%Y-%m-%d"! (время в формате ISO 8601); 
  \begin{verbatim}
  Time.local( 1990, 3, 31 ).strftime "Дата: %F"
  # -> "Дата: 1990-03-31"
  \end{verbatim}   
   
  \key{\%v} - форматная строке \verb!"%e-%b-%Y"! (время в формате VMS); 
  \begin{verbatim}
  Time.local( 1990, 3, 31 ).strftime "Дата: %v"
  # -> "Дата: 31-MAR-1990"
  \end{verbatim}   
   
  \key{\%c} - формат системы;
  \begin{verbatim}
  Time.local( 1990, 3, 31 ).strftime "Система: %c"
  # -> "Система: Sat Mar 31 00:00:00 1990"
  \end{verbatim}
   
  \key{\%r} - форматная строке \verb!"%I:%M:%S %p"!; 
  
  \verb!Time.local( 1990, 3, 31 ).strftime "%r" # -> "12:00:00 AM"!    
   
  \key{\%R} - форматная строке \verb!"%H:%M"!; 
  
  \verb!Time.local( 1990, 3, 31 ).strftime "%R" # -> "00:00"!    
   
  \key{\%T} - форматная строке \verb!"%H:%M:%S"!; 
  
  \verb!Time.local( 1990, 3, 31 ).strftime "%T" # -> "00:00:00"!    
\end{keylist}

\begin{keylist}{Форматирование:}
  \firstkey{\%n} - перевод строки;
  \begin{verbatim}
  Time.local( 1990, 3, 31 ).strftime "%D %n %F %n %c"
  # -> "03/31/90 \textbackslash n 1990-03-31 \textbackslash n Sat Mar 31 00:00:00 1990"
  \end{verbatim}    
   
  \key{\%t} - отступ;
  \begin{verbatim}
  Time.local( 1990, 3, 31 ).strftime "%D %t %F %t %c"
  # -> "03/31/90 \textbackslash t 1990-03-31 \textbackslash t Sat Mar 31 00:00:00 1990"
  \end{verbatim}   
\end{keylist}

\begin{keylist}{Остальное:}
  \firstkey{\%x} - только дата; 
  \\\verb!Time.local( 1990, 3, 31 ).strftime "%x" # -> "03/31/90"!    
   
  \key{\%X} - только время; 
  \\\verb!Time.local( 1990, 3, 31 ).strftime "%X" # -> "00:00:00"!    
   
  \key{\%z} - смещение часового пояса относительно UTC; 
  \\\verb!Time.local( 1990, 3, 31 ).strftime "%z" # -> "+0400"!
  \begin{description}
    \item[\%:z] - часы и минуты разделяются с помощью двоеточия;
    \\\verb!Time.local( 1990, 3, 31 ).strftime "%:z" # -> "+04:00"!
    \item[\%::z] - часы, минуты и секунды разделяются с помощью двоеточия;
    \\\verb!Time.local( 1990, 3, 31 ).strftime "%::z" # -> "+04:00:00"!
  \end{description}    
  
  \key{\%Z} - название временной зоны; 
  \\\verb!Time.local( 1990, 3, 31 ).strftime "%Z" # -> "MSD"!    
   
  \key{\%\%} - знак процента.
\end{keylist}
  \hypertarget{appio}{}
\chapter{Создание потоков}

Вид потока (mode) регулируется с помощью группы модификаторов.

\begin{keylist}{Модификаторы:}
  
  \firstkey{"r"} - только для чтения; 
  
  \key{"r+"} - как для чтения, так и для записи (новые данные вместо старых); 
  
  \key{"w"} - только для записи (новые данные вместо старых). При необходимости создается новый файл; 
  
  \key{"w+"} - как для чтения, так и для записи (новые данные вместо старых). При необходимости создается новый файл; 
  
  \key{"a"} - только для записи (новые данные добавляются к старым). При необходимости создается новый файл; 
  
  \key{"a+"} - как для чтения, так и для записи (новые данные добавляются к старым). При необходимости создается новый файл; 
  
  \key{"b"} - двоичный режим (может использоваться с другими). Чтение из файла выполняется в ASCII кодировке. Используется только для чтения двоичных файлов в Windows; 
  
  \key{"t"} - текстовый режим (может использоваться с другими).
\end{keylist}

При чтении или записи данных используются две кодировки: внутренняя и внешняя. Внешняя кодировка - это кодировка текста внутри потока. По умолчанию она совпадает с кодировкой ОС. Внутренняя кодировка - это кодировка для работы с полученными данными внутри программы. По умолчанию она совпадает с внешней кодировкой. Если внутренняя и внешняя кодировка отличаются, то при чтении данных выполняется автоматическое преобразование из внешней кодировки во внутреннюю, а при записи данных - из внутренней во внешнюю.

После модификатора могут быть указаны внешняя и внутренняя кодировки, разделяемые двоеточием: \verb!"w+:ascii:utf-8"!. Если внутренняя кодировка не указана, то по умолчанию используется внешняя кодировка.

Вид потока также может быть изменен с помощью дополнительного аргумента - массива ключей или набора констант из модуля File::Constants.

\begin{keylist}{Принимаемые элементы:}
  
  \firstkey{mode:} вид создаваемого потока; 
  
  \key{textmode: true} - текстовый режим; 
  
  \key{binmode: true} - двоичный режим; 
  
  \key{autoclose: true} - закрытие файла, после закрытия потока; 
  
  \key{external_encoding:} внешняя кодировка; 
  
  \key{internal_encoding:} внутренняя кодировка. Если внутренняя кодировка не указана, то по умолчанию используется внешняя кодировка; 
  
  \key{encoding:} внешняя и внутренняя кодировки в формате \verb!"external:internal"!;
\end{keylist}
   
Также принимаются элементы, влияющие на \hyperlink{appencode}{\underline{преобразование кодировок}}.
  \hypertarget{appfile}{}
\chapter{Файловая система Linux}

\subsubsection*{Типы файлов}

\begin{description}
  \item[Обычный файл] - файл, позволяющий вводить или выводить данные, а также перемещаться по ним с помощью буферизации (сохранения данных в буфер);

  \item[Жесткая ссылка] - файл, ссылающийся на другой файл. 

  Жесткие ссылки необходимы, чтобы использовать нескольких имен для одного файла. При этом все жесткие ссылки равноправны - файл будет удален только в том случае, если удалены все жесткие ссылки на него. 

  Изменение любой жесткой ссылки повлияет на связанный с ней файл. 

  Создание жестких ссылок возможно только на одном физическом носителе информации (жестком диске, карте памяти и т.д.);

  \item[Символьная ссылка (ярлык)] - файл, ссылающийся на другой файл. 

  Ярлыки необходимы, чтобы использовать нескольких имен для одного файла. При этом существует одно основное имя - файл будет удален только в том случае, если удален основной файл. 

  Изменение любой символьной ссылки повлияет только на саму ссылку. Изменение основного файла повлияет на все связанные с ним ссылки;

  \item[Блочное устройство] - файл, обеспечивающий интерфейс доступа к какому-либо устройству. 

  Ввод и вывод данных в блочные устройства выполняется в виде блоков, размер которых устанавливается блочным устройством. При этом существует возможность перемещаться по данным в пределах блочного устройства. Жесткий диск - это один из примеров блочных устройств;

  \item[Символьное устройство] - файл, обеспечивающий интерфейс доступа к какому-либо устройству. 

  Ввод и вывод данных в символьное устройство выполняется в виде отдельных байтов. Обычно ввод и вывод данных не буферизуется и не существует возможности перемещаться по данным в пределах символьного устройства. Терминалы или модемы - это примеры символьных устройств.

  \item[Сокет] - файл, обеспечивающий коммуникацию между различными процессами, которые могут выполняться на разных компьютерах.
\end{description}

\subsubsection*{Поиск файлов}

Поиск файлов выполняется с помощью путей, передаваемых методам. Путь - это текст или любой другой объект, отвечающий на вызов метода \method{.to_path}.

\itemtitle{Виды путей:}
\begin{description}
  \item[Абсолютный путь] - путь к файлу, начинающийся от корневого каталога или буквы диска. 
  \item[Относительный путь] - путь к файлу, относительно текущего рабочего каталога.
\end{description}

\itemtitle{Синтаксис путей:}
\begin{itemize} 
  \item Для обозначения верхнего уровня базового каталога используется \mono{..}; 
  \item Для обозначения базового каталога используется \mono{.}; 
  \item Для обозначения домашнего каталога используется \mono{\textasciitilde};
  \item Для разделения каталогов в Windows используется символ обратной косой черты (\textbackslash), а в Linux - символ косой черты (/);
  \item Расширение файла от его имени отделяется точкой.
\end{itemize}

Некоторые методы позволяют искать файлы по переданным образцам.
\begin{keylist}{Синтаксис образца:}
  
  \firstkey{*} - соответствует любому файлу или любой группе символов в имени файла; 
  
  \key{**} - соответствует любому каталогу в имени файла (включая символ разделителя); 
  
  \key{?} - соответствует любому одиночному символу в имени файла; 
  
  \key{[...]} - соответствует любому одному символу в имени файла из указанных в квадратных скобках; 
  
  \key{[\textasciicircum...]} - соответствует любому одному символу в имени файла, кроме указанных в квадратных скобках; 
  
  \key{\{ \}}{ - логическое или между символами, разделенными запятыми.}
\end{keylist}

Также существует набор констант, влияющих на поиск
\begin{keylist}{Константы:}
  
  \firstkey{File::FNM_SYSCASE} - чувствительность к регистру зависит от ОС (действует по умолчанию);
  
  \key{File::FNM_CASEFOLD} - игнорирование регистра; 
  
  \key{File::FNM_PATHNAME} - спецсимвол ?, не будет соответствовать косой черте; 
  
  \key{File::FNM_NOESCAPE} - обратная косая черта будет соответствовать самой себе, а не экранировать следующий символ; 
  
  \key{File::FNM_DOTMATCH} - знак точки в имени файла будет считаться частью имени, а не разделителем для расширения (необходимо для поиска скрытых файлов в Linux).
\end{keylist}

Несколько констант могут быть использованы в выражении побитового ИЛИ. Тем, кто хочет разобраться почему это работает, следует изучить двоичную арифметику.

\subsubsection*{Доступ к файлам}

В некоторых файловых системах предусмотрена возможность ограничения доступа пользователей к содержимому файла. При этом обычно выделяют три типа прав: право на чтение, право на запись и право на выполнения. Доступ может быть определен для владельца файла, группы владельцев и для всех остальных пользователей.

Права доступа объявляются с помощью четырех целых чисел. Каждая цифра соответствует двоичному биту, добавляемому к файлу. 
\begin{itemize}
  \item Первая цифра считается дополнительной и определяет либо различные способы запуска файла, либо дополнительное условие для каталогов;  
  \item Оставшиеся три цифры объявляют права доступа владельца, группы владельцев и всех остальных пользователей. 
\end{itemize}

Данные цифровые коды могут быть применены и в Windows. При этом допускается ограничивать доступ только для чтения и записи информации.

\begin{keylist}{Список прав (perm):}
  
  \firstkey{700} - владелец файла имеет право на чтение, запись и выполнение; 
  
  \key{400} - владелец файла имеет право на чтение;
  
  \key{200} - владелец файла имеет право на запись; 
  
  \key{100} - владелец файла имеет право на выполнение (или право на просмотр каталога); 
  
  \key{70} - группа владельцев имеет право на чтение, запись и выполнение; 
  
  \key{40} - группа владельцев имеет право на чтение; 
  
  \key{20} - группа владельцев имеет право на запись; 
  
  \key{10} - группа владельцев имеет право на выполнение (или право на просмотр каталога); 
  
  \key{7} - все остальные пользователи имеют право на чтение, запись и выполнение; 
  
  \key{4} - все остальные пользователи имеют право на чтение; 
  
  \key{2} - все остальные пользователи имеют право на запись; 
  
  \key{1} - все остальные пользователи имеют право на выполнение файла (или право на просмотр каталога).
\end{keylist}

По умолчанию файл выполняется от имени того пользователя, которым файл был запущен. 
\begin{keylist}{Список прав:}
  
  \firstkey{4000} - файл запускается от имени владельца файла; 
  
  \key{2000} - файл запускается от имени группы владельцев; 
  
  \key{1000} - из каталога можно удалить только те файлы, владельцем которых является пользователь.
\end{keylist}

Права доступа определяются после сложения чисел, объявляющих доступ для отдельных категорий пользователей. 

0444 - все пользователи имеют право только на чтение информации из файла. 

Привилегии пользователей определяются с помощью цифровых идентифкаиторов. Каждый пользователь имеет четыре вида идентификаторов;

\begin{description} 
  \item[UID] - реальный идентификатор пользователя, запустившего файл;
  \item[EUID] - действующий идентификатор пользователя, с которым файл выполняется;
  \item[GUID] - реальный идентификатор группы пользователей, запустивших файл;
  \item[EGUID] - действующий идентификатор группы пользователей, с которым файл выполняется;
\end{description}

Для проверки привелегий пользователей обычно используется действующий идентификатор.

Доступ также может быть ограничены с помощью набора констант.

\begin{keylist}{Константы}
  
  \firstkey{File::RDONLY} - открывается только для чтения; 
  
  \key{File::WRONLY} - открывается только для записи; 
  
  \key{File::RDWR} - открывается как для чтения, так и для записи; 
  
  \key{File::APPEND} - запись данных в конец файла (право на запись необходимо устанавливать отдельно); 
  
  \key{File::CREAT} - создание нового файла. Определение владельца выполняется по действующему идентификатору. Определение группы выполняется по идентификатору группы для программы или для базового каталога;
  
  \key{File::DIRECT} - ограничение кэширования содержимого файла;
  
  \key{File::EXCL} - ограничение возможности создания ярлыков; 
  
  \key{File::NONBLOCK} - при работе с файлом процесс выполнения не блокируется; 
  
  \key{File::TRUNC} - новые данные сохраняются вместо существующих (право на запись необходимо устанавливать отдельно); 
  
  \key{File::NOCTTY} - терминал открывается, но управление программой не передается; 
  
  \key{File::BINARY} - двоичный режим; 
  
  \key{File::SYNC, File::DSYNC, File::RSYNC} - cинхронное открытие файла. При записи в поток, он будет блокировать процесс выполнения до тех пор, пока информация не будет реально записана на устройство; 
  
  \key{File::NOFOLLOW} - открываются сами ярлыки, а не связанные с ними файлы; 
  
  \key{File::NOATIME} - время последнего доступа не обновляется. 
\end{keylist}

\backmatter
  \chapter*{Заключение}
  \addcontentsline{toc}{part}{Заключение}
  \epigraph
{Лучший способ разобраться в чем-то до конца - попробовать научить этому компьютер.}
{Дональд Кнут}

Здесь заканчивается введение в язык программирования Ruby. Не думайте, что прочитав эту книгу вы сразу станете писать высоко-нагруженные приложения. Максимум чему вы научились - это программирование небольших скриптов, способных немного облегчить вашу повседневную работу. Еще множество необходимых знаний о рефакторинге, тестировании и отладке, архитектуре и оптимизации (и т.д.) отделяет вас от гордого звания программиста. Могу лишь надеяться, что удовольствия от работы с Ruby поможет преодолеть все эти препятствия и сообщество получит еще одного человека, способного создавать изящные и полезные программы. 

Как вы могли заметить в этой книге приведено очень мало примеров. И это не случайно. Изучение исходного кода уже работающих и востребованных приложений поможет вам быстрее понять все особенности и возможности Ruby. Сделав свой вклад в развитие какой-либо существующей программы вы не только улучшите свои навыки по написанию кода, но также поможете развитию языка. Любые же примеры, которые могут быть приведены в книге изначально очень сильно ограничены. 

Дополнительную информацию о Ruby, можно найти:
\begin{description}
  \item[\href{www.google.com}{\underline{www.google.com}}] - здесь есть все что вам необходимо; 
  \item[\href{http://www.rubygems.org}{\underline{http://www.rubygems.org}}] - хранилище пакетов (использующих Ruby и менеджер пакетов RubyGems);
  \item[\href{http://www.ruby-toolbox.com}{\underline{http://www.ruby-toolbox.com}}] - удобный каталог, классифицирующий существующие gem-ы (пакеты);
  \item[\href{http://github.com}{\underline{http://github.com}}] - сайт для публикации исходного кода. В нем есть раздел и для Ruby-программистов; 
\end{description} 

\end{document}