\hypertarget{appfile}{}
\chapter{Файловая система Linux}

\subsubsection*{Типы файлов}

\begin{description}
  \item[Обычный файл] - файл, позволяющий вводить или выводить данные, а также перемещаться по ним с помощью буферизации (сохранения данных в буфер);

  \item[Жесткая ссылка] - файл, ссылающийся на другой файл. 

  Жесткие ссылки необходимы, чтобы использовать нескольких имен для одного файла. При этом все жесткие ссылки равноправны - файл будет удален только в том случае, если удалены все жесткие ссылки на него. 

  Изменение любой жесткой ссылки повлияет на связанный с ней файл. 

  Создание жестких ссылок возможно только на одном физическом носителе информации (жестком диске, карте памяти и т.д.);

  \item[Символьная ссылка (ярлык)] - файл, ссылающийся на другой файл. 

  Ярлыки необходимы, чтобы использовать нескольких имен для одного файла. При этом существует одно основное имя - файл будет удален только в том случае, если удален основной файл. 

  Изменение любой символьной ссылки повлияет только на саму ссылку. Изменение основного файла повлияет на все связанные с ним ссылки;

  \item[Блочное устройство] - файл, обеспечивающий интерфейс доступа к какому-либо устройству. 

  Ввод и вывод данных в блочные устройства выполняется в виде блоков, размер которых устанавливается блочным устройством. При этом существует возможность перемещаться по данным в пределах блочного устройства. Жесткий диск - это один из примеров блочных устройств;

  \item[Символьное устройство] - файл, обеспечивающий интерфейс доступа к какому-либо устройству. 

  Ввод и вывод данных в символьное устройство выполняется в виде отдельных байтов. Обычно ввод и вывод данных не буферизуется и не существует возможности перемещаться по данным в пределах символьного устройства. Терминалы или модемы - это примеры символьных устройств.

  \item[Сокет] - файл, обеспечивающий коммуникацию между различными процессами, которые могут выполняться на разных компьютерах.
\end{description}

\subsubsection*{Поиск файлов}

Поиск файлов выполняется с помощью путей, передаваемых методам. Путь - это текст или любой другой объект, отвечающий на вызов метода \method{.to_path}.

\itemtitle{Виды путей:}
\begin{description}
  \item[Абсолютный путь] - путь к файлу, начинающийся от корневого каталога или буквы диска. 
  \item[Относительный путь] - путь к файлу, относительно текущего рабочего каталога.
\end{description}

\itemtitle{Синтаксис путей:}
\begin{itemize} 
  \item Для обозначения верхнего уровня базового каталога используется \mono{..}; 
  \item Для обозначения базового каталога используется \mono{.}; 
  \item Для обозначения домашнего каталога используется \mono{\textasciitilde};
  \item Для разделения каталогов в Windows используется символ обратной косой черты (\textbackslash), а в Linux - символ косой черты (/);
  \item Расширение файла от его имени отделяется точкой.
\end{itemize}

Некоторые методы позволяют искать файлы по переданным образцам.
\begin{keylist}{Синтаксис образца:}
  
  \firstkey{*} - соответствует любому файлу или любой группе символов в имени файла; 
  
  \key{**} - соответствует любому каталогу в имени файла (включая символ разделителя); 
  
  \key{?} - соответствует любому одиночному символу в имени файла; 
  
  \key{[...]} - соответствует любому одному символу в имени файла из указанных в квадратных скобках; 
  
  \key{[\textasciicircum...]} - соответствует любому одному символу в имени файла, кроме указанных в квадратных скобках; 
  
  \key{\{ \}}{ - логическое или между символами, разделенными запятыми.}
\end{keylist}

Также существует набор констант, влияющих на поиск
\begin{keylist}{Константы:}
  
  \firstkey{File::FNM_SYSCASE} - чувствительность к регистру зависит от ОС (действует по умолчанию);
  
  \key{File::FNM_CASEFOLD} - игнорирование регистра; 
  
  \key{File::FNM_PATHNAME} - спецсимвол ?, не будет соответствовать косой черте; 
  
  \key{File::FNM_NOESCAPE} - обратная косая черта будет соответствовать самой себе, а не экранировать следующий символ; 
  
  \key{File::FNM_DOTMATCH} - знак точки в имени файла будет считаться частью имени, а не разделителем для расширения (необходимо для поиска скрытых файлов в Linux).
\end{keylist}

Несколько констант могут быть использованы в выражении побитового ИЛИ. Тем, кто хочет разобраться почему это работает, следует изучить двоичную арифметику.

\subsubsection*{Доступ к файлам}

В некоторых файловых системах предусмотрена возможность ограничения доступа пользователей к содержимому файла. При этом обычно выделяют три типа прав: право на чтение, право на запись и право на выполнения. Доступ может быть определен для владельца файла, группы владельцев и для всех остальных пользователей.

Права доступа объявляются с помощью четырех целых чисел. Каждая цифра соответствует двоичному биту, добавляемому к файлу. 
\begin{itemize}
  \item Первая цифра считается дополнительной и определяет либо различные способы запуска файла, либо дополнительное условие для каталогов;  
  \item Оставшиеся три цифры объявляют права доступа владельца, группы владельцев и всех остальных пользователей. 
\end{itemize}

Данные цифровые коды могут быть применены и в Windows. При этом допускается ограничивать доступ только для чтения и записи информации.

\begin{keylist}{Список прав (perm):}
  
  \firstkey{700} - владелец файла имеет право на чтение, запись и выполнение; 
  
  \key{400} - владелец файла имеет право на чтение;
  
  \key{200} - владелец файла имеет право на запись; 
  
  \key{100} - владелец файла имеет право на выполнение (или право на просмотр каталога); 
  
  \key{70} - группа владельцев имеет право на чтение, запись и выполнение; 
  
  \key{40} - группа владельцев имеет право на чтение; 
  
  \key{20} - группа владельцев имеет право на запись; 
  
  \key{10} - группа владельцев имеет право на выполнение (или право на просмотр каталога); 
  
  \key{7} - все остальные пользователи имеют право на чтение, запись и выполнение; 
  
  \key{4} - все остальные пользователи имеют право на чтение; 
  
  \key{2} - все остальные пользователи имеют право на запись; 
  
  \key{1} - все остальные пользователи имеют право на выполнение файла (или право на просмотр каталога).
\end{keylist}

По умолчанию файл выполняется от имени того пользователя, которым файл был запущен. 
\begin{keylist}{Список прав:}
  
  \firstkey{4000} - файл запускается от имени владельца файла; 
  
  \key{2000} - файл запускается от имени группы владельцев; 
  
  \key{1000} - из каталога можно удалить только те файлы, владельцем которых является пользователь.
\end{keylist}

Права доступа определяются после сложения чисел, объявляющих доступ для отдельных категорий пользователей. 

0444 - все пользователи имеют право только на чтение информации из файла. 

Привилегии пользователей определяются с помощью цифровых идентифкаиторов. Каждый пользователь имеет четыре вида идентификаторов;

\begin{description} 
  \item[UID] - реальный идентификатор пользователя, запустившего файл;
  \item[EUID] - действующий идентификатор пользователя, с которым файл выполняется;
  \item[GUID] - реальный идентификатор группы пользователей, запустивших файл;
  \item[EGUID] - действующий идентификатор группы пользователей, с которым файл выполняется;
\end{description}

Для проверки привелегий пользователей обычно используется действующий идентификатор.

Доступ также может быть ограничены с помощью набора констант.

\begin{keylist}{Константы}
  
  \firstkey{File::RDONLY} - открывается только для чтения; 
  
  \key{File::WRONLY} - открывается только для записи; 
  
  \key{File::RDWR} - открывается как для чтения, так и для записи; 
  
  \key{File::APPEND} - запись данных в конец файла (право на запись необходимо устанавливать отдельно); 
  
  \key{File::CREAT} - создание нового файла. Определение владельца выполняется по действующему идентификатору. Определение группы выполняется по идентификатору группы для программы или для базового каталога;
  
  \key{File::DIRECT} - ограничение кэширования содержимого файла;
  
  \key{File::EXCL} - ограничение возможности создания ярлыков; 
  
  \key{File::NONBLOCK} - при работе с файлом процесс выполнения не блокируется; 
  
  \key{File::TRUNC} - новые данные сохраняются вместо существующих (право на запись необходимо устанавливать отдельно); 
  
  \key{File::NOCTTY} - терминал открывается, но управление программой не передается; 
  
  \key{File::BINARY} - двоичный режим; 
  
  \key{File::SYNC, File::DSYNC, File::RSYNC} - cинхронное открытие файла. При записи в поток, он будет блокировать процесс выполнения до тех пор, пока информация не будет реально записана на устройство; 
  
  \key{File::NOFOLLOW} - открываются сами ярлыки, а не связанные с ними файлы; 
  
  \key{File::NOATIME} - время последнего доступа не обновляется. 
\end{keylist}