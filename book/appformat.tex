\hypertarget{appformat}{}
\chapter{Форматные строки}

Форматная строка - это текст, в котором обычные символы перемешаны с специальными синтаксическими конструкциями. Обычные символы переносятся в результат без изменений, а специальные влияют на форматируемые объекты.

Спецсимволы классифицируются по типам влияния на результат. Из них только устанавливающие тип форматирования являются обязательными.

Синтаксис спецсимволов (пробелы между ними не используются, а добавлены только для наглядности):
\\\verb!%модификатор размер .точность тип_форматирования!

Размер влияет на количество символов в результате. По умолчанию в начало текста добавляются дополнительные пробелы.
\\\verb!"%5d"%2 ->"    2"!

Точность влияет на количество цифр после десятичной точки (по умолчанию - 6).

\subsubsection*{Типы форматирования:}

\begin{keylist}{Для целых чисел:}

  \firstkey{b} – преобразует число в десятичную систему счисления. Для отрицательных чисел будет использоваться необходимое дополнение до 2 с символами \mono{..} в качестве приставки.
  \begin{verbatim}
  "%b" % 2 # -> "10"
  "%b" % -2 # -> "..10"
  \end{verbatim}
  
  \key{B} – аналогично предыдущему, но с приставкой 0B в альтернативной нотации.
  
  \key{d (или i или u)} - преобразует число в двоичную систему счисления.
  \\\verb!"%d" % 0x01 # -> "1"!
  
  \key{o} – преобразует число в восьмеричную систему счисления. Для отрицательных чисел будет использоваться необходимое дополнение до 2 с символами .. в качестве приставки.
  \begin{verbatim}
  "%o" % 2 # -> "2"
  "%o" % -2 # -> "..76"
  \end{verbatim}
  
  \key{x} – преобразует число в шестнадцатеричную систему счисления. Для отрицательных чисел будет использоваться необходимое дополнение до 2 с символами .. в качестве приставки.
  \begin{verbatim}
  "%x" % 2 -> "2"
  "%x" % -2 -> "..fe"
  \end{verbatim}
  
  \key{X} – аналогично предыдущему, но с приставкой 0X,  в альтернативной нотации.
\end{keylist}

\begin{keylist}{Для десятичных дробей:}
  \firstkey{e} – преобразует число в экспоненциальную нотацию.
  \\\verb!"%e" % 1.2 # -> "1.200000e+00"!
  
  \key{E} – аналогично предыдущему, но с использованием символа экспоненты E.
  
  \key{f} – округление числа.
  \\\verb!"%.3f" % 1.2 # -> "1.200"!
  
  \key{g} – преобразует число в экспоненциальную нотацию, если показатель степени будет меньше -4 или больше либо равен точности. В других случаях точность определяет количество значащих цифр.
  \begin{verbatim}
  "%.1g" % 1.2 # -> "1"
  "%g" % 123.4 # -> "1e+02"
  \end{verbatim}
  
  \key{G} – аналогично предыдущему, но с использованием символа экспоненты E.

  \key{a} - преобразование числа в виде: знак числа, число в шестнадцатеричной системе счисления с приставкой 0x, символ p, знак показателя степени и показатель степени в десятичной системе счисления.
  \\\verb!"%.3a" % 1.2 # -> "0x1.333p+0"!
  
  \key{A} – аналогично предыдущему, но с использованием приставки 0X и символа~P.
\end{keylist}

\begin{keylist}{Для других объектов:}
  \firstkey{c} - результат будет содержать один символ.
  \\\verb!"%с" % ?h # -> "h"!
  
  \key{p} – вызов метода .inspect для объекта.
  \\\verb!"%p" % ?h # -> "\"h\""!
  
  \key{s} – преобразование текста. Точность определяет количество символов.
  \\\verb!"%.3s" % "Ruby" # -> "Rub"!
\end{keylist}

\begin{keylist}{Модификаторы:}  
  \firstkey{-} - пробелы будут добавляться не в начало, а в конце результата.
  \\\verb!"%-4b" % 2 # -> "10  "!

  \key{0 (для чисел)} - вместо пробелов в результате будут использоваться нули.
  \\\verb!"%04b" % 2 # -> "0010"!
  
  \key{+ (для чисел)} - результат будет содержать знак плюса для положительных чисел. Для oxXbB используется обычная запись отрицательных чисел.
  \\\verb!"%+b" % -2 # -> "-10"!
  
  \key{\# (для эмблем bBoxXaAeEfgG)} - использование альтернативной нотации.
  \begin{itemize}
    \item Для o повышается точность результата до тех пор пока первая цифра не будет нулем, если не используется дополнительное форматирование.
    \\\verb!"%#o" % 2 # -> "02"!

    \item Для xXbB используются соответствующие приставки.
    \\\verb!"%#X" % 2 # -> "0X2"!

    \item Для aAeEfgG используется десятичная точка даже если в этом нет необходимости.
    \\\verb!"%#.0E" % 2 # -> "2.E+00"!

    \item Для gG используются конечные нули.
    \\\verb!"%#G" % 1.2 # -> "1.20000"!
  \end{itemize}
  
  \key{*} - соответствующий спецсимволу объект используется для определения размера. Для отрицательных чисел пробелы добавляются в конце результата.
  \\\verb!"%*d" % [ -2, 1 ] # -> "1 "!  
\end{keylist}

Если форматная строка содержит несколько спецсмволов, то они будут последовательно использоваться для форматируемых объектов, которые должны храниться в индексном или ассоциативном массивах.

Для индексных массивов модификатор \verb!цифра$! изменяет форматирование элемента с соответствующей позицией (начиная с 1). При этом модификатор должен быть указан для каждого спецсимвола.
\\\verb!"%3$d, %1$d, %1$d" % [ 1, 2, 3 ] # -> "3, 1, 1"!

Для ассоциативных массивов модификатор <идентификатор> после приставки применяет форматирование для элемента с соответсвующим ключом.
\\\verb!"%<two>d, %<one>d" % { one: 1, two: 2, three: 3 } # -> "2, 1"!