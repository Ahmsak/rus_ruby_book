\chapter{Дата и время}

Экземпляры класса Time - это абстрактные объекты, содержащие информацию о времени. Время хранится в секундах, начиная с 01.01.1970 00:00 UTC. Системы отсчета времени GMT (время по Гринвичу) и UTC (универсальное время) трактуются как эквивалентные.

При сравнении разных объектов необходимо помнить, что различные пояса могут иметь смещения по времени от UTC.

Добавленные модули: Comparable

\begin{keylist}{Аргументы:}
  
  \firstkey{year} - год; 
  
  \key{month} - месяц: либо целое число от 1 до 12, либо текст, содержащий первые три буквы английского названия месяца (аббревиатуру); 
  
  \key{day} - день месяца: целое число от 1 до 31; 
  
  \key{wday} - день недели: целое число от 0 до 6, начиная с воскресенья; 
  
  \key{yday} - день года: целое число от 1 до 366; 
  
  \key{isdst} - летнее время: логическая величина; 
  
  \key{zone} - временная зона: текст; 
  
  \key{hour} - час: целое число от 0 до 23; 
  
  \key{min} - минуты: целое число от 0 до 59; 
  
  \key{sec} - секунды: целое число или десятичная дробь от 0 до 60; 
  
  \key{usec} - микросекунды: целое число или десятичная дробь от 0 до 999;
\end{keylist}

\begin{methodlist}
  \declare{::new}{\# -> time}
  \begin{alltt}
  (year, month = 1, day = 1, hour = 0, min = 0, sec = 0, usec = 0, zone)
  # -> time\
  \end{alltt}
  Создание объекта. При вызове без аргументов возвращается текущее системное время. Последним аргументом передается смещение относительно UTC в виде текста "+00:00" или количества секунд. По умолчанию берется системное смещение часового пояса.
  \begin{verbatim}
  Time.new 1990, 3, 31, nil, nil, nil, "+04:00" 
  # -> 1990-03-31 00:00:00 +0400\
  \end{verbatim} 
   
  \declare{::now}{\# -> time} 
  Создание объекта для текущего системного времени. 
  \\\verb!Time.now -> 2011-09-17 10:36:26 +0400!
 
  \declare{::at(time)}{\# -> time} 
  \verb!( sec, usec = nil ) # -> time!

  Создание объекта. Принимаются секунды и микросекунды, прошедшие с начала точки отсчета UTC. Время вычисляется с учетом смещения часового пояса. 
  \\\verb!Time.at 1 -> 1970-01-01 03:00:01 +0300!

  \declare{::utc( year, month = 1, day = 1, hour = 0, min = 0, sec = 0, usec = 0)}{\# -> time} 
  \verb!( sec, min, hour, day, month, year, wday, yday, isdst ) # -> time!
  \alias{gm} 
  Создание объекта.   
  \\\verb!Time.utc 1990, 3, 31 -> 1990-03-31 00:00:00 UTC!
 
  \declare{::local( year, month = 1, day = 1, hour = 0, min = 0, sec = 0, usec = 0, zone )}{\# -> time} 
  \verb!(sec, min, hour, day, month, year, wday, yday, isdst, zone) # -> time! 

  \alias{time} 
  Версия предыдущего метода, вычисляющая время с учетом смещения часового пояса. 
  \\\verb!Time.local 1990, 3, 31 -> 1990-03-31 00:00:00 +0400!
\end{methodlist}

\subsection*{Приведение типов} 

\begin{methodlist}
  \declare{.to_s}{\# -> string} 
  \alias{inspect} 
  Преобразование в текст. 
  \\\verb!Time.local( 1990, 3, 31 ).to_s -> "1990-03-31 00:00:00 +0400"!
 
  \declare{.to_a}{\# -> array} 
  Возвращает индексный массив вида:
  \begin{verbatim}
  [ self.sec, self.min, self.hour,
    self.day, self.month, self.year,
    self.wday, self.yday,
    self.isdst, self.zone ]\
  \end{verbatim}   
  \verb!Time.local( 1990, 3, 31 ).to_a # -> [ 0, 0, 0, 31, 3, 1990, 6, 90, true, "MSD" ]!
 
  \declare{.to_i}{\# -> integer} 
  \alias{tv_sec} 
  Возвращает количество секунд прошедших начиная с 1970-01-01 00:00:00 UTC. 
  \\\verb!Time.local( 1990, 3, 31 ).to_i # -> 638827200!
 
  \declare{.to_r}{\# -> rational} 
  Возвращает количество секунд прошедших начиная с 1970-01-01 00:00:00 UTC в виде рациональной дроби.
  \\\verb!Time.local( 1990, 3, 31 ).to_r # -> (638827200/1)!
 
  \declare{.to_f}{\# -> float} 
  Возвращает количество секунд прошедших начиная с 1970-01-01 00:00:00 UTC в виде десятичной дроби.
  \\\verb!Time.local( 1990, 3, 31 ).to_f # -> 638827200.0!
\end{methodlist}

\subsection*{Операторы}

\begin{methodlist}
  \declare{time + sec}{\# -> time2} 
  Прибавляет переданное количество секунд. 
  \\\verb!Time.local( 1990, 3, 31 ) + 3600 # -> 1990-03-31 01:00:00 +0400!
 
  \declare{time - time2}{\# -> float}
  Возвращает разницу в секундах.
  \\\verb!Time.local( 1990, 3, 31 ) - Time.new( 1990, 3, 31 ) -> 0.0!

  \declare{time - sec}{\# -> time2} 
  Отнимает переданное количество секунд. 
  \\\verb!Time.local( 1990, 3, 31 ) - 3600 -> 1990-03-30 23:00:00 +0400 !
 
  \declare{time <=> time2}{} 
  Сравнение 
  \\\verb!Time.local( 1990, 3, 31 ) <=> Time.new( 1990, 3, 31 ) -> 0!
\end{methodlist}

\subsection*{Форматирование}

\begin{methodlist}
  \declare{.strftime(format)}{\# -> string}
  Форматирование времени на основе переданной \hyperlink{appdatetime}{\underline{форматной строки}}.
\end{methodlist}

\subsection*{Изменение времени}

\begin{methodlist}
  \declare{.getutc}{\# -> time}
  \alias{getgm} 
  Возвращает время относительно UTC (без смещения часовых поясов). 
  \\\verb!Time.local( 1990, 3, 31 ).getutc # -> 1990-03-30 20:00:00 UTC!
 
  \declare{.utc}{\# -> self} 
  \alias{gmtime} 
  Версия предыдущего метода, изменяющая значение объекта. 

  \declare{.getlocal( zone = nil )}{\# -> time} 
  Возвращает время, учитывая смещение часового пояса. Смещение может быть явно передано методу (по умолчанию используется системное смещение). 
  \\\verb!Time.local(1990, 3, 31).getutc.getlocal # -> 1990-03-31 00:00:00 +0400!
 
  \declare{.localtime( zone = nil )}{\# -> self} 
  Версия предыдущего метода, изменяющая значение объекта.
\end{methodlist}

\subsection*{Статистика}

\begin{methodlist}
  \declare{.asctime}{\# -> string} 
  \alias{ctime} 
  Возвращает время. 
  \\\verb!Time.local( 1990, 3, 31 ).asctime # -> "Sat Mar 31 00:00:00 1990"!
 
  \declare{.utc_offset}{\# -> integer}
  \alias{gmt_offset, gmtoff} 
  Интерпретатор возвращает смещение часового пояса относительно UTC в секундах. 
  \\\verb!Time.local( 1990, 3, 31 ).utc_offset # -> 14400!
 
  \declare{.zone}{\# -> string}
  Возвращает название временной зоны. 
  \\\verb!Time.local( 1990, 3, 31 ).zone # -> "MSD"!

  \declare{.year}{\# -> integer} 
  Возвращает год. 
  \\\verb!Time.local( 1990, 3, 31 ).year # -> 1990!
 
  \declare{.month}{\# -> integer} 
  \alias{mon} 
  Возвращает номер месяца. 
  \\\verb!Time.local( 1990, 3, 31 ).month # -> 3!
 
  \declare{.yday}{\# -> integer} 
  Возвращает номер дня в году от 1 до 366. 
  \\\verb!Time.local( 1990, 3, 31 ).yday # -> 90!
 
  \declare{.day}{\# -> integer} 
  \alias{mday} 
  Возвращает число. 
  \\\verb!Time.local( 1990, 3, 31 ).day # -> 31!
 
  \declare{.wday}{\# -> integer} 
  Возвращает день недели (от 0 до 6, начиная с воскресенья). 
  \\\verb!Time.local( 1990, 3, 31 ).wday # -> 6!

  \declare{.hour}{\# -> integer} 
  Возвращает час дня (число от 0 до 23). 
  \\\verb!Time.local( 1990, 3, 31 ).hour # -> 0!
 
  \declare{.min}{\# -> integer} 
  Возвращает количество минут (число от 0 до 59). 
  \\\verb!Time.local( 1990, 3, 31 ).min # -> 0!
 
  \declare{.sec}{\# -> integer} 
  Возвращает количество секунд (число от 0 до 60). 
  \\\verb!Time.local( 1990, 3, 31 ).sec # -> 0!
 
  \declare{.subsec}{\# -> integer} 
  Возвращает дробную часть секунд. 
  \\\verb!Time.local( 1990, 3, 31 ).subsec # -> 0!
 
  \declare{.usec}{\# -> integer} 
  \alias{tv_usec} 
  Возвращает количество микросекунд. 
  \\\verb!Time.local( 1990, 3, 31 ).usec # -> 0!
 
  \declare{.time}{}
  \alias{tv_nsec} 
  Возвращает количество наносекунд. 
  \\\verb!Time.local( 1990, 3, 31 ).nsec # -> 0!
\end{methodlist}

\subsection*{Предикаты}

\begin{methodlist}
  \declare{.monday?}{} 
  Проверяет является ли понедельник днем недели. 
  \\\verb!Time.local( 1990, 3, 31 ).monday? # -> false!

  \declare{.tuesday?}{} 
  Проверяет является ли вторник днем недели. 
  \\\verb!Time.local( 1990, 3, 31 ).tuesday? # -> false!
 
  \declare{.wednesday?}{} 
  Проверяет является ли среда днем недели. 
  \\\verb!Time.local( 1990, 3, 31 ).wednesday? # -> false!
 
  \declare{.thursday?}{} 
  Проверяет является ли четверг днем недели. 
  \\\verb!Time.local( 1990, 3, 31 ).saturday? # -> false!
 
  \declare{.friday?}{} 
  Проверяет является ли пятница днем недели. 
  \\\verb!Time.local( 1990, 3, 31 ).friday? # -> false!
 
  \declare{.saturday?}{} 
  Проверяет является ли суббота днем недели. 
  \\\verb!Time.local( 1990, 3, 31 ).saturday? # -> true!
 
  \declare{.sunday?}{} 
  Проверяет является ли воскресенье днем недели. 
  \\\verb!Time.local( 1990, 3, 31 ).saturday? # -> false!

  \declare{.utc?}{}
  \alias{gmt?} 
  Проверяет используется ли время относительно UTC. 
  \\\verb!Time.local( 1990, 3, 31 ).utc? # -> false!
 
  \declare{.dst?}{}
  \alias{isdst} 
  Проверяет используется ли переход на летнее время. 
  \\\verb!Time.local( 1990, 3, 31 ).dst? # -> true!
\end{methodlist}
 
\subsection*{Остальное}

\begin{methodlist}
  \declare{.hash}{\# -> integer} 
  Возвращает цифровой код объекта. 
  \\\verb!Time.local( 1990, 3, 31 ).hash # -> -494674000!

  \declare{.round( precise = 0 )}{ -> time}
  Округляет количество секунд с заданной точностью. Точность определяет размер дробной части. 
  \\\verb!Time.local( 1990, 3, 31 ).round 2 # -> 1990-03-31 00:00:00 +0400!
\end{methodlist}