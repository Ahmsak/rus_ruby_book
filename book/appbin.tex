\hypertarget{appbin}{}
\chapter{Запуск программы}

\begin{keylist}{Ключи:}
  \firstkey{\twominus copyright} - отображение сведений о копирайте; 
  
  \key{\twominus version} - отображение версии интерпретатора;
  
  \key{-(-h)elp} - отображение справочной информации;

  \key{-0[\italy{CODEPOINT}]} - символ перевода строки при чтении из потока (\verb!$/!). Кодовая позиция задается в восьмеричной системе счисления.
  \begin{itemize}
    \item Если число не указано, то строки разделятся не будут; 
    \item Если используется -00, то в качестве разделителя будут использоваться два символа перевода строки, идущих один за другим;
    \item Если используется -0777, то файлы буду обрабатываться как одна большая строка.
  \end{itemize}
   
  \key{-C (-X) \italy{DIR}} - базовый каталог;
  
  \key{\twominus encoding (-E) \italy{EXTERNAL[:INTERNAL]}} - внешняя и внутренняя кодировки;

  \key{-F \italy{PATTERN}} - разделитель для частей текста (\verb!$;!), использующийся при вызове метода \method{.split}. В качестве образца передаются символы или тело регулярного выражения;

  \key{-I \italy{DIRS}} - каталоги, добавляемые в начало \verb!\$LOAD_PATH ($:)! и использующиеся для поиска подключаемых библиотек. Каталоги разделяются двоеточием (Linux) или точкой с запятой (Windows);

  \key{-K \italy{ENCODING}} - внешняя и внутренняя кодировки:
  \begin{description}
    \item[e] - EUC-JP;
    \item[s] - Windows-31J (CP932);
    \item[u] - UTF-8;
    \item[n] - ASCII-8BIT (BINARY).
  \end{description} 
  
  \key{-S} - поиск программы с помощью переменной окружения PATH;
  
  \key{-T[\italy{SECURITY}]} - уровень безопасности для программы (по умолчанию 1);
  
  \key{-U} - UTF-8 в качестве внутренней кодировки;

  \key{-W[\italy{VERBOSE}]} - степень подробности отладочной информации:
  \begin{description}
    \item[0] - без предупреждений, \verb!$VERBOSE! ссылается на nil;
    \item[1] - средний уровень, \verb!$VERBOSE! ссылается на false;
    \item[2 (по умолчанию)] - выводятся все предупреждения, \verb!$VERBOSE! ссылается на true.
  \end{description}    
  
  \key{-a} - автоматическое разделение строк при использовании ключей \verb!-n! или \verb!-p!. В начале каждой итерации цикла выполняется код: \verb/$F = $_.split!/;

  \key{-c}- проверка синтаксиса. Если ошибок не найдено, то в стандартный поток для вывода передается \verb!"Syntax OK"!; 

  \key{-(-d)ebug} - режим отладки, \verb!$DEBUG! ссылается на true. Ключ позволяет писать код для отладки программы, который будет выполняться только если \verb!$DEBUG! ссылается на true;
  \\\verb!{verbatim} if $DEBUG! 
  
  \key{-e \italy{verbatim}} - выполнение произвольного кода;

  \key{-i [\italy{EXT}]} - режим редактирования, позволяющий записывать данные с помощью ARGF. Расширение используется для создания резервных копий; 

  \key{-l} - автоматическое изменение строк кода. При этом, во-первых, \verb!$\! копирует \verb!$/!, и, во-вторых, при чтении строк для каждой вызывается метод \method{.chop!};

  \key{-n} - программа выполняется в теле цикла (каждая строка программы выполняется отдельно):  
  \begin{verbatim}
  while gets 
    # code 
  end\
  \end{verbatim}    
  
  \key{-p} - программа выполняется в теле цикла (каждая строка программы выполняется отдельно, результат выполнения передается в стандартный поток для вывода):  
  \begin{verbatim}
  while gets 
    # code 
  end
  print\
  \end{verbatim}
  
  \key{-r \italy{LIB}} - подключение библиотеки перед выполнением с помощью \method{.require};

  \key{-s} - предварительная обработка аргументов, начинающихся с дефиса. Обработка выполняется до любого обычного аргумента или символов --. Обработанные аргументы будут удалены из ARGV.
  \begin{itemize}
    \item Для аргументов вида \verb!-x=y!, \verb!$x! в теле программы будет ссылаться на y;
    \item Для аргументов вида \verb!-x!, \verb!$x! в теле программы будет ссылаться на true.
  \end{itemize}    
  
  \key{-(-v)erbose} - программа запускается с ключом -w. Если имя программы не указано, то отображается версия интерпретатора. Ключ позволяет писать код для отладки программы, который будет выполняться только если \verb!$VERBOSE! ссылается на true;
  \\\verb!{verbatim} if $VERBOSE!; 

  \key{-w} - отображаются все возможные предупреждения, \verb!$VERBOSE! ссылается на true;

  \key{-x[\italy{DIR}]} - отображает код программы, начиная со строки, начинающейся с \verb/!#/ и содержащей \mono{ruby}. Конец программы должен быть указан с помощью EOF, \textasciicircum D (control-D), \textasciicircum Z (control-Z), или \verb!__END__!. Переданный каталог используется вместо базового;

  \key{-(-y)ydebug} - используется для отладки интерпретатора;

  \key{\twominus disable-... или \twominus enable-...} - включение или отключение:  
  \begin{description}
    \item[gems] - поиск подключаемых библиотек в пакетах;
    \item[rubyopt] - использования переменной окружения RUBYOPT;
    \item[all] - выполнение двух перечисленных действий.
  \end{description}
  
  \key{--dump \italy{TARGET}} - используется для отладки интерпретатора.
\end{keylist}

\begin{keylist}{Переменные окружения:}
  \firstkey{RUBYLIB} - список каталогов, используемых для поиска подключаемых библиотек; 

  \key{RUBYOPT} - список ключей, используемых для запуска программ. Могут быть добавлены только ключи -d, -E, -I, -K, -r, -T, -U, -v, -w, -W, \twominus debug, \twominus disable-... и \twominus enable-...; 

  \key{RUBYPATH} - список каталогов, в которых выполняется поиск программы; 
  
  \key{RUBYSHELL} - путь к оболочке ОС. Переменная действительна только для mswin32, mingw32, и OS/2; 
  
  \key{PATH} - путь к интерпретатору.
\end{keylist}

Доступ к перемнным окружения может быть получен с помощью ENV, объекта, подобный ассоциативному массиву. Для объекта определен метод \method{.to_hash}, преобразующий его в настоящий ассоциативный массив.

\begin{keylist}{Глобальные переменные:}
  \firstkey{\$F} - последний результат выполнения выражения \verb!$_.split!. Переменная определена, если программа запущена с ключами -a, -n или -p;
  \key{\$-W} - степень подробности предупреждений; 
  \key{\$-i} - расширения, используемое для резервных копий;
  \key{\$-d} -> bool; 
  \key{\$-l} -> bool; 
  \key{\$-v} -> bool; 
  \key{\$-p} -> bool; 
  \key{\$-a} -> bool;
  \key{\$; (\$-F)} - разделитель частей текста при вызове метода \method{.split} (по умолчанию nil);
  \key{\$/ (\$-0)} - символ перевода строки (по умолчанию - \verb!"\n"!);
  \key{\$\textbackslash} - разделитель, добавляемый при передаче объектов в поток (по умолчанию nil); 
  \key{\$,} - разделитель записываемых элементов(по умолчанию nil).
  \key{\$\textasciitilde} - экземпляр класса MatchData для последнего поиска совпадений; 
  \key{\$\&} - текст найденного совпадения; 
  \key{\$`} - текст перед найденным совпадением; 
  \key{\$'} - текст после найденного совпадения; 
  \key{\$+} - текст совпадения с группой; 
  \key{\$1, \$2, \$3, \$4, \$5, \$6, \$7, \$8, \$9} - текст совпадения с группой, имеющей соответствующий индекс;
  \key{\$LOAD_FEAUTURES (\$")} - массив всех подключенных библиотек;
  \key{\$LOAD_PATH (\$:, \$-I)} - массив каталогов для поиска подключаемых библиотек;
  \key{\$.} - позиция последней прочитанной строки из потока; 
  \key{\$_} - последняя прочитанная строка из потока;
  \key{\$!} - последнее вызванное событие; 
  \key{\$@} - расположение последнего вызванного события;
  \key{\$DEBUG} - true, если программа запущена с ключом \verb!-(-d)ebug!;
 
  \key{\$VERBOSE (\$-v, \$-w)}
  \begin{itemize}
    \item nil, если программа запущена с ключом \verb!-W0!;
    \item true, если программа запущена с ключами \verb!-w! или \verb!-(-v)erbose!;
    \item false в остальных случаях.
  \end{itemize}    
  
  \key{\$\$} - идентификатор текущего процесса;
  \key{\$?} - статус последнего выполненного процесса;
  \key{\$FILENAME} - относительный путь к текущему файлу в ARGF. При взаимодействии с стандартным потоком для ввода возвращается \verb!"-"!;
  \key{\$SAFE} - текущий уровень безопасности.
\end{keylist}    