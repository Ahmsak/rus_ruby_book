\epigraph
{Лучший способ разобраться в чем-то до конца - попробовать научить этому компьютер.}
{Дональд Кнут}

Здесь заканчивается введение в язык программирования Ruby. Не думайте, что прочитав эту книгу вы сразу станете писать высоко-нагруженные приложения. Максимум чему вы научились - это программирование небольших скриптов, способных немного облегчить вашу повседневную работу. Еще множество необходимых знаний о рефакторинге, тестировании и отладке, архитектуре и оптимизации (и т.д.) отделяет вас от гордого звания программиста. Могу лишь надеяться, что удовольствия от работы с Ruby поможет преодолеть все эти препятствия и сообщество получит еще одного человека, способного создавать изящные и полезные программы. 

Как вы могли заметить в этой книге приведено очень мало примеров. И это не случайно. Изучение исходного кода уже работающих и востребованных приложений поможет вам быстрее понять все особенности и возможности Ruby. Сделав свой вклад в развитие какой-либо существующей программы вы не только улучшите свои навыки по написанию кода, но также поможете развитию языка. Любые же примеры, которые могут быть приведены в книге изначально очень сильно ограничены. 

Дополнительную информацию о Ruby, можно найти:
\begin{description}
  \item[\href{www.google.com}{\underline{www.google.com}}] - здесь есть все что вам необходимо; 
  \item[\href{http://www.rubygems.org}{\underline{http://www.rubygems.org}}] - хранилище пакетов (использующих Ruby и менеджер пакетов RubyGems);
  \item[\href{http://www.ruby-toolbox.com}{\underline{http://www.ruby-toolbox.com}}] - удобный каталог, классифицирующий существующие gem-ы (пакеты);
  \item[\href{http://github.com}{\underline{http://github.com}}] - сайт для публикации исходного кода. В нем есть раздел и для Ruby-программистов; 
\end{description} 