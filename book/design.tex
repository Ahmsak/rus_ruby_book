\chapter{Подключение библиотек}

Для облегчения повторного использования кода программу принято разделять на библиотеки. Библиотека - это набор модулей и классов, которые планируется неоднократно использовать. Обычно каждая библиотека содержит один модуль или класс.

После подключения библиотеки ее можно использовать так, как если бы содержащийся в ней код был частью программы. Обычно подключение необходимых библиотек выполняется в самом начале кода, после объявления кодировки.

Поиск всех подключаемых библиотек происходит в заранее определенных каталогах. Каталоги должны быть сохранены в глобальном массиве \verb!$LOAD_PATH ($:)!. Поиск файла выполняется, начиная с первого элемента.

\begin{methodlist}
  \declare{.require(path)}{\# -> bool}
  Подключение переданной библиотеки. Подключенные библиотеки сохраняются в массиве \verb!$LOAD_FEAUTURES ($”)!. Одну библиотеку можно подключить только один раз. Уровень безопасности подключаемой библиотеки должен быть равен 0.

  При передаче библиотеки расширение обычно не указывается. По умолчанию обрабатывается расширение \verb!.rb!. Если файла с таким расширением не найдено, то будет произведен поиск бинарного файла с тем же именем (например, с расширениями \verb!.so! или \verb!.dll!).
   
  Возвращается результат подключения библиотеки. 

  \declare{.require_relative(path)}{\# -> bool}
  Версия метода, аналогичная предыдущему. Поиск библиотеки выполняется в базовом каталоге программы.
 
  \declare{.load( path, anonym = false )}{} 
  Подключение библиотеки. Одну библиотеку можно подключать бесконечное число раз. В имени файла должно быть указано его расширение.

  Библиотека может быть подключена в теле анонимного модуля. В этом случае она не будет влиять на глобальную область видимости основной программы. 
 
  \declare{.autoload( name, path )}{\# -> nil} 
  Подключение библиотеки при использовании переданной константы. Данный метод используется для автоматизации подключения библиотек.
 
  \declare{.autoload?(name)}{\# -> path} 
  Возвращает название библиотеки, которая будет подключена при использовании переданной константы. Если такая библиотека не объявлена, то возвращается nil. 
 
  \declare{module.autoload( name, path )}{\# -> nil} 
  Подключение библиотеки при использовании переданной константы в теле модуля. Данный метод используется для автоматизации подключения библиотек.
 
  \declare{module.autoload?(name)}{\# -> path} 
  Возвращает название библиотеки, которая будет подключена при использовании переданной константы  теле модуля. Если такая библиотека не объявлена, то возвращается nil. 
\end{methodlist}