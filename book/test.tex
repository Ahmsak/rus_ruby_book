\chapter{Тестирование}

\epigraph
{Если вы хотите улучшить программу, вы должны не тестировать больше, а программировать лучше.}{}

Любая, даже самая маленькая программа не может гарантировать правильной работы. Тестирование программ и исправление обнаруженных ошибок - один из самых трудоемких этапов разработки. Она не заканчивается и после публикации приложения. 

Исправив самые распространенные ошибки на этапе разработки программист сохранит огромное количество нервов и времени будущим пользователям. 

Обзор кода часто эффективнее, чем тестирование, но выполнение тестов позволяет проверить влияние внесенных изменений.

Главная задача тестирования - это выявление ошибок, а не их устранение. Устранение дефектов - это самый дорогой и длительный этап разработки. Легче сразу создать высококачественную программу, чем создать низкокачественную и исправлять ее. Повышение качества программы снижает затраты на ее разработку.

Существует множество подходов к тестированию приложения, но в основном грамотное тестирование - процесс прежде всего творческий.

В общем случае тестирование приложения разделяется на четыре уровня:
\begin{description}
  \item[Модульное тестирование] - тестирование минимально возможного фрагмента кода (он же блочный или unit test); 
  \item[Интеграционное тестирование] - тестирование взаимодействиями между различными элементами приложения; 
  \item[Функциональное тестирование] - тестирование задач, выполняемых с помощью приложения; 
  \item[Тестирование производительности.]
\end{description}

Одна из популярных техник тестирования - разработка приложения через его тестирование (TDD, Test-Drive Development). Использование этой техники разделено на следующие этапы: 
\begin{enumerate}
  \item Написание кода, тестирующего часть приложения; 
  \item Выполнение теста. Получение отрицательного результата. Это необходимо для проверки корректности теста; 
  \item Написание части кода приложения; 
  \item Выполнение теста. Получение положительного результата.
\end{enumerate}

Создание тестов перед кодом фокусирует внимание на требованиях к программе (т.е. понимания для чего она создается).

Для тестирования программы может быть использована команда \verb!testrb! (исполняемый файл на Ruby, использующий стандартную библиотеку Test::Unit). 

Программа принимает путь к файлу, выполняет содержащуюся в нем программу и выводит информацию о выполнении (время выполнения, количество выполненных тестов, результат их выполнения). 

Для проверки выполнения небольших кусков кода может быть использован интерактивный терминал irb. 