\hypertarget{appregexp}{}
\chapter{Синтаксис регулярных выражений}

\begin{keylist}{Модификаторы:}
  
  \firstkey{i} – поиск будет выполняться без учета регистра символов;
  
  \key{m} – поиск будет выполняться в многострочном режиме. Точка в теле регулярного выражения будет соответствовать также и символу перевода строки;
  
  \key{x} – пробельные символы (пробел, отступ, перевод строки) в теле регулярного выражения будут игнорироваться интерпретатором;
  
  \key{o} – интерполяция в теле регулярного выражения будет выполняться только один раз, перед началом поиска;
  
  \key{u, e, s, n} – тело регулярного выражения будет обрабатываться в указанной кодировке. Соответственно: u - UTF-8, e - EUC-JP , s - Windows-31J , n - ASCII-8BIT.
\end{keylist}

\begin{keylist}{ASCII символы:}
  
  \firstkey{.} - соответствует любому символу в тексте (кроме символа перевода строки в однострочном режиме поиска);
  
  \key{\textbackslash w} - соответствует любой букве, цифре или знаку подчеркивания;  
  \key{\textbackslash W} - соответствует любому символу, кроме букв, цифр или знаков подчеркивания;
  
  \key{\textbackslash s} - соответствует любому пробельному символу (пробел, отступ, перевод строки);
  
  \key{\textbackslash S} - соответствует любому символу, кроме пробельных;
  
  \key{\textbackslash d} - соответствует любой десятичной цифре;
  
  \key{\textbackslash D} - соответствует любому символу, кроме десятичных цифр;
  
  \key{\textbackslash h} - соответствует любой шестнадцатеричной цифре;
  
  \key{\textbackslash H} - соответствует любому символу, кроме шестнадцатеричных цифр.
\end{keylist}

\begin{keylist}{Юникод символы:}

\firstkey{[[:класс:]]} -  соответствует любому символу, входящему в класс:
  \begin{description}
    \item[alnum]  – буквы и цифры;
    \item[alpha]  – буквы;
    \item[ascii]  – ASCII символы;
    \item[blank]  – пробел и отступ;
    \item[cntrl]  – эмблемы составного текста;
    \item[digit]  – десятичные цифры;
    \item[graph]  – буквы, цифры и знаки препинания;
    \item[lower]  – строчные буквы;
    \item[print]  – буквы, цифры, знаки препинания и пробел;
    \item[punct]  – знаки препинания;
    \item[space]  – пробельные символы (пробел, отступ, перевод строки);
    \item[upper]  – прописные буквы;
    \item[word]   – буквы, цифры и специальные знаки препинания (знак подчеркивания);
    \item[xdigit] – шестнадцатеричные цифры;
  \end{description}

\key{\textbackslash p\{класс\}} - соответствует любому символу, входящему в класс.

{\bf \textbackslash p\{\textasciicircum класс\}} - соответствует любому символу, кроме входящих в класс.
  \begin{description}
    \item[Alnum]    – буквы и цифры; 
    \item[Alpha]    – буквы;
    \item[Any]      – Unicode символы;
    \item[ASCII]    – ASCII символы;
    \item[Assigned] – свободные цифровые коды;
    \item[Blank]    – пробел и отступ;
    \item[Cntrl]    – эмблемы составного текста;
    \item[Digit]    – десятичные цифры;
    \item[Graph]    – буквы, цифры и знаки препинания;
    \item[Lower]    – строчные буквы;
    \item[Print]    – буквы, цифры, знаки препинания и пробел;
    \item[Punct]    – знаки препинания;
    \item[Space]    – пробельные символы (пробел, отступ, перевод строки);
    \item[Upper]    – прописные буквы;
    \item[Word]     – буквы, цифры и специальные знаки препинания (знак подчеркивания);
    \item[Xdigit]   – шестнадцатеричные цифры.
  \end{description}
\end{keylist}

Юникод-классы символов:

C - остальные символы; Cc - спецсимволы; Cf - спецсимволы, влияющие на форматирование; Сn – свободные цифровые коды; Co - логотипы; Cs - символы-заменители;

L – буквы; Ll - строчные буквы; Lm - особые символы; Lo - остальные символы; Lt - буквы в начале слова; Lu - прописные буквы;

M – символы, использующиеся в связке; Mn - символы, изменяющие другие символы; Mc - специальные модификаторы, занимающие отдельную позицию в тексте; Me - символы, внутри которых могут находиться другие символы;

N - цифры; Nd - десятичные цифры; Nl - римские цифры; No - остальные цифры;

P - знаки препинания; Pc - специальные знаки препинания; Pd - дефисы и тире; Ps - открывающие скобки; Pe - закрывающие скобки; Pi - открывающие кавычки; Pf - закрывающие кавычки; Po - остальные знаки препинания;

S - декоративные символы; Sm - математические символы; Sc - символы денежных единиц;	Sk - составные декоративные символы; So - остальные декоративные символы;

Z - разделители, не имеющие графического представления; Zs - пробелы; Zl - перевод строки; Zp - перевод параграфа.

Также можно указать класс, определяющий алфавит. Поддерживаемые алфавиты: Arabic, Armenian, Balinese, Bengali, Bopomofo, Braille, Buginese, Buhid, Canadian_Aboriginal, Carian, Cham, Cherokee, Common, Coptic, Cuneiform, Cypriot, Cyrillic, Deseret, Devanagari, Ethiopic, Georgian, Glagolitic, Gothic, Greek, Gujarati, Gurmukhi, Han, Hangul, Hanunoo, Hebrew, Hiragana, Inherited, Kannada, Katakana, Kayah_Li, Kharoshthi, Khmer, Lao, Latin, Lepcha, Limbu, Linear_B, Lycian, Lydian, Malayalam, Mongolian, Myanmar, New_Tai_Lue, Nko, Ogham, Ol_Chiki, Old_Italic, Old_Persian, Oriya, Osmanya, Phags_Pa, Phoenician, Rejang, Runic, Saurashtra, Shavian, Sinhala, Sundanese, Syloti_Nagri, Syriac, Tagalog, Tagbanwa, Tai_Le, Tamil, Telugu, Thaana, Thai, Tibetan, Tifinagh, Ugaritic, Vai, and Yi.

\begin{keylist}{Группировка символов:}
  
  \firstkey{...} - cоответствует любому символу из ограниченных квадратными скобками. Внутри квадратных скобок могут быть использованы диапазоны символов (a-z);
  
  \key{\textasciicircum...} - соответствует любому символу, кроме ограниченных квадратными скобками.
  
  \key{(?:...)} - символы объединяются в группу и используются как одна логическая единица;
  
  \key{(...)} - символы объединяются в группу и используются как одна логическая единица. Группе будет присвоен порядковый номер (от 1 до 9);
  
  \key{(?<идентификатор>...)} - символы объединяются в группу и используются как одна логическая единица. Группе будет присвоен указанный идентификатор.
\end{keylist}

Если лексема регулярного выражения использовалось на месте левого операнда, тогда, после выполнения выражения, идентификаторы групп ссылаются на текст совпадения, или на nil, если совпадений не найдено. Идентификаторы групп объявляются как локальные переменные.

Глобальные переменные от \$1 до \$9 также ссылаются на текст совпадения с группой, имеющей указанный номер.

\begin{keylist}{Группы символов:}
  
  \firstkey{\textbackslash целое_число} - соответствует тексту совпадения с группой, имеющей указанный номер;
  
  \key{\textbackslash K <идентификатор>} - соответствует тексту совпадения с группой, имеющей указанный идентификатор;
  
  \key{\textbackslash g <...>} - соответствует тексту совпадения с группой, имеющей указанный порядковый номер или идентификатор.
\end{keylist}

\begin{keylist}{Повторы:}
  
  \firstkey{…?} - соответствует от 0 до 1 повторам символа или группы;
  
  \key{...*} - соответствует 0 и более повторам символа или группы. Результат поиска содержит максимально возможное совпадение (жадный алгоритм);
  
  \key{...*?} - соответствует 0 и более повторам символа или группы. Результат поиска содержит минимально возможное совпадение (не жадный алгоритм);
  
  \key{...+} - соответствует 1 и более повторам символа или группы. Результат поиска содержит максимально возможное совпадение (жадный алгоритм);
  
  \key{...+?} - соответствует 1 и более повторам символа или группы. Результат поиска содержит минимально возможное совпадение (не жадный алгоритм);
  
  \key{...\{a, b\}} - соответствует от a до b повторам символа или группы. Результат поиска содержит максимально возможное совпадение (жадный алгоритм);
  
  \key{...\{a, b\}?} - соответствует от a до b повторам символа или группы. Результат поиска содержит минимально возможное совпадение (не жадный алгоритм).
  
  В обоих случаях допускается отсутствие a, b, или запятой.
\end{keylist}

\begin{keylist}{Положение в тексте:}
  
  \firstkey{\textasciicircum...} - соответствует символу или группе в начале строки;
  
  \key{...\$} - соответствует символу или группе в конце строки;
  
  \key{\textbackslash A...} - соответствует символу или группе в начале текста;
  
  \key{...\textbackslash z} - соответствует символу или группе в конце текста;
  
  \key{...\textbackslash Z} - соответствует символу или группе в конце текста или перед последним символом перевода строки, замыкающим текст;
  
  \key{...\textbackslash b} - соответствует символу или группе в конце слова;
  
  \key{\textbackslash b...} - соответствует символу или группе в начале слова;
  
  \key{...\textbackslash B} - соответствует символу или группе в любом месте, кроме конца слова;
  
  \key{\textbackslash B...} - соответствует символу или группе в любом месте, кроме начала слова.
\end{keylist}

\begin{keylist}{Логические условия:}
  
  \firstkey{№1|№2} - соответствует либо символам №1, либо символам №2;
  
  \key{№1(?=№2)} - соответствует символам №1, если символы №2 встречаются далее по тексту (позитивное заглядывание вперед);
  
  \key{№1(?!№2)} - соответствует символам №1, если символы №2 не встречаются далее по тексту (негативное заглядывание вперед);
  
  \key{№1(?<=№2)} - соответствует символам №1, если символы №2 встречаются в предыдущей части текста (позитивное заглядывание назад);
  
  \key{№1(?<!№2)} - соответствует символам №1, если символы №2 не встречаются в предыдущей части текста (негативное заглядывание вперед).
\end{keylist}

\begin{keylist}{Остальное:}
  
  \firstkey{(?\#...)} - игнорируемый комментарий;
  
  \key{(?>...)} - в любом случае соответствует указанным символам;
  
  \key{(?№1-№2)} - применяет модификаторы №1 и отменяет модификаторы №2 для дальнейшего поиска;
  
  \key{(?№1-№2:...)} - применяет модификаторы №1 и отменяет модификаторы №2 для символов или групп.
\end{keylist}