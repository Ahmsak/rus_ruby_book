\hypertarget{appencode}{}
\chapter{Преобразование кодировок}

\begin{keylist}{Принимаемые элементы:}
  
  \firstkey{replace:} текст, использующийся для замены символов. По умолчанию используется \mono{"uFFFD"} для символов Unicode и \mono{"?"} для других символов;
  
  \key{:invalid => :replace} - заменяются ошибочные байты;
  
  \key{:undef => :replace} - заменяются отсутствующие символы;
  
  \key{:fallback => encoding} - изменяется кодировка отсутствующие символов;
  
  \key{:xml => :text} - символы из XML CharData экранируются. Результат может быть использован в HTML 4.0: 
    \begin{itemize}
      \item \verb!'&'! на \verb!'&amp;'! 
      \item \verb!'<'! на \verb!'&lt;'!
      \item \verb!'>'! на \verb!'&gt;'! 
      \item отсутствующие символы на байты вида \verb!&x**! (где * - цифра в шестнадцатеричной системе счисления).
    \end{itemize}
  
  \key{:xml => :attr} - символы из XML AttValue экранируются. Результат ограничивается двойными кавычками и может быть использован для значений свойств в HTML 4.0: 
    \begin{itemize}
      \item \verb!'&'! на \verb!'&amp;'! 
      \item \verb!'<'! на \verb!'&lt;'!
      \item \verb!'>'! на \verb!'&gt;'! 
      \item \verb!'""! на \verb!’&quot;'!
      \item отсутствующие символы на байты вида \verb!&x**! (где * - цифра в шестнадцатеричной системе счисления).
    \end{itemize}
  
  \key{cr_newline: true} - символы LF (\verb!\n!) заменяются на CR (\verb!\r!);
  
  \key{crlf_newline: true} - символы LF (\verb!\n!) заменяются на CRLF (\verb!\r\n!);
  
  \key{universal_newline: true} - символы CR (\verb!\r!) и CRLF (\verb!\r\n!) заменяются на LF (\verb!\n!).  
\end{keylist}