\chapter{Псевдослучайные числа}

Для генерации псевдослучайных чисел в Ruby используется алгоритм "Вихрь Мерсена", предложенный в 1997 году Мацумото и Нисимурой. Его достоинствами являются колоссальный период \verb!(2 ** 19937 - 1)!, равномерное распределение в 623 измерениях, быстрая генерация случайных чисел (в 2-3 раза быстрее, чем стандартные генераторы). Однако, существуют алгоритмы, распознающие последовательность, порождаемую "Вихрем Мерсенна", как неслучайную.

Для получения псевдослучайных чисел в Ruby предоставлен класс Random.

\begin{methodlist}
  \declare{::new( seed = Random.new_seed )}{\# -> random} 
  Создание генератора. Объект, переданный методу, необходим для исключения повторяющихся чисел. Его также называют "соль". 
  \\\verb!Random.new # -> #<Random:0xa0a1fa4>! 

  \declare{::rand( number = 0 )}{\# -> number2} 
  Проверяет результат \verb!number.to_i.abs!: если результат выполнения равен нулю или ссылается на nil, то в результате возвращается псевдослучайная десятичная дробь в диапазоне \verb!0.0...1.0!. В другом случае возвращается псевдослучайное число в диапазоне \verb!0...number.to_i.abs!. 
  \\\verb!Random.rand # -> 0.8736231696463861!

  \declare{rand( number = 0 )}{\# -> number2}
  Частный метод экземпляров из модуля Kernel, аналогичный предыдущему.

  \declare{::new_seed}{\# -> integer} 
  Возвращает новое число для генерации псевдослучайных чисел. 
  \\\verb!Random.new_seed # -> 69960780063826734370396971659065074316!
 
  \declare{::srand( number = 0 )}{\# -> number} 
  Возвращает новое число для генерации псевдослучайных чисел. Если аргумент равен нулю, то используется время вызова, идентификатор процесса и порядковый номер вызова. С помощью аргумента программа может быть детерминирована во время тестирования. В результате возвращается предыдущее значение.

  \declare{srand( number = 0 )}{\# -> number}
  Частный метод экземпляров из модуля Kernel, аналогичный предыдущему.
\end{methodlist}

\subsection*{Генераторы}

\begin{methodlist}
  \declare{.bytes(bytesize)}{\# -> string} 
  Возвращает случайный двоичный текст. 
  \\\verb!Random.new.bytes 2 -> "\xA1W"!

  \declare{.rand( object = nil )}{\# -> number}
  \begin{itemize}
    \item Если передано целое число, то возвращается псевдослучайное число в диапазоне \verb!0...object!;
    \item Если передано отрицательное число или ноль, то вызывается ошибка;
    \item Если передана десятичная дробь, то возвращается псевдослучайная десятичная дробь в диапазоне \verb!0.0...object!;
    \item Если передан диапазон, то возвращается случайный элемент диапазона. Для начальной и конечной границ должны быть определены операторы - (разность) и + (сумма);
    \item В остальных случаях вызывается ошибка.
  \end{itemize}   

  \declare{.seed}{\# -> integer} 
  Возвращает число, использующееся для генерации псевдослучайных чисел. 
  \\\verb!Random.new.seed # -> 173038287409845379387953855893202182131!
\end{methodlist}